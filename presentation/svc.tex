\section{Serial Verb Constructions}

\begin{frame}{Serial Verb Constructions: Terms and Definitions}
\begin{itemize}
\item Generally difficult to define what counts as a serial verb construction (SVC) cross-linguistically \citep{aikhenvalddixon2006}
\item Easier to make a definition applicable to a specific language \pause
\item My functional definition:
\end{itemize}
\ex~[exno=15]
Any clause containing two or more verbs without an overt coordinator and where the verbs share the semantic interpretation of the second position clausal inflection is a serial verb construction.
\xe
\vspace{-20pt}
\begin{itemize}
\item A clause is bounded by the scope of the second position enclitics
\end{itemize}
\end{frame}

\begin{frame}{Serial Verb Constructions: Type I}
Type I: Simultaneity

\begin{itemize}
\item Actions must occur at the same time
\item Often but not always include a verb of motion
\item No ordering restrictions
\end{itemize}

\ex[exno=16]
\begingl
\glpreamble \textcolor{blue}{ʔuucuʔuk}w̓it̓asaḥ \textcolor{blue}{yaacuk} c̓uumaʕas. //
\gla \textcolor{blue}{ʔuucuʔuk}-w̓it̓as=(m)aˑḥ \textcolor{blue}{yaacuk} c̓uumaʕas //
\glb \textcolor{blue}{go.to.\textsc{dr}}-going.to=\textsc{real.1sg} \textcolor{blue}{walk.\textsc{dr}} Port.Alberni //
\glft `I'm going to walk to Port Alberni.' (\textbf{B}, Bob Mundy) //
\endgl \label{ex:walktoalberni}
\xe

\ex[exno=17]~
\begingl
\glpreamble \textcolor{blue}{ʔanasł}intwaʔš \textcolor{blue}{t̓awiłšƛ}. //
\gla \textcolor{blue}{ʔana-siła}=int=waˑʔš \textcolor{blue}{t̓awił-šiƛ} //
\glb \textcolor{blue}{only-do}=\textsc{pst}=\textsc{hrsy.3} \textcolor{blue}{lie.down-\textsc{pf}} //
\glft `He just laid down.' (\textbf{Q}, Sophie Billy) //
\endgl \label{ex:justliedown}
\xe
\end{frame}

\begin{frame}[fragile]{Serial Verb Constructions: Type I}

\begin{itemize}
\item A verb can be separated from its object by the intervening VP.
\end{itemize}

\ex[exno=16]
\begingl
\glpreamble \textcolor{red}{ʔuucuʔuk}w̓it̓asaḥ \textcolor{blue}{yaacuk} \textcolor{red}{c̓uumaʕas}. //
\gla \textcolor{red}{ʔuucuʔuk}-w̓it̓as=(m)aˑḥ \textcolor{blue}{yaacuk} \textcolor{red}{c̓uumaʕas} //
\glb \textcolor{red}{go.to.\textsc{dr}}-going.to=\textsc{real.1sg} \textcolor{blue}{walk.\textsc{dr}} \textcolor{red}{Port.Alberni} //
\glft `I'm going to walk to Port Alberni.' (\textbf{B}, Bob Mundy) //
\endgl \label{ex:walktoalberni}
\xe


\begin{comment}
\ex
\begingl
\glpreamble \textcolor{red}{ʔuʔiis}ʔaƛ̓in \textcolor{blue}{haʔuk} \textcolor{red}{suuḥaa}. //
\gla \textcolor{red}{ʔu-!iis}=!aƛ=!in \textcolor{blue}{haʔuk} \textcolor{red}{suuḥaa} //
\glb \textcolor{red}{\textsc{x}-eat}=\textsc{now}=\textsc{cmmd.1pl} \textcolor{blue}{eat.\textsc{dr}} \textcolor{red}{spring.salmon} //
\glft `Let's eat spring salmon!' (\textbf{B}, Bob Mundy and Marjorie Touchie) //
\endgl \label{ex:eateat2}
\xe

\ex~
\adjustbox{max width=\textwidth - 0.7in}{
\begingl
\glpreamble \textcolor{red}{hiniic}intiisʔinł \textcolor{blue}{ʔucičƛ ciquuwłi} \textcolor{red}{t̓aatn̓aʔiskqs}. //
\gla \textcolor{red}{hina-iic}=int=(y)iis=ʔinł \textcolor{blue}{ʔu-ci-čiƛ} \textcolor{blue}{ciq-uwił=ʔiˑ} \textcolor{red}{L.<t>-t̓an̓a=ʔis=uk=qaˑs}  //
\glb \textcolor{red}{\textsc{empty}-carry}=\textsc{pst}=\textsc{weak.1sg}=\textsc{habit} \textcolor{blue}{\textsc{x}-go.to-\textsc{pf}} \textcolor{blue}{pray-building} \textcolor{red}{\textsc{pl}-child=\textsc{dim}=\textsc{poss}=\textsc{defn.1sg}} //
\glft `I would always take my children to church.' (\textbf{Q}, Sophie Billy) //
\endgl \label{ex:takechildrentochurch}
}
\xe
\end{comment}

\pause

\begin{itemize}
\item Speakers seemed to prefer verbs to match in perfectiveness, but elicitation results were mixed
\item I turned to corpus study
\end{itemize}

\end{frame}

\begin{frame}{Serial Verb Constructions: Type I}
\begin{table}[H]
\centering
\caption{Type I SVCs and Perfectivity}
\label{table:svctype1}
\begin{tabular}{cllll}
\multicolumn{1}{l}{} &  & \multicolumn{1}{c}{Word count} & \multicolumn{1}{c}{Type 1 SVCs} & \multicolumn{1}{c}{\begin{tabular}[c]{@{}c@{}}Perfectivity\\  mismatches\end{tabular}} \\ \hline
\multicolumn{1}{|l|}{1910--1914} & \multicolumn{1}{l|}{Nootka Texts} & \multicolumn{1}{l|}{2220} & \multicolumn{1}{l|}{22} & \multicolumn{1}{l|}{1} \\ \hline
\multicolumn{1}{|c|}{\multirow{4}{*}{2010--2019}} & \multicolumn{1}{l|}{Barkley speakers} & \multicolumn{1}{l|}{942} & \multicolumn{1}{l|}{10} & \multicolumn{1}{l|}{3} \\ \cline{2-5} 
\multicolumn{1}{|c|}{} & \multicolumn{1}{l|}{Central speakers} & \multicolumn{1}{l|}{2456} & \multicolumn{1}{l|}{26} & \multicolumn{1}{l|}{9.5} \\ \cline{2-5} 
\multicolumn{1}{|c|}{} & \multicolumn{1}{l|}{Northern speakers} & \multicolumn{1}{l|}{1621} & \multicolumn{1}{l|}{12} & \multicolumn{1}{l|}{3.5} \\ \cline{2-5} 
\multicolumn{1}{|c|}{} & \multicolumn{1}{l|}{\begin{tabular}[c]{@{}l@{}}Kyuquot-Checleseht\\ speakers\end{tabular}} & \multicolumn{1}{l|}{6928} & \multicolumn{1}{l|}{36} & \multicolumn{1}{l|}{11} \\ \hline
\end{tabular}
\end{table}
\end{frame}

\begin{frame}[fragile]{Serial Verb Constructions: Type II}
Type II: Location + Action

\begin{itemize}
\item Imperfective location must occur first
\item Same object separation permissible as in Type I
\item No perfectivity matching
\end{itemize}

\ex[exno=17]
\begingl
\glpreamble \textcolor{red}{mačiił}ʔaƛniš mamuuk. //
\gla \textcolor{red}{mačiił}=!aƛ=niˑš mamuuk  //
\glb \textcolor{red}{inside.\textsc{dr}}=\textsc{now}=\textsc{real.1pl} work.\textsc{dr} //
\glft `I am working inside.' (\textbf{C}, \textit{tupaat} Julia Lucas) //
\endgl \label{ex:insideworking}
\xe

\ex[exno=18]~
\begingl
\glpreamble *mamuuk̓aƛniš \textcolor{red}{mačiił}. //
\gla mamuuk=!aƛ=niˑš \textcolor{red}{mačiił} //
\glb work.\textsc{dr}-\textsc{now}=\textsc{real.1pl} \textcolor{red}{inside.\textsc{dr}} //
\glft Intended: `I am working inside.' (\textbf{C}, \textit{tupaat} Julia Lucas) //
\endgl \label{ex:*insideworking}
\xe

\begin{comment}
\ex~
\begingl
\glpreamble \textcolor{red}{hiłqii}mitʔišʔał \textcolor{blue}{huuxsʔatu} \textcolor{red}{nučii}. //
\gla \textcolor{red}{hił-qii}=(m)it=ʔiˑš=ʔaˑł \textcolor{blue}{huuxsʔatu} \textcolor{red}{nuč-iˑ} //
\glb \textcolor{red}{be.at-on.top}=\textsc{pst}=\textsc{strg.3}=\textsc{habit} \textcolor{blue}{rest.\textsc{dr}} \textcolor{red}{mountain-\textsc{nmlz}} //
\glft `He rests on top of mountains.' (\textbf{N}, Fidelia Haiyupis) //
\endgl \label{ex:restonmountains}
\xe
\end{comment}
\end{frame}

\begin{frame}{Serial Verb Constructions: Type III}
Type III: Adposition-like Verbs + Action

\begin{itemize}
\item \cite{woo2007b} defines a set of verbs with adposition-like meanings which adjoin to a main verb in a clause
\item Same object separation allowed
\item No ordering restriction
\item No perfectivity matching
\end{itemize}

\ex[exno=19]
\begingl
\glpreamble \textcolor{red}{ʔuuʔatupšiƛ}waʔiš \textcolor{blue}{mamuuk} \textcolor{red}{Friendship Center}. //
\gla \textcolor{red}{ʔu-L.ʔatup-šiƛ}=waˑʔiš \textcolor{blue}{mamuuk} \textcolor{red}{Friendship Center} //
\glb \textcolor{red}{\textsc{x}-do.for-\textsc{pf}}=\textsc{hrsy.3} \textcolor{blue}{work.\textsc{dr}} \textcolor{red}{Friendship Center} //
\glft `I hear she started to work for the Friendship Center.' (\textbf{C}, \textit{tupaat} Julia Lucas) //
\endgl \label{ex:starttoworkfor2}
\xe

\end{frame}

\begin{frame}{Serial Verb Constructions: Type IV}
Type IV: Sequential or Separable Action

\begin{itemize}
\item Can be interpreted as sequential ``and then"
\item No perfectivity matching
\end{itemize}

\ex[exno=20]
\begingl
\glpreamble sukʷiʔi k̓ašsaap //
\gla su-kʷiƛ=!iˑ k̓aš-saˑp //
\glb hold-\textsc{pf}=\textsc{cmmd.2sg} put.away-\textsc{pf.caus} //
\glft `Take it and put it away.' (\textbf{C}, \textit{tupaat} Julia Lucas) //
\endgl \label{ex:foldputaway1}
\xe

\ex[exno=21]~
\begingl
\glpreamble \# k̓ašsaap̓i sukʷiƛ //
\gla k̓aš-saˑp=!iˑ su-kʷiƛ //
\glb put.away-\textsc{pf.caus}=\textsc{cmmd.2sg} hold-\textsc{pf} //
\glft \# `Put it away, then take it.' (\textbf{C}, \textit{tupaat} Julia Lucas) //
\endgl \label{ex:foldputaway2}
\xe

\end{frame}

\begin{frame}[noframenumbering]{Serial Verb Constructions: Type IV}

\begin{itemize}
\item Object separation is ungrammatical
\end{itemize}

\ex
\begingl
\glpreamble \textcolor{red}{cassaap}s \textcolor{red}{ʕiniiƛ} \textcolor{blue}{č̓axʷaciis}. //
\gla \textcolor{red}{cas-saˑp}=s \textcolor{red}{ʕiniiƛ} \textcolor{blue}{č̓axʷac-iis} //
\glb \textcolor{red}{chase-\textsc{pf.caus}}=\textsc{strg.1sg} \textcolor{red}{dog} \textcolor{blue}{bucket-hold.\textsc{dr}} //
\glft `I chased the dog, (I) carrying the bucket.' (\textbf{C}, \textit{tupaat} Julia Lucas) //
\endgl \label{ex:chasedog1}
\xe

\ex~
\begingl
\glpreamble *\textcolor{red}{cassaap}s \textcolor{blue}{č̓axʷaciis} \textcolor{red}{ʕiniiƛ}. //
\gla \textcolor{red}{cas-saˑp}=s \textcolor{blue}{č̓axʷac-iis} \textcolor{red}{ʕiniiƛ} //
\glb \textcolor{red}{chase-\textsc{pf.caus}}=\textsc{strg.1sg} \textcolor{blue}{bucket-hold.\textsc{dr}} \textcolor{red}{dog} //
\glft Intended: `Carrying the bucket, I chased the dog.' (\textbf{C}, \textit{tupaat} Julia Lucas) //
\endgl \label{ex:chasedog3}
\xe

\end{frame}


\begin{frame}{Serial Verb Constructions: Type Overview}
\begin{table}[H]
\centering
\caption{Summary of SVC Types}
\label{table:svcsummary}
\adjustbox{max width=\textwidth}{
\begin{tabular}{l|llll}
 & Description & Perfectivity matching & Verb-object splitting & Ordering restriction \\ \hline
Type I & Simultaneous & (\cmark) & \cmark & None \\ \hline
Type II & Location & \xmark & \cmark & Location first \\ \hline
Type III & Adposition-like & \xmark & \cmark & None \\ \hline
Type IV & Separable / Sequential & \xmark & \xmark & Temporal ordering \\ \hline
\end{tabular}}
\end{table}
\end{frame}

\begin{frame}{Serial Verb Constructions: Valence Operations}
Some of the clausal enclitics alter valence. What happens under serialization?

\begin{table}[h]
\centering
\label{table:2pclitics}
\adjustbox{max width=\textwidth}{
\begin{tabular}{c|c|c|c|c|c|c|c|c|c|c|}
\cline{2-11}
morph & =ʔaaqƛ & \textcolor{red}{=!ap}      & =!aƛ & \textcolor{red}{=!at}    & \begin{tabular}[c]{@{}c@{}}=uk\\ =ʔak\end{tabular} & =(m)it & \begin{tabular}[c]{@{}c@{}}=ʔiˑš\\ =maˑ\\ =ḥaˑ\\=$\emptyset$\\ ...\end{tabular} & =ʔaała   & =ʔał   & =ƛaˑ \\ \cline{2-11}
meaning  & \textsc{fut} & \textcolor{red}{\textsc{caus}} & \textsc{now}  & \textcolor{red}{\textsc{pass}} & \textsc{poss}  & \textsc{pst} & \begin{tabular}[c]{@{}c@{}}subject-mood\\ portmanteaus\end{tabular} & \textsc{habit} & \textsc{pl} & also \\ \cline{2-11}
\end{tabular}
}
\end{table}

\end{frame}

\begin{frame}{Serial Verb Constructions: Valence Operations}
These enclitics scope narrowly over the individual coordinated verb, while subject-mood scopes over the whole clause.

\ex[exno=22]
\begingl
\glpreamble ʔaḥʔaaʔaƛna ƛ̓ičiƛ ʔuca\textcolor{red}{ap} ḥaa hupałʔi. //
\gla ʔaḥʔaaʔaƛ=naˑ ƛ̓i-čiƛ ʔu-ca\textcolor{red}{=!ap} ḥaa hupał=ʔiˑ //
\glb and.then=\textsc{neut.1pl} shoot-\textsc{pf} \textsc{x}-go\textcolor{red}{=\textsc{caus}} \textsc{d3} sun.or.moon=\textsc{art} //
\glft `Then we shoot them toward the moon.' (\textbf{C}, \textit{tupaat} Julia Lucas) //
\endgl \label{ex:shootatthemoon}
\xe

\ex[exno=23]~
\begingl
\glpreamble čimqstuƛitaḥ nanaʔiiči\textcolor{red}{ʔat}. //
\gla čimqstuƛ=(m)it=(m)aˑḥ nanaʔiičiƛ\textcolor{red}{=!at} //
\glb be.happy.\textsc{pf}=\textsc{pst}=\textsc{real.1sg} understand.\textsc{pf}\textcolor{red}{=\textsc{pass}} //
\glft `I was happy being understood.' (\textbf{B}, Bob Mundy) //
\endgl \label{ex:happyunderstood}
\xe

\end{frame}

\begin{frame}{HPSG Implemenation: SVCs}
Location verbs, adpositive verbs, and other verbs are distinguished from each other.

I track this through a new \textsc{head} property, \textsc{htype}.

\vspace{10pt}

\begin{avm}
\[ head.htype & location | adpositive | normal \]
\end{avm}

\vspace{10pt}

Every SVC will specify which type of verbs are allowed.

\end{frame}

\begin{frame}{HPSG Implemenation: SVCs}
Valence changing needs to occur in two places:
\begin{enumerate}
\item With the rest of the enclitics on the first word in the clause
\item On the first word in the VP coordinated in SVCs
\end{enumerate}

I have two versions then of these enclitics which distinguish clause-heading from coordinator.

\vspace{10pt}

\centering
\begin{avm}
\[ head & \[ aux & + \\
             prd & + \\
             form & finite | nonfinite \] \]
\end{avm}

\vspace{10pt}

\end{frame}


\begin{frame}[fragile]{HPSG Implemenation: SVCs}
\begin{itemize}
\item I model serial verbs as coordination, either with the semantics of \textsc{meanwhile} or \textsc{and}
\item Coordination strategy is developed off of \cite{drellishakbender2005}
\item A supertype defines the leftmost verb as containing the enclitic complex, and identifies its subject, tense, and mood with later coordinated verbs.
\item Subtypes specify SVC-specfic restrictions
\end{itemize}

\end{frame}

\begin{comment}
\begin{frame}{HPSG Implemenation: SVCs}
Specific rules for SVC Type I (perfective version)

\ex
\begin{avm}
\[\asort{svc1-perf-bottom-coord-rule}
synsem.local & \[ coord-rel.pred & \textsc{meanwhile} \\
                  coord-strat & ``1-perf" \\
                  cont.hook.index.e.aspect & perfective \] \\
nonconj-dtr$\ldots$htype & normal \]
\end{avm} \label{ex:svc1-perf-bottom-coord-rule}
\xe

\ex~
\begin{avm}
\[\asort{svc1-perf-top-coord-rule}
synsem.local.coord-strat & ``1-perf" \\
lcoord-dtr.synsem.local & \[ cat.head.htype & normal \\
                             cont.hook.index.e.aspect & perfective \] \]
\end{avm} \label{ex:svc1-perf-top-coord-rule}
\xe
\end{frame}
\end{comment}

\begin{frame}{HPSG Implemenation: SVCs}
Specific rules for SVC Type II (locations)

\ex[exno=25]
\begin{avm}
\[\asort{svc2-bottom-coord-rule}
synsem.local & \[ coord-rel.pred & \textsc{meanwhile} \\
                  coord-strat & ``2" \] \]
\end{avm} \label{ex:svc2-bottom-coord-rule}
\xe

\ex~[exno=26]
\begin{avm}
\[\asort{svc2-top-coord-rule}
synsem.local & \[ coord-strat & ``2" \\
                  cat.head.htype & location \] \]
\end{avm} \label{ex:svc2-top-coord-rule}
\xe
\end{frame}

\begin{comment}
\begin{frame}{HPSG Implemenation: SVCs}
Specific rules for SVC Type IV (separable/sequential)

\ex
\begin{avm}
\[\asort{svc4-bottom-coord-rule}
synsem.local & \[ coord-rel.pred & \textsc{and} \\
                  coord-strat & ``4" \] \\
nonconj-dtr$\ldots$htype & normal \]
\end{avm} \label{ex:svc4-bottom-coord-rule}
\xe

\ex~
\begin{avm}
\[\asort{svc4-bottom-coord-rule}
synsem.local & \[ coord-strat & ``4" \\
                  cat & \[ val.comps & \< \> \\
                           head.htype & normal \] \] \]
\end{avm} \label{ex:svc4-top-coord-rule}
\xe
\end{frame}
\end{comment}

\begin{frame}{HPSG Implementation: Parsing Results}

\begin{itemize}
\item Examples all came from speakers, either in elicitation sessions or in text
\item Coverage: 11.3\%
\item Overgeneration: 16\% (4\%)
\end{itemize}

\begin{table}[]
\centering
\caption{}
\label{tab:my-table}
\begin{tabular}{l|llll}
 & Total & Parsed & Unparsed & Avg \# of parses \\ \hline
Grammatical sentences & 284 & 32 & 252 & 5.78 \\ \hline
Ungrammatical sentences & 25 & 4 (1) & 21 (24) & 2.75
\end{tabular}
\end{table}
	
\end{frame}


