\section{Conclusion}

\begin{frame}{Conclusion}

Serial Verb Construction
\begin{itemize}
\item Two or more verbs coordinating	 covertly
\item 3-4 types of constructions depending on verbs involved
\end{itemize}

Linker Construction
\begin{itemize}
\item Two or more predicates coordinating overtly
\item Belongs to a category of second position suffixes
\item Supports interpretation of a broad predicate category
\end{itemize}

\end{frame}

\begin{frame}{Conclusion}

Shared phenomenon:
\begin{itemize}
\item Coordination strategies allow for complement-separation
\item The syntactic domain of the second position enclitics is not identical
\begin{itemize}
\item Minimally, passive and causative scope at the level of a ``maximal predicate phrase" (non-coordinated PredP)
\item Minimally, the subject-mood portmanteaus and tense scope at the level of the clause (includes coordination)
\end{itemize}
\end{itemize}

These coordination strategies illuminate the syntactic categories and constituents in the language.

\end{frame}

