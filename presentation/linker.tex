\section{The Linker}

\begin{frame}{The Linker: Morphological properties}

\begin{itemize}
\item The morpheme \textit{-(q)ḥ}	
\item Translated as `meanwhile' in \cite{sapir1939}
\item Wide variety of attachment properties
\end{itemize}

\end{frame}

\begin{frame}{The Linker: Morphological properties}

\begin{itemize}
\item Attaches to contentful parts of speech
\begin{itemize}
\item \textbf{Verb}: \textit{ciqink̓aƛna \textcolor{blue}{ƛiḥaa\textbf{qḥ}}} \\ \hspace{5pt} `We talked while driving' (\textbf{C}, \textit{tupaat} Julia Lucas)
\item \textbf{Adjective}: \textit{t̓ikʷaamitwaʔiš čims \textcolor{blue}{ḥaaʔak\textbf{qḥ}}} \\ \hspace{5pt} `The bear was digging and strong'  (\textbf{C}, \textit{tupaat} Julia Lucas)
\item \textbf{Noun}: \textit{\textcolor{blue}{łuucma\textbf{qḥ}}itqač̓aʔaał taakšiƛ p̓iišmita} \\ \hspace{5pt} `There was a woman who kept gossiping' (\textbf{C}, \textit{tupaat} Julia Lucas)
\item \textbf{Adverb}: \textit{\textcolor{blue}{y̓uuqʷaa\textbf{qḥ}}s ʕasqii ʔaanaḥi wik hinʔałšiƛ} \\ \hspace{5pt} `I'm also bald but I don't know it.' (\textbf{C}, \textit{tupaat} Julia Lucas)
\end{itemize}
\item But not functional parts of speech
\begin{itemize}
\item \textbf{Complementizer}: \textit{*ʔuušcukʔisit \textcolor{blue}{ʔani\textbf{qḥ}} ʔunaḥʔisitqa} \\ \hspace{5pt} Intended: `It was a little difficult (to do) because it's small.' (\textbf{B}, Bob Mundy)
\end{itemize}
\item The linker may either be initial (with the second position enclitics) or later.
\end{itemize}

\end{frame}

\begin{frame}{Interpretive effects of linker attachment}

\begin{itemize}
\item Non-verbal elements with a linker attached \textit{must} have a subject interpretation.
\end{itemize}

\ex[exno=27]
\begingl
\glpreamble ʔuuwaʔaƛ ʔuuš.//
\gla ʔu-L.waƛ=!aƛ ʔuuš //
\glb \textsc{x}-find=\textsc{now} some //
\glft `He/she found something.' (*? Someone found it) (\textbf{C}, \textit{tupaat} Julia Lucas) //
\endgl \label{ex:findsomething}
\xe

\ex~[exno=28]
\begingl
\glpreamble ʔuuwaʔaƛ ʔuuš\textcolor{blue}{qḥ}.//
\gla ʔu-L.waƛ=!aƛ ʔuuš\textcolor{blue}{-qḥ} //
\glb \textsc{x}-find=\textsc{now} some\textcolor{blue}{-\textsc{link}} //
\glft `Someone found it.' (*He/she found something) (\textbf{C}, \textit{tupaat} Julia Lucas) //
\endgl \label{ex:findsomeone}
\xe
	
\end{frame}

\begin{frame}{Predicates and the linker}

All of these things can be explained by two properties:

\begin{enumerate}
\item The linker attaches to predicates
\item Elements coordinated by the linker share a subject (second position enclitics)
\end{enumerate}

When the linker attaches to a non-verbal predicate as in (28), it is interpreted as a predicate (i.e., with a subject) which is shared with the other predicate (typically verb) in the clause.

\end{frame}

\begin{frame}{Predicates and the linker}

\begin{itemize}
\item \textit{ʔuuwaƛ}: \textsc{find}(\textit{e}, \textsc{subj}:\textit{x}, \textsc{obj}:\textit{y})
\item \textit{ʔuuš}: \textsc{some}(\textit{e}, \textsc{subj}:\textit{x})
\end{itemize}

\pause

\ex[exno=27]
\begingl
\gla {[ʔuuwaʔaƛ]\textsubscript{pred}} {[ʔuuš]\textsubscript{part}} //
\glb find some //
\endgl \label{ex:findsomething2}
\xe

\vspace{-15pt}

\hspace{25pt} \textsc{find}(\textit{e1}, \textit{x}, \textit{y}) $\land$ (\textbf{\textsc{subj}}: $\exists$\textit{x} \textsc{some}(\textit{e2}, \textit{x}) $\lor$ \textbf{\textsc{obj}}: $\exists$\textit{y} \textsc{some}(\textit{e2}, \textit{y}))

\pause

\ex[exno=28]
\begingl
\gla {[ʔuuwaʔaƛ]\textsubscript{pred}} {[ʔuuš\textcolor{blue}{qḥ}]\textsubscript{pred}} //
\glb find some\textcolor{blue}{-\textsc{link}} //
\endgl \label{ex:findsomeone}
\xe

\vspace{-15pt}

\hspace{25pt} \textsc{find}(\textit{e1}, \textit{x}, \textit{y}) $\land$ \textsc{some}(\textit{e2}, \textit{x})

\end{frame}

\begin{frame}{Predicates and second position}

\begin{itemize}
\item The linker can also attach to adverbs (non-predicates)
\item What's it linking? \pause
\item Two ``maximal predicate phrases" (predicate + complements + modifiers)
\end{itemize}

\ex[exno=29]
\begingl
\glpreamble {[}\textcolor{red}{ʔeʔim}\textcolor{blue}{qḥ}ʔaƛquuweʔin \textcolor{red}{hitaḥtačiƛ}{]\textsubscript{pred1}} {[}\textcolor{Purple}{sukʷiʔaƛ} \textcolor{Purple}{puuʔakʔiʔał}{]\textsubscript{pred2}} //
\gla {[}\textcolor{red}{ʔeʔim}-\textcolor{blue}{(q)ḥ}=!aƛ=quu=weˑʔin \textcolor{red}{hitaḥta-čiƛ}{]\textsubscript{pred1}} {[}\textcolor{Purple}{su-kʷiƛ=!aƛ} \textcolor{Purple}{puu=ʔak=ʔiˑ=ʔał}{]\textsubscript{pred2}} //
\glb {[}\textcolor{red}{first}-\textcolor{blue}{\textsc{link}}=\textsc{now}=\textsc{pssb.3}=\textsc{hrsy.3} \textcolor{red}{go.out.to.sea-\textsc{pf}}{]\textsubscript{pred1}} {[}\textcolor{Purple}{hold-\textsc{pf}=\textsc{now}} \textcolor{Purple}{gun=\textsc{poss}=\textsc{art}=\textsc{pl}}{]\textsubscript{pred2}} //
\glft `As soon as they left the land, they would take their guns.' (\textbf{B}, \citealt[395]{sapir1955}) //
\endgl
\xe \label{ex:takeoutguns}

\end{frame}

\begin{frame}{The linker: Similarities with SVCs}

\begin{itemize}
\item Like SVCs, a linked predicate can separate verb and object	
\end{itemize}

\ex[exno=30]~
\begingl
\glpreamble \textcolor{red}{hiłqḥ}sʔaał \textcolor{blue}{načaał} \textcolor{red}{ƛiisuwił}. //
\gla \textcolor{red}{hił-(q)ḥ}=s=ʔaał \textcolor{blue}{načaał} \textcolor{red}{ƛiisuwił} //
\glb \textcolor{red}{be.at-\textsc{link}}=\textsc{strg.1sg}=\textsc{habit} \textcolor{blue}{read} \textcolor{red}{school} //
\glft `I read at school.' (\textbf{C}, \textit{tupaat} Julia Lucas) //
\endgl \label{ex:readatschool}
\xe

\pause

\begin{itemize}
\item Like SVCs, passive and causative scope narrowly over the local predicate in linker constructions	
\end{itemize}

\ex[exno=31]~
\begingl
\glpreamble {[}ƛawiiči\textcolor{blue}{ʔat}aḥ t̓an̓eʔis{]\textsubscript{PredP\_1}} {[}hiłḥ maḥt̓iiʔakqas{]\textsubscript{PredP\_2}}. //
\gla {[}ƛaw-iˑčiƛ=\textcolor{blue}{!at}=(m)aˑḥ t̓an̓a=ʔis{]\textsubscript{PredP\_1}} {[}hił-(q)ḥ maḥt̓ii=ʔak=qaˑs{]\textsubscript{PredP\_2}} //
\glb {[}near-\textsc{pf}=\textcolor{blue}{\textsc{pass}}=\textsc{real.1sg} child=\textsc{dim}{]\textsubscript{PredP\_1}} {[}be.at-\textsc{link} house=\textsc{poss}=\textsc{defn.1sg}{]\textsubscript{PredP\_2}} //
\glft `A child came up to me at at my house.' (\textbf{B}, Bob Mundy) //
\endgl \label{ex:approachedat}
\xe

\end{frame}


\begin{frame}{HPSG Implementation: Linker}
\begin{itemize}
\item My analysis for the linker shares the same foundation as second-position suffix verbs.
\item The linker is an incorporating suffix and how it behaves depends on how incorporation works (predicate vs adverb).
\item I need two versions of the linker:
\begin{enumerate}
\item A version that occurs on the first coordinand.
\item A version that occurs on the second coordinand.
\end{enumerate}
\end{itemize}
\end{frame}

\begin{frame}{HPSG Implementation: Parsing Results}

\begin{itemize}
\item Examples all came from speakers, either in elicitation sessions or in text
\item Coverage: 12.4\%
\item Overgeneration: 7.1\% (3.6\%)
\end{itemize}

\begin{table}[]
\centering
\caption{}
\label{tab:my-table}
\begin{tabular}{l|llll}
 & Total & Parsed & Unparsed & Avg \# of parses \\ \hline
Grammatical sentences & 177 & 22 & 155 & 2 \\ \hline
Ungrammatical sentences & 28 & 2 (1*) & 26 (27*) & 2
\end{tabular}
\end{table}
	
\end{frame}
