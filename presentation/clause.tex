\section{Clause Structure}

\begin{frame}{Clause: Terms and definitions}
Semantics
\begin{itemize}
\item \textbf{Relation}: An atomic unit that represents semantic meaning, e.g.\ \textsc{live}, \textsc{yellow}, \textsc{submarine}
\item \textbf{Argument}: An element that is involved in a relation. Minimally they come in the flavors of an \textit{event} or an \textit{individual}, e.g.,\ \textsc{see}(\textit{e}, \textit{x}, \textit{y})
\end{itemize}
Syntax
\begin{itemize}
\item \textbf{Predicate}: The unit in the syntax which provides the main semantic relation \textit{and event} of a clause, whose semantic arguments are filled through syntactic relations like subject and object
\item \textbf{Participant}: A unit in the syntax which serves as a subject or object in relation to a predicate
\end{itemize}
\end{frame}

\begin{frame}{Clause: Terms and definitions}
English:

\ex[exno=1] \label{ex:dogbarks}
[\textcolor{blue}{The dog}]\textsubscript{participant} [\textcolor{red}{barks}]\textsubscript{predicate}
\xe

\pause

\hspace{20pt} \textsc{bark}(\textit{e}, \textit{x}), \textsc{dog}(\textit{x})

\pause

\ex[exno=2] \label{ex:grassgreen}
[\textcolor{blue}{The grass}]\textsubscript{participant} [\textcolor{red}{appears}]\textsubscript{predicate} [\textcolor{blue}{green}]\textsubscript{participant}
\xe


\pause

\hspace{20pt} \textsc{appear}(\textit{e}, \textit{x}, \textit{y}), \textsc{grass}(\textit{x}), \textsc{green}(\textit{y})

\end{frame}

\begin{frame}{Clause: Terms and definitions}
Nuuchahnulth:

\ex[exno=3] 
\begingl
\glpreamble $[$\textcolor{red}{n̓aacsiičiƛ}$]$\textsubscript{pred}ʔiš $[$\textcolor{blue}{hałmiiḥa quuʔas}$]$\textsubscript{part} //
\gla $[$\textcolor{red}{n̓aacsa-iˑčiƛ}$]$\textsubscript{pred}=ʔiˑš $[$\textcolor{blue}{hałmiiḥa quuʔas}$]$\textsubscript{part} //
\glb $[$\textcolor{red}{see.\textsc{cv}-\textsc{in}}$]$\textsubscript{pred}=\textsc{strg.3sg} $[$\textcolor{blue}{drowning person}$]$\textsubscript{part}	 //
\glft `He sees a drowning person.' (\textbf{N}, Fidelia Haiyupis) //
\endgl \label{ex:verbpred}
\xe

\pause

\ex[exno=4]~
\begingl
\glpreamble $[$\textcolor{red}{qʷac̓ał}$]$\textsubscript{pred}ʔiš $[$\textcolor{blue}{ḥaakʷaaƛ}$]$\textsubscript{part}ʔi //
\gla $[$\textcolor{red}{qʷac̓ał}$]$\textsubscript{pred}=ʔiˑš $[$\textcolor{blue}{ḥaakʷaaƛ}$]$\textsubscript{part}=ʔiˑ //
\glb $[$\textcolor{red}{beautiful}$]$\textsubscript{pred}=\textsc{strg.3} $[$\textcolor{blue}{young.girl}$]$\textsubscript{part}=\textsc{art} //
\glft `The young girl is beautiful.' (\textbf{C}, \textit{tupaat} Julia Lucas) //
\endgl \label{ex:adjpred}
\xe

\pause

\ex[exno=5]~
\begingl
\glpreamble $[$\textcolor{red}{pisatuwił}$]$\textsubscript{pred}ma ʔaanaḥi //
\gla $[$\textcolor{red}{pisatuwił}$]$\textsubscript{pred}=maˑ ʔaanaḥi //
\glb $[$\textcolor{red}{gym}$]$\textsubscript{pred}=\textsc{real.3} only //
\glft `It's only a gym.' (\textbf{B}, Marjorie Touchie) //
\endgl \label{ex:nounpred}
\xe

\begin{textblock*}{5cm}(10cm,3cm) % {block width} (coords) 
   \textcolor{red}{Verb}
\end{textblock*}

\only<2->{
\begin{textblock*}{5cm}(10cm,5.4cm) % {block width} (coords) 
   \textcolor{red}{Adjective}
\end{textblock*}}

\only<3->{
\begin{textblock*}{5cm}(10cm,7.8cm) % {block width} (coords) 
   \textcolor{red}{Noun}
\end{textblock*}}

\end{frame}

\begin{frame}{Clause: Terms and definitions}

\ex[exno=6]
\begingl
\glpreamble $[$\textcolor{red}{ʔuḥ}$]$ʔiiš $[$\textcolor{red}{ʕiḥak}$]$\textsubscript{pred} $[$\textcolor{blue}{kamatquk}$]$\textsubscript{part}ʔi //
\gla $[$\textcolor{red}{ʔuḥ}$]$=ʔiˑš $[$\textcolor{red}{ʕiḥak}$]$\textsubscript{pred} $[$\textcolor{blue}{kamatq-uk}$]$\textsubscript{part}=ʔiˑ //
\glb $[$\textcolor{red}{be}$]$=\textsc{strg.3} $[$\textcolor{red}{cry.\textsc{dr}}$]$\textsubscript{pred} $[$\textcolor{blue}{run-\textsc{dr}}$]$\textsubscript{part}=\textsc{art} //
\glft `The running one is crying.' (\textbf{C}, \textit{tupaat} Julia Lucas) //
\endgl \label{ex:verbpart}
\xe

\pause

\ex[exno=7]~ 
\begingl
\glpreamble $[$\textcolor{red}{wik̓}$]$iičʔaał $[$\textcolor{red}{ƛ̓iixc̓us}$]$\textsubscript{pred} $[$\textcolor{blue}{ƛaƛuu}$]$\textsubscript{part}ʔi //
\gla $[$\textcolor{red}{wik}$]$=!iˑč=ʔaał $[$\textcolor{red}{ƛ̓iixc̓us}$]$\textsubscript{pred} $[$\textcolor{blue}{ƛaƛuu}$]$=\textsubscript{part}ʔiˑ //
\glb $[$\textcolor{red}{\textsc{neg}}$]$=\textsc{cmmd.2pl}=\textsc{habit} $[$\textcolor{red}{laugh.at.\textsc{dr}}$]$\textsubscript{pred} $[$\textcolor{blue}{other.\textsc{pl}}$]$\textsubscript{part}=\textsc{art} //
\glft `Don't laugh at others.' (\textbf{C}, \textit{tupaat} Julia Lucas) //
\endgl \label{ex:adjpart}
\xe

\pause

\ex[exno=4]~
\begingl
\glpreamble $[$\textcolor{red}{qʷac̓ał}$]$\textsubscript{pred}ʔiš $[$\textcolor{blue}{ḥaakʷaaƛ}$]$\textsubscript{part}ʔi //
\gla $[$\textcolor{red}{qʷac̓ał}$]$\textsubscript{pred}=ʔiˑš $[$\textcolor{blue}{ḥaakʷaaƛ}$]$\textsubscript{part}=ʔiˑ //
\glb $[$\textcolor{red}{beautiful}$]$\textsubscript{pred}=\textsc{strg.3} $[$\textcolor{blue}{young.girl}$]$\textsubscript{part}=\textsc{art} //
\glft `The young girl is beautiful.' (\textbf{C}, \textit{tupaat} Julia Lucas) //
\endgl \label{ex:nounpart}
\xe

\begin{textblock*}{5cm}(10cm,3cm) % {block width} (coords) 
   \textcolor{blue}{Verb}
\end{textblock*}

\only<2->{
\begin{textblock*}{5cm}(10cm,5.4cm) % {block width} (coords) 
   \textcolor{blue}{Adjective}
\end{textblock*}}

\only<3->{
\begin{textblock*}{5cm}(10cm,7.8cm) % {block width} (coords) 
   \textcolor{blue}{Noun}
\end{textblock*}}

\end{frame}

\begin{frame}{Clause: Terms and definitions}
\centering
\begin{tikzpicture}[sibling distance=5em,
  every node/.style = {anchor=mid,shape=rectangle,align=center,fill=white}]
\node (noun) at (0,3) {Noun};
\node (verb) at (3,3) {Verb};
\node (adjective) at (6,3) {Adjective};
\node (adverb) at (8,3) {Adverb};
\node (predicate) at (3,6) {\textcolor{red}{Syntactic Predicate}};
\node (participant) at (3,0) {\textcolor{blue}{Syntactic Participant}};
\draw[->] (noun) -- (predicate);
\draw[->] (verb) -- (predicate);
\draw[->] (adjective) -- (predicate);
\draw[->] (noun) -- (participant) node[midway,fill=white] {(=ʔiˑ)};
\draw[->] (verb) -- (participant) node[midway,fill=white] {=ʔiˑ};
\draw[->] (adjective) -- (participant) node[midway,fill=white] {=ʔiˑ};
\draw[->,out=90,in=0] (adverb) to node[midway]{modifies} (predicate);
\end{tikzpicture}
\end{frame}


\begin{frame}[noframenumbering]{Clause: Terms and definitions}
\centering
\begin{tikzpicture}[sibling distance=5em,
  every node/.style = {anchor=mid,shape=rectangle,align=center,fill=white}]
\node (noun) at (0,3) {NP};
\node (verb) at (3,3) {VP};
\node (adjective) at (6,3) {AdjP};
\node (adverb) at (8,3) {Adv(P)};
\node (predicate) at (3,6) {\textcolor{red}{PredP}};
\node (participant) at (3,0) {\textcolor{blue}{PartP}};
\draw[->] (noun) -- (predicate) node[midway, fill=white] {is a};
\draw[->] (verb) -- (predicate) node[midway, fill=white] {is a};
\draw[->] (adjective) -- (predicate) node[midway, fill=white] {is a};
\draw[->] (noun) -- (participant) node[midway, fill=white] {is a\\(+ =ʔiˑ)};
\draw[->] (verb) -- (participant) node[midway,fill=white] {+ =ʔiˑ};
\draw[->] (adjective) -- (participant) node[midway,fill=white] {+ =ʔiˑ};
\draw[->,out=90,in=0] (adverb) to node[midway]{modifies a} (predicate);
\end{tikzpicture}
\end{frame}

\begin{frame}{Clause: Second position enclitics}

Each clause a second position enclitic complex which falls on the first word, including on preceding modifiers (adverbs) and coordinators.

\ex[exno=8]
\begingl
\glpreamble \textcolor{red}{y̓uuqʷaa}ʔaqƛs \textcolor{red}{n̓aačuk}. //
\gla \textcolor{red}{y̓uuqʷaa}=!aqƛ=s \textcolor{red}{n̓aačuk}  //
\glb \textcolor{red}{also}=\textsc{fut}=\textsc{1sg} \textcolor{red}{look.for.\textsc{dr}} //
\glft `I will also look for it.' (\textbf{C}, \textit{tupaat} Julia Lucas) //
\endgl \label{ex:2padvpred}
\xe

\ex[exno=9]~
\begingl
\glpreamble \textcolor{red}{ʔaḥʔaaʔaƛ}na \textcolor{red}{huʔacačiƛ} \textcolor{blue}{ʔaḥkuu}. //
\gla \textcolor{red}{ʔaḥʔaaʔaƛ}=naˑ \textcolor{red}{huʔa-ca-čiƛ} \textcolor{blue}{ʔaḥkuu}  //
\glb \textcolor{red}{and.then}=\textsc{strg.1pl} \textcolor{red}{back-go-\textsc{pf}} \textcolor{blue}{\textsc{d1}} //
\glft `And then we came back here.' (\textbf{C}, \textit{tupaat} Julia Lucas) //
\endgl \label{ex:2pconjpred}
\xe

\end{frame}

\begin{frame}{Clause: Second position enclitics}


\begin{table}[h]
\centering
\label{table:2pclitics}
\caption{Template for clausal enclitics}
\adjustbox{max width=\textwidth}{
\begin{tabular}{c|c|c|c|c|c|c|c|c|c|c|}
\cline{2-11}
morph & =ʔaaqƛ & =!ap      & =!aƛ & =!at    & \begin{tabular}[c]{@{}c@{}}=uk\\ =ʔak\end{tabular} & =(m)it & \begin{tabular}[c]{@{}c@{}}=ʔiˑš\\ =maˑ\\ =ḥaˑ\\=$\emptyset$\\ ...\end{tabular} & =ʔaała   & =ʔał   & =ƛaˑ \\ \cline{2-11}
meaning  & \textsc{fut} & \textsc{caus} & \textsc{now}  & \textsc{pass} & \textsc{poss}  & \textsc{pst} & \begin{tabular}[c]{@{}c@{}}subject-mood\\ portmanteaus\end{tabular} & \textsc{habit} & \textsc{pl} & also \\ \cline{2-11}
\end{tabular}
}
\end{table}

\end{frame}

\begin{frame}{Clause: Second position suffixes}

\begin{itemize}
\item Transitive verbal suffixes appear in second position with respect to their object.
\end{itemize}

\pause

\ex[exno=10]
\begingl
\glpreamble \textcolor{orange}{nuuk}\textcolor{Purple}{naak}s. //
\gla \textcolor{orange}{nuuk}-\textcolor{Purple}{naˑk}=s //
\glb \textcolor{orange}{song}-\textcolor{Purple}{have}=\textsc{strg.1sg} //
\glft `I have a song/songs.' (\textbf{N}, \textit{yuułnaak} Simon Lucas) //
\endgl \label{ex:havesong}
\xe

\ex~[exno=11]
\begingl
\glpreamble \textcolor{orange}{ʔaƛa}\textcolor{Purple}{nak}s \textcolor{orange}{nuuk}. //
\gla \textcolor{orange}{ʔaƛa}-\textcolor{Purple}{naˑk}=s \textcolor{orange}{nuuk} //
\glb \textcolor{orange}{two}-\textcolor{Purple}{have}=\textsc{strg.1sg} \textcolor{orange}{song} //
\glft `I have two songs.' (\textbf{N}, \textit{yuułnaak} Simon Lucas) //
\endgl \label{ex:havetwosongs}
\xe

%A semantically empty root can serve as an attachment site as well.

\ex~[exno=12]
\begingl
\glpreamble \textcolor{orange}{ʔu}\textcolor{Purple}{naak}s \textcolor{orange}{c̓iiqy̓ak}. //
\gla \textcolor{orange}{ʔu}\textcolor{Purple}{-naˑk}=s \textcolor{orange}{c̓iiq-y̓ak} //
\glb \textcolor{orange}{\textsc{x}}-have=\textsc{strg.1sg} \textcolor{orange}{chant-for} //
\glft `I have a chant.' (\textbf{N}, \textit{yuułnaak} Simon Lucas) //
\endgl  \label{ex:havechant}
\xe

\end{frame}


\begin{frame}{Clause: Second position suffixes}

\begin{itemize}
\item Some of these suffixes take predicate complements
\end{itemize}

\ex[exno=13]
\begingl
\glpreamble \textcolor{red}{sukʷiƛ}\textcolor{Purple}{maḥsa}niš. //
\gla \textcolor{red}{su-kʷiƛ}\textcolor{Purple}{-maḥsa}=niˑš //
\glb {hold-\textsc{pf}}\textcolor{Purple}{-want.to}=\textsc{strg.1pl} //
\glft `We want to take (her).' (\textbf{N}, \textit{yuułnaak} Simon Lucas) //
\endgl \label{ex:wanttograb}
\xe

\ex[exno=14]~
\begingl
\glpreamble \textcolor{red}{ʔaani}\textcolor{Purple}{maḥsa}s \textcolor{red}{waa} ʔin ... //
\gla \textcolor{red}{ʔaani}\textcolor{Purple}{-maḥsa}=s \textcolor{red}{waa} ʔin  ... //
\glb \textcolor{red}{only}\textcolor{Purple}{-want.to}=\textsc{real.1sg} \textcolor{red}{say} \textsc{comp} ... //
\glft `I only want to say that ...' (\textbf{N}, \textit{yuułnaak} Simon Lucas) //
\endgl \label{ex:onlywanttosay}
\xe
	
\end{frame}

\begin{comment}
\begin{frame}[fragile]{Clause: Summary}
\begin{itemize}
\item Clause: Predicate =inflection (participants)
\item Nouns, verbs, and adjectives can be predicates or participants
\item Some verbs are second-position suffixes
\item Modifiers precede what they modify, second position elements move accordingly
\end{itemize}

\begin{table}[]
\centering
\label{tab:2p}
\caption {Summary of second positions}
\adjustbox{max width=\textwidth}{
\begin{tabular}{llll}
 & \textbf{Syntactic domain} & \textbf{Attaches to} & \textbf{Meaning} \\ \hline
\textbf{Clausal enclitics} & Clause & First (non-extracted) word & tense, subject, mood \\ \hline
\textbf{Main predicate suffixes} & VP & First word in object (noun or adj) & transitive verbs \\ \hline
\textbf{Auxiliary predicate suffixes} & VP & First word in object (pred or adv) & modal-like verbs
\end{tabular}
}
\end{table}
	
\end{frame}
\end{comment}

\begin{frame}{HPSG Analysis}
\begin{itemize}
\item HPSG is a lexicalist framework that represents grammatical rules through unification of constraints
\item There is no movement: ordering accounted for by altering constraint properties
\item Analysis was implemented computationally
\end{itemize}
\end{frame}

\begin{frame}{HPSG Analysis: Clause}
Predicate vs Participant:
\begin{itemize}
\item A \textsc{head} feature \textsc{prd}: + | --
\item Keeps track of semantic eventiveness
\item Nouns, adjectives, verbs are \textsc{prd} +
\end{itemize}

Clausal enclitics:
\begin{itemize}
\item Head of the sentence
\item Select for a predicate complement and inherit its subject and complements
\item Add subject information and other inflectional material
\end{itemize}
\end{frame}

\begin{frame}{HPSG Analysis: Suffix verbs}

I treat suffix verbs as incorporation which occurs in two steps:

\begin{enumerate}
\item A rule that prepares a root for incorporation
\item A rule that attaches the contentful suffix	
\end{enumerate}

\centering
\adjustbox{max width=\textwidth, max totalheight=2.4in}{
\begin{forest}
[ \begin{avm}
  \[\asort{suffix-attachment} rel & {\textsc{suffix-meaning}(\textnormal{\textit{e2}},\textnormal{\textit{x}},\avmbox{1})} \]
  \end{avm}
  [ \begin{avm}
 	\[\asort{noun/verb/adj-incorporation} ????? \]
    \end{avm}
    [ \begin{avm}
 	  \[\asort{noun/verb/adj-root} subj & \< \avmbox{1} \> \\
 	     rel & {\textsc{meaning}(\textnormal{\textit{e1}},\avmbox{1})} \]
       \end{avm}
    ]
  ] 
]
\end{forest}
}

\end{frame}

\begin{frame}{HPSG Implementation: Parsing Results}

\begin{itemize}
\item Hand-crafted set of ``basic" clausal phenomena
\item Coverage: 84.4\%
\item Overgeneration: 1.3\%
\end{itemize}

\begin{table}[]
\centering
\label{tab:clause-coverage}
\begin{tabular}{l|llll}
 & Total & Parsed & Unparsed & Avg \# of parses \\ \hline
Grammatical sentences & 167 & 141 & 26 & 1.12 \\ \hline
Ungrammatical sentences & 79 & 1 & 78 & 2
\end{tabular}
\end{table}
	
\end{frame}
