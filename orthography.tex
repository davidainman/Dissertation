\chapter{Orthography}

Nuuchahnulth orthography is phonemically transparent. The writing system is fairly recent and is within the Americanist phonetic alphabet (APA) tradition, and bears a resemblance to that loose set of standards.

Nuuchahnulth has five vowel qualities, /a, e, i, o, u/ with a short/long distinction. Mid-vowels typically only occur long, although the Barkley and Central dialects have umlaut rules that derives short /e/ from /a/.\footnote{The Barkley Sound rule is regressive, /aʔi/ $\rightarrow$ /eʔi/, and it applies consistently across the whole language. The Central rule is progressive, /iʔa/ $\rightarrow$ /iʔe/, and applies more irregularly, although it appears to occur most in frequent morpheme combinations.} The consonant inventory is quite large and shown in \Cref{table:cons}.

\begin{table}[ht]
\centering
\caption{Nuuchahnulth consonants}
\label{table:cons}
\begin{tabular}{llllllllllll}
\noindent\parbox[c]{40pt}{\textbf{plain \\ plosives}} & p  & t  & ƛ  & c  & č  & k  & kʷ  & q  & qʷ \\
\noindent\parbox[c]{40pt}{\textbf{glottalized \\ plosives}} & p̓ & t̓ & ƛ̓ & c̓ & č̓ & k̓ & k̓ʷ & &  & ʕ  &  ʔ  \\
\textbf{fricatives} &   &    & ł  & s  & š  & x  & xʷ  & x̣ & x̣ʷ & ḥ & h \\
\textbf{resonants} & m  &  n  &  &  & y   & w  &     &    &     &    &   \\
\noindent\parbox[c]{40pt}{\textbf{glottalized \\ resonants}} & m̓ &   n̓ &  &  & y̓  & w̓ &     &    &     &    &  
\end{tabular}
\end{table}

I list below the cases where the Nuuchahnulth symbols have a value other than their expected IPA interpretation:

\begin{itemize}
	\item ł is the voiceless lateral fricative, \textbeltl
	\item ƛ is the voiceless lateral affricate, \texttoptiebar{t\textbeltl}
	\item ƛ̓ is the corresponding ejective, \texttoptiebar{t\textbeltl}'
	\item c is the voiceless alveolar sibilant affricate, \texttoptiebar{ts}
	\item c̓ is the corresponding ejective, \texttoptiebar{ts}'
	\item š is the voiceless postalveolar sibilant ʃ
	\item č is the voiceless postalveolar sibilant affricate \texttoptiebar{tʃ}
	\item č̓ is the corresponding ejective, \texttoptiebar{tʃ}'
	\item x̣ is the voiceless uvular fricative, \textchi
	\item x̣ʷ is the corresponding labialized fricative, {\textchi}ʷ
	\item ḥ is the voiceless pharyngeal fricative, {\textcrh}
	\item y is the voiced palatal glide, j
	\item m̓, n̓, y̓, w̓, are preglottalized: ʔm, ʔn, ʔy, ʔw
	\item ʕ is the so-called pharyngeal stop, which has been claimed to be a pre-glottalized pharyngeal [ʔʕ] \citep{shank2000}, or in the most complete study, an epiglottal stop with a pharyngeal off-glide [\textbarglotstop$^{\text{ʕ}}$] \citep{carlson2001, esling2005}. To my ears it has multiple realizations, and it is difficult for me to distinguish from ʔ before /a/. It patterns in the phonology with the ejective series,\footnote{In places where the grammar would generate q̓ or q̓ʷ, ʕ is always found instead. ʕ also occurs where cognate Makah words have either q̓ or q̓ʷ.} thus its placement in the chart.
\end{itemize}