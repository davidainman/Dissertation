\chapter{The Linker} \label{ch:link}

The linker morpheme in Nuuchahnulth \textit{-(q)ḥ}, like serial verb constructions (\S\ref{ch:sv}, is a method by which the language can combine multiple verbs into a single clause. In this section I will examine how this construction behaves and its differences from serialization.

\section{Data} \label{ch:link:data}

In this section I give my collected data on the linker morpheme. I present how the construction is used and draw some conclusions about how it behaves. Like with serial verbs, I will try to keep this section fairly theory-neutral, saving the specifics of an HPSG analysis for \S\ref{ch:link:analysis}.

The morpheme \textit{-(q)ḥ} is the last possible suffix on a word. It is typically pronounced as the sequence \textit{qḥ} following a vowel or nasal, and otherwise as \textit{ḥ}. The Central Ahousaht elder \textit{tupaat} Julia Lucas almost always pronounces the linker as the full \textit{qḥ} regardless of the phonological environment, with the exception of certain light verbs. I do not know if this reflects a sub-dialect of Ahousaht, or if this pronunciation is unique to her, but I transcribe her speech faithfully.

The suffix is translated as `meanwhile' in \cite{sapir1939}, and was first dubbed the ``linker" by Adam Werle (\textit{p.c.}), on the understanding that it ``links" two predicates together. In some sense, it is coordinating two elements with each other, below the syntactic scope of the second position clitics. I will first look at the morphological attachment properties of this special coordinator (\S\ref{ch:link:attach}), followed by its syntactic properties (\S\ref{ch:link:clause}--\ref{ch:link:dangling}).

\subsection{Comparison with other coordination}

The linker differs from the other coordinators in the language. Various types of `and' coordination are done with the word \textit{ʔaḥʔaaʔaƛ} and participant coordination with \textit{ʔuḥʔi(i)š}. These words coordinate clauses, VPs, and participant phrases, while I will claim that the linker coordinates predicates.

Much like English \textit{and}, the coordinator \textit{ʔaḥʔaaʔaƛ} may occur at the beginning or middle of a sentence. I distinguish sentence-initial and sentence-medial \textit{ʔaḥʔaaʔaƛ} by prosody, pause, and the presence of clausal clitics.

In its most common use at the beginning of a sentence, \textit{ʔaḥʔaaʔaƛ} can host the clausal clitics (\ref{ex:sentinitandclitics}, \ref{ex:sentinitandclitics2}) or that can be deferred to the predicate (\ref{ex:sentinitandnoclitics}).

\begin{comment}
Context for (): Describing a picture-story.

ʔaanamtqač̓a ʔuusuqtack̓in.
And then he got hurt a little bit.

ʔaḥʔaaʔaƛƛa ƛakišiʔeƛƛa.
And then he stands back up.
BM
\end{comment}

\ex \label{ex:sentinitandclitics}
\begingl
\glpreamble ʔaḥʔaaʔaƛitweʔinʔaała wiinapi haʔukw̓it̓asin waaʔat, nay̓iiʔak̓aƛquuč t̓iqsčiił haʔumʔi. //
\gla ʔaḥʔaaʔaƛ=(m)it=weˑʔin=ʔaała wiinapi haʔuk-w̓it̓as=(m)in waa=!at nay̓iiʔak=!aƛ=quu=č t̓iq-sči-°ił haʔum=ʔiˑ //
\glb and=\textsc{pst}=\textsc{hrsy.3}=\textsc{habit} hold.still.\textsc{dr} eat-going.to=\textsc{strg.1pl} say=\textsc{pass} immediately=\textsc{now}=\textsc{pssb.3}=\textsc{hrsy} sit-beside-indoors.\textsc{dr} food=\textsc{art} //
\glft `Then he would stop and wait for someone to say ``We are going to eat," and immediately he would sit down by the food.' (\textbf{B}, Marjorie Touchie) //
\endgl
\xe


\ex~ \label{ex:sentinitandclitics2}
\begingl
\glpreamble ʔaḥʔaaʔaƛsa huʔaas n̓aacsiičiƛ naani. //
\gla ʔaḥʔaaʔaƛ=saˑ huʔaas n̓aacsa-iˑčiƛ naani //
\glb and=\textsc{neut.1sg} again see.\textsc{dr}-\textsc{incep} grizzly.bear  //
\glft `And then I also saw a grizzly bear used.' (\textbf{C}, \textit{tupaat} Julia Lucas) //
\endgl
\xe

[[TODO: Find a better example of below]]

\ex~ \label{ex:sentinitandnoclitics}
\begingl
\glpreamble ʔaḥʔaaʔaƛ ʔuk̓ʷič̓ap̓aƛsuuk ʔiiḥ ciyapuxs. //
\gla ʔaḥʔaaʔaƛ ʔu-k̓ʷič=!ap=!aƛ=suuk ʔiiḥ ciyapuxs //
\glb and \textsc{x}-wear=\textsc{caus}=\textsc{now}=\textsc{neut.2sg} big hat //
\glft `And you wear a big hat.' (\textbf{C}, \textit{tupaat} Julia Lucas) //
\endgl
\xe

%ʔaḥʔaaʔaƛ always permits a change of subject, so it coordinates two clauses
%ʔaḥʔaaʔaƛ muučiiłšiƛna hił siy̓a ʔaḥʔaaʔaƛ ḥaakʷaaƛuk Matthew, kʷaaʔuucukqs.
%We were there for four days, me and Matthew's daughter, my granddaughter.
%JL

Sentence-intermediate \textit{ʔaḥʔaaʔaƛ} can coordinate two VPs, which share the clitic subjects (\ref{ex:intermediateand}, \ref{ex:intermediateand2}).

\ex \label{ex:intermediateand}
\begingl
\glpreamble ʔaa nunuukšiƛnišʔaał ʔaḥʔaaʔaƛ huułhuuła huuuu tuupšiʔeƛquu. //
\gla ʔaa nunuuk-šiƛ=niˑš=ʔaał ʔaḥʔaaʔaƛ huł-LR2L.a huuuu tup-šiƛ-LS=!aƛ=quu //
\glb oh sing.\textsc{dr}-\textsc{mo}=\textsc{strg.1pl}=\textsc{habit} and dance-\textsc{rp} whoa.long.time dark-\textsc{mo}-\textsc{gr}=\textsc{now}=\textsc{pssb.3} //
\glft `Oh, we sing and dance, hey for a long time, when it gets dark.' (\textbf{C}, \textit{tupaat} Julia Lucas) //
\endgl
\xe

%ʔuʔatup̓aƛin nunuuk ʔaḥʔaaʔaƛ huyaał. BM

\ex~ \label{ex:intermediateand2}
\begingl
\glpreamble ʔaƛa čaakupiiḥ čaaniʔišʔaałʔał t̓aaqyiił ʔaḥʔaaʔaƛ ʕapkšiƛ ʔuukʷił. //
\gla ʔaƛa čakup-L.iiḥ čaani=ʔiˑš=ʔaał=ʔał t̓aaqyiił ʔaḥʔaaʔaƛ ʕapk-šiƛ ʔu-L.(č)ił //
\glb two man-\textsc{pl} little.while=\textsc{strg.3}=\textsc{habit}=\textsc{pl} stand.inside.\textsc{dr} and grapple-\textsc{mo} \textsc{x}-do.to //
\glft `Two men stand inside for a little while and try to grapple each other [in wrestling games].' (\textbf{C}, \textit{tupaat} Julia Lucas) //
\endgl
\xe

As with English \textit{and}, \textit{ʔaḥʔaaʔaƛ} can be used in this way to imply order (\ref{ex:firstandthen}).

\ex \label{ex:firstandthen}
\begingl
\glpreamble ʔutwiiʔaqƛ̓in nunuuk ʔaḥʔaaʔaƛ haʔukšiƛ. //
\gla ʔu-(t)wii=!aqƛ=(y)in nunuuk ʔaḥʔaaʔaƛ haʔuk-šiƛ //
\glb \textsc{x}-first=\textsc{fut}=\textsc{weak.1pl} sing.\textsc{dr} and eat.\textsc{dr}-\textsc{mo} //
\glft `First we will sing and then eat.' (\textbf{C}, \textit{tupaat} Julia Lucas) //
\endgl
\xe

Though less common, \textit{ʔaḥʔaaʔaƛ} can also be used to coordinate participants (\ref{ex:andnp}).

\ex \label{ex:andnp}
\begingl
\glpreamble ʔaƛamitʔišʔaałʔał ʕaaḥuusʔatḥ ʔaḥʔaaʔaƛ ḥiškʷiiʔatḥ. //
\gla ʔaƛa=(m)it=ʔiˑš=ʔaał=ʔał ʕaaḥuusʔatḥ ʔaḥʔaaʔaƛ ḥiškʷiiʔatḥ //
\glb two=\textsc{pst}=\textsc{strg.3}=\textsc{habit}=\textsc{pl} Ahousaht and Hesquiaht //
\glft `There were two, the Ahousahts and the Hesquiahts.' (\textbf{C}, \textit{tupaat} Julia Lucas) //
\endgl
\xe

The coordinator \textit{ʔuḥʔi(i)š} is more constrained. It only coordinates participants (\ref{ex:7uh7is1}).

\ex \label{ex:7uh7is1}
\begingl
\glpreamble ʔuḥintʔinł ʔukʷiił n̓uw̓iiqsknaqs ʔuuḥw̓ał ḥumiis ʔuḥʔiiš c̓istuup. //
\gla ʔuḥ=int=ʔinł ʔu-(č)iił n̓uw̓iiqsu=ʔak=naqs ʔu-L.ḥw̓ał ḥumiis ʔuḥʔiiš c̓is-(š)tuˑp //
\glb be=\textsc{pst}=\textsc{habit} \textsc{x}-make father=\textsc{poss}=\textsc{pst.defn.1sg} \textsc{x}-use red.cedar and line-kind //
\glft `It was my dad that made it using red cedar and rope.' (\textbf{Q}, Sophie Billy) //
\endgl
\xe

\textit{ʔaḥʔaaʔaƛ} coordinates clauses (which differ in subject), VPs (which share a subject), and participants. \textit{ʔuḥʔi(i)š} coordinates only participants. The linker, as I will argue below, coordinates a different syntactic category: predicates. As a suffix, it has a greater degree of freedom in its sites of attachment, and its scope of coordination differs from the free morpheme coordinators.

\subsection{Attachment properties} \label{ch:link:attach}

The linker shows considerable flexibility in the stems it attaches to, attaching to nouns (\ref{ex:womangossiping}), adjectives (\ref{ex:strongbear}), verbs (\ref{ex:talkdriving}), and adverbs (\ref{ex:alsobald}).

\ex \label{ex:womangossiping}
\begingl
\glpreamble łuucmaqḥitqač̓aʔaał taakšiƛ p̓iišmita. //
\gla łuucma-(q)ḥ=(m)it=qaˑč̓a=ʔaał taakšiƛ p̓iišmita //
\glb woman-\textsc{link}=\textsc{pst}=\textsc{dubv}=\textsc{habit} always gossip.\textsc{cv} //
\glft `There was a woman who kept gossiping.' (\textbf{C}, \textit{tupaat} Julia Lucas) //
\endgl
\xe

\ex~ \label{ex:strongbear}
\begingl
\glpreamble t̓ikʷaamitwaʔiš čims ḥaaʔakqḥ. //
\gla t̓ikʷ-(y)aˑ=mit=waˑʔiš čims ḥaaʔak-(q)ḥ //
\glb dig-\textsc{cv}=\textsc{pst}=\textsc{hrsy.3} bear strong-\textsc{link} //
\glft `The bear was digging and strong.' (\textbf{C}, \textit{tupaat} Julia Lucas) //
\endgl
\xe

\ex~ \label{ex:talkdriving}
\begingl
\glpreamble ciqink̓aƛna ƛiḥaaqḥ. //
\gla ciq-(č)ink=!aƛ=naˑ ƛiḥ-(y)aˑ-(q)ḥ //
\glb speak-with=\textsc{now}=\textsc{neut.1pl} drive-\textsc{dr}-\textsc{link} //
\glft `We talked while driving.' (\textbf{C}, \textit{tupaat} Julia Lucas) //
\endgl
\xe



\vspace{5pt}

\ex~ \label{ex:alsobald}
\begingl
\glpreamble y̓uuqʷaaqḥs ʕasqii ʔaanaḥi wik hinʔałšiƛ. //
\gla y̓uuqʷaa-(q)ḥ=s ʕasqii ʔaanaḥi wik hinʔał-šiƛ //
\glb also-\textsc{link}=\textsc{strg.1sg} bald only \textsc{neg} realize-\textsc{mo} //
\glft `I'm also bald but I don't know it.' (\textbf{C}, \textit{tupaat} Julia Lucas) 

[Context: My friend is going bald. I'm also going bald but I don't look in the mirror much and haven't noticed.\footnotemark] //
\endgl
\xe

\footnotetext{This scenario was constructed to mirror an example present in \cite{sapir1939}.}

However, the linker cannot attach complementizers (\ref{ex:hardsmall1}, \ref{ex:hardsmall2}).

\ex~ \label{ex:hardsmall1}
\begingl
\glpreamble ʔuušcukʔisit ʔani ʔunaḥʔisitqa. //
\gla ʔuušcuk=ʔis=(m)it ʔani ʔunaḥ=ʔis=(m)it=qaˑ //
\glb difficiult=\textsc{dim}=\textsc{pst} \textsc{comp} small=\textsc{dim}=\textsc{pst}=\textsc{embd} //
\glft `It was a little difficult (to do) because it's small.' (\textbf{B}, Bob Mundy) //
\endgl
\xe

\ex~ \label{ex:hardsmall2}
\begingl
\glpreamble *ʔuušcukʔisit ʔaniqḥ ʔunaḥʔisitqa. //
\gla ʔuušcuk=ʔis=(m)it ʔani-(q)ḥ ʔunaḥ=ʔis=(m)it=qaˑ //
\glb difficult=\textsc{dim}=\textsc{pst} \textsc{comp}-\textsc{link} small=\textsc{dim}=\textsc{pst}=\textsc{embd} //
\glft Intended: `It was a little difficult (to do) because it's small.' (\textbf{B}, Bob Mundy) //
\endgl
\xe

From only this data, the linker appears to distinguish morphologically between content and function categories. Another way of expressing this content/function division is by appealing to what can serve as a syntactic predicate in Nuuchahnulth (see \ref{ch:clause}). Nouns, adjectives, and verbs may all be predicative, and while adverbs are not syntactic predicates themselves, they along with their verb create a main predicate. I will return to the matter of adverbs in \S\ref{ch:link:2p}. Complementizers, on the other hand, are only connective material and cannot be the main predicate of a clause, nor can they be part of the predicative phrase. In following sections, I will refer to the predicate in linker constructions that hosts the linker as the ``linked predicate" and the predicate that lacks it as the ``unlinked" or ``non-linked" predicate.

%\subsection{Syntactic properties of linked predicates} \label{ch:link:syn}

\subsection{Clause Heading} \label{ch:link:clause}

A predicate with a linker attached may not head a matrix or dependent clause. I first give some evidence on the flexibility of the relative ordering of the linker, and then examine when they are and are not allowed in matrix and dependent clauses.

In a sentence with two predicates, one with the linker and one without, the ordering does not typically make a difference.\footnote{There are some cases where altering the ordering affects grammaticality judgments. I believe this has to do with a preference for the predicate with the linker attached to come first and, between two predicates, for certain semantic classes to host the linker over others. I address these in \S\ref{ch:link:preferences}.} It is possible for either predicate in an utterance to host the linker, as in (\ref{ex:speakoutsidebob1}, \ref{ex:speakoutsidebob2}).

\ex \label{ex:speakoutsidebob1}
\begingl
\glpreamble hitaasḥitaḥ ciiqciiqa. //
\gla hitaas-(q)ḥ=(m)it=(m)aˑḥ ciq-LR2L.a //
\glb be.outside-\textsc{link}=\textsc{pst}=\textsc{real.1sg} speak-\textsc{rp} //
\glft `I was speaking outside.' (\textbf{B}, Bob Mundy) //
\endgl
\xe

\ex~ \label{ex:speakoutsidebob2}
\begingl
\glpreamble ciiqciiqaqḥitaḥ hitaas. //
\gla ciq-LR2L.a-(q)ḥ=(m)it=(m)aˑḥ  hitaas  //
\glb speak-\textsc{rp}-\textsc{link}=\textsc{pst}=\textsc{real.1sg} be.outside //
\glft `I was speaking outside.' (\textbf{B}, Bob Mundy) //
\endgl
\xe

Just as either predicate in a construction may take the linker, the linker may occur either on the first (\ref{ex:speakoutsidefidelia1}) or second (\ref{ex:speakoutsidefidelia2}) predicate in the utterance.

\ex \label{ex:speakoutsidefidelia1}
\begingl
\glpreamble ƛ̓aaʔaasḥintniš ciiqciiqa. //
\gla ƛ̓aaʔaas-(q)ḥ=int=niš ciq-LR2L.a //
\glb be.outside-\textsc{link}=\textsc{pst}=\textsc{strg.1pl} speak-\textsc{rp} //
\glft `We were speaking outside.' (\textbf{N}, Fidelia Haiyupis) //
\endgl
\xe

\ex~ \label{ex:speakoutsidefidelia2}
\begingl
\glpreamble ciiqciiqamitniš ƛ̓aaʔaasḥ. //
\gla ciq-LR2L.a=mit=niˑš ƛ̓aaʔaas-(q)ḥ //
\glb speak-\textsc{rp}=\textsc{pst}=\textsc{strg.1pl} be.outside-\textsc{link} //
\glft `We were speaking outside.' (\textbf{N}, Fidelia Haiyupis) //
\endgl
\xe

Although there is flexibility as to which predicate takes the linker, clauses may not be headed by a single predicate with a linker. This can be seen for main clauses in (\ref{ex:longawake}, \ref{*ex:longawake}) below.

\ex \label{ex:longawake}
\begingl
\glpreamble qiiʔiłs ƛupkaaqḥ. //
\gla qiiʔił=s ƛupk-(y)aˑ-(q)ḥ //
\glb lie.in.bed.a.long.time=\textsc{strg.1sg} awake-\textsc{dr}-\textsc{link} //
\glft `I lay awake inside for a long time.' (\textbf{N}, \textit{yuułnaak} Simon Lucas) //
\endgl
\xe

\ex~ \label{*ex:longawake}
\begingl
\glpreamble *ƛupkaaqḥs qii. //
\gla ƛupk-(y)aˑ-(q)ḥ=s qii //
\glb awake-\textsc{dr}-\textsc{link}=\textsc{strg.1sg} long.time //
\glft Intended: `I lay awake for a long time.' (\textbf{N}, \textit{yuułnaak} Simon Lucas) //
\endgl
\xe

(\ref{*ex:longawake}) has undergone two changes relative to (\ref{ex:longawake}): (i) the words have been rearranged, and (ii) the ending \textit{-°ił}, a predicative location (Davidson, \textit{forthcoming}) has been taken off the adverb \textit{qii}. The former change should not affect the grammaticality of the sentence, as demonstrated in (\ref{ex:speakoutsidefidelia1}, \ref{ex:speakoutsidefidelia2}). But the latter change creates an utterance with ``linked" predicate followed by the syntactically non-predicative adverb \textit{qii} (\ref{*ex:longawake}). In contrast, (\ref{ex:longawake}) contains two full predicates. Because the adverb \textit{qii} cannot be a syntactic predicate, (\ref{*ex:longawake}) only has one predicative word with a linker morpheme, and no further predicate for that linker to coordinate with.

Like main clauses, a dependent clause may not be headed by a single predicate with a linker morpheme, as shown in (\ref{ex:yourehere}, \ref{*ex:yourehere}).

\ex \label{ex:yourehere}
\begingl
\glpreamble ʔuuʕaqstuƛaḥ ʔanik hił ʔaḥkuu. //
\gla ʔuuʕaqstuƛ=(m)aˑḥ ʔani=k hił ʔaḥkuu //
\glb be.happy.\textsc{mo}=\textsc{real.1sg} \textsc{comp}=\textsc{2sg} be.at \textsc{d1} //
\glft `I'm happy you're here.' (\textbf{B}, Bob Mundy) //
\endgl
\xe

\ex~ \label{*ex:yourehere}
\begingl
\glpreamble *ʔuuʕaqstuƛaḥ ʔanik hiłḥ ʔaḥkuu. //
\gla ʔuuʕaqstuƛ=(m)aˑḥ ʔani=k hił-(q)ḥ ʔaḥkuu //
\glb be.happy.\textsc{mo}=\textsc{real.1sg} \textsc{comp}=\textsc{2sg} be.at-\textsc{link} \textsc{d1} //
\glft Intended: `I'm happy you're here.' (\textbf{B}, Bob Mundy) //
\endgl
\xe

Although the word \textit{hił} `be at' frequently takes the linker in texts, it is ungrammatical in (\ref{*ex:yourehere}), where it is the sole predicate of the dependent clause. I was able to replicate a similar example with a Checleseht speaker from the other end of the dialect continuum (\ref{ex:sawabear}, \ref{*ex:sawabear}).

\ex \label{ex:sawabear}
\begingl
\glpreamble n̓aacsiičƛintiis ʔin hił čimsʔii maḥt̓eekitk. //
\gla n̓aaca-iˑčiƛ=int=(y)iis ʔin hił čims=ʔiˑ maḥt̓ii=ʔak=ʔiˑtk //
\glb see.\textsc{dr}-\textsc{in}=\textsc{pst}=\textsc{weak.1sg} \textsc{comp} be.at bear=\textsc{art} house=\textsc{poss}=\textsc{defn.2sg} //
\glft `I saw there was a bear at your house.' (\textbf{Q}, Sophie Billy) //
\endgl
\xe

\ex~ \label{*ex:sawabear}
\begingl
\glpreamble *n̓aacsiičƛintiis ʔin hiłḥ čimsʔii maḥt̓eekitk. //
\gla n̓aaca-iˑčiƛ=int=(y)iis ʔin hił-(q)ḥ čims=ʔiˑ maḥt̓ii=ʔak=ʔiˑtk //
\glb see.\textsc{dr}-\textsc{in}=\textsc{pst}=\textsc{weak.1sg} \textsc{comp} be.at-\textsc{link} bear=\textsc{art} house=\textsc{poss}=\textsc{defn.2sg} //
\glft Intended: `I saw there was a bear at your house.' (\textbf{Q}, Sophie Billy) //
\endgl
\xe

From these examples, I conclude that the linker requires two predicates to coordinate. This means that the syntactic head of a clause cannot be a predicate with linker morphology. The head must either be the linker itself, or the predicate without linker morphology.

\subsection{Sharing second position suffixes and clitics} \label{ch:link:second}

Nuuchahnulth has a series of clausal second-position enclitics, which include tense and subject-mood portmanteaus. In a linker construction, both predicates share the same subject, mood, and tense.

\ex \label{ex:stopatmyhouse}
\begingl
\glpreamble hiłḥʔum maḥt̓iiʔakqs wiinapuƛ. //
\gla hił-(q)ḥ=!um maḥt̓iˑ=ʔak=qs wiinapuƛ //
\glb be.at-\textsc{link}=\textsc{cmfu.2sg} house=\textsc{poss}=\textsc{defn.1sg} stop.\textsc{mo} //
\glft `Stop at my house.' (\textbf{N}, Fidelia Haiyupis) //
\endgl
\xe

The command portmanteau \textit{=!um} in (\ref{ex:stopatmyhouse}) syntactically scopes\footnote{Because of the utility of the concept of scoping in this discussion, I will use the word ``scope" from here on to refer to a syntactic element that has an effect over another syntactic element. This should not be confused with scopal semantics.} over both predicates. My consultant did not accept this as possibly meaning that someone else was stopping. If these clitics belong to the clause as a whole, which there is good independent reason to believe (\citealt[35--36]{rose1981}, \citealt[42--50]{woo2007b}), the linker coordinates predicates below the level of the clause.

In addition to the clausal second-positions, there are some suffixes which I claim appear in a predicative second position \citep{inman2018}. [[TODO: Regurgitate a summary of the ``predicate position" argument in the clause chapter!]] These include modals and, importantly, the linker itself. The modals in this predicative second position seem to be shared across linked predicates, in a similar fashion to the clitics.

\vspace{5pt}

\noindent Context for (\ref{ex:drivinghome}): I am taking a friend home and we are leaving a gathering.

\ex \label{ex:drivinghome}
\begingl
\glpreamble waałšiƛw̓it̓asniš ƛiḥaaqḥ. //
\gla wał-šiƛ-LS-w̓it̓as=niˑš ƛiḥ-(y)aˑ-qḥ //
\glb go.home-\textsc{mo}-\textsc{grad}-going.to=\textsc{strg.1pl} drive-\textsc{cv}-\textsc{link} //
\glft `We're going to drive home.' (\textbf{C}, \textit{tupaat} Julia Lucas) //
\endgl
\xe

Both verbs in (\ref{ex:drivinghome}) share the semantics of the modal suffix \textit{-w̓it̓as}, because both the driving and the going home are intentional, not-yet-occurred events. I confirmed the sharing of the subject portmanteau \textit{=niˑš} by asking if it were possible to say (\ref{ex:drivinghome}) to mean that we were going to walk home but someone else was driving elsewhere. My consultant said no: (\ref{ex:drivinghome}) must mean that it is we who are going to go home and we who are doing it driving in a car.

(\ref{ex:readrain}) and (\ref{ex:readrain2}) provide a situation where the obligatory subject sharing creates an odd interpretation. I was asking about different activities depending on the weather. The felicitous expression without the linker is in (\ref{ex:readrain}). My rephrase in (\ref{ex:readrain2}) with the linker was met with an immediate laugh.

\ex \label{ex:readrain}
\begingl
\glpreamble n̓ačaałaḥʔaała m̓iƛaaʔaƛquu. //
\gla n̓ačaał=(m)aˑḥ=ʔaała m̓iƛ-(y)aˑ=!aƛ=quu //
\glb read=\textsc{real.1pl}=\textsc{habit} rain-\textsc{cv}=\textsc{now}=\textsc{pssb.3} //
\glft `I read whenever it rains.' (\textbf{B}, Bob Mundy) //
\endgl
\xe

\ex \label{ex:readrain2}
\begingl
\glpreamble \#n̓ačaałaḥʔaała m̓iƛaaqḥ. //
\gla n̓ačaał=(m)aˑḥ=ʔaała m̓iƛ-(y)aˑ-(q)ḥ //
\glb read=\textsc{real.1pl}=\textsc{habit} rain-\textsc{cv}-\textsc{link} //
\glft \# `I read and I am raining.' (\textbf{B}, Bob Mundy) //
\endgl
\xe

Both predicates in a linker construction share the semantics of the second-position clitics, which importantly means they share a subject. They also share at least modal suffixes from what I term the second-position predicate position.

%TODO, and a sentence may not be composed of two predicates, both with linkers (\ref{ex:*someonespoke}).

\begin{comment}
\ex \label{ex:someonespoke}
\begingl
\glpreamble ʔuušqḥʔaƛ ciqšiƛ.//
\gla ʔuuš-qḥ=ʔaƛ ciq-šiƛ //
\glb some-\textsc{link}=\textsc{now} speak-\textsc{mo} //
\glft `Someone spoke.' //
\endgl
\xe

\ex~ \label{ex:*someonespoke}
\begingl
\glpreamble *ʔuušqḥʔaƛ ciqšiƛḥ.//
\gla *ʔuuš-(q)ḥ=ʔaƛ ciq-šiƛ-(q)ḥ //
\glb *some-\textsc{link}=\textsc{now} speak-\textsc{mo}-\textsc{link} //
\glft Intended: `Someone spoke.' //
\endgl
\xe
\end{comment}

%The above examples suggest that the predicate the linker attaches to (along with its complements) may not be a complete sentence, and is dependent on another clause.

\subsection{Linkers on non-verbs} \label{ch:link:nonverb}

The examples so far have focused on linkers attached to verbs. For English speakers, verbal coordination is perhaps the easiest example of syntactic predicates sharing inflectional properties. However, as detailed in \S\ref{ch:link:attach}, it is possible for the linker to attach to a wide variety of non-verbs. The properties of the linker are identical on non-verbs, but it is worthwhile to look at how this works.

Perhaps the most common type of non-verbal predicate that receives the linker is quantificational adjectives (henceforth, quantifiers). The presence or absence of the linker on a quantifier significantly changes the possible interpretations for the sentence. With a bare (non-linked) quantifier, the quantifier may be interpreted as a syntactic object (\ref{ex:findsomething}) and may not come before the verb (\ref{ex:*findsomething}). When a linker is attached, the quantifier must be interpreted as the subject and may either come before (\ref{ex:findsomeone}) or after the verb (\ref{ex:someonefind}).

\vspace{5pt}

\noindent Context for (\ref{ex:findsomething}--\ref{ex:someonefind}): My family and I are looking for a Christmas present for my sister.

\ex \label{ex:findsomething}
\begingl
\glpreamble ʔuuwaʔaƛ ʔuuš.//
\gla ʔu-L.waƛ=!aƛ ʔuuš //
\glb \textsc{x}-find=\textsc{now} some //
\glft `He/she found something.' (*? Someone found it) (\textbf{C}, \textit{tupaat} Julia Lucas) //
\endgl
\xe

\ex~ \label{ex:*findsomething}
\begingl
\glpreamble *ʔuuš ʔuuwaʔaƛ.//
\gla ʔuuš ʔu-L.waƛ=!aƛ //
\glb some \textsc{x}-find=\textsc{now} //
\glft Intended: `He/she found something.' (\textbf{C}, \textit{tupaat} Julia Lucas) //
\endgl
\xe

\ex~ \label{ex:findsomeone}
\begingl
\glpreamble ʔuuwaʔaƛ ʔuušqḥ.//
\gla ʔu-L.waƛ=!aƛ ʔuuš-qḥ //
\glb \textsc{x}-find=\textsc{now} some-\textsc{link} //
\glft `Someone found it.' (*He/she found something) (\textbf{C}, \textit{tupaat} Julia Lucas) //
\endgl
\xe

\ex~ \label{ex:someonefind}
\begingl
\glpreamble ʔuušqḥʔaƛ ʔuuwaƛ.//
\gla ʔuuš-qḥ=!aƛ ʔu-L.waƛ //
\glb some-\textsc{link}=\textsc{now} \textsc{x}-find //
\glft `Someone found it.' (*He/she found something) (\textbf{C}, \textit{tupaat} Julia Lucas) //
\endgl
\xe

\begin{comment}
\ex \label{ex:someonefind2}
\begingl
\glpreamble ʔuušqḥ ʔuuwaʔaƛ.//
\gla ʔuuš-qḥ ʔu-L.waƛ=!aƛ //
\glb some-\textsc{link} \textsc{x}-find=\textsc{now} //
\glft `Someone found it.' (*He/she found something) //
\endgl
\xe
\end{comment}

In (\ref{ex:findsomeone}, \ref{ex:someonefind}), the two predicates being linked are \textit{some} and \textit{find}. Because quantifiers are possible predicates in Nuuchahnulth, the same analysis applied to two linked verbs can apply here: These are two predicates that share a subject. That is, there is a (null) third-person subject that is shared between the predicates \textit{some} and \textit{find}: ``There exists an \textit{x} such that \textit{some(x)} and \textit{find(x,y)}." This subject sharing makes the objective reading impossible in (\ref{ex:findsomeone}, \ref{ex:someonefind}).

Julia rejected an interpretation of (\ref{ex:findsomething}) where non-linked \textit{ʔuuš} `some' was interpreted as the subject. However, in another context she produced (\ref{ex:talented}), where \textit{ʔuuš} `some' is in fact given a subjective interpretation.

\ex \label{ex:talented}
\begingl
\glpreamble ʔuušʔiišʔaał wic̓ik, ʔuuš ʕac̓ik, ʔuuš ʔum̓aaqƛ ʔuuy̓ip. //
\gla ʔuuš=ʔiˑš=ʔaał wic̓ik, ʔuuš ʕac̓ik, ʔuuš ʔum̓aaqƛ ʔu-iˑy̓ip //
\glb some=\textsc{strg.3}=\textsc{habit} not.talented, some talented, some able.to \textsc{x}-get //
\glft ‘Some are not talented, some are talented, some are able to get (the challenge).’ (\textbf{C}, \textit{tupaat} Julia Lucas) //
\endgl
\xe

In (\ref{ex:talented}), the first two verbs are intransitive, so there is no other syntactic interpretation for \textit{ʔuuš} `some' other than the subjective one. The final verb is transitive, but the parallelism with the first two clauses primes the listener to interpret \textit{ʔuuš} as subjective. The fact that Julia did not add a linker in (\ref{ex:talented}) shows that a subjective interpretation is possible for non-linked quantifiers. %However, when there is an ambiguity, as in (\ref{ex:findsomething}), the absence of the linker is a clue that the speaker had an objective interpretation in mind because the presence of a linker would force an unambiguous subjective reading.

This observation about quantifiers holds true for other adjectives and also nouns, as seen in (\ref{ex:canoesink1}--\ref{ex:canoesink3}). The initial sentence puts two clauses together with a complementizer (\ref{ex:canoesink1}), but can be rephrased without a complementizer by using the linker (\ref{ex:canoesink2}, \ref{ex:canoesink3}).

\vspace{5pt}

\noindent Context for (\ref{ex:canoesink1}--\ref{ex:canoesink3}): I arrived on the beach in a canoe. I left my canoe and went into town. While I'm inside, my canoe is carried out on the tide and capsizes. One person left behind on the beach sees it. (\ref{ex:canoesink1}) was suggested by my consultant, and we worked to rephrase it as (\ref{ex:canoesink2}) and (\ref{ex:canoesink3}). My consultant was adamant that (\ref{ex:canoesink1}) and (\ref{ex:canoesink2}) meant exactly the same thing. If this is true, then the linker is not adding any deep semantic content.\footnote{My analysis ends up putting in a relation \textsc{and}. While this may not be totally meaningless, it is virtually meaningless.}

\ex \label{ex:canoesink1}
\begingl
\glpreamble c̓awaakitwaʔiš n̓aacsa niiʔatu č̓apac.//
\gla c̓awaak=it=waˑʔiš n̓aacsa niiʔatu č̓apac //
\glb one=\textsc{pst}=\textsc{hrsy.3} see.\textsc{cv} sink canoe //
\glft `I hear that one (person) saw the canoe sink.' (\textbf{C}, \textit{tupaat} Julia Lucas) //
\endgl
\xe

\ex~ \label{ex:canoesink2}
\begingl
\glpreamble c̓awaakḥitwaʔiš n̓aacsa niiʔatu č̓apac.//
\gla c̓awaak-(q)ḥ=it=waˑʔiš n̓aacsa.\textsc{cv} niiʔatu č̓apac //
\glb one-\textsc{link}=\textsc{pst}=\textsc{hrsy.3} see.\textsc{dr} sink canoe //
\glft `I hear that one (person) saw the canoe sink.' (\textbf{C}, \textit{tupaat} Julia Lucas) //
\endgl
\xe

\ex~ \label{ex:canoesink3}
\begingl
\glpreamble quuʔasqḥitwaʔiš n̓aacsa niiʔatu č̓apacʔi.//
\gla quuʔas-(q)ḥ=it=waˑʔiš n̓aacsa niiʔatu č̓apac=ʔiˑ //
\glb person-\textsc{link}=\textsc{pst}=\textsc{hrsy.3} see.\textsc{cv} sink canoe=\textsc{art} //
\glft `I hear that a person saw the canoe sink.' (\textbf{C}, \textit{tupaat} Julia Lucas) //
\endgl
\xe

%It is important that the complementizer in (\ref{ex:canoesink1}) creates an overt subordinate clause for , while in the rephrase with the linker (\ref{ex:canoesink2}), there is no complementizer. This supports the data from \S\ref{ch:link:clause} suggesting that the linker itself forms a subordinate (and not a matrix) clause. (\ref{ex:canoesink3}) simply shows, again, that nouns are valid hosts for the linker, just as much as adjectives.

Using the same setup as (\ref{ex:canoesink1}--\ref{ex:canoesink3}), I elicited sentences from another speaker. This consultant initially proposed the sentence in (\ref{ex:canoesink4}). I proposed (\ref{ex:canoesink5}) by removing the linker, which he rejected, and then (\ref{ex:canoesink6}), which he accepted.

\ex \label{ex:canoesink4}
\begingl
\glpreamble n̓aacsiičiƛweʔin c̓awaakḥ niiʔatu č̓apac. //
\gla n̓aacsa-iˑčiƛ=weˑʔin c̓awaak-(q)ḥ niiʔatu č̓apac //
\glb see.\textsc{cv}-\textsc{in}=\textsc{hrsy.3} one-\textsc{link} sink canoe //
\glft `I hear that one (person) saw the canoe sink.' (\textbf{B}, Bob Mundy) //
\endgl
\xe

\ex~ \label{ex:canoesink5}
\begingl
\glpreamble *n̓aacsiičiƛweʔin c̓awaak niiʔatu č̓apac. //
\gla n̓aacsa-iˑčiƛ=weˑʔin c̓awaak niiʔatu č̓apac //
\glb see.\textsc{cv}-\textsc{in}=\textsc{hrsy.3} one sink canoe //
\glft Intended: `I hear that one saw the canoe sink.' (\textbf{B}, Bob Mundy) //
\endgl
\xe

\ex~ \label{ex:canoesink6}
\begingl
\glpreamble n̓aacsiičiƛweʔin c̓awaakḥ quuʔas niiʔatu č̓apac. //
\gla n̓aacsa.\textsc{cv}-iˑčiƛ=weˑʔin c̓awaak-(q)ḥ quuʔas niiʔatu č̓apac //
\glb see-\textsc{in}=\textsc{hrsy.3} one-\textsc{link} person sink canoe //
\glft `I hear that one person saw the canoe sink.' (\textbf{B}, Bob Mundy) //
\endgl
\xe

Bob's response to removing the linker in (\ref{ex:canoesink5}) was to say, ``It's not complete. One what? What did one see?" Following the basic structure of the Nuuchahnulth clause (\S\ref{ch:clause}), the participants of the syntactic predicate \textit{n̓aacsiičiƛ} `see' should be \textit{c̓awaak} `one' and \textit{niiʔatu č̓apac} `sink canoe'. But \textit{c̓awaak}, as an adjective, cannot be a full NP participant without an article \citep{jacobsen1979}. So it is stranded and the utterance (\ref{ex:canoesink5}) is nonsensical. The presence of the linker in my consultant's initial proposed sentence (\ref{ex:canoesink4}) forces `one' to be coreferenced with the subject of `see', as already shown for the quantifiers in (\ref{ex:findsomething}--\ref{ex:someonefind}). The other participant of the seeing act (what is seen) is the dependent clause `sink canoe'.

Example (\ref{ex:canoesink6}) shows that the linked clause not headed by a verb can include more than one word. Here \textit{c̓awaak} `one' is a predicate with a subject \textit{quuʔas} `person'.  This dependent clause also interrupts the matrix predicate \textit{n̓aacsiičiƛ} `see' and its clausal object \textit{niiʔatu č̓apac} `the canoe sink.' In \S\ref{ch:link:clause} I argued that linker constructions were either headed by the predicate lacking linker morphology or the linker itself. The syntax of (\ref{ex:canoesink6}), where a predicate, linker, and its participant can all interrupt another predicate with and its participant is evidence in favor of an analysis where the non-linked predicate is the sentential head, and the linker forms a dependent clause. A rough bracketing of (\ref{ex:canoesink6}) based on this preliminary analysis is given in (\ref{ex:canoesink6.2}).

\ex \label{ex:canoesink6.2}
\begingl
\gla {[}n̓aacsa.\textsc{cv}-iˑčiƛ=weˑʔin {[}c̓awaak-(q)ḥ quuʔas{]\textsubscript{linked\_clause}} {[}niiʔatu č̓apac{]\textsubscript{participant\_of\_see} ]} //
\glb see-\textsc{in}=\textsc{hrsy.3} one-\textsc{link} person sink canoe //
%\glft `I hear that one person sees the canoe sink.'
\endgl
\xe

[[TODO: track down subscript rendering problem above]]

\begin{comment}
ACTUALLY*2: This works quite well for showing a deictic predicate. Unfortunately it is XL so I cannot use it. Oh well!

ACTUALLY! I think the below is wrong. If you look at XL's sentence, the predicate is deictic ʔaḥʔaa, to which the linker still attaches! This would mean the only counterexample would be Adv V+link. Investigate this further.

But occasionally the linker may occur on the sole predicate in a sentence. This appears to contradict examples (\ref{ex:someonespoke}) and (\ref{ex:*someonespoke}), but the translation provided for these ``dangling" linkers always indicates they are notionally attached to the preceding sentence. I have 1 (TODO: look for more, update number) example from my corpus, involving the word \textit{qʷis} `do so'.\footnote{I am not here counting examples from \textit{tupaat} Julia Lucas, who is unique in always uses the the conjunction \textit{ʔunʔuuƛ} with a linker attached. I believe she has a different lexical entry for the word, and will explain in section (TODO).} I give one example below:

TODO, this is from Charlie Lucas, who I do not have permissions to share. Update it with a sharable example.

Context: \textit{łačiƛni wa. ʔuušciłʔap̓aƛukni nunuuk. ʔuušciłʔap̓aƛukni huyaał.} `We've let it go, haven't we? It has become hard for us to sing. It has become hard for us to dance.'

\ex \label{ex:danglinglinker}
\begingl
\glpreamble ʔaḥʔaa qʷisḥnii.//
\gla ʔaḥʔaa qʷis-(q)ḥ=niˑ //
\glb DDYN do.so-\textsc{link}=\textsc{neut.1pl} //
\glft `That's what happened to us.' //
\endgl
\xe

Although the one predicate is 

\end{comment}

\subsection{Ordering in linker constructions} \label{ch:link:participants}

I have already demonstrated that the non-linked predicate may be separated from its complement by an intervening linked predicate (\ref{ex:canoesink4}, \ref{ex:canoesink6}, \ref{ex:canoesink6.2}). The reverse ordering is also possible. The linked predicate may be separated from its direct object by the non-linked predicate. In (\ref{ex:workathome}) the verb \textit{hił} `be at' and its object `my house' are contiguous, but in (\ref{ex:workathome2}) they are separated by the non-linked predicate \textit{mamuuk} `work'.

\ex \label{ex:workathome}
\begingl
\glpreamble hiłḥitin maḥt̓iiʔakqas mamuuk. //
\gla hił-(q)ḥ=(m)it=(m)in maḥt̓ii=ʔak=qas mamuuk //
\glb be.at-\textsc{link}=\textsc{pst}=\textsc{real.1pl} house=\textsc{poss}=\textsc{defn.1sg} work //
\glft `We worked at my house.' (\textbf{B}, Bob Mundy) //
\endgl
\xe

\ex~ \label{ex:workathome2}
\begingl
\glpreamble hiłḥitin mamuuk maḥt̓iiʔakqas. //
\gla hił-(q)ḥ=(m)it=(m)in mamuuk maḥt̓ii=ʔak=qas //
\glb be.at-\textsc{link}=\textsc{pst}=\textsc{real.1pl} work house=\textsc{poss}=\textsc{defn.1sg} //
\glft `We worked at my house.' (\textbf{B}, Bob Mundy) //
\endgl
\xe

Not only is (\ref{ex:workathome2}) grammatical but this is often the structure speakers prefer. For one of my consultants, Northern dialect speaker Fidelia Haiyupis, this kind of object separation was acceptable when the linked predicate was separated from its object (\ref{ex:fhhil}) but not when the non-linked predicate was separated from its object (\ref{ex:fhqc}, \ref{*ex:fhqc}). I can only note that this may be a feature of Northern dialects, but it is unclear from the small amount of data that I have. %In the above examples, the linked predicate is the one separated from its direct object, but it can also be the non-linked predicate that is separated from its object, as already seen in (\ref{ex:canoesink4}, \ref{ex:canoesink6}).

\ex \label{ex:fhhil}
\begingl
\glpreamble hiłḥsiiš ʔuukʷiił č̓upč̓upšumł maḥt̓iiʔakʔik. //
\gla hił-(q)ḥ=siˑš ʔu-L.(č)iił č̓upč̓upšumł maḥt̓ii=ʔak=ʔik //
\glb be.at-\textsc{link}=\textsc{strg.1sg} \textsc{x}-make sweater house=\textsc{poss}=\textsc{defn.2sg} //
\glft `I am making a sweater at your house.' (\textbf{N}, Fidelia Haiyupis) //
\endgl
\xe

\ex~ \label{ex:fhqc}
\begingl
\glpreamble ʔuuct̓iiḥs Queens Cove ƛiḥaaqḥ. //
\gla ʔuuct̓iiḥ=s Queens Cove ƛiḥ-(y)aˑ-(q)ḥ //
\glb go.toward.\textsc{dr}=\textsc{strg.1sg} Queens Cove drive-\textsc{cv}-\textsc{link} //
\glft `I am driving to Queens Cove.' (\textbf{N}, Fidelia Haiyupis) //
\endgl
\xe

\ex~ \label{*ex:fhqc}
\begingl
\glpreamble *ʔuuct̓iiḥs ƛiḥaaqḥ Queens Cove. //
\gla ʔuuct̓iiḥ=s ƛiḥ-(y)aˑ-(q)ḥ Queens Cove //
\glb go.toward.\textsc{dr}=\textsc{strg.1sg} drive-\textsc{cv}-\textsc{link} Queens Cove //
\glft Intended: `I am driving to Queens Cove.' (\textbf{N}, Fidelia Haiyupis) //
\endgl
\xe

\noindent For most speakers, however, both types of ``interruption" are possible.

\subsection{The linker and the predicate complex} \label{ch:link:2p}

Like many bound morphemes in Nuuchahnulth, the linker appears to attach to the first word in some clause. This has already been seen in (\ref{ex:alsobald}), repeated as (\ref{ex:alsobald2}) below.

\ex \label{ex:alsobald2}
\begingl
\glpreamble y̓uuqʷaaqḥs ʕasqii ʔaanaḥi wik hinʔałšiƛ. //
\gla y̓uuqʷaa-qḥ=s ʕasqii ʔaanaḥi wik hinʔał-šiƛ //
\glb also-\textsc{link}=\textsc{strg.1sg} bald only \textsc{neg} aware-\textsc{mo} //
\glft `I'm also bald but I don't know it.' (\textbf{C}, \textit{tupaat} Julia Lucas) //
\endgl
\xe

The two predicates being tied together in (\ref{ex:alsobald2}) sentence are `also bald' and `only not know (it).' The linker appears on the preposed adverb \textit{y̓uuqʷaa} of the first predicate.	 Examples like this are difficult to gather directly as they require special context and it is possible to express the same meaning without the linker, but a few examples occur in the Nootka Texts. In (\ref{takeoutguns}) the linker also attaches to the preceding adverb of its linked predicate `still at war', and links that to the still later predicate `grab their guns.'

\ex \label{takeoutguns}
\begingl
\glpreamble ʔeʔimqḥʔaƛquuweʔin hitaḥtačiƛ sukʷiʔaƛ puuʔakʔiʔał. //
\gla ʔeʔim-(q)ḥ=!aƛ=quu=weˑʔin hitaḥta-čiƛ su-kʷiƛ=!aƛ puu=ʔak=ʔiˑ=ʔał //
\glb first-\textsc{link}=\textsc{now}=\textsc{pssb.3}=\textsc{hrsy.3} go.out.to.sea-\textsc{mo} hold-\textsc{mo}=\textsc{now} gun=\textsc{poss}=\textsc{art}=\textsc{pl} //
\glft `As soon as they left the land, they would take their guns.' (\textbf{B}, \citealt[395]{sapir1955}) //
\endgl
\xe

In (\ref{stillatwar}), the linker again attaches to an adverb \textit{ʔiiqḥii} `still', and links the entire predicate `still doing war' to the earlier predicate \textit{qʷis} `do thus.'

\ex \label{stillatwar}
\begingl
\glpreamble qiiḥsn̓aakck̓in ʔaḥ qʷiyiič [[qʷis] [ʔiiqḥii\textbf{qḥ} hitačink maatmaasʔi]] qaḥsaap̓aƛquuweʔin č̓amuʔałʔaƛquu yuułuʔiłʔatqḥ huuʕiiʔatḥuʔałʔaƛquu. //
\gla qiiḥsn̓aak-ck̓in ʔaḥ qʷiyi=(y)ii=č [[qʷis] [ʔiiqḥii-\textbf{(q)ḥ} hitačink maatmaas=ʔiˑ]] qaḥ-saˑp=!aƛ=quu=weˑʔin  č̓am-uʔał=!aƛ=quu yuułuʔiłʔatḥ-(q)ḥ huuʕiiʔatḥ-uʔał=!aƛ=quu. //
\glb long.time-\textsc{dim} \textsc{d1} when=\textsc{weak.3}=\textsc{hrsy} do.thus still-\text{link} go.against tribe.\textsc{pl}=\textsc{art} kill-\textsc{mo.caus}=\textsc{now}=\textsc{pssb.3}=\textsc{hrsy.3} canoe-see=\textsc{now}=\textsc{pssb.3} Ucluelet-\textsc{link} Huuayaht-see=\textsc{pssb.3}=\textsc{hrsy.3} //
\glft `For a little longer after this happened, while the tribes were still at war, the Ucluelets would kill Huu-ay-ahts when they saw their canoes.' (\textbf{B}, \citealt[392]{sapir1955}) //
\endgl
\xe

TODO: NOTE: What is ʔiiqḥiiqḥ linking? It isn't qʷis, it's a sentence shoved in a longer sentence. Go look at preceding sentences. Is there a shorter example?

These examples, as well the case of modal suffix scoping have led me to believe there is a phrasal unit between the clause (where the second position clitics scope) and the main predicate. I have dubbed this the ``predicate phrase." This phrase consists maximally of the predicate word and preceding adverbs. The predicate linker will attach to the first word in the predicate phrase, whether that is the predicate word itself or a preceding adverb. [[TODO: Move the main arguments up to the clause section]]

\subsection{Dangling linkers} \label{ch:link:dangling}

There is one case I know of where the linker does not appear to be linking its predicate to anything. I believe that the interpretation shows that there is an elided phrase (\ref{takecare}).

\ex \label{takecare}
\begingl
\glpreamble ʔuʔaałukḥʔiʔał. //
\gla ʔu-!aałuk-(q)ḥ=!iˑ=ʔał //
\glb \textsc{x}-look.after-\textsc{link}=\textsc{cmmd.2sg}=\textsc{habit} //
\glft `Take care!' (\textbf{N}, Fidelia Haiyupis) //
\endgl
\xe

The meaning of (\ref{takecare}) is ``Farewell, look after yourself in whatever you're doing." But ``whatever you're doing" is dropped from the sentence. I think that the linker is a leftover from the elided phrase. %These kinds of ``dangling" linkers are uncommon, and in my experience speakers won't accept them out of the blue unless it is a formulaic expression.

Adam: It's a typical farewell in Barkley dialects but in the north it is not typical.

\subsection{Semantic and ordering preferences} \label{ch:link:preferences}

Despite the relative flexibility of which predicate in a construction gets the linker (\S\ref{ch:link:clause}), there are some cases where speakers strongly prefer the linker to go on one or the other predicate.

In a sentence expressing action at a location, speakers I worked with preferred to put the linker on the location word, and not on the action word. Sometimes speakers rejected other orderings, as in (\ref{speakingoutside1}--\ref{*speakingoutside3}).

\ex \label{speakingoutside1}
\begingl
\glpreamble ƛ̓aaʔaasḥiis ciiqmałap. //
\gla ƛ̓aaʔaas-(q)ḥ=(y)iis ciiqmałap //
\glb outide-\textsc{link}=\textsc{weak.1sg} speak.publicly //
\glft `I'm speaking outside.' (\textbf{Q}, Sophie Billy) //
\endgl
\xe

\ex~ \label{speakingoutside2}
\begingl
\glpreamble ciiqmałapiis hiłḥ ƛ̓aaʔaas. //
\gla ciiqmałap=(y)iis hił-(q)ḥ ƛ̓aaʔaas //
\glb speak.publicly=\textsc{weak.1sg} be.at-\textsc{link} outside //
\glft `I'm speaking outside.' (\textbf{Q}, Sophie Billy) //
\endgl
\xe

\ex~ \label{*speakingoutside3}
\begingl
\glpreamble *ciiqmałapḥiis ƛ̓aaʔaas. //
\gla ciiqmałap-(q)ḥ=(y)iis hił-(q)ḥ ƛ̓aaʔaas //
\glb speak.publicly-\textsc{link}=\textsc{weak.1sg} be.at-\textsc{link} outside //
\glft Intended: `I'm speaking outside.' (\textbf{Q}, Sophie Billy) //
\endgl
\xe

[[TODO: ?ƛ̓aaʔaasḥiis ciiqmałap]]

I was unable to find a case where Sophie would use a linker in such cases on any word other than the location word, and in the (small) corpus of speech I have from her, there are no instances of her doing so. Sophie uses the linker construction much less than all other language consultants I worked with, and rejected many constructions that other speakers used. She is the youngest known fluent speaker, and her speech represents a very innovative Checkleseht dialect. In the data I collected, she most readily attached the linker to quantificational adjectives and location words, and rarely used it elsewhere.

With other consultants who used the linker more widely, they would sometimes reject reorderings or sample sentences that occurred within a set. The following series is from Bob Mundy, a Ucluelet elder, who preferred linked predicates to be the first predicate in the sentence. (\ref{speakingoutside4}) and (\ref{speakingoutside5}) are repeated from (\ref{ex:speakoutsidebob2}) and (\ref{ex:speakoutsidebob1}) respectively.

\ex \label{speakingoutside4}
\begingl
\glpreamble ciiqciiqaqḥitaḥ hitaas. //
\gla ciq-LR2L.a-(q)ḥ=(m)it=(m)aˑḥ hitaas //
\glb speak-\textsc{rp}-\textsc{link}=\textsc{pst}=\textsc{real.1sg} be.outside //
\glft `I'm speaking outside.' (\textbf{B}, Bob Mundy) //
\endgl
\xe

\ex~ \label{speakingoutside5}
\begingl
\glpreamble hitaasḥitaḥ ciiqciiqa. //
\gla hitaas-(q)ḥ=(m)it=(m)aˑḥ ciq-LR2L.a //
\glb be.outside-\textsc{link}=\textsc{pst}=\textsc{real.1sg} speak-\textsc{rp} //
\glft `I'm speaking outside.' (\textbf{B}, Bob Mundy) //
\endgl
\xe

\ex~ \label{speakingoutside6}
\begingl
\glpreamble *hitaasitaḥ ciiqciiqaqḥ. //
\gla hitaas=(m)it=(m)aˑḥ ciq-LR2L.a-(q)ḥ //
\glb be.outside=\textsc{pst}=\textsc{real.1sg} speak-\textsc{rp}-\textsc{link} //
\glft Intended: `I'm speaking outside.' (\textbf{B}, Bob Mundy) //
\endgl
\xe

[[TODO?: Repeat with 4th option in mix, ciiqciiqamitaḥ hitaasḥ]]

While Bob was adamant about his ungrammatical judgment, I think the context of rephrasing is important, as this transforms the grammaticality question into something like a ranked choice task. I do not think (\ref{speakingoutside6}) is truly ungrammatical, as Bob would still generate this kind of ordering in fluent speech. Despite his judgment about here, in another context Bob unprompted produced sentences with the second-predicate linked, as in (\ref{ex:canoesink4}) and (\ref{ex:uupaalh}).

Both the rephrasing data from Bob and the restricted use of the linker by Sophie suggests some general preferences: all else being equal, a location word should not be the one linked, and the first word should be the one with the linker. [[TODO: Get some numbers over the example sentences collected so far]]

\subsection{Data Summary}

The data presented so far leads to the following conclusions:

\begin{enumerate}
	\item The linker may attach to any content word of Nuuchahnulth. This includes nouns, adjectives (including quantifiers), verbs, and adverbs, and excludes complementizers.\footnote{There is more to say about a possible class of adpositions. This is addressed in \S\ref{ch:link:adpositive}.} (\S\ref{ch:link:attach})
	%\item The only apparent non-predicate that the linker may attach to is \textit{ʔuunuuƛ} `because'.
	\item A clause may not consist of only a linked predicate. (\S\ref{ch:link:clause})
	\item Both predicates in a linker construction shares the second-position inflectional information, including subject. (\S\ref{ch:link:second})
	\item The linker does not add semantic content to a predicate. (\S\ref{ch:link:second})
	\item The properties of the linker do not alter depending on whether it attaches to a verb or other part of speech. (\S\ref{ch:link:nonverb})
	\item It is possible for either predicate in a linker construction to be separated from their complement by the other predicate. (\S\ref{ch:link:participants})
	\item The linker attaches to the first word in its predicate complex, even if that first word is an adverb that precedes the predicate. (\S\ref{ch:link:2p})
	\item In certain pragmatically restricted environments, the linker can be used without attaching to a matrix clause. A plausible interpretation in this context is of an elided predicate. (\S\ref{ch:link:dangling})
	\item There seems to be a preference for linked predicates to occur first and on location words (\S\ref{ch:link:preferences}).
\end{enumerate}

\section{Application of the linker to categoricity questions} \label{ch:link:application}

There are some words in Nuuchahnulth whose part of speech properties are not entirely clear. \cite{woo2007b} examines Nuuchahnulth's large (but closed) set of adpositive-like words, and ends up categorizing them as special types of verbs (some of them little-\textit{v}, from a Minimalist perspective). There are other words whose status is somewhat unclear, such as \textit{ʔuunuuƛ}/\textit{ʔunw̓iiƛ} `because of an event', \textit{ʔuusaaḥi} `because of a thing', and \textit{ʔuyi} `at a time'. Some of these words accept the linker and others do not. Recall that the linker typically occurs freely on content words such as verbs (\ref{ch:link:attach}), so if these words are verbs, or at least normal verbs, the linker should be able to attach.

Briefly, I show here that \textit{ʔuunuuƛ}/\textit{ʔunw̓iiƛ} `because of an event' do accept the linker, while \textit{ʔuusaaḥi} `because of a thing' may not (\ref{ch:link:because}). Similarly, \textit{ʔuyi} `at the time' only accepts the linker marginally (\ref{ch:link:uyi}). Most of the adpositive-like verbs can also accept the linker (\ref{ch:link:adpositive}), but not the special non-subject marking\footnote{The marking properties of these words and are somewhat more complex than this simple story. [[TODO: Put this in the clause section -- It's just non-ARG1, cite Woo.]]} adpositives \textit{ʔuukʷił} and \textit{ʔuḥta}. This aligns with \citeauthor{woo2007b}'s findings, where these words are functional and non-predicative.

The marginal cases of \textit{ʔuusaaḥi} and \textit{ʔuyi} suggest words moving from a simple verb to another category, either a restricted verb type or an incipient category of prepositions. On the other hand, evidence from the linker suggests that \textit{ʔuukʷił} and \textit{ʔuḥta} are members of a special syntactic category, either a very small class of prepositions or little-\textit{v}, depending on one's syntactic framework.

\subsection{`Because' words} \label{ch:link:because}

There are three words in Nuuchahnulth that roughly translate to English `because': \textit{ʔuusaaḥi} (all dialects), \textit{ʔuunuuƛ}\footnote{Elder \textit{tupaat} Julia Lucas, who is an Ahousaht speaker, consistently pronounces this word as \textit{ʔunʔuuƛ}. I do not know whether this is a feature of her particular idiolect or a sub-Ahousaht dialect feature of which she is the only known (to me) speaker. I transcribe the word as she pronounces it.} (Barkley and Central, recognized but rare in Northern and Kyuquot-Checleseht) and \textit{ʔunw̓iiƛ} (Northern and Kyuquot-Checleseht only).

To lay some terminological groundwork, I will be using the technical terms \textit{protasis} and \textit{apodosis}. The \textit{protasis} is the part of the sentence describing the condition, and the \textit{apodosis} is the part of the sentence describing the consequence or result. I will call the words relating these propositions \textit{becausitives}.

\textit{ʔuunuuƛ} and \textit{ʔunw̓iiƛ} appear to be dialectal variants with the same meaning and use patterns. The most straightforward way to use the words is as the first word, or main predicate of the sentence (\ref{ex:uunuutl1}, \ref{ex:unwiitl2}), where they take the second position clitic complex, including the subject portmanteau. It is hard to conceive of the relation \textsc{because} having a subject, and indeed the subject agreement marks the subject of the apodosis. Argument-dropping is common for Nuuchahnulth verbs, and these constructions can often drop the apodosis and realize it in a later clause (\ref{ex:uunuutl1}), if at all.

%Context for (\ref{ex:uunuutl1}, \ref{ex:uunuutl2}): A baby was crying last night. I didn't sleep well, and am explaining it to someone.

\ex \label{ex:uunuutl1}
\begingl
\glpreamble ʔuunuuƛitaḥ wik ƛuł weʔič. ʕiḥakita nay̓aqak. //
\gla ʔuunuuƛ=(m)it=(m)aˑḥ wik ƛuł weʔič. ʕiḥak=(m)it=maˑ nay̓aqak //
\glb because=\textsc{pst}=\textsc{real.1sg} \textsc{neg} good sleep. cry=\textsc{pst}=\textsc{real.3} baby //
\glft `I didn't sleep well because (of it); the baby was crying.' (\textbf{B}, Bob Mundy) //
\endgl
\xe

\ex~ \label{ex:unwiitl2}
\begingl
\glpreamble ʔunw̓iiƛiis mačiił ʔin m̓iƛaa. //
\gla ʔunw̓iiƛ=(y)iis mačiił ʔin m̓iƛ-(y)aˑ //
\glb because=\textsc{weak.1sg} inside.\textsc{dr} \textsc{comp} rain-\textsc{dr} //
\glft `I'm inside because it is raining.' (\textbf{Q}, Sophie Billy) //
\endgl
\xe

The apodosis can be introduced with a complementizer, as in (\ref{ex:unwiitl2}) above and (\ref{ex:uunuutl5}) below. The complementizer may not be used to introduce the protasis (\ref{ex:uunuutl6}, \ref{ex:uunuutl7}).

\ex \label{ex:uunuutl5}
\begingl
\glpreamble ʔuunuuƛs hiniiʔiƛ ʔin m̓iƛaa. //
\gla ʔuunuuƛ=s hiniiʔiƛ ʔin m̓iƛ-(y)aˑ //
\glb because=\textsc{strg.1sg} inside.\textsc{mo} \textsc{comp} rain-\textsc{cv} //
\glft `I came inside because it is raining.' (\textbf{N}, Fidelia Haiyupis) //
\endgl
\xe

\vspace{5pt}

Context for (\ref{ex:uunuutl6}, \ref{ex:uunuutl7}): There are two teams playing tug-of-war. One has access to supernatural medicine and they are the winners.

\ex~ \label{ex:uunuutl6}
\begingl
\glpreamble ʔunʔuuƛḥitqač̓aʔał hitaʔap ʔin ʕuʔinak. //
\gla ʔunʔuuƛ-(q)ḥ=(m)it=qač̓a=ʔał hitaʔap ʔin ʕuʔi-naˑk //
\glb because-\textsc{link}=\textsc{pst}=\textsc{dubv}=\textsc{pl} win \textsc{comp} medicine-have //
\glft `They won because they had medicine.' (\textbf{C}, \textit{tupaat} Julia Lucas) //
\endgl
\xe

\ex~ \label{ex:uunuutl7}
\begingl
\glpreamble \# ʔunʔuuƛḥitqač̓aʔał ʕuʔinak ʔin hitaʔap. //
\gla ʔunʔuuƛḥitqač̓aʔał ʕuʔi-naˑk ʔin hitaʔap //
\glb because-\textsc{link}=\textsc{pst}=\textsc{dubv}=\textsc{pl} medicine-have \textsc{comp} win //
\glft Intended: `They won because they had medicine.'\footnotemark (\textbf{C}, \textit{tupaat} Julia Lucas) //
\endgl
\xe

\footnotetext{The actual meaning of (\ref{ex:uunuutl7}), `they had medicine because they won' would be the opposite of what makes sense in the story. ``It's backwards," in my consultant's words.}

As demonstrated in (\ref{ex:uunuutl6}, \ref{ex:uunuutl7}), the becausative can have a linker attached, in which case the linker must be linking the becausative to the following apodosis, since the protasis is explicitly subordinated by the complementizer. The complementizer is optional in this linker construction, and the order of becausative and apodosis is flexible (\ref{ex:unwiitllink1}).

\ex \label{ex:unwiitllink1}
\begingl
\glpreamble hiniiʔiƛs ʔunw̓iiƛḥ m̓iƛšiƛ. //
\gla hiniiʔiƛ=s ʔunw̓iiƛ-(q)ḥ m̓iƛ-šiƛ //
\glb inside.\textsc{mo}=\textsc{real.1sg} because-\textsc{link} rain-\textsc{mo} //
\glft `I am inside because it started raining.' (\textbf{N}, Fidelia Haiyupis) //
\endgl
\xe

One of my consultants, Bob Mundy (Ucluelet), translated the linker attachment in this way: \textit{ʔuunuuƛ} is `because' and \textit{ʔuunuuƛḥ} is `that's why.' This is a fairly succinct way of translating the presence of the linker.

So far, the evidence suggests that these becausatives have at least one argument, the protasis, which can optionally be introduced with a complementizer. The apodosis is more complicated since it is the argument that the linker morpheme ``links" the becausative to (\ref{ex:unwiitllink1}). If the linker is behaving here as it has in other constructions, that would mean that the apodosis is not an argument of the becausative in those constructions. So is the apodosis an argument in the because constructions without the linker in (\ref{ex:uunuutl1}--\ref{ex:uunuutl5})?

I believe the answer is no: The apodosis is never an argument of the becausative. (\ref{ex:uunuutl1}--\ref{ex:uunuutl5}) have the same argument ordering as in (\ref{ex:unwiitllink1}), but without a linker attached. The protasis is also explicitly subordinated with a subordinate subject-mood portmanteau (the definite and the weak moods, respectively). The apodosis is the main predicate of the sentence and takes the main clause's subject-mood portmanteau. If the becausative is a verb, the structure in (\ref{ex:uunuutl1}--\ref{ex:uunuutl5}) is very like the SVCs of adpositive-like verbs (\S\ref{ch:sv:data}).

 %This can be seen in (\ref{ex:uunuutl3}--\ref{ex:unwiitl1}), where the ordering of the clause is reversed from (\ref{ex:uunuutl1}--\ref{ex:uunuutl5}). In (\ref{ex:uunuutl1}--\ref{ex:uunuutl5}), the becausative is the first word, followed by the apodosis, then the protasis. These constructions are highly analogous to the SVCs of adpositive-like verbs (\S\ref{ch:sv:data}).

\vspace{5pt}

\noindent Context for (\ref{ex:uunuutl3}, \ref{ex:unwiitl1}): Two teams are playing tug of war. Our team is strongest and we won.

\ex \label{ex:uunuutl3}
\begingl
\glpreamble hiteʔitapin ʔuunuuƛ našukqin. //
\gla hiteʔitap=(m)in ʔuunuuƛ našuk=qin //
\glb win=\textsc{real.1pl} because strong=\textsc{defn.1pl} //
\glft `We won because we are strong.' (\textbf{B}, Marjorie Touchie) //
\endgl
\xe

\ex~ \label{ex:unwiitl1}
\begingl
\glpreamble tuunuumitniš ʔunw̓iiƛ ḥaaʔakin. //
\gla tuunuu=(m)it=niˑš ʔunw̓iiƛ ḥaaʔak=(y)in //
\glb win=\textsc{pst}=\textsc{strg.1pl} because strong=\textsc{weak.1pl} //
\glft `We won because we are strong.' (\textbf{N}, Fidelia Haiyupis) //
\endgl
\xe

Examples like (\ref{ex:uunuutl3}) and (\ref{ex:unwiitl1}) are in some ways the rarest form of the because construction. My consultant Marjorie Touchie (Ucluelet) freely and frequently produced constructions like this, but Fidelia Haiyupis (Ehatesaht) and Julia Lucas (Ahousaht) rejected examples like this, insisting that these cases needed to contain a linker. However, both Fidelia and Julia produced such a sentence in fluent speech. If I had to make a guess about why these sentences sounded strange out of the blue, it would be that the becausative-first construction is the older and more conservative pattern, while the apodosis-first construction is newer, possibly under the influence of the English word order. However, this is speculation.

Finally, \textit{ʔuunuuƛ}/\textit{ʔunw̓iiƛ} must take a protasis that is verbal, not nominal (\ref{ex:uunuutl2}, which is from the same context as \ref{ex:uunuutl1}) or adjectival (\ref{ex:uunuutl4}). This is somewhat unusual, given the language's flexibility around predication (\S\ref{ch:clause}).

Move because of the baby 214 up here

\ex \label{ex:uunuutl2}
\begingl
\glpreamble *wikitaḥ ƛuł weʔič ʔuunuuƛ nay̓aqakʔisʔi. //
\gla wik=(m)it=(m)aˑḥ ƛuł weʔič ʔuunuuƛ nay̓aqak=ʔis=ʔiˑ //
\glb \textsc{neg}=\textsc{pst}=\textsc{real.1sg} good sleep because baby=\textsc{dim}=\textsc{art} //
\glft Intended: `I didn't sleep well because of the baby.' (\textbf{B}, Bob Mundy) //
\endgl
\xe

%\noindent Context for (\ref{ex:uunuutl4}): A bunch of kids are racing. A fast boy wins the race.

\ex~ \label{ex:uunuutl4}
\begingl
\glpreamble *hitaʔapweʔin kaatkimqsuptaał t̓an̓eʔisʔi ʔuunuuƛ našuk. //
\gla hitaʔap=weˑʔin kaatkimqsuptaał t̓an̓a=ʔis=ʔiˑ ʔuunuuƛ našuk //
\glb win=\textsc{hrsy.3} race child=\textsc{dim}=\textsc{art} because strong //
\glft Intended: `The kid won the race because he is strong.' (\textbf{B}, Bob Mundy) //
\endgl
\xe

\vspace{5pt}


\begin{comment}
Becausitives also follow the typical verbal pattern of being able to freely drop arguments, already seen in (\ref{ex:uunuutl1}) and again in (\ref{ex:becausechanged}).

\vspace{5pt}

Context for (\ref{ex:becausechanged}): A retelling of traditional ways of life. This followed an explanation of how this isn't done anymore, and a lengthy pause.

\ex~ \label{ex:becausechanged}
\begingl
\glpreamble ʔunʔuuƛ̓aƛʔał kʷiisḥin. //
\gla ʔunʔuuƛ=!aƛ=ʔał kʷis-L.ḥin //
\glb because=\textsc{now}=\textsc{pl} different-\textsc{dr} //
\glft `Because they're different now.' (\textbf{C}, \textit{tupaat} Julia Lucas) //
\endgl
\xe
\end{comment}

%There was some difference between speakers about the grammaticality of non-initial becausitives. One of my Ucluelet consultants, Marjorie Touchie produced non-initial becausitives without the linker (\ref{ex:uunuutl3}), and Fidelia Haiyupis, an Ehattesaht woman, produced such an example once (\ref{ex:unwiitl1}). However, on other occasions Fidelia rejected such examples without the linker (\ref{ex:unwiitllink2}, \ref{ex:unwiitllink3}), as did Julia Lucas, an Ahousaht speaker (\ref{ex:uunuutllink1}, \ref{ex:uunuutllink2}). My guess is that the obligatorily-linked version is the older pattern, and this reflects a change in progress that is at different points of progression for different speakers and different dialects.

\begin{comment}
\ex~ \label{ex:because2}
\begingl
\glpreamble ʔuunuuƛḥs hiniiʔiƛ ʔin m̓iƛaa. //
\gla ʔuunuuƛ-(q)ḥ=s mačiił ʔin m̓iƛ-(y)aˑ //
\glb because-\textsc{link}=\textsc{strg.1sg} inside.\textsc{mo} \textsc{comp} rain-\textsc{dr} //
\glft `I came inside because it was raining.' (\textbf{N}, Fidelia Haiyupis) //
\endgl
\xe

\ex \label{ex:unwiitllink2}
\begingl
\glpreamble hitaʔapintniš ʔunw̓iiƛḥ ʕuuy̓aałintin. //
\gla hitaʔap=int=niš ʔunw̓iiƛ-(q)ḥ ʕuuy̓aał=int=(y)in //
\glb inside.\textsc{mo}=\textsc{real.1sg} because-\textsc{link} take.medicine=\textsc{pst}=\textsc{weak.1pl} //
\glft `We won because we had medicine.' (\textbf{N}, Fidelia Haiyupis) //
\endgl
\xe

\ex~ \label{ex:unwiitllink3}
\begingl
\glpreamble *hitaʔapintniš ʔunw̓iiƛ ʕuuy̓aałintin. //
\gla hitaʔap=int=niš ʔunw̓iiƛ ʕuuy̓aał=int=(y)in //
\glb inside.\textsc{mo}=\textsc{real.1sg} because take.medicine=\textsc{pst}=\textsc{weak.1pl} //
\glft Intended: `We won because we had medicine.' (\textbf{N}, Fidelia Haiyupis) //
\endgl
\xe

\ex~ \label{ex:uunuutllink1}
\begingl
\glpreamble wikits ƛuł waʔič ʔunʔuuƛḥ wawaałwiqa ʕiniiƛ. //
\gla wik=(m)it=s ƛuł waʔič ʔunʔuuƛ-(q)ḥ wawaałwiqa ʕiniiƛ //
\glb \textsc{neg}=\textsc{pst}=\textsc{real.1sg} good sleep because-\textsc{link} bark dog //
\glft `I didn't sleep well because the dog was barking.' (\textbf{C}, Julia Lucas) //
\endgl
\xe

\ex~ \label{ex:uunuutllink2}
\begingl
\glpreamble *wikits ƛuł waʔič ʔunʔuuƛ wawaałwiqa ʕiniiƛ. //
\gla wik=(m)it=s ƛuł waʔič ʔunʔuuƛ wawaałwiqa ʕiniiƛ //
\glb \textsc{neg}=\textsc{pst}=\textsc{real.1sg} good sleep bark dog //
\glft Intended: `I didn't sleep well because the dog was barking.' (\textbf{C}, Julia Lucas) //
\endgl
\xe
\end{comment}

The evidence suggests something like the following for \textit{ʔuunuuƛ} and \textit{ʔunw̓iiƛ}. These words are verbs that take a single clausal complement, a protasis, which must be verbal and may be optionally introduced by a complementizer. The way the \textsc{because} relation is syntactically related to its apodosis is either through a SVC, which behaves much like the adpositive-like SVCs, or via a linker construction which links the apodosis and becausative.

%The evidence so far suggests that the words \textit{ʔuunuuƛ} and \textit{ʔunw̓iiƛ} behave like verbs. There are two constructions that link these words to their arguments. There is an SVC where the becausative and its apodosis appear in the matrix clause, and the protasis appears as a complement of the becausative, optionally with an overt complementizer. There is also a linker construction, where the becausative has a linker attached and coordinates with the apodosis in the matrix clause, and again the protasis is a subordinate clause with an optional complementizer. For some speakers, the SVC construction is the only one in which the becausative can appear without a linker: that is, the first word in the sentence. The apodosis shares its subject with the becausitive, and when the predicate linker appears on the becausative it must link it with with the apodosis. In keeping with this specialness of the apodosis argument, the protasis (but not the apodosis) can be introduced with a complementizer.

Where \textit{ʔuunuuƛ} and \textit{ʔunw̓iiƛ} behave as verbs with a verbal complement representing the protasis, \textit{ʔuusaaḥi} requires its nominal complement protasis. Examples (\ref{ex:uusahi1}, \ref{ex:uusahi2}) below are a rephrasing of (\ref{ex:uunuutl1}), demonstrating that, opposite from \textit{ʔuunuuƛ}/\textit{ʔunw̓iiƛ}, \textit{ʔuusaaḥi} must take a noun phrase protasis and not a verbal clause. 

[[TODO this one needs to be checked]]

\ex \label{ex:uusahi1}
\begingl
\glpreamble ʔuusaaḥimta nay̓aqakʔi. wikitaḥ ƛuł weʔič. //
\gla ʔuusaaḥi=imt=(m)aˑ nay̓aqak=ʔiˑ. wik=(m)it=(m)aˑḥ ƛuł weʔič //
\glb because.of=\textsc{pst}=\textsc{real.3} baby=\textsc{art} \textsc{neg}=\textsc{pst}=\textsc{real.1sg} good sleep //
\glft `It was because of the baby; I didn't sleep well.' (\textbf{B}, Bob Mundy) //
\endgl
\xe

\ex~ \label{ex:uusahi2}
\begingl
\glpreamble *ʔuusaaḥimta ʕiḥak nay̓aqakʔi. wikitaḥ ƛuł weʔič. //
\gla ʔuusaaḥi=imt=(m)aˑ ʕiḥak nay̓aqak=ʔiˑ. wik=(m)it=(m)aˑḥ ƛuł weʔič //
\glb because.of=\textsc{pst}=\textsc{real.3} cry.\textsc{dr} baby \textsc{neg}=\textsc{pst}=\textsc{real.1sg} good sleep //
\glft Intended: `It was because of the crying baby; I didn't sleep well.' (\textbf{B}, Bob Mundy) //
\endgl
\xe

[[TODO: Julia has this example pace adam ʔuusaḥiniš tiič č̓aʔak]]

The noun phrase protasis must also occur immediately following \textit{ʔuusaaḥi}, as shown in (\ref{ex:uusahi3}, \ref{ex:uusahi4}).

\ex \label{ex:uusahi3}
\begingl
\glpreamble ʔuusaaḥi ʕuʔi hitaʔap. //
\gla ʔuusaaḥi ʕuʔi hitaʔap //
\glb because.of medicine win //
\glft `They won because of the medicine.' (\textbf{C}, \textit{tupaat} Julia Lucas) //
\endgl
\xe

\ex~ \label{ex:uusahi4}
\begingl
\glpreamble *ʔuusaaḥi hitaʔap ʕuʔi. //
\gla ʔuusaaḥi hitaʔap ʕuʔi //
\glb because.of win medicine //
\glft Intended: `They won because of the medicine.' (\textbf{C}, \textit{tupaat} Julia Lucas) //
\endgl
\xe

\textit{ʔuusaaḥi} may only take a clausal protasis if the protasis is preceded by the complementizer (\ref{ex:uusahi5}, \ref{ex:uusahi6}).

\ex \label{ex:uusahi5}
\begingl
\glpreamble ʔuusaaḥi hitaʔap ʔin ʕuy̓inak. //
\gla ʔuusaaḥi hitaʔap ʔin ʕuy̓i-naˑk //
\glb because.of win \textsc{comp} medicine-have  //
\glft `They won because they had medicine.' (\textbf{C}, \textit{tupaat} Julia Lucas) //
\endgl
\xe

\ex~ \label{ex:uusahi6}
\begingl
\glpreamble ʔuusaaḥis wik ƛuł waʔič ʔin waawaałyuqʷa ʕiniiƛ. //
\gla ʔuusaaḥi=s wik ƛuł waʔič ʔin wałyuq-LR2L.a ʕiniiƛ //
\glb because.of=\textsc{strg.1sg} \textsc{neg} good sleep \textsc{comp} bark-\textsc{rp} dog  //
\glft `I didn't sleep well because of the dog.' (\textit{C}, \textit{tupaat} Julia Lucas) //
\endgl
\xe

\begin{comment}
[[TODO: uusahi plus linker ]]
\textit{ʔuusaaḥi} may only be able to take the linker when it is non-initial. Both consultants with whom I attempted to add a linker to an ʔuusaaḥi-initial sentence were uncertain if it was okay or not but felt it was weird (\ref{ex:uusahi7}, \ref{ex:uusahi8}).

\ex \label{ex:uusahi7}
\begingl
\glpreamble ?? ʔuusaaḥiqḥita nay̓aqakʔi wikitaḥ ƛuł weʔič. //
\gla ʔuusaaḥi-(q)ḥ=(m)it=(m)aˑ nay̓aqak=ʔiˑ wik=(m)it=(m)aˑḥ ƛuł weʔič //
\glb because.of-\textsc{link}=\textsc{pst}=\textsc{real.3} baby=\textsc{art} \textsc{neg}=\textsc{pst}=\textsc{real.1sg} good sleep //
\glft Intended: `I didn't sleep well because of the baby.' (\textbf{B}, Bob Mundy) //
\endgl
\xe

*? ʔuusaaḥiqḥʔiš ʔuusaqta wik̓aałukʷint

ʔuusaqtumtʔiš ʔuusaaḥiqḥ wik̓aałukʷint
\end{comment}

\textit{ʔuusaaḥi} is able to take the linker, although like the use of the complementizer, this changes the syntactic category of its complement, from a noun or participant to a clause.

\ex \label{ex:uusahiqh}
\begingl
\glpreamble ʔuusuqtumtʔiš ʔuusaaḥiqḥ wik̓aałukʷint. //
\gla ʔuusuqta=umt=ʔiˑš ʔuusaaḥi-(q)ḥ wik-!aałuk=int //
\glb hurt=\textsc{pst}=\textsc{strg.3} because-\textsc{link} \textsc{neg}-look.after=\textsc{pst}  //
\glft `He got hurt because he wasn't paying attention.' (\textit{N}, Fidelia Haiyupis) //
\endgl
\xe

%Like \textit{ʔuunuuƛ}/\textit{ʔunw̓iiƛ}, \textit{ʔuusaaḥi} behaves in many ways like other verbs. It has two complements, one of which must be a noun phrase protasis (unlike \textit{ʔuunuuƛ}/\textit{ʔunw̓iiƛ}, which must have clausal protases). Like \textit{ʔuunuuƛ}/\textit{ʔunw̓iiƛ}, \textit{ʔuusaaḥi} shares its subject with its apodosis complement. It may be open to linker attachment, but this is unclear. The word does not occur in the Nootka Texts \citep{sapir1939, sapir1955}, so appeals to published historical Nuuchahnulth cannot resolve the matter. If \textit{ʔuusaaḥi} cannot accept the linker, it is one of very few verbs (if any) with this property, and is perhaps in the midst of a change in progress, from verb-like to preposition or conjunction-like.

Like \textit{ʔuunuuƛ}/\textit{ʔunw̓iiƛ}, \textit{ʔuusaaḥi} appears to be a verb taking a single argument, a protasis. This is associated with the apodosis of the \textsc{because} relation via either a serial verb construction with the clausal apodosis, or with a linker. Unlike \textit{ʔuunuuƛ}/\textit{ʔunw̓iiƛ}, \textit{ʔuusaaḥi} takes a nominal protasis, but this can be changed into a verbal protasis with either the introduction of the complementizer or by attaching the linker to \textit{ʔuusaaḥi}.

\subsection{\textit{ʔuyi}} \label{ch:link:uyi}

Of the possibly-verbal, possibly-adpositional words in Nuuchahnulth, \textit{ʔuyi} and \textit{ʔuukʷił} are perhaps the most ambiguous cases (Adam Werle, \textit{p.c.}). The meaning of \textit{ʔuyi} is `at (a time)' and it typically cooccurs with another predicative word in a sentence. In this case, the clausal clitics scope over both predicates (\ref{ex:uyin}--\ref{ex:uyidrop}). The temporal complement of \textit{ʔuyi} can be a nominal either occurring after (\ref{ex:uyin}) or before (\ref{ex:uyiobj}) \textit{ʔuyi} itself, it can be expressed in a clause with a dependent mood such as the possible mood (\ref{ex:uyipssb}) or the definite mood (\ref{ex:uyidef}), or it can be dropped from the clause entirely (\ref{ex:uyidrop}).

\ex \label{ex:uyin}
\begingl
\glpreamble ʔuyaw̓it̓siis saantii ʔucičƛ ciquułi. //
\gla ʔuya-w̓it̓s=(y)iis saantii ʔu-ci-čiƛ ciquwił=ʔiˑ //
\glb at.a.time-going.to=\textsc{weak.1sg} Sunday \textsc{x}-go.to-\textsc{mo} church=\textsc{art} //
\glft `I'm going to church on Sunday.' (\textbf{Q}, Sophie Billy) //
\endgl
\xe

\ex~ \label{ex:uyiobj}
\begingl
\glpreamble waałakin yuułuʔiłʔatḥ kuʔał ʔuyi. //
\gla wałaak-LS=(m)in yuułuʔiłʔatḥ kuʔał ʔuyi //
\glb go.to-\textsc{gr}=\textsc{real.1pl} Ucluelet morning at.a.time //
\glft `We're going to Ucluelet in the morning.' (\textbf{B}, Bob Mundy) //
\endgl
\xe

\ex~ \label{ex:uyipssb}
\begingl
\glpreamble ʔuyimaḥʔaała n̓an̓aan̓ič kuʔiičiʔeƛquu. //
\gla ʔuyi=maˑḥ=ʔaała n̓an̓aan̓ič kuʔał-iˑčiƛ=!aƛ=quu //
\glb at.a.time=\textsc{real.1sg}=\textsc{habit} read morning-\textsc{in}=\textsc{now}=\textsc{pssb.3} //
\glft `I read in the mornings.' (\textbf{B}, Bob Mundy) //
\endgl
\xe

\ex~ \label{ex:uyidef}
\begingl
\glpreamble ʔuyimtaḥ ʕimtnaakšiƛ čakupšiʔeƛqas. //
\gla ʔuyi=imt=(m)aˑḥ ʕimt-naˑk-šiƛ čakup-šiƛ=!aƛ=qaˑs //
\glb at.a.time=\textsc{pst}=\textsc{real.1sg} name-have-\textsc{mo} man-\textsc{mo}=\textsc{now}=\textsc{defn.1sg} //
\glft `I was a full man when I got my name.' (\textbf{B}, Bob Mundy) //
\endgl
\xe

\ex~ \label{ex:uyidrop}
\begingl
\glpreamble ʔuyiʔum kitḥšiƛ siičił. //
\gla ʔuyi=!um kitḥ-šiƛ si-L.(č)ił //
\glb at.a.time=\textsc{cmfu.2sg} ring-\textsc{mo} \textsc{1sg}-do.to //
\glft `Call me then.' (\textbf{C}, \textit{tupaat} Julia Lucas) //
\endgl
\xe

\textit{ʔuyi} has a tendency to double in fluent speech: as the first predicate of a two-utterance, then later following its object (\ref{ex:uyidouble1}, \ref{ex:uyidouble2}). This could be described grammatically as the first \textit{ʔuyi} occurring with a dropped argument and the second with its object.  Note that the sentence in (\ref{ex:uyidouble2}) is grammatical without the doubling (\ref{ex:uyidouble3}).

\ex \label{ex:uyidouble1}
\begingl
\glpreamble ʔuyimtinʔaała wałaak May ʔuyiʔeƛ. //
\gla ʔuyi=imt=(m)in=ʔaała wałaak May ʔuyi=!aƛ //
\glb at.a.time=\textsc{pst}=\textsc{real.1pl}=\textsc{habit} go.to May at.a.time=\textsc{now} //
\glft `We would go (there) in May.' (\textbf{B}, Bob Mundy) //
\endgl
\xe

\ex~ \label{ex:uyidouble2}
\begingl
\glpreamble ʔuyisʔaał yaacuk kuʔał ʔuyi. //
\gla ʔuyi=s=ʔaał yaacuk kuʔał ʔuyi //
\glb at.a.time=\textsc{strg.1sg}=\textsc{habit} walk.\textsc{dr} morning at.a.time //
\glft `I walk in the morning.' (\textbf{C}, \textit{tupaat} Julia Lucas) //
\endgl
\xe

\ex~ \label{ex:uyidouble3}
\begingl
\glpreamble ʔuyisʔaał yaacuk kuʔał. //
\gla ʔuyi=s=ʔaał yaacuk kuʔał //
\glb at.a.time=\textsc{strg.1sg}=\textsc{habit} walk.\textsc{dr} morning //
\glft `I walk in the morning.' (\textbf{C}, \textit{tupaat} Julia Lucas) //
\endgl
\xe

The features of \textit{ʔuyi} so far are in line with other verbs. The clitic-sharing across predicates and the structure of (\ref{ex:uyidouble3}) in particular is identical to other serial verb constructions (see TODO serial verb section). However, the doubling in (\ref{ex:uyidouble1}, \ref{ex:uyidouble2}) is unique. One point of differentiation is that \textit{ʔuyi} only marginally accepts the linker. After attempting to elicit and construct examples of linked \textit{ʔuyiqḥ}, Barkley speakers Bob Mundy and Marjorie Touchie said that \textit{ʔuyiqḥ} was not a word. They rejected a construction that added a linker to an expression for `tomorrow' (\ref{ex:uyiqh1}), as did Central speaker Julia Lucas when I presented her with the same construction (\ref{ex:uyiqh2}). Marjorie Touchie immediately corrected (\ref{ex:uyiqh1}) by telling me that the way to say this would be with \textit{ʔuyi ʔam̓ii}.

\ex \label{ex:uyiqh1}
\begingl
\glpreamble *ʔuyiqḥʔaƛaḥ ʔam̓ii mamuuk hił makuwił. //
\gla ʔuyi-(q)ḥ=!aƛ=(m)aˑḥ ʔam̓ii mamuuk hił makuwił //
\glb at.a.time-\textsc{link}=\textsc{now}=\textsc{real.1sg} one.day.away work at.a.location store //
\glft Intended: `I will go to work at the store tomorrow.' (\textbf{B}, Bob Mundy \& Marjorie Touchie) //
\endgl
\xe

\ex~ \label{ex:uyiqh2}
\begingl
\glpreamble *ʔuyiqḥʔaƛs ʔam̓ii mamuuk hił makuwił. //
\gla ʔuyi-(q)ḥ=!aƛ=s ʔam̓ii mamuuk hił makuwił //
\glb at.a.time-\textsc{link}=\textsc{now}=\textsc{strg.1sg} one.day.away work at.a.location store //
\glft Intended: `I will go to work at the store tomorrow.' (\textbf{C}, \textit{tupaat} Julia Lucas) //
\endgl
\xe

Unlike Bob and Marjorie, Julia did believe that \textit{ʔuyiqḥ} was a possible word and offered up this sentence as an example case:

[[Do not even cite this]]

\ex \label{ex:uyiqh3}
\begingl
\glpreamble ʔuyiqḥw̓it̓ass ʔaƛp̓it tinʕaƛ huʔacačiƛ. //
\gla ʔuyi-(q)ḥ-w̓it̓as=s ʔaƛ-p̓it tin-ʕaƛ huʔa-ca-čiƛ //
\glb at.a.time-\textsc{link}-going.to=\textsc{strg.1sg} two-times bell-sound.of back-go-\textsc{mo} //
\glft `I will come back at two o'clock.' (\textbf{C}, \textit{tupaat} Julia Lucas) //
\endgl
\xe

[[TODO: below is wrong, should be ʔuʔuyaqḥ]]

I am unable to explain why (\ref{ex:uyiqh3}) is grammatical and (\ref{ex:uyiqh2}) is not. In all of the Nootka Texts, there is only one example of linked \textit{ʔuyiqḥ}, out of approximately 746 instances of \textit{ʔuyi}.

\ex \label{ex:uyiqh4}
\begingl
\glpreamble minkšiʔaƛquu č̓inaaqḥčik nunuuk ʔuʔuyiqḥ ʔuʔuuštaqyuqʷałšy̓akukʔi. //
\gla mink-šiƛ=!aƛ=quu č̓in-(y)aˑ-(q)ḥčik nunuuk R-ʔuyi-(q)ḥ R-ʔuuštaqyu-qałš-y̓ak=uk=ʔiˑ //
\glb around-\textsc{mo}-\textsc{now}=\textsc{pssb.3} pull.hair-\textsc{dr}-along.the.way sing.\textsc{dr} \textsc{pl}-at.the.time.of-\textsc{link} \textsc{pl}-doctor-take.action.on-for.the.purpose.of=\textsc{poss}=\textsc{art} //
\glft `As they make the circuit, dragging them along by the hair, they sing his doctoring songs.' (\citealt[105]{sapir1939}) //
\endgl
\xe

The marginality of linkers on \textit{ʔuyi} -- and its capacity for grammatical doubling -- suggests that there is something special about this word, although it behaves in most other ways like a verb entering into a serial verb construction. Like \textit{ʔuusaaḥi} (\S\ref{ch:link:because}), \textit{ʔuyi} may be a change-in-progress, from a verb to something preposition-like.

\subsection{Adpositive-like words} \label{ch:link:adpositive}

In her dissertation, \cite{woo2007b} examines the syntax of what she terms ``prepositional predicates" and, ultimately, agrees with previous researchers that these words are verbs. The words she considers are: (1) \textit{ʔuuḥw̓ał} `using', (2) \textit{ʔuuʔink} `using', (3) \textit{ʔuucḥin} benefactive, (4) \textit{ʔuʔatup} benefactive/recipient, (5) \textit{ʔuukčamałčiqḥ} `do together with someone', (6) \textit{ʔukʷink} `go with', (7) \textit{ʔuukʷił} `do to', (8) \textit{ʔuḥta} `do to', and (9) \textit{ʔuḥ} subject marker.

\citeauthor{woo2007b} separates out the last three of the list from the rest. The first six of these prepositional predicates introduce an extra argument into the clause, and using the Minimal Framework, \citeauthor{woo2007b} categorizes them as full verbs (V) which, when working in concert with a main verb, coordinate at the level of \textit{v}P. This is supported in part by the first set of words can occur as the sole predicate of a sentence.

However, the  latter three words (\textit{ʔuukʷił}, \textit{ʔuḥta}, and \textit{ʔuḥ}) optionally mark arguments already inherent in the main verb. They require a main predicate to form a grammatical sentence (or may only be used alone in special circumstances, like question-answering). These \citeauthor{woo2007b} categorizes as flavors of \textit{v}.

Although I approach my analysis from within a different framework, I agree with \citeauthor{woo2007b}'s broad categorization. I checked speaker's intuitions about attaching the linker \textit{-(q)ḥ} to these adpositive-like words and the judgments I received support \citeauthor{woo2007b}'s bifurcation into two categories, and importantly that the first category are in fact verbs. Not all speakers recognize or use all of these adpositive-like words, so I was only able to test a subset. There is also a morphophonological problem testing \textit{ʔuḥ} (which would be a *?\textit{ʔuḥḥ} with the linker). However, I have collected data on (1) \textit{ʔuuḥw̓ał}, (3) \textit{ʔuucḥin}, (4) \textit{ʔuʔatup}, (not in \citeauthor{woo2007b}'s list) \textit{ʔuupaał}, (7) \textit{ʔuukʷił}, and (8) \textit{ʔuḥta}. In short, the words \citeauthor{woo2007b}'s calls verbs mostly accept the linker, while all of her ``little-\textit{v}" words do not.

\paragraph{\textit{ʔuuḥw̓ał}} \label{ch:link:uuhwal} The adpositive verb \textit{ʔuuḥw̓ał} `using' can accept the linker in a sentence without any change of meaning.

\ex \label{ex:uuhwal}
\begingl
\glpreamble wikcuk̓ʷapʔic ƛiisƛiisa ʔuuḥw̓ał ƛiisc̓uuy̓ak. //
\gla wikcuk=!ap=ʔic ƛis-LR2L.a ʔuuḥw̓ał ƛiisc̓uuy̓ak //
\glb easy=\textsc{caus}=\textsc{strg.2sg} write-\textsc{rp} using computer //
\glft `It's easy for you to write using a computer.' (\textbf{N}, Fidelia Haiyupis) //
\endgl
\xe

\ex~ \label{ex:uuhwalh}
\begingl
\glpreamble wikcuk̓ʷapʔic ƛiisƛiisa ʔuuḥw̓ałḥ ƛiisc̓uuy̓ak. //
\gla wikcuk=!ap=ʔic ƛis-LR2L.a ʔuuḥw̓ał-(q)ḥ ƛiisc̓uuy̓ak //
\glb easy=\textsc{caus}=\textsc{strg.2sg} write-\textsc{rp} using-\textsc{link} computer //
\glft `It's easy for you to write using a computer.' (\textbf{N}, Fidelia Haiyupis) //
\endgl
\xe

\paragraph{\textit{ʔucḥin}} \label{ch:link:uuchin} The adpositive verb \textit{ʔucḥin} `for, on the behalf of' can also accept the linker, although my consultant was less sure about it. She said that I could ``get away with" (\ref{ex:uuchinqh}) but thought it was unnecessary.

\ex \label{ex:uuchin}
\begingl
\glpreamble ʔucḥins mamuuk ʔuušḥy̓umsukqs. //
\gla ʔucḥin=s mamuuk ʔuuš-(q)ḥy̓uˑ-mis=uk=qs //
\glb \textsc{benef}=\textsc{strg.1sg} work some-related.or.friend-\textsc{nmlz}=\textsc{poss}=\textsc{defn.1sg} //
\glft `I'm working for my friend.' (\textbf{N}, Fidelia Haiyupis) //
\endgl
\xe

\ex~ \label{ex:uuchinqh}
\begingl
\glpreamble ʔucḥinqḥʔaƛs mamuuk ʔuušḥy̓umsukqs. //
\gla ʔucḥin-(q)ḥ=!aƛ=s mamuuk ʔuuš-(q)ḥy̓uˑ-mis=uk=qs //
\glb \textsc{benef}-\textsc{link}=\textsc{now}=\textsc{strg.1sg} work some-related.or.friend-\textsc{nmlz}=\textsc{poss}=\textsc{defn.1sg} //
\glft `I'm working for my friend.' (\textbf{N}, Fidelia Haiyupis) //
\endgl
\xe

\paragraph{\textit{ʔuuʔatup}} \label{ch:link:uatup} There is speaker disagreement on whether the adpositive verb \textit{ʔuuʔatup} `on the behalf of, for the benefit of' freely accepts the linker. My consultant \textit{tupaat} Julia Lucas, a Central speaker, accepted it (\ref{ex:uatup}, \ref{ex:uatuph}) but my Barkley Sound consultants Bob Mundy and Marjorie Touchie did not (\ref{ex:uatup2}, \ref{ex:uatuph2}). This may be another case of a change in progress, where for my Barkley consultants, \textit{ʔuʔatup} is coming to more closely resemble \textit{ʔuukʷił} grammatically (\S\ref{ch:link:uukwil}), something approaching a true adposition.

\ex \label{ex:uatup}
\begingl
\glpreamble ʔak̓ułis suw̓a ḥiy̓aḥi č̓apac ʔuuʔatup ḥaakʷaaƛukʔitk. //
\gla ʔak̓ułi=s suw̓a ḥiy̓aḥi č̓apac ʔuuʔatup ḥaakʷaaƛ=uk=ʔitk. //
\glb loan=\textsc{strg.1sg} \textsc{2sg} \textsc{d1} canoe \textsc{benef} girl=\textsc{poss}=\textsc{defn.2sg} //
\glft `I'm loaning you that canoe for your daughter.' (\textbf{C}, \textit{tupaat} Julia Lucas) //
\endgl
\xe

\ex~ \label{ex:uatuph}
\begingl
\glpreamble ʔak̓ułis suw̓a ḥiy̓aḥi č̓apac ʔuuʔatupḥ ḥaakʷaaƛukʔitk. //
\gla ʔak̓ułi=s suw̓a ḥiy̓aḥi č̓apac ʔuuʔatup-(q)ḥ ḥaakʷaaƛ=uk=ʔitk. //
\glb loan=\textsc{strg.1sg} \textsc{2sg} \textsc{d1} canoe \textsc{benef}-\textsc{link} girl=\textsc{poss}=\textsc{defn.2sg} //
\glft `I'm loaning you that canoe for your daughter.' (\textbf{C}, \textit{tupaat} Julia Lucas) //
\endgl
\xe

\ex~ \label{ex:uatup2}
\begingl
\glpreamble huyaałaḥ ʔuuʔatup t̓aatn̓eʔis. //
\gla huyaał=(m)aˑḥ ʔuuʔatup t̓aatn̓a=ʔis. //
\glb dance=\textsc{real.1sg} \textsc{benef} child.\textsc{pl}=\textsc{dim} //
\glft `I dance for the children.' (\textbf{B}, Bob Mundy, Marjorie Touchie) //
\endgl
\xe

\ex~ \label{ex:uatuph2}
\begingl
\glpreamble *huyaałaḥ ʔuuʔatupḥ t̓aatn̓eʔis. //
\gla huyaał=(m)aˑḥ ʔuuʔatup-(q)ḥ t̓aatn̓a=ʔis //
\glb dance=\textsc{real.1sg} \textsc{benef}-\textsc{link} child.\textsc{pl}=\textsc{dim} //
\glft Intended: `I am dancing for the children.' (\textbf{B}, Bob Mundy, Marjorie Touchie) //
\endgl
\xe

\paragraph{\textit{ʔuupaał}} Though this does not appear in \cite{woo2007b}, it is another adpositive-like verb that appears to have the same meaning as \textit{ʔukʷink} `with'. My consultants familiar with the word used it both with and without the linker.

\ex \label{ex:uupaal1}
\begingl
\glpreamble ciiqmałapiw̓it̓asniš ʔuupaał y̓ukʷiiqsakqs. //
\gla ciq-mał-L.api-w̓it̓as=niˑš ʔuupaał y̓ukʷiiqsu=ʔak=qs. //
\glb speak-move.\textsc{dr}-above-going.to=\textsc{strg.1pl} with younger.sibling=\textsc{poss}=\textsc{defn.1sg} //
\glft `I am going to speak along with my younger sister.' (\textbf{C}, \textit{tupaat} Julia Lucas) //
\endgl
\xe

\ex~ \label{ex:uupaalqh1}
\begingl
\glpreamble ciiqmałapiw̓it̓asniš ʔuupaałqḥ y̓ukʷiiqsakqs. //
\gla ciq-mał-L.api-w̓it̓as=niˑš ʔuupaał-(q)ḥ y̓ukʷiiqsu=ʔak=qs. //
\glb speak-move.\textsc{dr}-above-going.to=\textsc{strg.1pl} with-\textsc{link} younger.sibling=\textsc{poss}=\textsc{defn.1sg} //
\glft `I am going to speak along with my younger sister.' (\textbf{C}, \textit{tupaat} Julia Lucas) //
\endgl
\xe

\ex~ \label{ex:uupaal2}
\begingl
\glpreamble ʔuupaałw̓it̓asaḥ yaaqsčaʕinqas k̓anisw̓it̓as. //
\gla ʔuupaał-w̓it̓as=(m)aˑḥ yaaq-L.sčaʕin=qas k̓anis-w̓it̓as. //
\glb with-going.to=\textsc{real.1sg} who-be.friend=\textsc{defn.1sg} camp-going.to //
\glft `I'm going camping with my friends.' (\textbf{B}, Marjorie Touchie) //
\endgl
\xe

\ex~ \label{ex:uupaalqh2}
\begingl
\glpreamble waałakaḥ nam̓int ʔuupaałḥ yaaqsčaʕinqas k̓anisw̓it̓as. //
\gla wałaak-LS=(m)aˑḥ nam̓int ʔuupaał-(q)ḥ yaaq-L.sčaʕin=qas k̓anis-w̓it̓as. //
\glb go.to.\textsc{mo}-\textsc{grad}=\textsc{real.1pl} Namint with-\textsc{link} who-be.friend=\textsc{defn.1sg} camp-going.to //
\glft `I am going to go with my friends to camp at Namint.' (\textbf{B}, Bob Mundy) //
\endgl
\xe
	
\paragraph{\textit{ʔuukʷił}} \label{ch:link:uukwil} Unlike the fully predicative verbs above, \textit{ʔuukʷił} `do to' does not accept the linker.

\ex \label{ex:tugofwar1}
\begingl
\glpreamble hałiiłintʔiš ʔiiḥatisʔatḥ ʔuukʷił c̓išaaʔatḥ čiict̓ałw̓it̓as. //
\gla hałiił=int=ʔiˑš ʔiiḥatisʔatḥ ʔu-L.(č)ił c̓išaaʔatḥ čiict̓ał-w̓it̓as //
\glb ask=\textsc{pst}=\textsc{strg.3} Ehattisaht \textsc{do.to} Tseshaht do.tug.of.war-going.to //
\glft `The Ehattesahts invited the Tseshahts to play tug of war.' (\textbf{N}, Fidelia Haiyupis) //
\endgl
\xe

\ex~ \label{ex:tugofwar2}
\begingl
\glpreamble *hałiiłintʔiš ʔiiḥatisʔatḥ ʔuukʷiłḥ c̓išaaʔatḥ čiict̓ałw̓it̓as. //
\gla hałiił=int=ʔiˑš ʔiiḥatisʔatḥ ʔu-L.(č)ił-(q)ḥ c̓išaaʔatḥ čiict̓ał-w̓it̓as //
\glb ask=\textsc{pst}=\textsc{strg.3} Ehattisaht \textsc{do.to}-\textsc{link} Tseshaht do.tug.of.war-going.to //
\glft Intended: `The Ehattesahts invited the Tseshahts to play tug of war.' (\textbf{N}, Fidelia Haiyupis) //
\endgl
\xe

\paragraph{\textit{ʔuḥta}} \label{ch:link:uhta} Like the more common object marker \textit{ʔuukʷił}, the marker \textit{ʔuḥtaa} also does not accept the linker.

[[TODO: this is infrequent and seems to be being lost]]

\vspace{5pt}

\noindent Context for (\ref{ex:uhta}, \ref{ex:uhtaqh}), discussing family relations.

\ex \label{ex:uhta}
\begingl
\glpreamble ʔuḥtaa Jane ʔuʔukʷił Alexandra y̓ukʷiiqsu. //
\gla ʔuḥtaa Jane ʔuʔukʷił Alexandra y̓ukʷiiqsu //
\glb \textsc{do.to} Jane call Alexandra younger.sibling //
\glft `Only Jane can call Alexandra younger.' (\textbf{C}, \textit{tupaat} Julia Lucas) //
\endgl
\xe

\ex~ \label{ex:uhtaqh}
\begingl
\glpreamble *ʔuḥtaaqḥ Jane ʔuʔukʷił Alexandra y̓ukʷiiqsu. //
\gla ʔuḥtaa-(q)ḥ Jane ʔuʔukʷił Alexandra y̓ukʷiiqsu //
\glb \textsc{do.to}-\textsc{link} Jane call Alexandra younger.sibling //
\glft Intended: `Only Jane can call Alexandra younger.' (\textbf{C}, \textit{tupaat} Julia Lucas) //
\endgl
\xe

\subsection{Summary of the linker and class-ambiguous words}

I believe that this data about the attachment of the predicate linker can help shed light on the categoricity of these words. \textit{ʔuunuuƛ} and \textit{ʔunw̓iiƛ} `because' behave like verbs, and I believe they should be treated as such. \textit{ʔuyi} appears verbal but more marginally so, and is possibly in the process of transitioning to a preposition. The adpositive-like words that can accept the linker seem to be clearly verbal, which agrees with \cite{woo2007b}'s categorization. However the argument-marking words \textit{ʔuukʷił} and \textit{ʔuḥta} behave differently, as befitting non-predicative words belonging to a different category.

\section{HPSG Analysis and Implementation} \label{ch:link:analysis}

%\subsection{Implementation} \label{ch:link:implementation}