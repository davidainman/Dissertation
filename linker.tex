\chapter{The Linker} \label{ch:link}

The linker morpheme in Nuuchahnulth \textit{-(q)ḥ}, like serial verb constructions (\cref{ch:sv}), is a method by which the language can combine multiple predicates into a single clause. In this chapter I will examine how this construction behaves: how it differs from serialization (\S\ref{ch:link:data}), how the linker can be applied to answer questions about syntactic categories in Nuuchahnulth (\S\ref{ch:link:application}), and finally how I analyze it within the HPSG framework (\S\ref{ch:link:analysis}).

\section{Data} \label{ch:link:data}

In this section I will present the data I have collected on the linker morpheme and how the construction is used. As with serial verbs, I will keep this section fairly theory-neutral, saving the specifics of an HPSG analysis for \S\ref{ch:link:analysis}.

The morpheme \textit{-(q)ḥ} is one of the last possible suffixes on a word.\footnote{Exceptions are the auxiliary predicate suffixes, which follow it (\S\ref{ch:clause:2pv:auxiliary}).} It is typically pronounced as the sequence \textit{qḥ} following a vowel or nasal, and otherwise as \textit{ḥ}. The Central Ahousaht elder \textit{tupaat} Julia Lucas almost always pronounces the linker as the full \textit{qḥ} regardless of the phonological environment, with the exception of certain light verbs. This frequent full pronunciation of the linker is an Ahousaht and Tla-o-qui-aht feature (Werle, \textit{p.c.}).

The suffix is translated as `meanwhile' in \citet{sapir1939}, and was first dubbed the ``linker" by Adam Werle (\textit{p.c.}), on the understanding that it ``links" two predicates together. It coordinates two elements with each other within the syntactic domain marked by the second-position enclitics. I will first compare the linker coordination strategy to other coordination strategies (\S\ref{ch:link:others}), then examine the morphological attachment properties of this special coordinator (\S\ref{ch:link:attach}), and finally look at its syntactic properties (\S\ref{ch:link:clause}--\ref{ch:link:dangling}).

\subsection{Comparison with other coordination} \label{ch:link:others}

The linker morpheme is not the only form of overt coordination in Nuuchahnulth. The two words associated with `and' coordination are \textit{ʔaḥʔaaʔaƛ}, which coordinates sentences and VPs, and \textit{ʔuḥʔi(i)š}/\textit{ʔiš}, which coordinates participants.  %These words coordinate clauses, VPs, and participant phrases, while I will claim that the linker coordinates predicates.

Much like English \textit{and}, \textit{ʔaḥʔaaʔaƛ} may occur at the beginning or in the middle of a sentence. I distinguish sentence-initial and sentence-medial \textit{ʔaḥʔaaʔaƛ} by prosody, pause, and the presence of clausal enclitics.

When introducing a sentence, \textit{ʔaḥʔaaʔaƛ} can host the clausal clitics (\ref{ex:sentinitandclitics}, \ref{ex:sentinitandclitics2}) or the clitics can be deferred to the following predicate (\ref{ex:sentinitandnoclitics}).

\begin{comment}
Context for (): Describing a picture-story.

ʔaanamtqač̓a ʔuusuqtack̓in.
And then he got hurt a little bit.

ʔaḥʔaaʔaƛƛa ƛakišiʔeƛƛa.
And then he stands back up.
BM
\end{comment}

\ex \label{ex:sentinitandclitics}
\begingl
\glpreamble ʔaḥʔaaʔaƛitweʔinʔaała wiinapi haʔukw̓it̓asin waaʔat, nay̓iiʔak̓aƛquuč t̓iqsčiił haʔumʔi. //
\gla ʔaḥʔaaʔaƛ=(m)it=weˑʔin=ʔaała wiinapi haʔuk-w̓it̓as=(m)in waa=!at nay̓iiʔak=!aƛ=quu=č t̓iq-sči-°ił haʔum=ʔiˑ //
\glb and=\textsc{pst}=\textsc{hrsy.3}=\textsc{habit} hold.still.\textsc{dr} eat-going.to=\textsc{strg.1pl} say=\textsc{pass} immediately=\textsc{now}=\textsc{pssb.3}=\textsc{hrsy} sit-beside-indoors.\textsc{dr} food=\textsc{art} //
\glft `Then he would stop and wait for someone to say, ``We are going to eat," and immediately he would sit down by the food.' (\textbf{B}, Marjorie Touchie) //
\endgl
\xe


\ex~ \label{ex:sentinitandclitics2}
\begingl
\glpreamble ʔaḥʔaaʔaƛsa huʔaas n̓aacsiičiƛ naani. //
\gla ʔaḥʔaaʔaƛ=saˑ huʔaas n̓aacsa-iˑčiƛ naani //
\glb and=\textsc{neut.1sg} again see.\textsc{cv}-\textsc{in} grizzly.bear  //
\glft `And then I also saw a grizzly bear (costume used in a ceremony).' (\textbf{C}, \textit{tupaat} Julia Lucas) //
\endgl
\xe

\ex~ \label{ex:sentinitandnoclitics}
\begingl
\glpreamble ʔaḥʔaaʔaƛ ʔuk̓ʷič̓ap̓aƛsuuk ʔiiḥ ciyapuxs. //
\gla ʔaḥʔaaʔaƛ ʔu-k̓ʷič=!ap=!aƛ=suuk ʔiiḥ ciyapuxs //
\glb and \textsc{x}-wear=\textsc{caus}=\textsc{now}=\textsc{neut.2sg} big hat //
\glft `And you have them wear a big hat.' (\textbf{C}, \textit{tupaat} Julia Lucas) //
\endgl
\xe

%ʔaḥʔaaʔaƛ always permits a change of subject, so it coordinates two clauses
%ʔaḥʔaaʔaƛ muučiiłšiƛna hił siy̓a ʔaḥʔaaʔaƛ ḥaakʷaaƛuk Matthew, kʷaaʔuucukqs.
%We were there for four days, me and Matthew's daughter, my granddaughter.
%JL

Sentence-intermediate \textit{ʔaḥʔaaʔaƛ} coordinates two VPs, which share the semantics of the subject-mood enclitic (\ref{ex:intermediateand}, \ref{ex:intermediateand2}).

\ex \label{ex:intermediateand}
\begingl
\glpreamble ʔaa nunuukšiƛnišʔaał ʔaḥʔaaʔaƛ huułhuuła huuuu tuupšiʔeƛquu. //
\gla ʔaa nunuuk-šiƛ=niˑš=ʔaał ʔaḥʔaaʔaƛ huł-LR2L.a huuuu tuupšiƛ=!aƛ=quu //
\glb oh sing.\textsc{dr}-\textsc{mo}=\textsc{strg.1pl}=\textsc{habit} and dance-\textsc{rp} whoa.long.time get.dark.\textsc{mo}=\textsc{now}=\textsc{pssb.3} //
\glft `Oh, we sing and dance, hey for a long time, when it gets dark.' (\textbf{C}, \textit{tupaat} Julia Lucas) //
\endgl
\xe

%ʔuʔatup̓aƛin nunuuk ʔaḥʔaaʔaƛ huyaał. BM

\ex~ \label{ex:intermediateand2}
\begingl
\glpreamble ʔaƛa čaakupiiḥ čaaniʔišʔaałʔał t̓aaqyiił ʔaḥʔaaʔaƛ ʕapkšiƛ ʔuukʷił. //
\gla ʔaƛa čakup-L.iiḥ čaani=ʔiˑš=ʔaał=ʔał t̓aaqyiił ʔaḥʔaaʔaƛ ʕapk-šiƛ ʔu-L.(č)ił //
\glb two man-\textsc{pl} little.while=\textsc{strg.3}=\textsc{habit}=\textsc{pl} stand.inside.\textsc{dr} and grapple-\textsc{mo} \textsc{x}-do.to //
\glft `Two men stand inside for a little while and try to grapple each other [in wrestling games].' (\textbf{C}, \textit{tupaat} Julia Lucas) //
\endgl
\xe

As with English \textit{and}, \textit{ʔaḥʔaaʔaƛ} can be used in this way to imply order (\ref{ex:firstandthen}).

\ex \label{ex:firstandthen}
\begingl
\glpreamble ʔutwiiʔaqƛ̓in nunuuk ʔaḥʔaaʔaƛ haʔukšiƛ. //
\gla ʔu-(t)wii=!aqƛ=!in nunuuk ʔaḥʔaaʔaƛ haʔuk-šiƛ //
\glb \textsc{x}-first=\textsc{fut}=\textsc{cmmd.1pl} sing.\textsc{dr} and eat.\textsc{dr}-\textsc{mo} //
\glft `First we will sing and then eat.' (\textbf{C}, \textit{tupaat} Julia Lucas) //
\endgl
\xe

\textit{ʔaḥʔaaʔaƛ} is sometimes used to coordinate participants (\ref{ex:andnp}).

\ex \label{ex:andnp}
\begingl
\glpreamble ʔaƛamitʔišʔaałʔał ʕaaḥuusʔatḥ ʔaḥʔaaʔaƛ ḥiškʷiiʔatḥ. //
\gla ʔaƛa=(m)it=ʔiˑš=ʔaał=ʔał ʕaaḥuusʔatḥ ʔaḥʔaaʔaƛ ḥiškʷiiʔatḥ //
\glb two=\textsc{pst}=\textsc{strg.3}=\textsc{habit}=\textsc{pl} Ahousaht and Hesquiaht //
\glft `There were two, the Ahousahts and the Hesquiahts.' (\textbf{C}, \textit{tupaat} Julia Lucas) //
\endgl
\xe

The coordinator \textit{ʔuḥʔi(i)š} (and sometimes \textit{ʔiš} in Barkley Sound) is more constrained. It typically only coordinates participants (\ref{ex:7uh7is1}), although innovative speakers sometimes use it to coordinate other constituents.

\ex \label{ex:7uh7is1}
\begingl
\glpreamble ʔuḥintʔinł ʔukʷiił n̓uw̓iiqsknaqs ʔuuḥw̓ał ḥumiis ʔuḥʔiiš c̓istuup. //
\gla ʔuḥ=int=ʔinł ʔu-(č)iił n̓uw̓iiqsa=ʔak=naqs ʔu-L.ḥw̓ał ḥumiis ʔuḥʔiiš c̓is-(š)tuˑp //
\glb be=\textsc{pst}=\textsc{habit} \textsc{x}-make father=\textsc{poss}=\textsc{pst.defn.1sg} \textsc{x}-use red.cedar and line-kind //
\glft `It was my dad that made it using red cedar and rope.' (\textbf{Q}, Sophie Billy) //
\endgl
\xe

\textit{ʔaḥʔaaʔaƛ} coordinates clauses (which differ in subject), VPs (which share a subject), and participants. \textit{ʔuḥʔi(i)š} coordinates participants. I will end up arguing that the linker coordinates a different syntactic category: \textit{maximal predicate phrases} (\S\ref{ch:link:2p}), a category that includes VPs but is not identical to a VP.

One of the elements of the second-position enclitic complex, \textit{=ƛaˑ} (see \S\ref{ch:clause:cliticnormal}) also serves a coordinating function. I gloss this element as `also' although its semantics are broader than that. It can be used alone to link a sentence to what came before (\ref{ex:tla1}), and can also be used along with clausal coordinating \textit{ʔaḥʔaaʔaƛ} (\ref{ex:tla2}).

\noindent Context for (\ref{ex:tla1}): A person is giving a gift to a young woman at a coming of age ceremony.

\ex~ \label{ex:tla1}
\begingl
\glpreamble ʔuucsasaƛƛa ḥaa ḥaakʷaaƛʔi. //
\gla ʔu-iic-LS.sa=!aƛ=ƛa ḥaa ḥaakʷaaƛ=ʔiˑ //
\glb \textsc{x}-own-\textsc{aug}=\textsc{now}=also \textsc{d3} young.woman=\textsc{art} //
\glft `And it now belongs to that young woman.' (\textbf{C}, \textit{tupaat} Julia Lucas) //
\endgl
\xe

\noindent Context for (\ref{ex:tla2}): The speaker is recounting a story where she is waiting for a phone call.

\ex~ \label{ex:tla2}
\begingl
\glpreamble ʔaḥʔaaʔaƛƛa tuupšiʔeƛ. //
\gla ʔaḥʔaaʔaƛ=ƛaˑ tuupšiƛ=!aƛ //
\glb and=also get.dark.\textsc{mo}=\textsc{now}  //
\glft `And then it became night.' (\textbf{C}, \textit{tupaat} Julia Lucas) //
\endgl
\xe

There is also the coordinating suffix \textit{-L.p̓ičḥ}, which is translated as `doing while $\ldots$-ing' in \citet{sapir1939}. It generally attaches only to verbs, and its attachment properties are highly lexicalized. For instance, all speakers I worked with recognized the word \textit{ʕiiḥp̓ičḥ}, as in (\ref{ex:whilecrying}).

\ex \label{ex:whilecrying}
\begingl
\glpreamble ʕiiḥp̓ičḥweʔin Bob mamuuk. //
\gla ʕiḥ-L.p̓ičḥ=weˑʔin Bob mamuuk //
\glb cry-while=\textsc{hrsy.3} Bob work //
\glft `Bob is crying while working.'\footnotemark{}\ (\textbf{B}, Marjorie Touchie) //
\endgl
\xe

\footnotetext{Bob was not actually crying. This was an example sentence Marj used in a joint work session, and was a joke.}

The attachment properties of \textit{-L.p̓ičḥ} are fairly unpredictable. \textit{ʕiiḥp̓ičḥ} shows a case where the suffix attaches to a bare root (ʕiḥ-), and it is ungrammatical for it to attach to inflected \textit{ʕiiḥšiƛ} (\ref{*ex:whilecrying}). Speakers accepted the bare root \textit{nuuk-} combining with \textit{-L.p̓ičḥ} to form \textit{nuukp̓ičḥ} as well (\ref{ex:whilesinging}). However, at least for my consultant Julia Lucas, the inflected form \textit{nunuuk} was acceptable as well (\ref{ex:whilesinging}).

\ex \label{*ex:whilecrying}
\begingl
\glpreamble *ʕiiḥšiƛp̓ičḥweʔin Bob mamuuk. //
\gla ʕiiḥšiƛ-L.p̓ičḥ=weˑʔin Bob mamuuk //
\glb cry.\textsc{mo}-while=\textsc{hrsy.3} Bob work //
\glft Intended: `Bob was crying while working.' (\textbf{B}, Marjorie Touchie) //
\endgl
\xe

\ex~ \label{ex:whilesinging}
\begingl
\glpreamble nuukp̓ičḥ tuuxtuuxʷa waaʕitʔis. //
\gla nuuk-L.p̓ičḥ tuxʷ-LR2L.a waaʕit=ʔis //
\glb sing-while jump-\textsc{rp} frog=\textsc{dim} //
\glft `The little frogs are singing while they jump.' (\textbf{C}, \textit{tupaat} Julia Lucas) //
\endgl
\xe

\ex~ \label{ex:whilesinging2}
\begingl
\glpreamble nunuukp̓ičḥ tuuxtuuxʷa waaʕitʔis. //
\gla nunuuk-L.p̓ičḥ tuxʷ-LR2L.a waaʕit=ʔis //
\glb sing.\textsc{dr}-while jump-\textsc{rp} frog=\textsc{dim} //
\glft `The little frogs are singing while they jump.' (\textbf{C}, \textit{tupaat} Julia Lucas) //
\endgl
\xe

This bound root/inflected stem alternation is not predictable. Marjorie Touchie volunteered \textit{w̓aaʔakp̓ičḥ} (\ref{ex:whilecryingperf}), which Bob Mundy agreed to as well. Neither speaker accepted bare root \textit{*w̓aap̓ičḥ}, although this is the version of the word that occurs in \citet{sapir1939}. Fidelia Haiyupis (a Northern dialect speaker) rejected both forms.

\ex \label{ex:whilecryingperf}
\begingl
\glpreamble w̓aaʔakp̓ičḥʔaƛma ʔamawatuʔa quuquuʔaca. //
\gla w̓aa-ʔak-L.p̓ičḥ=!aƛ=maˑ ʔamawatuʔa quuquuʔaca //
\glb shy-\textsc{dr}-while=\textsc{now}=\textsc{real.3} Bob.Mundy speak.Nuuchahnulth //
\glft `Bob is shy to speak Nuuchahnulth.' (\textbf{B}, Marjorie Touchie) //
\endgl
\xe

There is no constraint on ordering for the word containing \textit{-L.p̓ičḥ}, as shown in (\ref{ex:whilesilent}, \ref{ex:whilesilent2}).

\ex \label{ex:whilesilent}
\begingl
\glpreamble wiikʕaƛp̓ičḥʔi haʔuk. //
\gla wik-ʕaƛ-L.p̓ičḥ=!iˑ haʔuk //
\glb \textsc{neg}-make.a.sound.\textsc{dr}-while=\textsc{cmmd.2sg} eat.\textsc{dr} //
\glft `Eat quietly.' (\textbf{T}, Fidelia Haiyupis) //
\endgl
\xe

\ex~ \label{ex:whilesilent2}
\begingl
\glpreamble haʔuk̓ʷi wiikʕaƛp̓ičḥ. //
\gla haʔuk=!iˑ wik-ʕaƛ-L.p̓ičḥ //
\glb eat.\textsc{dr}=\textsc{cmmd.2sg} \textsc{neg}-make.a.sound.\textsc{dr}-while //
\glft `Eat quietly.' (\textbf{T}, Fidelia Haiyupis) //
\endgl
\xe

The ending \textit{-L.p̓ičḥ} is also not fully productive. There are some words to which it simply does not attach, such as \textit{*ʔuʔiicp̓ičḥ} `eat', \textit{*haʔukp̓ičḥ} or \textit{*haʔup̓ičḥ} `eat.' All speakers rejected attempts to attach the ending to nouns, \textit{*quuʔasp̓ičḥ} `while being a person', \textit{*qʷayac̓iikp̓ičḥ} `while being a wolf.'

The data around contemporary uses of \textit{-L.p̓ičḥ} is complex and contradictory. In the grammar represented in the Sapir-Thomas Nootka Texts, \textit{-L.p̓ičḥ} appears to attach only to verbal roots. In the modern system, the morpheme occasionally attaches to non-roots (\ref{ex:whilecrying}, \ref{ex:whilesinging2}), but still does not attach to nouns. Which stems the morpheme attaches to is highly idiosyncratic and varies speaker to speaker, and likely dialect to dialect. The linker, on the other hand, has a much greater degree of freedom in its sites of attachment (\S\ref{ch:link:attach}), it is for most speakers completely productive, and the scope of its coordination goes beyond that of the word it attaches to (\S\ref{ch:link:2p}).

\subsection{Attachment properties} \label{ch:link:attach}

The linker shows considerable flexibility in the stems it attaches to, attaching to nouns (\ref{ex:womangossiping}), adjectives (\ref{ex:strongbear}), verbs (\ref{ex:talkdriving}), and adverbs (\ref{ex:alsobald}).

\ex \label{ex:womangossiping}
\begingl
\glpreamble łuucma\textbf{qḥ}itqač̓aʔaał taakšiƛ p̓iišmita. //
\gla łuucma-\textbf{(q)ḥ}=(m)it=qaˑč̓a=ʔaał taakšiƛ p̓iišmita //
\glb woman-\textbf{\textsc{link}}=\textsc{pst}=\textsc{dubv}=\textsc{habit} always gossip.\textsc{cv} //
\glft `There was a woman who kept gossiping.' (\textbf{C}, \textit{tupaat} Julia Lucas) //
\endgl
\xe

\ex~ \label{ex:strongbear}
\begingl
\glpreamble t̓ikʷaamitwaʔiš čims ḥaaʔak\textbf{qḥ}. //
\gla t̓ikʷ-(y)aˑ=mit=waˑʔiš čims ḥaaʔak-\textbf{(q)ḥ} //
\glb dig-\textsc{cv}=\textsc{pst}=\textsc{hrsy.3} bear strong-\textbf{\textsc{link}} //
\glft `The bear was digging and strong.' (\textbf{C}, \textit{tupaat} Julia Lucas) //
\endgl
\xe

\ex~ \label{ex:talkdriving}
\begingl
\glpreamble ciqink̓aƛna ƛiḥaa\textbf{qḥ}. //
\gla ciq-(č)ink=!aƛ=naˑ ƛiḥ-(y)aˑ-\textbf{(q)ḥ} //
\glb speak-with=\textsc{now}=\textsc{neut.1pl} drive-\textsc{dr}-\textbf{\textsc{link}} //
\glft `We talked while driving.' (\textbf{C}, \textit{tupaat} Julia Lucas) //
\endgl
\xe

\noindent Context for (\ref{ex:alsobald}): My friend is going bald. I'm also going bald but I don't look in the mirror much and haven't noticed.\footnote{This scenario was constructed to mirror an example present in \citet{sapir1939}.}


\ex~ \label{ex:alsobald}
\begingl
\glpreamble y̓uuqʷaa\textbf{qḥ}s ʕasqii ʔaanaḥi wik hinʔałšiƛ. //
\gla y̓uuqʷaa-\textbf{(q)ḥ}=s ʕasqii ʔaanaḥi wik hinʔał-šiƛ //
\glb also-\textbf{\textsc{link}}=\textsc{strg.1sg} bald only \textsc{neg} realize-\textsc{mo} //
\glft `I'm also bald but I don't know it.' (\textbf{C}, \textit{tupaat} Julia Lucas) //
\endgl
\xe

However, the linker cannot attach to complementizers (\ref{ex:hardsmall1}, \ref{ex:hardsmall2}).

\ex \label{ex:hardsmall1}
\begingl
\glpreamble ʔuušcukʔisit ʔani ʔunaḥʔisitqa. //
\gla ʔuušcuk=ʔis=(m)it ʔani ʔunaḥ=ʔis=(m)it=qaˑ //
\glb difficiult=\textsc{dim}=\textsc{pst} \textsc{comp} small=\textsc{dim}=\textsc{pst}=\textsc{embd} //
\glft `It is a little difficult (to do) because it's small.' (\textbf{B}, Bob Mundy) //
\endgl
\xe

\ex~ \label{ex:hardsmall2}
\begingl
\glpreamble *ʔuušcukʔisit ʔani\textbf{qḥ} ʔunaḥʔisitqa. //
\gla ʔuušcuk=ʔis=(m)it ʔani-\textbf{(q)ḥ} ʔunaḥ=ʔis=(m)it=qaˑ //
\glb difficult=\textsc{dim}=\textsc{pst} \textsc{comp}-\textbf{\textsc{link}} small=\textsc{dim}=\textsc{pst}=\textsc{embd} //
\glft Intended: `It is a little difficult (to do) because it's small.' (\textbf{B}, Bob Mundy) //
\endgl
\xe

From only this data, the linker appears to distinguish morphologically between content and function categories. Another way of expressing this content/function division is by appealing to what can serve as a syntactic predicate in Nuuchahnulth (\S\ref{ch:clause:predp}). Nouns, adjectives, and verbs may all be predicative, and while adverbs are not syntactic predicates themselves, they directly modify a predicate. I will return to the matter of adverbs in \S\ref{ch:link:2p}. Complementizers, on the other hand, are only connective material and cannot be the main predicate of a clause, nor can they modify or otherwise be part of a predicate phrase. In the following sections, I will as a terminological convenience occasionally refer to the predicate in a linker construction that hosts the linker as the ``linked predicate" and the predicate that lacks it as the ``unlinked" or ``non-linked predicate."

%\subsection{Syntactic properties of linked predicates} \label{ch:link:syn}

\subsection{Clause heading} \label{ch:link:clause}

In a sentence with two predicates, one with the linker and one without, the ordering does not typically make a difference.\footnote{There are some cases where altering the ordering affects grammaticality judgments. I address these in \S\ref{ch:link:preferences}.} It is possible for either predicate in an utterance to host the linker, as in (\ref{ex:speakoutsidebob1}, \ref{ex:speakoutsidebob2}).

\ex \label{ex:speakoutsidebob1}
\begingl
\glpreamble hitaasḥitaḥ ciiqciiqa. //
\gla hitaas-(q)ḥ=(m)it=(m)aˑḥ ciq-LR2L.a //
\glb be.outside-\textsc{link}=\textsc{pst}=\textsc{real.1sg} speak-\textsc{rp} //
\glft `I was speaking outside.' (\textbf{B}, Bob Mundy) //
\endgl
\xe

\ex~ \label{ex:speakoutsidebob2}
\begingl
\glpreamble ciiqciiqaqḥitaḥ hitaas. //
\gla ciq-LR2L.a-(q)ḥ=(m)it=(m)aˑḥ  hitaas  //
\glb speak-\textsc{rp}-\textsc{link}=\textsc{pst}=\textsc{real.1sg} be.outside //
\glft `I was speaking outside.' (\textbf{B}, Bob Mundy) //
\endgl
\xe

Just as either predicate in a construction may take the linker, the linker may occur either on the first (\ref{ex:speakoutsidefidelia1}) or second (\ref{ex:speakoutsidefidelia2}) predicate in the utterance.

\ex \label{ex:speakoutsidefidelia1}
\begingl
\glpreamble ƛ̓aaʔaasḥintniš ciiqciiqa. //
\gla ƛ̓aaʔaas-(q)ḥ=int=niš ciq-LR2L.a //
\glb be.outside-\textsc{link}=\textsc{pst}=\textsc{strg.1pl} speak-\textsc{rp} //
\glft `We were speaking outside.' (\textbf{T}, Fidelia Haiyupis) //
\endgl
\xe

\ex~ \label{ex:speakoutsidefidelia2}
\begingl
\glpreamble ciiqciiqamitniš ƛ̓aaʔaasḥ. //
\gla ciq-LR2L.a=mit=niˑš ƛ̓aaʔaas-(q)ḥ //
\glb speak-\textsc{rp}=\textsc{pst}=\textsc{strg.1pl} be.outside-\textsc{link} //
\glft `We were speaking outside.' (\textbf{T}, Fidelia Haiyupis) //
\endgl
\xe

Although there is flexibility as to which predicate takes the linker, clauses may not be headed by a single linked predicate. This can be seen for main clauses in (\ref{ex:longawake}--\ref{*ex:longawake2}) below.

\ex \label{ex:longawake}
\begingl
\glpreamble qiiʔiłitaḥ ƛupkaa. //
\gla qii-°ił=(m)it=(m)aˑḥ ƛupk-(y)aˑ //
\glb long.time-indoors=\textsc{pst}=\textsc{strg.1sg} awake-\textsc{cv} //
\glft `I was awake a long time.' (\textbf{B}, Bob Mundy) //
\endgl
\xe

\ex~ \label{*ex:longawake1}
\begingl
\glpreamble *qiiʔiłitaḥ ƛupkaaqḥ. //
\gla qii-°ił=(m)it=(m)aˑḥ ƛupk-(y)aˑ-(q)ḥ //
\glb long.time-indoors=\textsc{pst}=\textsc{strg.1sg} awake-\textsc{cv}-\textsc{link} //
\glft Intended: `I was awake a long time.' (\textbf{B}, Bob Mundy) //
\endgl
\xe

\ex~ \label{*ex:longawake2}
\begingl
\glpreamble *qiiʔiłḥitaḥ ƛupkaa. //
\gla qii-°ił-(q)ḥ=(m)it=(m)aˑḥ ƛupk-(y)aˑ //
\glb long.time-indoors-\textsc{link}=\textsc{pst}=\textsc{strg.1sg} awake-\textsc{cv} //
\glft Intended: `I was awake a long time.' (\textbf{B}, Bob Mundy) //
\endgl
\xe

\begin{comment}
\ex \label{ex:longawake}
\begingl
\glpreamble qiiʔiłs ƛupkaaqḥ. //
\gla qiiʔił=s ƛupk-(y)aˑ-(q)ḥ //
\glb lie.in.bed.a.long.time=\textsc{strg.1sg} awake-\textsc{dr}-\textsc{link} //
\glft `I lay awake inside for a long time.' (\textbf{T}, \textit{yuułnaak} Simon Lucas) //
\endgl
\xe

\ex~ \label{*ex:longawake}
\begingl
\glpreamble *ƛupkaaqḥs qii. //
\gla ƛupk-(y)aˑ-(q)ḥ=s qii //
\glb awake-\textsc{dr}-\textsc{link}=\textsc{strg.1sg} long.time //
\glft Intended: `I lay awake for a long time.' (\textbf{T}, \textit{yuułnaak} Simon Lucas) //
\endgl
\xe
\end{comment}

In my analysis, (\ref{ex:longawake}) contains one predicate, \textit{ƛupkaa}, and a modifying adverb \textit{qiiʔił}. Despite the two words available on which to attach a linker morpheme, there is only one predicate phrase in the utterance (here headed by a verb). There is nothing for the linker to coordinate this predicate phrase with. A linked predicate with nothing to link to (or coordinate with) is ungrammatical.

%(\ref{*ex:longawake}) has undergone two changes relative to (\ref{ex:longawake}): (i) the words have been rearranged, and (ii) the ending \textit{-°ił}, a predicative location (Davidson, \textit{forthcoming}) has been taken off the adverb \textit{qii}. The former change should not affect the grammaticality of the sentence, as demonstrated in (\ref{ex:speakoutsidefidelia1}, \ref{ex:speakoutsidefidelia2}). But the latter change creates an utterance with ``linked" predicate followed by the syntactically non-predicative adverb \textit{qii} (\ref{*ex:longawake}). In contrast, (\ref{ex:longawake}) contains two full predicates. Because the adverb \textit{qii} cannot be a syntactic predicate, (\ref{*ex:longawake}) only has one predicative word with a linker morpheme, and no further predicate for that linker to coordinate with.

This pattern can be seen in dependent clauses as well, where a single predicate with a linker morpheme is also ungrammatical (\ref{ex:yourehere}, \ref{*ex:yourehere}).

\ex \label{ex:yourehere}
\begingl
\glpreamble ʔuuʕaqstuƛaḥ ʔanik hił ʔaḥkuu. //
\gla ʔuuʕaqstuƛ=(m)aˑḥ ʔani=k hił ʔaḥkuu //
\glb be.happy.\textsc{mo}=\textsc{real.1sg} \textsc{comp}=\textsc{2sg} be.at \textsc{d1} //
\glft `I'm happy you're here.' (\textbf{B}, Bob Mundy) //
\endgl
\xe

\ex~ \label{*ex:yourehere}
\begingl
\glpreamble *ʔuuʕaqstuƛaḥ ʔanik hiłḥ ʔaḥkuu. //
\gla ʔuuʕaqstuƛ=(m)aˑḥ ʔani=k hił-(q)ḥ ʔaḥkuu //
\glb be.happy.\textsc{mo}=\textsc{real.1sg} \textsc{comp}=\textsc{2sg} be.at-\textsc{link} \textsc{d1} //
\glft Intended: `I'm happy you're here.' (\textbf{B}, Bob Mundy) //
\endgl
\xe

Although the word \textit{hił} `be at' frequently takes the linker in texts, in (\ref{*ex:yourehere}) it is the sole predicate of the dependent clause. I was able to replicate a similar example with my Checleseht consultant Sophie Billy, from the other end of the dialect continuum (\ref{ex:sawabear}, \ref{*ex:sawabear}).

\ex \label{ex:sawabear}
\begingl
\glpreamble n̓aaciičƛintiis ʔin hił čimsʔii maḥt̓eekitk. //
\gla n̓aaca-iˑčiƛ=int=(y)iis ʔin hił čims=ʔiˑ maḥt̓ii=ʔak=ʔiˑtk //
\glb see.\textsc{cv}-\textsc{in}=\textsc{pst}=\textsc{weak.1sg} \textsc{comp} be.at bear=\textsc{art} house=\textsc{poss}=\textsc{defn.2sg} //
\glft `I saw there was a bear at your house.' (\textbf{Q}, Sophie Billy) //
\endgl
\xe

\ex~ \label{*ex:sawabear}
\begingl
\glpreamble *n̓aaciičƛintiis ʔin hiłḥ čimsʔii maḥt̓eekitk. //
\gla n̓aaca-iˑčiƛ=int=(y)iis ʔin hił-(q)ḥ čims=ʔiˑ maḥt̓ii=ʔak=ʔiˑtk //
\glb see.\textsc{cv}-\textsc{in}=\textsc{pst}=\textsc{weak.1sg} \textsc{comp} be.at-\textsc{link} bear=\textsc{art} house=\textsc{poss}=\textsc{defn.2sg} //
\glft Intended: `I saw there was a bear at your house.' (\textbf{Q}, Sophie Billy) //
\endgl
\xe

From these examples, I conclude that the linker obligatorily coordinates two predicates.

\subsection{Linkers on non-verbs} \label{ch:link:nonverb}

The examples so far have focused on linkers attached to verbs. Verbal coordination has a straightforward analog with English. However, as detailed in \S\ref{ch:link:attach}, it is possible for the linker to attach to a wide variety of non-verbs. I will claim that the linker performs the same function in all cases---that is, it links syntactic predicates of any type (\S\ref{ch:clause:predp}), not just verbal predicates.

Anecdotally, the most common type of non-verbal predicate that receives the linker is quantificational adjectives (henceforth, quantifiers). The presence or absence of the linker on a quantifier significantly changes the possible interpretations for the sentence. With a bare (non-linked) quantifier, the quantifier may be interpreted as a syntactic object (\ref{ex:findsomething}) and may not come before the verb (\ref{ex:*findsomething}). When a linker is attached, the quantifier must be interpreted as the subject and may either come before (\ref{ex:findsomeone}) or after the verb (\ref{ex:someonefind}).

\vspace{5pt}

\noindent Context for (\ref{ex:findsomething}--\ref{ex:someonefind}): My family and I are looking for a Christmas present for my sister.

\ex \label{ex:findsomething}
\begingl
\glpreamble ʔuuwaƛintʔiš ʔuuš. //
\gla ʔu-L.waƛ=int=ʔiˑš ʔuuš //
\glb \textsc{x}-find=\textsc{pst}=\textsc{strg.3} some //
\glft `He/she found something.' (\textbf{T}, Fidelia Haiyupis) //
\endgl
\xe

\ex~ \label{ex:*findsomething}
\begingl
\glpreamble *ʔuušintʔiš ʔuuwaƛ. //
\gla ʔuuš=int=ʔiˑš ʔu-L.waƛ //
\glb some=\textsc{pst}=\textsc{strg.3} \textsc{x}-find //
\glft Intended:`He/she found something.' (\textbf{T}, Fidelia Haiyupis) //
\endgl
\xe

\ex~ \label{ex:findsomeone}
\begingl
\glpreamble ʔuuwaƛintʔiš ʔuušḥ. //
\gla ʔu-L.waƛ=int=ʔiˑš ʔuuš-(q)ḥ //
\glb \textsc{x}-find=\textsc{pst}=\textsc{strg.3} some-\textsc{link} //
\glft `Someone found it.' (*He/she found something) (\textbf{T}, Fidelia Haiyupis) //
\endgl
\xe

\ex~ \label{ex:someonefind}
\begingl
\glpreamble ʔuušḥintʔiš ʔuuwaƛ. //
\gla ʔuuš-(q)ḥ=int=ʔiˑš ʔu-L.waƛ //
\glb some-\textsc{link}=\textsc{pst}=\textsc{strg.3} \textsc{x}-find //
\glft `Someone found it.' (*He/she found something) (\textbf{T}, Fidelia Haiyupis) //
\endgl
\xe


\begin{comment}
\ex \label{ex:findsomething}
\begingl
\glpreamble ʔuuwaʔaƛ ʔuuš.//
\gla ʔu-L.waƛ=!aƛ ʔuuš //
\glb \textsc{x}-find=\textsc{now} some //
\glft `He/she found something.' (*? Someone found it) (\textbf{C}, \textit{tupaat} Julia Lucas) //
\endgl
\xe

\ex~ \label{ex:*findsomething}
\begingl
\glpreamble *ʔuuš ʔuuwaʔaƛ.//
\gla ʔuuš ʔu-L.waƛ=!aƛ //
\glb some \textsc{x}-find=\textsc{now} //
\glft Intended: `He/she found something.' (\textbf{C}, \textit{tupaat} Julia Lucas) //
\endgl
\xe

\ex~ \label{ex:findsomeone}
\begingl
\glpreamble ʔuuwaʔaƛ ʔuušqḥ.//
\gla ʔu-L.waƛ=!aƛ ʔuuš-(q)ḥ //
\glb \textsc{x}-find=\textsc{now} some-\textsc{link} //
\glft `Someone found it.' (*He/she found something) (\textbf{C}, \textit{tupaat} Julia Lucas) //
\endgl
\xe

\ex~ \label{ex:someonefind}
\begingl
\glpreamble ʔuušqḥʔaƛ ʔuuwaƛ.//
\gla ʔuuš-(q)ḥ=!aƛ ʔu-L.waƛ //
\glb some-\textsc{link}=\textsc{now} \textsc{x}-find //
\glft `Someone found it.' (*He/she found something) (\textbf{C}, \textit{tupaat} Julia Lucas) //
\endgl
\xe

\ex \label{ex:someonefind2}
\begingl
\glpreamble ʔuušqḥ ʔuuwaʔaƛ.//
\gla ʔuuš-qḥ ʔu-L.waƛ=!aƛ //
\glb some-\textsc{link} \textsc{x}-find=\textsc{now} //
\glft `Someone found it.' (*He/she found something) //
\endgl
\xe
\end{comment}

In (\ref{ex:findsomeone}, \ref{ex:someonefind}), the two predicates being coordinated are \textit{some} and \textit{find}. Because quantifiers are predicates in Nuuchahnulth, the same analysis applied to two verbs coordinated with the linker can apply here: These are two syntactic predicates that share a subject. That is, there is a (null) third person subject that is shared between the predicates \textit{some} and \textit{find}: ``There exists an \textit{x} such that \textsc{some}(\textit{x}) and \textsc{find}(\textit{x},\textit{y})." This subject sharing makes the objective reading impossible in (\ref{ex:findsomeone}, \ref{ex:someonefind}).

Though a subjective interpretation of \textit{ʔuuš} in (\ref{ex:findsomething}) is dispreferred, in other contexts this kind of reading is licensed. In (\ref{ex:talented}), my consultant Julia Lucas used a parallel construction where an unlinked \textit{ʔuuš} `some' is given a subjective interpretation.

\ex \label{ex:talented}
\begingl
\glpreamble ʔuušʔiišʔaał wic̓ik, ʔuuš ʕac̓ik, ʔuuš ʔum̓aaqƛ ʔuuy̓ip. //
\gla ʔuuš=ʔiˑš=ʔaał wic̓ik, ʔuuš ʕac̓ik, ʔuuš ʔum̓aaqƛ ʔu-iˑy̓ip //
\glb some=\textsc{strg.3}=\textsc{habit} not.talented, some talented, some able.to \textsc{x}-get //
\glft `Some are not talented, some are talented, some are able to get (the challenge).' (\textbf{C}, \textit{tupaat} Julia Lucas) //
\endgl
\xe

In (\ref{ex:talented}), the first two verbs are intransitive, so there is no other syntactic interpretation for \textit{ʔuuš} `some' other than the subjective one. The final verb is transitive, but the parallelism with the first two clauses primes the listener to interpret \textit{ʔuuš} as subjective. The fact that Julia did not add a linker in (\ref{ex:talented}) shows that a subjective interpretation is possible for non-linked quantifiers, in the right context. %However, when there is an ambiguity, as in (\ref{ex:findsomething}), the absence of the linker is a clue that the speaker had an objective interpretation in mind because the presence of a linker would force an unambiguous subjective reading.

This observation about quantifiers holds true for other adjectives and also nouns, as seen in (\ref{ex:canoesink1}--\ref{ex:canoesink3}). The initial sentence puts two clauses together with a complementizer (\ref{ex:canoesink1}), but can be rephrased without a complementizer by using the linker (\ref{ex:canoesink2}, \ref{ex:canoesink3}).

\vspace{5pt}

\noindent Context for (\ref{ex:canoesink1}--\ref{ex:canoesink3}): I arrived on the beach in a canoe. I left my canoe and went into town. While I'm inside, my canoe is carried out on the tide and capsizes. One person left behind on the beach sees it. (\ref{ex:canoesink1}) was suggested by my consultant, and we worked to rephrase it as (\ref{ex:canoesink2}) and (\ref{ex:canoesink3}).

\ex \label{ex:canoesink1}
\begingl
\glpreamble c̓awaakitwaʔiš n̓aacsa niiʔatu č̓apac.//
\gla c̓awaak=it=waˑʔiš n̓aacsa niiʔatu č̓apac //
\glb one=\textsc{pst}=\textsc{hrsy.3} see.\textsc{cv} sink canoe //
\glft `I hear that one (person) saw the canoe sink.' (\textbf{C}, \textit{tupaat} Julia Lucas) //
\endgl
\xe

\ex~ \label{ex:canoesink2}
\begingl
\glpreamble c̓awaakḥitwaʔiš n̓aacsa niiʔatu č̓apac.//
\gla c̓awaak-(q)ḥ=it=waˑʔiš n̓aacsa niiʔatu č̓apac //
\glb one-\textsc{link}=\textsc{pst}=\textsc{hrsy.3} see.\textsc{cv} sink canoe //
\glft `I hear that one (person) saw the canoe sink.' (\textbf{C}, \textit{tupaat} Julia Lucas) //
\endgl
\xe

\ex~ \label{ex:canoesink3}
\begingl
\glpreamble quuʔasqḥitwaʔiš n̓aacsa niiʔatu č̓apacʔi.//
\gla quuʔas-(q)ḥ=it=waˑʔiš n̓aacsa niiʔatu č̓apac=ʔiˑ //
\glb person-\textsc{link}=\textsc{pst}=\textsc{hrsy.3} see.\textsc{cv} sink canoe=\textsc{art} //
\glft `I hear that a person saw the canoe sink.' (\textbf{C}, \textit{tupaat} Julia Lucas) //
\endgl
\xe



%It is important that the complementizer in (\ref{ex:canoesink1}) creates an overt subordinate clause for , while in the rephrase with the linker (\ref{ex:canoesink2}), there is no complementizer. This supports the data from \S\ref{ch:link:clause} suggesting that the linker itself forms a subordinate (and not a matrix) clause. (\ref{ex:canoesink3}) simply shows, again, that nouns are valid hosts for the linker, just as much as adjectives.

Julia Lucas was adamant that (\ref{ex:canoesink1}) and (\ref{ex:canoesink2}) meant exactly the same thing. If this is true, then the linker is not adding any deep semantic content.\footnote{My analysis ends up putting in a relation \textsc{and}. While this may not be totally meaningless, it is nearly	 meaningless.} Using the same setup, I elicited sentences from Barkley speaker Bob Mundy. He initially proposed the sentence in (\ref{ex:canoesink4}). I proposed (\ref{ex:canoesink5}) by removing the linker, which he rejected, and then (\ref{ex:canoesink6}), which he accepted.

\ex \label{ex:canoesink4}
\begingl
\glpreamble n̓aacsiičiƛweʔin c̓awaakḥ niiʔatu č̓apac. //
\gla n̓aacsa-iˑčiƛ=weˑʔin c̓awaak-(q)ḥ niiʔatu č̓apac //
\glb see.\textsc{cv}-\textsc{in}=\textsc{hrsy.3} one-\textsc{link} sink canoe //
\glft `I hear that one (person) saw the canoe sink.' (\textbf{B}, Bob Mundy) //
\endgl
\xe

\ex~ \label{ex:canoesink5}
\begingl
\glpreamble *n̓aacsiičiƛweʔin c̓awaak niiʔatu č̓apac. //
\gla n̓aacsa-iˑčiƛ=weˑʔin c̓awaak niiʔatu č̓apac //
\glb see.\textsc{cv}-\textsc{in}=\textsc{hrsy.3} one sink canoe //
\glft Intended: `I hear that one saw the canoe sink.' (\textbf{B}, Bob Mundy) //
\endgl
\xe

\ex~ \label{ex:canoesink6}
\begingl
\glpreamble n̓aacsiičiƛweʔin c̓awaakḥ quuʔas niiʔatu č̓apac. //
\gla n̓aacsa.\textsc{cv}-iˑčiƛ=weˑʔin c̓awaak-(q)ḥ quuʔas niiʔatu č̓apac //
\glb see-\textsc{in}=\textsc{hrsy.3} one-\textsc{link} person sink canoe //
\glft `I hear that one person saw the canoe sink.' (\textbf{B}, Bob Mundy) //
\endgl
\xe

Bob's response to removing the linker in (\ref{ex:canoesink5}) was to say, ``It's not complete. One what? What did one see?" Following the basic structure of the Nuuchahnulth clause (\cref{ch:clause}), the two syntactic participants of the predicate \textit{n̓aacsiičiƛ} `see' in (\ref{ex:canoesink5}) should be \textit{c̓awaak} `one' and \textit{niiʔatu č̓apac} `sink canoe'. But \textit{c̓awaak}, as an adjective, cannot be a full NP participant without an article \citep{jacobsen1979}. So it is a syntactically disconnected word and the utterance is nonsensical. The presence of the linker in my consultant's initial proposed sentence (\ref{ex:canoesink4}) forces `one' to be a predicate with the same subject as the predicate `see'. That is, ``There is an \textit{x} such that \textsc{see}(\textit{x},\textit{y}) and \textsc{one}(\textit{x})." The other participant in (\ref{ex:canoesink4}) \textit{niiʔatu č̓apac} `a canoe sink' is the clausal complement of the seeing act.

(\ref{ex:canoesink6}) shows that the coordinated elements may be more than one word. \textit{c̓awaak} `one' is a syntactic predicate taking the subject \textit{quuʔas} `person'.  This dependent linked clause also interrupts the matrix predicate \textit{n̓aacsiičiƛ} `see' and its clausal object \textit{niiʔatu č̓apac} `the canoe sink' in a manner similar to SVC interruptions (\S\ref{ch:sv:data}). A rough bracketing of (\ref{ex:canoesink6}) based on this preliminary analysis is given in (\ref{ex:canoesink6.2}).

\ex \label{ex:canoesink6.2}
\begingl
\gla {[}n̓aacsa.\textsc{cv}-iˑčiƛ=weˑʔin {[}c̓awaak-(q)ḥ quuʔas{]\textsubscript{linked\_clause}} {[}niiʔatu č̓apac{]\textsubscript{participant\_of\_see} ]} //
\glb see-\textsc{in}=\textsc{hrsy.3} one-\textsc{link} person sink canoe //
%\glft `I hear that one person sees the canoe sink.'
\endgl
\xe

\begin{comment}
ACTUALLY*2: This works quite well for showing a deictic predicate. Unfortunately it is XL so I cannot use it. Oh well!

ACTUALLY! I think the below is wrong. If you look at XL's sentence, the predicate is deictic ʔaḥʔaa, to which the linker still attaches! This would mean the only counterexample would be Adv V+link. Investigate this further.

But occasionally the linker may occur on the sole predicate in a sentence. This appears to contradict examples (\ref{ex:someonespoke}) and (\ref{ex:*someonespoke}), but the translation provided for these ``dangling" linkers always indicates they are notionally attached to the preceding sentence. I have 1 (look for more, update number) example from my corpus, involving the word \textit{qʷis} `do so'.\footnote{I am not here counting examples from \textit{tupaat} Julia Lucas, who is unique in always uses the the conjunction \textit{ʔunʔuuƛ} with a linker attached. I believe she has a different lexical entry for the word, and will explain in section.} I give one example below:

this is from Charlie Lucas, who I do not have permissions to share. Update it with a sharable example.

Context: \textit{łačiƛni wa. ʔuušciłʔap̓aƛukni nunuuk. ʔuušciłʔap̓aƛukni huyaał.} `We've let it go, haven't we? It has become hard for us to sing. It has become hard for us to dance.'

\ex \label{ex:danglinglinker}
\begingl
\glpreamble ʔaḥʔaa qʷisḥnii.//
\gla ʔaḥʔaa qʷis-(q)ḥ=niˑ //
\glb DDYN do.so-\textsc{link}=\textsc{neut.1pl} //
\glft `That's what happened to us.' //
\endgl
\xe

Although the one predicate is 

\end{comment}


\subsection{Complement ordering} \label{ch:link:participants}

Briefly addressed already, like serial verb constructions (\S\ref{ch:sv:data}), the linker construction allows predicdates to be separated from their complements by the coordinated phrase. I have already shown that the non-linked predicate may be separated from its complement by the intervening linked predicate (\ref{ex:canoesink4}, \ref{ex:canoesink6}, \ref{ex:canoesink6.2}). The reverse ordering is also possible: The linked predicate may be separated from its direct object by the non-linked predicate. In (\ref{ex:workathome}) the verb \textit{hił} `be at' and its object `my house' are contiguous, but in (\ref{ex:workathome2}) they are separated by the non-linked predicate \textit{mamuuk} `work'.

\ex \label{ex:workathome}
\begingl
\glpreamble hiłḥitin maḥt̓iiʔakqas mamuuk. //
\gla hił-(q)ḥ=(m)it=(m)in maḥt̓ii=ʔak=qas mamuuk //
\glb be.at-\textsc{link}=\textsc{pst}=\textsc{real.1pl} house=\textsc{poss}=\textsc{defn.1sg} work //
\glft `We worked at my house.' (\textbf{B}, Bob Mundy) //
\endgl
\xe

\ex~ \label{ex:workathome2}
\begingl
\glpreamble hiłḥitin mamuuk maḥt̓iiʔakqas. //
\gla hił-(q)ḥ=(m)it=(m)in mamuuk maḥt̓ii=ʔak=qas //
\glb be.at-\textsc{link}=\textsc{pst}=\textsc{real.1pl} work house=\textsc{poss}=\textsc{defn.1sg} //
\glft `We worked at my house.' (\textbf{B}, Bob Mundy) //
\endgl
\xe

Not only is (\ref{ex:workathome2}) grammatical but this is often the structure speakers prefer. For one of my consultants, Ehattesaht speaker Fidelia Haiyupis, this kind of object separation was acceptable when the linked predicate was separated from its object (\ref{ex:fhhil}) but not when the non-linked predicate was separated from its object (\ref{ex:fhqc}, \ref{*ex:fhqc}). This may be a feature of Northern dialects, the Ehattesaht dialect, or my presentation of the material. It is unclear from the small amount of data that I have. %In the above examples, the linked predicate is the one separated from its direct object, but it can also be the non-linked predicate that is separated from its object, as already seen in (\ref{ex:canoesink4}, \ref{ex:canoesink6}).

\ex \label{ex:fhhil}
\begingl
\glpreamble hiłḥsiiš ʔuukʷiił č̓upč̓upšumł maḥt̓iiʔakʔik. //
\gla hił-(q)ḥ=siˑš ʔu-L.(č)iił č̓upč̓upšumł maḥt̓ii=ʔak=ʔik //
\glb be.at-\textsc{link}=\textsc{strg.1sg} \textsc{x}-make sweater house=\textsc{poss}=\textsc{defn.2sg} //
\glft `I am making a sweater at your house.' (\textbf{T}, Fidelia Haiyupis) //
\endgl
\xe

\ex~ \label{ex:fhqc}
\begingl
\glpreamble ʔuuct̓iiḥs Queens Cove ƛiḥaaqḥ. //
\gla ʔuuct̓iiḥ=s Queens Cove ƛiḥ-(y)aˑ-(q)ḥ //
\glb go.toward.\textsc{dr}=\textsc{strg.1sg} Queens Cove drive-\textsc{cv}-\textsc{link} //
\glft `I am driving to Queens Cove.' (\textbf{T}, Fidelia Haiyupis) //
\endgl
\xe

\ex~ \label{*ex:fhqc}
\begingl
\glpreamble *ʔuuct̓iiḥs ƛiḥaaqḥ Queens Cove. //
\gla ʔuuct̓iiḥ=s ƛiḥ-(y)aˑ-(q)ḥ Queens Cove //
\glb go.toward.\textsc{dr}=\textsc{strg.1sg} drive-\textsc{cv}-\textsc{link} Queens Cove //
\glft Intended: `I am driving to Queens Cove.' (\textbf{T}, Fidelia Haiyupis) //
\endgl
\xe

\noindent For most speakers, however, both types of ``interruption" are possible, as in (\ref{ex:readatschool}, \ref{ex:readatschool2}).

\ex \label{ex:readatschool}
\begingl
\glpreamble hiłqḥsʔaał n̓ačaał ƛiisuwił. //
\gla hił-(q)ḥ=s=ʔaał n̓ačaał ƛiisuwił //
\glb be.at-\textsc{link}=\textsc{strg.1sg}=\textsc{habit} read school //
\glft `I read at school.' (\textbf{C}, \textit{tupaat} Julia Lucas) //
\endgl
\xe

\ex~ \label{ex:readatschool2}
\begingl
\glpreamble hiłqḥsʔaał ʔaḥkuu n̓ačaał. //
\gla hił-(q)ḥ=s=ʔaał ʔaḥkuu n̓ačaał //
\glb be.at-\textsc{link}=\textsc{strg.1sg}=\textsc{habit} \textsc{d1} read //
\glft `I read here.' (\textbf{C}, \textit{tupaat} Julia Lucas) //
\endgl
\xe

\subsection{Semantic interpretations of suffixes and clitics} \label{ch:link:second}

Nuuchahnulth has a series of clausal second-position enclitics, which include tense and subject-mood portmanteaus (\S\ref{ch:clause:cliticnormal}). In a linker construction, as in a serial verb construction (\S\ref{ch:sv:data}), both predicates share the same subject, mood, and tense.

\ex \label{ex:stopatmyhouse}
\begingl
\glpreamble hiłḥʔum maḥt̓iiʔakqs wiinapuƛ. //
\gla hił-(q)ḥ=!um maḥt̓iˑ=ʔak=qs wiinapuƛ //
\glb be.at-\textsc{link}=\textsc{cmfu.2sg} house=\textsc{poss}=\textsc{defn.1sg} stop.\textsc{mo} //
\glft `Stop at my house.' (\textbf{T}, Fidelia Haiyupis) //
\endgl
\xe

The command portmanteau \textit{=!um} in (\ref{ex:stopatmyhouse}) syntactically scopes\footnote{Because of the utility of the concept of scoping in this discussion, I will use the word ``scope" from here on to refer to a syntactic element that has an effect over another syntactic element. This should not be confused with semantic scope.} over both predicates. Fidelia did not accept this as possibly meaning that someone else was stopping. If these clitics belong to the clause as a whole, which there is good independent reason to believe (\citealt[p.~35--36]{rose1981}; \citealt[p.~42--50]{woo2007b}), the linker coordinates predicates within the clause, just as SVCs do.

(\ref{ex:readrain}) and (\ref{ex:readrain2}) show a situation where this obligatory subject sharing creates an odd interpretation. I was asking about different activities depending on the weather. The felicitous expression without the linker is in (\ref{ex:readrain}). My rephrase in (\ref{ex:readrain2}) with the linker was met with an immediate laugh.

\ex \label{ex:readrain}
\begingl
\glpreamble n̓ačaałaḥʔaała m̓iƛaaʔaƛquu. //
\gla n̓ačaał=(m)aˑḥ=ʔaała m̓iƛ-(y)aˑ=!aƛ=quu //
\glb read=\textsc{real.1pl}=\textsc{habit} rain-\textsc{cv}=\textsc{now}=\textsc{pssb.3} //
\glft `I read whenever it rains.' (\textbf{B}, Bob Mundy) //
\endgl
\xe

\ex~ \label{ex:readrain2}
\begingl
\glpreamble \#n̓ačaałaḥʔaała m̓iƛaaqḥ. //
\gla n̓ačaał=(m)aˑḥ=ʔaała m̓iƛ-(y)aˑ-(q)ḥ //
\glb read=\textsc{real.1pl}=\textsc{habit} rain-\textsc{cv}-\textsc{link} //
\glft \# `I read and I am raining.' (\textbf{B}, Bob Mundy) //
\endgl
\xe

The causative \textit{=!ap} and passive \textit{=!at} scope narrowly in linker constructions the same way they do in serial verb constructions (\S\ref{ch:sv:valence}). Example (\ref{ex:causedie}) is from a story describing a ceremony where, under the right circumstances, someone ``dies" and is brought back to life. It shows a causative morpheme only applying to the verb with the linker (\textit{qaḥak} `die') and not to the following verb (\textit{hiniis} `carry').

\ex \label{ex:causedie}
\begingl
\glpreamble qaḥakḥʔap̓aƛ hiniis ʔucaʔap hiłḥʔiitq c̓aac̓aayiqš. //
\gla qaḥ-ak-(q)ḥ=!ap=!aƛ hina-iis ʔu-ca=!ap hił-(q)ḥ=ʔiˑtq c̓aayiq-R2.š //
\glb die-\textsc{dr}-\textsc{link}=\textsc{caus}=\textsc{now} \textsc{empty}-carry.\textsc{dr} \textsc{x}-go=\textsc{caus} be.at-\textsc{link}=\textsc{defn.3} do.Tsayik.ceremony-\textsc{it} //
\glft `Making him dead they carry him along to the place where they do Tsayik.' (\textbf{B}, Hamilton George, \citealt[p.~106]{sapir1939}) //
\endgl
\xe

Example (\ref{ex:approachedat}) shows passive \textit{=!at} behaving similarly in an example from Bob Mundy. \textit{ƛawiičiʔat} `be approached by' is modified by a passive valence change, but the linked predicate \textit{hiłḥ} is not.

\ex \label{ex:approachedat}
\begingl
\glpreamble ƛawiičiʔataḥ t̓an̓eʔis hiłḥ maḥt̓iiʔakqas. //
\gla ƛaw-iˑčiƛ=!at=(m)aˑḥ t̓an̓a=ʔis hił-(q)ḥ maḥt̓ii=ʔak=qaˑs //
\glb near-\textsc{in}=\textsc{pass}=\textsc{real.1sg} child=\textsc{dim} be.at-\textsc{link} house=\textsc{poss}=\textsc{defn.1sg} //
\glft `A child came up to me at at my house.' (\textbf{B}, Bob Mundy) //
\endgl
\xe

Together with the evidence of scoping in serial verb constructions, I take this as good reason to believe the enclitics are split with respect to their syntactic domain. Some enclitics scope only over the predicate they attach to: minimally this set of enclitics includes causative \textit{=!ap} and passive \textit{=!at}. These enclitics still occur in second position with respect to their phrase, a phrase which includes a predicate, its modifiers, and valence-changing enclitics, but not subject, tense, and mood. In \cref{ch:sv} at the end of \S\ref{ch:sv:valence} this concept occurred as well and I termed it a ``maximal predicate phrase." These maximal predicate phrases are the units I believe are coordinated in linker constructions.\footnote{They are also the units coordinated in serial verb constructions: The maximal predicate phrase in this case simply is a verbal predicate phrase.} While the valence-changing enclitics have a domain of the maximal predicate phrase, other enclitics scope over the entire clause, including all coordinated structures: minimally this set of enclitics includes tense, the subject-mood portmanteaus, and the habitual morpheme.

In addition to the clausal second-position enclitics, some of the suffix verbs---the auxiliary predicate suffixes---modify predicates and have an interpretive scope beyond the word they attach to (\S\ref{ch:clause:2pv:auxiliary}). The modals in this position seem to be shared across linked predicates, in a similar fashion to the non-valence-changing enclitics.

\vspace{5pt}

\noindent Context for (\ref{ex:drivinghome}): I am taking a friend home and we are leaving a gathering.

\ex \label{ex:drivinghome}
\begingl
\glpreamble waałšiƛw̓it̓asniš ƛiḥaaqḥ. //
\gla wał-šiƛ-LS-w̓it̓as=niˑš ƛiḥ-(y)aˑ-qḥ //
\glb go.home-\textsc{mo}-\textsc{gr}-going.to=\textsc{strg.1pl} drive-\textsc{cv}-\textsc{link} //
\glft `We're going to drive home.' (\textbf{C}, \textit{tupaat} Julia Lucas) //
\endgl
\xe

Both verbs in (\ref{ex:drivinghome}) share the semantics of the modal suffix \textit{-w̓it̓as}, because both the driving and the going home are intentional, not-yet-occurred events. I was unable to find a context where the modal interpretation attached to only one predicate. I confirmed the sharing of the subject portmanteau \textit{=niˑš} by asking if it were possible to say (\ref{ex:drivinghome}) to mean that we were going to walk home but someone else was driving elsewhere. My consultant said no: (\ref{ex:drivinghome}) must mean that it is we who are going to go home and we who are doing it driving in a car.

Both predicates in a linker construction share the semantics of the second-position enclitics, which means they share a subject. They also share at least auxiliary predicate suffixes. The causative and passive, on the other hand, scope narrowly over the predicate they attach to.

% and a sentence may not be composed of two predicates, both with linkers (\ref{ex:*someonespoke}).

\begin{comment}
\ex \label{ex:someonespoke}
\begingl
\glpreamble ʔuušqḥʔaƛ ciqšiƛ.//
\gla ʔuuš-qḥ=ʔaƛ ciq-šiƛ //
\glb some-\textsc{link}=\textsc{now} speak-\textsc{mo} //
\glft `Someone spoke.' //
\endgl
\xe

\ex~ \label{ex:*someonespoke}
\begingl
\glpreamble *ʔuušqḥʔaƛ ciqšiƛḥ.//
\gla *ʔuuš-(q)ḥ=ʔaƛ ciq-šiƛ-(q)ḥ //
\glb *some-\textsc{link}=\textsc{now} speak-\textsc{mo}-\textsc{link} //
\glft Intended: `Someone spoke.' //
\endgl
\xe
\end{comment}

%The above examples suggest that the predicate the linker attaches to (along with its complements) may not be a complete sentence, and is dependent on another clause.

\subsection{The linker and the predicate phrase} \label{ch:link:2p}

Like many bound morphemes in Nuuchahnulth, the linker appears to attach to the first word in some clause. This has already been seen in (\ref{ex:alsobald}), repeated as (\ref{ex:alsobald2}) below.

\ex \label{ex:alsobald2}
\begingl
\glpreamble y̓uuqʷaaqḥs ʕasqii ʔaanaḥi wik hinʔałšiƛ. //
\gla y̓uuqʷaa-qḥ=s ʕasqii ʔaanaḥi wik hinʔał-šiƛ //
\glb also-\textsc{link}=\textsc{strg.1sg} bald only \textsc{neg} aware-\textsc{mo} //
\glft `I'm also bald but I don't know it.' (\textbf{C}, \textit{tupaat} Julia Lucas) //
\endgl
\xe

The two predicates being coordinated in (\ref{ex:alsobald2}) sentence are `also bald' and `only not know (it).' The linker appears on the preposed adverb \textit{y̓uuqʷaa} of the first predicate \textit{y̓uuqʷaa ʕasqii}.	This syntactic domain is once again the maximal predicate phrase: a predicate plus its modifiers (and any valence-changing enclitics). The linker is a second-position element attaching not to the first word in a clause but the first word in a maximal predicate phrase. This is typically the predicate itself but it may, as in (\ref{ex:alsobald2}), be a preceding adverb. Again, the domain of a maximal predicate phrase is distinguished from a clause only by the absence of the widest-scoping second-position enclitics: mood, tense, and subject.

Examples like (\ref{ex:alsobald2}), where the linker attaches to a preceding adverb, are difficult to gather directly as they require special context and it is possible to express the same meaning without the linker. However, this is not a unique case conjured by a linguist coercing his consultants. A few examples of this kind of construction occur in the Sapir-Thomas Nootka Texts. In (\ref{ex:takeoutguns}) the linker also attaches to the preceding adverb of its maximal predicate phrase `still at war', and links that to the still later predicate phrase `grab their guns.'

\ex \label{ex:takeoutguns}
\begingl
\glpreamble ʔeʔimqḥʔaƛquuweʔin hitaḥtačiƛ sukʷiʔaƛ puuʔakʔiʔał. //
\gla ʔeʔim-(q)ḥ=!aƛ=quu=weˑʔin hitaḥta-čiƛ su-kʷiƛ=!aƛ puu=ʔak=ʔiˑ=ʔał //
\glb first-\textsc{link}=\textsc{now}=\textsc{pssb.3}=\textsc{hrsy.3} go.out.to.sea-\textsc{mo} hold-\textsc{mo}=\textsc{now} gun=\textsc{poss}=\textsc{art}=\textsc{pl} //
\glft `As soon as they left the land, they would take their guns.' (\textbf{B}, Qwishanishim, \citealt[p.~395]{sapir1955}) //
\endgl
\xe

\begin{comment}
In (\ref{stillatwar}), the linker again attaches to an adverb \textit{ʔiiqḥii} `still', and links the entire predicate `still doing war' to the earlier predicate \textit{qʷis} `do thus.'

\ex \label{stillatwar}
\begingl
\glpreamble qiiḥsn̓aakck̓in ʔaḥ qʷiyiič qʷis, [ʔiiqḥii\textbf{qḥ} hitačink maatmaasʔi] [qaḥsaap̓aƛquuweʔin č̓amuʔałʔaƛquu yuułuʔiłʔatqḥ huuʕiiʔatḥuʔałʔaƛquu]]. //
\gla qiiḥsn̓aak-ck̓in ʔaḥ qʷiyi=(y)ii=č [[qʷis] [ʔiiqḥii-\textbf{(q)ḥ} hitačink maatmaas=ʔiˑ]] qaḥ-saˑp=!aƛ=quu=weˑʔin  č̓am-uʔał=!aƛ=quu yuułuʔiłʔatḥ-(q)ḥ huuʕiiʔatḥ-uʔał=!aƛ=quu. //
\glb long.time-\textsc{dim} \textsc{d1} when=\textsc{weak.3}=\textsc{hrsy} do.thus still-\text{link} go.against tribe.\textsc{pl}=\textsc{art} kill-\textsc{mo.caus}=\textsc{now}=\textsc{pssb.3}=\textsc{hrsy.3} canoe-see=\textsc{now}=\textsc{pssb.3} Ucluelet-\textsc{link} Huuayaht-see=\textsc{pssb.3}=\textsc{hrsy.3} //
\glft `For a little longer after this happened, while the tribes were still at war, the Ucluelets would kill Huu-ay-ahts when they saw their canoes.' (\textbf{B}, \citealt[p.~392]{sapir1955}) //
\endgl
\xe
\end{comment}

In (\ref{ex:longtimesingle}), the two predicate phrases being coordinated are `single a long time' and `going to the river to bathe'. As in (\ref{ex:takeoutguns}), the linker attaches to the preceding adverb of the first predicate phrase.

\ex \label{ex:longtimesingle}
\begingl
\glpreamble qiiqḥʔaƛ x̣ačłaa hat̓inʕasʔaƛ ḥaakʷaaƛʔi ʔucačiʔaƛ c̓aʔakʔisʔi. //
\gla qii-(q)ḥ=!aƛ x̣ačłaa hat̓inq-!as=!aƛ ḥaakʷaaƛ=ʔiˑ ʔu-ca-čiƛ=!aƛ c̓aʔak=ʔis=ʔiˑ //
\glb long.time-\textsc{link}=\textsc{now} single.woman bathe-in.order.to=\textsc{now} young.woman=\textsc{art} \textsc{x}-go-\textsc{mo}=\textsc{now} river=\textsc{dim}=\textsc{art} //
\glft `Having been single a long time the girl went to a little stream in order to bathe.' (\textbf{B}, Big Fred \citealt[p.~68]{sapir1939}) //
\endgl
\xe

%The linker morpheme attaches to the first word in the maximal predicate phrase in an utterance: either the predicate itself, or to a preceding modifier.

%These examples, as well the case of modal suffix scoping have led me to believe there is a phrasal unit between the clause (where the second-position clitics scope) and the main predicate. I have dubbed this the ``predicate phrase." This phrase consists maximally of the predicate word and preceding adverbs. The predicate linker will attach to the first word in the predicate phrase, whether that is the predicate word itself or a preceding adverb. [[Move the main arguments up to the clause section]]

\subsection{Dangling linkers} \label{ch:link:dangling}

There are a handful of cases where the linker does not appear to be linking its predicate to anything. I believe that the interpretation shows that there is an elided phrase, as in (\ref{ex:takecare}).

\ex \label{ex:takecare}
\begingl
\glpreamble ʕaʕałḥʔiʔaała. //
\gla ʕaʕał-(q)ḥ=!iˑ=ʔaała //
\glb be.comforted-\textsc{link}=\textsc{cmmd.2sg}=\textsc{habit} //
\glft `Take care!' (\textbf{B}, Bob Mundy) //
\endgl
\xe

The literal meaning of (\ref{ex:takecare}) is ``Be comforted, in whatever you're doing." But ``whatever you're doing" is dropped from the sentence. This kind of farewell construction has been noted by researchers Adam Werle and Henry Kammler, who have understood the linker here as linking to a dropped element (\textit{p.c.}). This is my explanation for this kind of construction as well. Especially when compared with the assertions that other such examples are ungrammatical (\S\ref{ch:link:clause}), I believe that examples like (\ref{ex:takecare}) are formulaic expressions and hide an elided coordinand. %These kinds of ``dangling" linkers are uncommon, and in my experience speakers won't accept them out of the blue unless it is a formulaic expression.

\subsection{Ordering preferences} \label{ch:link:preferences}

Despite the relative flexibility of which predicate in a construction gets the linker (\S\ref{ch:link:clause}), there are some cases where speakers have a preference for one ordering over another.

In a forced choice test, when speakers had a preference they always erred on the side of expressing a linked location word first (\cref{table:orderinglinkloc}). Unlike in the serial verb case (\S\ref{ch:sv:analysis:type2}), all speakers believed both forms sounded like good Nuuchahnulth, even if they had a preference for one.

\begin{table}[H]
\centering
\caption{Ordering of linked location predicates}
\label{table:orderinglinkloc}
\begin{tabular}{cllllll}
 &  & Total & SB & FH & JL & BM+MT \\ \hline
\multicolumn{1}{|c|}{\multirow{3}{*}{Pair 1}} & \multicolumn{1}{l|}{1 mamuukw̓it̓asniš hiłḥ maʔasukqs} & \multicolumn{1}{l|}{} & \multicolumn{1}{l|}{} & \multicolumn{1}{l|}{} & \multicolumn{1}{l|}{} & \multicolumn{1}{l|}{} \\ \cline{2-7} 
\multicolumn{1}{|c|}{} & \multicolumn{1}{l|}{2 hiłḥw̓it̓asniš maʔasukqas mamuuk} & \multicolumn{1}{l|}{3} & \multicolumn{1}{l|}{1} & \multicolumn{1}{l|}{} & \multicolumn{1}{l|}{1} & \multicolumn{1}{l|}{1} \\ \cline{2-7} 
\multicolumn{1}{|c|}{} & \multicolumn{1}{c|}{equally good} & \multicolumn{1}{l|}{1} & \multicolumn{1}{l|}{} & \multicolumn{1}{l|}{1} & \multicolumn{1}{l|}{} & \multicolumn{1}{l|}{} \\ \hline \hline
\multicolumn{1}{|c|}{\multirow{3}{*}{Pair 2}} & \multicolumn{1}{l|}{1 ciiqciiqamitniš mačiiłḥ} & \multicolumn{1}{l|}{} & \multicolumn{1}{l|}{} & \multicolumn{1}{l|}{} & \multicolumn{1}{l|}{} & \multicolumn{1}{l|}{} \\ \cline{2-7} 
\multicolumn{1}{|c|}{} & \multicolumn{1}{l|}{2 mačiiłḥitniš ciiqciiqa} & \multicolumn{1}{l|}{2} & \multicolumn{1}{l|}{} & \multicolumn{1}{l|}{1} & \multicolumn{1}{l|}{1} & \multicolumn{1}{l|}{} \\ \cline{2-7} 
\multicolumn{1}{|c|}{} & \multicolumn{1}{c|}{equally good} & \multicolumn{1}{l|}{2} & \multicolumn{1}{l|}{1} & \multicolumn{1}{l|}{} & \multicolumn{1}{l|}{} & \multicolumn{1}{l|}{1} \\ \hline
\end{tabular}
\end{table}

In a rephrasing test, Bob Mundy expressed a preference for the linker to be both on the location word, as well as on the first predicate. (\ref{ex:speakingoutside1}--\ref{ex:speakingoutside4}) are the versions I tried, in order. He found them all intelligible, but not equally good. (\ref{ex:speakingoutside3}) with the linker on the location and on the first word was best. (\ref{ex:speakingoutside2}) with the linker on the location word was okay but he remarked that it was a little off. (\ref{ex:speakingoutside1}) Bob described as `backwards', and (\ref{ex:speakingoutside4}) he rejected.

\ex \label{ex:speakingoutside1}
\begingl
\glpreamble ??ciiqciiqaqḥitaḥ hitaas. //
\gla ciq-LR2L.a-(q)ḥ=(m)it=(m)aˑḥ hitaas //
\glb speak-\textsc{rp}-\textsc{link}=\textsc{pst}=\textsc{real.1sg} be.outside //
\glft ?? `I'm speaking outside.' (\textbf{B}, Bob Mundy) //
\endgl
\xe

\ex~ \label{ex:speakingoutside2}
\begingl
\glpreamble ?ciiqciiqamitaḥ hitaasḥ. //
\gla ciq-LR2L.a=(m)it=(m)aˑḥ hitaas-(q)ḥ //
\glb speak-\textsc{rp}=\textsc{pst}=\textsc{real.1sg} be.outside-\textsc{link} //
\glft ? `I'm speaking outside.' (\textbf{B}, Bob Mundy) //
\endgl
\xe

\ex~ \label{ex:speakingoutside3}
\begingl
\glpreamble hitaasḥitaḥ ciiqciiqa. //
\gla hitaas-(q)ḥ=(m)it=(m)aˑḥ ciq-LR2L.a //
\glb be.outside-\textsc{link}=\textsc{pst}=\textsc{real.1sg} speak-\textsc{rp} //
\glft `I'm speaking outside.' (\textbf{B}, Bob Mundy) //
\endgl
\xe

\ex~ \label{ex:speakingoutside4}
\begingl
\glpreamble *hitaasitaḥ ciiqciiqaqḥ. //
\gla hitaas=(m)it=(m)aˑḥ ciq-LR2L.a-(q)ḥ //
\glb be.outside=\textsc{pst}=\textsc{real.1sg} speak-\textsc{rp}-\textsc{link} //
\glft Intended: `I'm speaking outside.' (\textbf{B}, Bob Mundy) //
\endgl
\xe

\vspace{-15pt}

In these examples, the preference for the linker to occur on a location word is strongest, and then second to that is the preference for the linked predicate to occur first. Evidence from Checleseht speaker Sophie Billy suggests that the preference for location verbs to host the linker is a feature across Nuuchahnulth dialects. Sophie has the least productive use of the linker in her fluent speech. I mainly have examples from her using the linker on location verbs, quantifiers, and because words. There is only one word outside of these categories I have ever seen her apply a linker morpheme to, the verb \textit{ƛawaa} `be near,' which is very nearly a location word. She rejected linker constructions that other speakers used, such as on adjectives like \textit{ḥaaʔak} `strong' (\ref{ex:strongbear}) or on numerals like \textit{c̓awaak} `one' (\ref{ex:canoesink2}). For Sophie, linkers are not just preferred to be on locations but they are ungrammatical in many other cases.

\begin{comment}
\ex \label{speakingoutside1}
\begingl
\glpreamble ƛ̓aaʔaasḥiis ciiqmałap. //
\gla ƛ̓aaʔaas-(q)ḥ=(y)iis ciiqmałapa //
\glb outide-\textsc{link}=\textsc{weak.1sg} speak.publicly //
\glft `I'm speaking outside.' (\textbf{Q}, Sophie Billy) //
\endgl
\xe

\ex~ \label{speakingoutside2}
\begingl
\glpreamble ciiqmałapayiis hiłḥ ƛ̓aaʔaas. //
\gla ciiqmałapa=(y)iis hił-(q)ḥ ƛ̓aaʔaas //
\glb speak.publicly=\textsc{weak.1sg} be.at-\textsc{link} outside //
\glft `I'm speaking outside.' (\textbf{Q}, Sophie Billy) //
\endgl
\xe

\ex~ \label{*speakingoutside3}
\begingl
\glpreamble *ciiqmałapḥiis ƛ̓aaʔaas. //
\gla ciiqmałapa-(q)ḥ=(y)iis hił-(q)ḥ ƛ̓aaʔaas //
\glb speak.publicly-\textsc{link}=\textsc{weak.1sg} be.at-\textsc{link} outside //
\glft Intended: `I'm speaking outside.' (\textbf{Q}, Sophie Billy) //
\endgl
\xe
\end{comment}

Annotating natural texts reveals a different set of facts from ranked choice tests. Using the same corpus for annotating serial verb constructions in  \cref{table:svctype1} and consisting of 14167 words, I annotated for the presence and ordering of linker morphemes in linker constructions. I split data according to attachment sites: verbs, adjectives, nouns, adverbs, and wh-words. The only wh-words in my corpus were \textit{qʷis} `do thus', \textit{qʷaa} `how' and \textit{ʔaaqin} `how/why.' I split the verbal category into locations, because words, and others, and the adjective category into quantifiers, numbers, and durations (numbers that are inflected for perfective aspect). The results are in \cref{table:linker}.

\vspace{-10pt}

\begin{table}[H]
\centering
\caption{Occurrence of linker constructions in naturally occurring Nuuchahnulth}
\label{table:linker}
\adjustbox{max width=\textwidth - 0.2in}{
\begin{tabular}{lcccccccccccccccccc}
\multicolumn{1}{c}{} & \multicolumn{6}{c}{\textbf{Verbs}} & \multicolumn{6}{c}{\textbf{Adjectives}} & \multicolumn{2}{c}{\multirow{2}{*}{\textbf{Nouns}}} & \multicolumn{2}{c}{\multirow{2}{*}{\textbf{Adverbs}}} & \multicolumn{2}{c}{\multirow{2}{*}{\textbf{Wh-words}}} \\
\multicolumn{1}{c}{} & \multicolumn{2}{c}{\textbf{Location}} & \multicolumn{2}{c}{\textbf{Because}} & \multicolumn{2}{c}{\textbf{Others}} & \multicolumn{2}{c}{\textbf{Quantifier}} & \multicolumn{2}{c}{\textbf{Number}} & \multicolumn{2}{c}{\textbf{Duration}} & \multicolumn{2}{c}{} & \multicolumn{2}{c}{} & \multicolumn{2}{c}{} \\ \cline{2-19} 
\multicolumn{1}{l|}{Linker 1st/2nd} & \multicolumn{1}{c|}{1st} & \multicolumn{1}{c|}{2nd} & \multicolumn{1}{c|}{1st} & \multicolumn{1}{c|}{2nd} & \multicolumn{1}{c|}{1st} & \multicolumn{1}{c|}{2nd} & \multicolumn{1}{c|}{1st} & \multicolumn{1}{c|}{2nd} & \multicolumn{1}{c|}{1st} & \multicolumn{1}{c|}{2nd} & \multicolumn{1}{c|}{1st} & \multicolumn{1}{c|}{2nd} & \multicolumn{1}{c|}{1st} & \multicolumn{1}{c|}{2nd} & \multicolumn{1}{c|}{1st} & \multicolumn{1}{c|}{2nd} & \multicolumn{1}{c|}{1st} & \multicolumn{1}{c|}{2nd} \\ \cline{2-19} 
\multicolumn{1}{l|}{\textbf{Nootka Texts}} & \multicolumn{1}{c|}{2} & \multicolumn{1}{c|}{6} & \multicolumn{1}{c|}{0} & \multicolumn{1}{c|}{0} & \multicolumn{1}{c|}{0} & \multicolumn{1}{c|}{2} & \multicolumn{1}{c|}{0} & \multicolumn{1}{c|}{0} & \multicolumn{1}{c|}{0} & \multicolumn{1}{c|}{1} & \multicolumn{1}{c|}{3} & \multicolumn{1}{c|}{0} & \multicolumn{1}{c|}{1} & \multicolumn{1}{c|}{0} & \multicolumn{1}{c|}{2} & \multicolumn{1}{c|}{0} & \multicolumn{1}{c|}{2} & \multicolumn{1}{c|}{0} \\ \cline{2-19} 
\multicolumn{1}{l|}{\textbf{Barkley}} & \multicolumn{1}{c|}{4} & \multicolumn{1}{c|}{2} & \multicolumn{1}{c|}{1} & \multicolumn{1}{c|}{1} & \multicolumn{1}{c|}{3} & \multicolumn{1}{c|}{0} & \multicolumn{1}{c|}{1} & \multicolumn{1}{c|}{0} & \multicolumn{1}{c|}{0} & \multicolumn{1}{c|}{0} & \multicolumn{1}{c|}{0} & \multicolumn{1}{c|}{0} & \multicolumn{1}{c|}{0} & \multicolumn{1}{c|}{0} & \multicolumn{1}{c|}{0} & \multicolumn{1}{c|}{0} & \multicolumn{1}{c|}{0} & \multicolumn{1}{c|}{0} \\ \cline{2-19} 
\multicolumn{1}{l|}{\textbf{Central}} & \multicolumn{1}{c|}{1} & \multicolumn{1}{c|}{12} & \multicolumn{1}{c|}{8} & \multicolumn{1}{c|}{0} & \multicolumn{1}{c|}{0} & \multicolumn{1}{c|}{0} & \multicolumn{1}{c|}{6} & \multicolumn{1}{c|}{2} & \multicolumn{1}{c|}{1} & \multicolumn{1}{c|}{1} & \multicolumn{1}{c|}{0} & \multicolumn{1}{c|}{0} & \multicolumn{1}{c|}{0} & \multicolumn{1}{c|}{1} & \multicolumn{1}{c|}{0} & \multicolumn{1}{c|}{0} & \multicolumn{1}{c|}{0} & \multicolumn{1}{c|}{0} \\ \cline{2-19} 
\multicolumn{1}{l|}{\textbf{Northern}} & \multicolumn{1}{c|}{5} & \multicolumn{1}{c|}{19} & \multicolumn{1}{c|}{1} & \multicolumn{1}{c|}{0} & \multicolumn{1}{c|}{2} & \multicolumn{1}{c|}{0} & \multicolumn{1}{c|}{1} & \multicolumn{1}{c|}{1} & \multicolumn{1}{c|}{0} & \multicolumn{1}{c|}{0} & \multicolumn{1}{c|}{0} & \multicolumn{1}{c|}{0} & \multicolumn{1}{c|}{0} & \multicolumn{1}{c|}{0} & \multicolumn{1}{c|}{0} & \multicolumn{1}{c|}{0} & \multicolumn{1}{c|}{2} & \multicolumn{1}{c|}{1} \\ \cline{2-19} 
\multicolumn{1}{l|}{\textbf{Kyuq.-Checl.}} & \multicolumn{1}{c|}{2} & \multicolumn{1}{c|}{8} & \multicolumn{1}{c|}{0} & \multicolumn{1}{c|}{0} & \multicolumn{1}{c|}{0} & \multicolumn{1}{c|}{1} & \multicolumn{1}{c|}{5} & \multicolumn{1}{c|}{0} & \multicolumn{1}{c|}{0} & \multicolumn{1}{c|}{0} & \multicolumn{1}{c|}{1} & \multicolumn{1}{c|}{0} & \multicolumn{1}{c|}{0} & \multicolumn{1}{c|}{0} & \multicolumn{1}{c|}{0} & \multicolumn{1}{c|}{0} & \multicolumn{1}{c|}{0} & \multicolumn{1}{c|}{0} \\ \cline{2-19} 
\multicolumn{1}{l|}{\textbf{Subtotal}} & \multicolumn{1}{c|}{14} & \multicolumn{1}{c|}{47} & \multicolumn{1}{c|}{10} & \multicolumn{1}{c|}{1} & \multicolumn{1}{c|}{5} & \multicolumn{1}{c|}{3} & \multicolumn{1}{c|}{13} & \multicolumn{1}{c|}{3} & \multicolumn{1}{c|}{1} & \multicolumn{1}{c|}{2} & \multicolumn{1}{c|}{4} & \multicolumn{1}{c|}{0} & \multicolumn{1}{c|}{1} & \multicolumn{1}{c|}{1} & \multicolumn{1}{c|}{2} & \multicolumn{1}{c|}{0} & \multicolumn{1}{c|}{4} & \multicolumn{1}{c|}{1} \\ \cline{2-19} 
\multicolumn{1}{l|}{\textbf{Total}} & \multicolumn{2}{c|}{61} & \multicolumn{2}{c|}{11} & \multicolumn{2}{c|}{8} & \multicolumn{2}{c|}{16} & \multicolumn{2}{c|}{3} & \multicolumn{2}{c|}{4} & \multicolumn{2}{c|}{2} & \multicolumn{2}{c|}{2} & \multicolumn{2}{c|}{5} \\ \cline{2-19} 
\end{tabular}
}
\end{table}

Fully half of linkers in the corpus appear on location words, mostly \textit{hił}. After locations, the most common uses are on quantifiers and because words (\textit{ʔuunuuƛ}, \textit{ʔunw̓iiƛ}). The use on nouns and adverbs is the least common, with no instances of linked adverbs in my corpus of modern Nuuchahnulth, although speakers do recognize and understand examples of this type.

Ordering preferences are not clear from this sample. Because words, quantifiers, and wh-words with a linker are all strongly likely to occur first in this data set. However, these categories are more often the first word in a clause: quantifiers and wh-words tend to front, and because words, as main predicates, tend to occur initially (\S\ref{ch:link:because}). The distribution of linked location words is directly opposite Bob Mundy's opinion in a forced choice task. There is an asymmetry that may be causing this, however: Without the linker present, locations almost always appear first (\S\ref{ch:sv:data:type2}). So in most cases where a location word occurs second in a clause, there is a strong likelihood that it will have a linker attached to it, while if it occurs initially the linker may be present or absent. From the remaining categories there may be a slight preference for linkers to occur initially, but it is in no way clear.

\subsection{Data summary}

The data presented so far leads me to the following conclusions:

\begin{enumerate}[nolistsep]
	\item The linker may attach to any content word in Nuuchahnulth. This includes nouns, adjectives (including quantifiers), verbs, and adverbs, and excludes complementizers.\footnote{There is more to say about a possible class of adpositions. This is addressed in \S\ref{ch:link:adpositive}.} (\S\ref{ch:link:attach})
	%\item The only apparent non-predicate that the linker may attach to is \textit{ʔuunuuƛ} `because'.
	\item A clause may not consist of only a linked predicate. (\S\ref{ch:link:clause})
	\item The syntactic properties of the linker do not alter depending on whether it attaches to a verb or other part of speech. (\S\ref{ch:link:nonverb})
	\item It is possible for either predicate in a linker construction to be separated from its complement by the other predicate. (\S\ref{ch:link:participants})
	\item Both predicates in a linker construction share second-position clausal inflectional information, including subject, but valence-altering enclitics scope over predicates individually. (\S\ref{ch:link:second})
	\item The linker does not add semantic content to a predicate. (\S\ref{ch:link:second})
	\item The linker attaches to the first word in a maximal predicate phrase, even if that first word is an adverb that precedes the predicate. The maximal predicate phrase is a predicate phrase with its modifiers that has not yet picked up the clausal information of the second-position enclitics. (\S\ref{ch:link:2p})
	\item In certain pragmatically restricted environments, the linker can be used without attaching to a matrix clause. A plausible interpretation in this context is the presence of an elided predicate. (\S\ref{ch:link:dangling})
	\item There is a preference for linked predicates to occur on location words, and in some cases to occur on the first predicate (\S\ref{ch:link:preferences}).
\end{enumerate}

I will ultimately account for these facts by modeling the predicate linker as a suffixing coordinator with the semantics of \textsc{and}. The relative insensitivity of the linker morpheme to category is additional evidence that verbs, adjectives, and nouns are part of a natural predicate class in Nuuchahnulth (\S\ref{ch:clause:predp}). The linker is one of the places in the grammar that is insensitive to which member of this predicate class it picks out. When the linker attaches to adverbs, it is always still coordinating the following predicate which the adverb is modifying. Before moving on to my full analysis (\S\ref{ch:link:analysis}), I will first use syntactic facts about the linker to answer questions about syntactic categories.

\section{Application of the linker to categoricity questions} \label{ch:link:application}

There are some words in Nuuchahnulth whose part of speech properties are not entirely clear. \citet{woo2007b} examines Nuuchahnulth's large (but closed) set of adposition-like words, and ends up categorizing them as special types of verbs (some of them little-\textit{v}, from a Minimalist perspective). There are other words whose status is somewhat unclear, such as \textit{ʔuunuuƛ}/\textit{ʔunw̓iiƛ} `because of an event', \textit{ʔuusaaḥi} `because of a thing', and \textit{ʔuyi} `at a time'. Some of these words accept the linker and others do not. Since the linker typically occurs freely on content words such as verbs (\S\ref{ch:link:attach}), if these words are normal verbs, the linker should be able to attach.

Briefly, I show here that \textit{ʔuunuuƛ}/\textit{ʔunw̓iiƛ} `because of an event' and \textit{ʔuusaaḥi} `because of a thing' accept the linker and are verbs (\S\ref{ch:link:because}). \textit{ʔuyi} `at a time' does not accept the linker, and appears to be an incipient adposition (\S\ref{ch:link:uyi}). Most of the adposition-like verbs can also accept the linker (\S\ref{ch:link:adpositive}), but not the special non-subject marking adpositives \textit{ʔuukʷił} and \textit{ʔuḥtaa}. This aligns with \citeauthor{woo2007b}'s findings, where these words are part of the functional little-\textit{v} category and thus non-predicative.

%The marginal cases of \textit{ʔuusaaḥi} and \textit{ʔuyi} suggest words moving from a simple verb to another category, either a restricted verb type or an incipient category of prepositions. On the other hand, evidence from the linker suggests that \textit{ʔuukʷił} and \textit{ʔuḥtaa} are members of a special syntactic category, either a very small class of prepositions or little-\textit{v}, depending on one's syntactic framework.

\subsection{`Because' words} \label{ch:link:because}

There are at least three words in Nuuchahnulth that roughly translate to English `because': \textit{ʔuusaaḥi} (all dialects), \textit{ʔuunuuƛ}\footnote{Ahousaht speaker \textit{tupaat} Julia Lucas consistently pronounces this word as \textit{ʔunʔuuƛ}. I do not know whether this is a feature of her particular idiolect or a sub-Ahousaht dialect feature of which she is the only known (to me) speaker. I transcribe the word as she pronounces it.} (Barkley and Central, recognized but rare in Northern and Kyuquot-Checleseht) and \textit{ʔunw̓iiƛ} (Northern and Kyuquot-Checleseht only). To distinguish the two arguments of the because semantic relation, I'll refer to the \textit{result} and the \textit{cause}. The because words themselves I'll call \textit{becausitives}.

\textit{ʔuunuuƛ} and \textit{ʔunw̓iiƛ} appear to be dialect variants with the same meaning and use patterns. They may be used as the first word or main predicate in a clause (\ref{ex:insidebecauserain}--\ref{ex:whydidntsleep}), where they take the second-position enclitic complex, including the subject-mood portmanteau. It is hard to conceive of the relation {\textsc{because}} having a subject, and indeed the subject agreement marks the subject of the result. The result in these cases follows the becausative (\ref{ex:insidebecauserain}--\ref{ex:whydidntsleep}), and the cause occurs either after a complementizer (\ref{ex:insidebecauserain}--\ref{ex:uunuutltlihasum}) or is dropped and realized it in a later clause, if at all (\ref{ex:whydidntsleep}).

\begin{comment}
\ex \label{ex:uunuutl0}
\begingl
\glpreamble wiiksinḥimaḥ teʔiłšiƛ ʔuunuuƛ wałaakqas c̓uumuʕas. //
\gla wik-L.sinḥi=(m)aˑḥ teʔił-šiƛ ʔuunuuƛ wałaak=qaˑs c̓uumuʕas //
\glb \textsc{neg}-try.to.do=\textsc{real.1sg} sick.\textsc{dr}-\textsc{mo} because go.\textsc{mo}=\textsc{defn.1sg} Port.Alberni //
\glft `I’m trying not to get sick because I am going to Port Alberni.' (\textbf{B}, Bob Mundy) //
\endgl
\xe
\end{comment}

%Context for (\ref{ex:uunuutl1}, \ref{ex:uunuutl2}): A baby was crying last night. I didn't sleep well, and am explaining it to someone.

\ex \label{ex:insidebecauserain}
\begingl
\glpreamble ʔunw̓iiƛiis mačiił ʔin m̓iƛaa. //
\gla ʔunw̓iiƛ=(y)iis mačiił ʔin m̓iƛ-(y)aˑ //
\glb because=\textsc{weak.1sg} inside.\textsc{dr} \textsc{comp} rain-\textsc{dr} //
\glft `I'm inside because it is raining.' (\textbf{Q}, Sophie Billy) //
\endgl
\xe

\ex~ \label{ex:inisidebecauserain2}
\begingl
\glpreamble ʔuunuuƛs hiniiʔiƛ ʔin m̓iƛaa. //
\gla ʔuunuuƛ=s hiniiʔiƛ ʔin m̓iƛ-(y)aˑ //
\glb because=\textsc{strg.1sg} inside.\textsc{mo} \textsc{comp} rain-\textsc{cv} //
\glft `I came inside because it is raining.' (\textbf{T}, Fidelia Haiyupis) //
\endgl
\xe

\ex~ \label{ex:uunuutltlihasum}
\begingl
\glpreamble ʔunʔuuƛ ƛ̓iḥasum ʔukłinuƛ ʔin ʔuḥʔatqač̓a ʔuʔaałuk witwaak. //
\gla ʔunʔuuƛ ƛ̓iḥasum ʔu-kłinuƛ ʔin ʔuḥ=!at=qaˑč̓a ʔu-!aałuk witwaak //
\glb because Red.Mist \textsc{x}-call.\textsc{mo} \textsc{comp} be=\textsc{pass}=\textsc{dubv} \textsc{x}-look.after warrior.\textsc{pl} //
\glft `The reason why he was called Red Mist is that he led warriors.' (\textbf{C}, \textit{tupaat} Julia Lucas) //
\endgl
\xe

\ex~ \label{ex:whydidntsleep}
\begingl
\glpreamble ʔuunuuƛitaḥ wik ƛuł weʔič. ʕiḥakita nay̓aqak. //
\gla ʔuunuuƛ=(m)it=(m)aˑḥ wik ƛuł weʔič. ʕiḥak=(m)it=maˑ nay̓aqak //
\glb because=\textsc{pst}=\textsc{real.1sg} \textsc{neg} good sleep. cry=\textsc{pst}=\textsc{real.3} baby //
\glft `I didn't sleep well because (of it); the baby was crying.' (\textbf{B}, Bob Mundy) //
\endgl
\xe

It is also possible for the becausative to occur second in the construction after the result, in which case the cause may (\ref{ex:becausewontdry}) or may not (\ref{ex:becausestrong1}, \ref{ex:becausestrong2}) be introduced by a complementizer.

\ex \label{ex:becausewontdry}
\begingl
\glpreamble wikʔaaqƛeʔicuu c̓ukʷiƛ ʔuunuuƛ ʔani wikʔaała čamiḥta ƛ̓uššiƛ. //
\gla wik=ʔaaqƛ=(m)eˑʔicuu c̓u-kʷiƛ ʔuunuuƛ ʔani wik=ʔaała čamiḥta ƛ̓uš-šiƛ //
\glb \textsc{neg}=\textsc{fut}=\textsc{real.2pl} wash-\textsc{mo} because \textsc{comp} \textsc{neg}=\textsc{hab} proper dry-\textsc{mo} //
\glft `You're not going to wash because it won't dry properly.' (\textbf{B}, Marjorie Touchie) //
\endgl
\xe

\vspace{5pt}

\noindent Context for (\ref{ex:becausestrong1}, \ref{ex:becausestrong2}): Two teams are playing tug of war. Our team is strongest and we won.

\ex \label{ex:becausestrong1}
\begingl
\glpreamble hiteʔitapin ʔuunuuƛ našukqin. //
\gla hiteʔitap=(m)in ʔuunuuƛ našuk=qin //
\glb win=\textsc{real.1pl} because strong=\textsc{defn.1pl} //
\glft `We won because we are strong.' (\textbf{B}, Marjorie Touchie) //
\endgl
\xe

\ex~ \label{ex:becausestrong2}
\begingl
\glpreamble tuunuumitniš ʔunw̓iiƛ ḥaaʔakin. //
\gla tuunuu=(m)it=niˑš ʔunw̓iiƛ ḥaaʔak=(y)in //
\glb win=\textsc{pst}=\textsc{strg.1pl} because strong=\textsc{weak.1pl} //
\glft `We won because we are strong.' (\textbf{T}, Fidelia Haiyupis) //
\endgl
\xe


While the cause can be introduced with a complementizer, as seen in (\ref{ex:insidebecauserain}, \ref{ex:inisidebecauserain2}, \ref{ex:uunuutltlihasum}, \ref{ex:becausewontdry}), the complementizer may never be used to introduce the result (\ref{ex:wonbecausemedicine}, \ref{ex:*wonbecausemedicine}).

\vspace{5pt}

\noindent Context for (\ref{ex:wonbecausemedicine}, \ref{ex:*wonbecausemedicine}): There are two teams playing tug-of-war. One has access to supernatural medicine and they are the winners.

\ex~ \label{ex:wonbecausemedicine}
\begingl
\glpreamble ʔunʔuuƛḥitqač̓aʔał hitaʔap ʔin ʕuʔinak. //
\gla ʔunʔuuƛ-(q)ḥ=(m)it=qač̓a=ʔał hitaʔap ʔin ʕuʔi-naˑk //
\glb because-\textsc{link}=\textsc{pst}=\textsc{dubv}=\textsc{pl} win \textsc{comp} medicine-have //
\glft `They won because they had medicine.' (\textbf{C}, \textit{tupaat} Julia Lucas) //
\endgl
\xe

\ex~ \label{ex:*wonbecausemedicine}
\begingl
\glpreamble \# ʔunʔuuƛḥitqač̓aʔał ʕuʔinak ʔin hitaʔap. //
\gla ʔunʔuuƛḥitqač̓aʔał ʕuʔi-naˑk ʔin hitaʔap //
\glb because-\textsc{link}=\textsc{pst}=\textsc{dubv}=\textsc{pl} medicine-have \textsc{comp} win //
\glft Intended: `They won because they had medicine.'\footnotemark\ (\textbf{C}, \textit{tupaat} Julia Lucas) //
\endgl
\xe

\footnotetext{The actual meaning of (\ref{ex:*wonbecausemedicine}), `they had medicine because they won' would be the opposite of what makes sense in the story. ``It's backwards," in Julia's words.}

As seen in (\ref{ex:wonbecausemedicine}), the becausative can have a linker attached.\footnote{Bob Mundy translated the linker attachment in this way: \textit{ʔuunuuƛ} is `because' and \textit{ʔuunuuƛḥ} is `that's why.' This is a fairly succinct way of translating the presence of the linker.} The exact same types of constructions that have been seen so far can also be produced with a linker on the becausative: with the becausative occurring first (\ref{ex:becausenochildren}), with it occurring second (\ref{ex:insidebecauserainlink}), and with and without the complementizer.\footnote{I have no instances of a dropped complementizer when the becausative is first: that is, a theoretical example of \textit{becausative =inflection result cause}. This construction may be ungrammatical or simply dispreferred, as it makes it difficult to determine which complement is the cause. A fuller explanation would require further analysis and work with speakers.}

\ex \label{ex:becausenochildren}
\begingl
\glpreamble ʔuunuuƛḥʔaƛitweʔin ʕiḥak ʔani wik̓iituk t̓aatn̓eʔis. //
\gla ʔuunuuƛ-(q)ḥ=!aƛ=(m)it=weˑʔin ʕiḥ-ak ʔani wik̓iit=uk L.<t>-t̓an̓a=ʔis //
\glb because-\textsc{link}=\textsc{now}=\textsc{pst}=\textsc{hrsy.3} cry-\textsc{dr} \textsc{comp} none=\textsc{poss} \textsc{pl}-child=\textsc{dim} //
\glft `She cried because she had no children.' (\textbf{B}, Marjorie Touchie) //
\endgl
\xe

\ex~ \label{ex:insidebecauserainlink}
\begingl
\glpreamble hiniiʔiƛs ʔunw̓iiƛḥ m̓iƛšiƛ. //
\gla hiniiʔiƛ=s ʔunw̓iiƛ-(q)ḥ m̓iƛ-šiƛ //
\glb inside.\textsc{mo}=\textsc{real.1sg} because-\textsc{link} rain-\textsc{mo} //
\glft `I am inside because it started raining.' (\textbf{T}, Fidelia Haiyupis) //
\endgl
\xe

All these sentences have similar constructions. The result is expressed adjacent to the becausative (before or after) and its subject is marked in the second-position enclitic complex, which may fall on either the becausative or the result. The cause is an embedded clause that is optionally introduced by a complementizer. The sentential cause complement may have its own subject-marking enclitics, which are always in a dependent mood (\ref{ex:becausestrong1}--\ref{ex:becausestrong2}).

It is tempting to analyze becausatives as having two syntactic complements: a result and a cause. However, the fact that the linker freely attaches to the becausative complicates this. What does the linker actually link the becausative \textit{to}? It can't be the cause, since the cause can be marked with a complementizer that explicitly subordinates the phrase. The only possibilities are that this is a ``dangling" linker (\S\ref{ch:link:dangling}), coordinating the becausative with something in the discourse context that is syntactically dropped, or that it is coordinating the becausative with the result.

The best analysis is the second one. Becausatives have a single complement (the cause), and the linker, when present, is coordinating the becausative with the result. This makes the structure of because expressions look like adposition-like verbs in a serial verb construction (\S\ref{ch:sv:analysis:type3}). Like adposition-like verbs, becausatives can appear with or without a linker without changing the meaning of the utterance.\footnote{To compare with adposition-like verbs, see the following section \S\ref{ch:link:adpositive}.} Becausatives then are typically in a structure where they are coordinated with their result. This can be achieved either covertly through a serial verb construction or overtly with the linker.

% Becausatives are unlike adposition-like verbs in that their complement is explicitly subordinated with a complementizer, but in the linker construction this complementizer can disappear.\footnote{More work needs to be done on this. I only became aware of a possible complementizer requirement going through my data after I left the field. I have not elicited an ungrammatical example demonstrating that the complementizer is required in a sentence structure like \textit{becausative} \textit{result} (?\textit{complementizer}) \textit{cause}. The lack of such an example in my corpus makes me suspect the complementizer is necessary, but I have not confirmed this.}

 %This can be seen in (\ref{ex:uunuutl3}--\ref{ex:unwiitl1}), where the ordering of the clause is reversed from (\ref{ex:uunuutl1}--\ref{ex:uunuutl5}). In (\ref{ex:uunuutl1}--\ref{ex:uunuutl5}), the becausative is the first word, followed by the apodosis, then the protasis. These constructions are highly analogous to the SVCs of adposition-like verbs (\S\ref{ch:sv:data}).

Finally, \textit{ʔuunuuƛ}/\textit{ʔunw̓iiƛ} must take a result that is a verbal predicate. A nominal complement is ungrammatical, as shown in (\ref{ex:whydidntsleeprepeat}) (repeated from (\ref{ex:whydidntsleep}) and (\ref{ex:whydidntsleep2}).

\ex \label{ex:whydidntsleeprepeat}
\begingl
\glpreamble ʔuunuuƛitaḥ wik ƛuł weʔič. ʕiḥakita nay̓aqak. //
\gla ʔuunuuƛ=(m)it=(m)aˑḥ wik ƛuł weʔič. ʕiḥak=(m)it=maˑ nay̓aqak //
\glb because=\textsc{pst}=\textsc{real.1sg} \textsc{neg} good sleep. cry=\textsc{pst}=\textsc{real.3} baby //
\glft `I didn't sleep well because (of it); the baby was crying.' (\textbf{B}, Bob Mundy) //
\endgl
\xe

\ex~ \label{ex:whydidntsleep2}
\begingl
\glpreamble *wikitaḥ ƛuł weʔič ʔuunuuƛ nay̓aqakʔisʔi. //
\gla wik=(m)it=(m)aˑḥ ƛuł weʔič ʔuunuuƛ nay̓aqak=ʔis=ʔiˑ //
\glb \textsc{neg}=\textsc{pst}=\textsc{real.1sg} good sleep because baby=\textsc{dim}=\textsc{art} //
\glft Intended: `I didn't sleep well because of the baby.' (\textbf{B}, Bob Mundy) //
\endgl
\xe

%\noindent Context for (\ref{ex:uunuutl4}): A bunch of kids are racing. A fast boy wins the race.
\begin{comment}
\ex~ \label{ex:uunuutl4}
\begingl
\glpreamble *hitaʔapweʔin kaatkimqsuptaał t̓an̓eʔisʔi ʔuunuuƛ našuk. //
\gla hitaʔap=weˑʔin kaatkimqsuptaał t̓an̓a=ʔis=ʔiˑ ʔuunuuƛ našuk //
\glb win=\textsc{hrsy.3} race child=\textsc{dim}=\textsc{art} because strong //
\glft Intended: `The kid won the race because he is strong.' (\textbf{B}, Bob Mundy) //
\endgl
\xe

\vspace{5pt}
\end{comment}

\begin{comment}
Becausitives also follow the typical verbal pattern of being able to freely drop arguments, already seen in (\ref{ex:uunuutl1}) and again in (\ref{ex:becausechanged}).

\vspace{5pt}

Context for (\ref{ex:becausechanged}): A retelling of traditional ways of life. This followed an explanation of how this isn't done anymore, and a lengthy pause.

\ex~ \label{ex:becausechanged}
\begingl
\glpreamble ʔunʔuuƛ̓aƛʔał kʷiisḥin. //
\gla ʔunʔuuƛ=!aƛ=ʔał kʷis-L.ḥin //
\glb because=\textsc{now}=\textsc{pl} different-\textsc{dr} //
\glft `Because they're different now.' (\textbf{C}, \textit{tupaat} Julia Lucas) //
\endgl
\xe
\end{comment}

%There was some difference between speakers about the grammaticality of non-initial becausitives. One of my Ucluelet consultants, Marjorie Touchie produced non-initial becausitives without the linker (\ref{ex:uunuutl3}), and Fidelia Haiyupis, an Ehattesaht woman, produced such an example once (\ref{ex:unwiitl1}). However, on other occasions Fidelia rejected such examples without the linker (\ref{ex:unwiitllink2}, \ref{ex:unwiitllink3}), as did Julia Lucas, an Ahousaht speaker (\ref{ex:uunuutllink1}, \ref{ex:uunuutllink2}). My guess is that the obligatorily-linked version is the older pattern, and this reflects a change in progress that is at different points of progression for different speakers and different dialects.

\begin{comment}
\ex~ \label{ex:because2}
\begingl
\glpreamble ʔuunuuƛḥs hiniiʔiƛ ʔin m̓iƛaa. //
\gla ʔuunuuƛ-(q)ḥ=s mačiił ʔin m̓iƛ-(y)aˑ //
\glb because-\textsc{link}=\textsc{strg.1sg} inside.\textsc{mo} \textsc{comp} rain-\textsc{dr} //
\glft `I came inside because it was raining.' (\textbf{T}, Fidelia Haiyupis) //
\endgl
\xe

\ex \label{ex:unwiitllink2}
\begingl
\glpreamble hitaʔapintniš ʔunw̓iiƛḥ ʕuuy̓aałintin. //
\gla hitaʔap=int=niš ʔunw̓iiƛ-(q)ḥ ʕuuy̓aał=int=(y)in //
\glb inside.\textsc{mo}=\textsc{real.1sg} because-\textsc{link} take.medicine=\textsc{pst}=\textsc{weak.1pl} //
\glft `We won because we had medicine.' (\textbf{T}, Fidelia Haiyupis) //
\endgl
\xe

\ex~ \label{ex:unwiitllink3}
\begingl
\glpreamble *hitaʔapintniš ʔunw̓iiƛ ʕuuy̓aałintin. //
\gla hitaʔap=int=niš ʔunw̓iiƛ ʕuuy̓aał=int=(y)in //
\glb inside.\textsc{mo}=\textsc{real.1sg} because take.medicine=\textsc{pst}=\textsc{weak.1pl} //
\glft Intended: `We won because we had medicine.' (\textbf{T}, Fidelia Haiyupis) //
\endgl
\xe

\ex~ \label{ex:uunuutllink1}
\begingl
\glpreamble wikits ƛuł waʔič ʔunʔuuƛḥ wawaałwiqa ʕiniiƛ. //
\gla wik=(m)it=s ƛuł waʔič ʔunʔuuƛ-(q)ḥ wawaałwiqa ʕiniiƛ //
\glb \textsc{neg}=\textsc{pst}=\textsc{real.1sg} good sleep because-\textsc{link} bark dog //
\glft `I didn't sleep well because the dog was barking.' (\textbf{C}, Julia Lucas) //
\endgl
\xe

\ex~ \label{ex:uunuutllink2}
\begingl
\glpreamble *wikits ƛuł waʔič ʔunʔuuƛ wawaałwiqa ʕiniiƛ. //
\gla wik=(m)it=s ƛuł waʔič ʔunʔuuƛ wawaałwiqa ʕiniiƛ //
\glb \textsc{neg}=\textsc{pst}=\textsc{real.1sg} good sleep bark dog //
\glft Intended: `I didn't sleep well because the dog was barking.' (\textbf{C}, Julia Lucas) //
\endgl
\xe
\end{comment}

The evidence suggests the following for \textit{ʔuunuuƛ} and \textit{ʔunw̓iiƛ}. These words are verbs that take a single sentential causal complement, which is verbal and may optionally be introduced by a complementizer. The semantic relation of the becausatives is \textsc{cause} and has a single argument (the cause) which is related to a result through syntactic coordination, either in an SVC in a manner analogous to the behavior of adposition-like SVCs, or via a linker construction.

%The evidence so far suggests that the words \textit{ʔuunuuƛ} and \textit{ʔunw̓iiƛ} behave like verbs. There are two constructions that link these words to their arguments. There is an SVC where the becausative and its apodosis appear in the matrix clause, and the protasis appears as a complement of the becausative, optionally with an overt complementizer. There is also a linker construction, where the becausative has a linker attached and coordinates with the apodosis in the matrix clause, and again the protasis is a subordinate clause with an optional complementizer. For some speakers, the SVC construction is the only one in which the becausative can appear without a linker: that is, the first word in the sentence. The apodosis shares its subject with the becausitive, and when the predicate linker appears on the becausative it must link it with with the apodosis. In keeping with this specialness of the apodosis argument, the protasis (but not the apodosis) can be introduced with a complementizer.

While \textit{ʔuunuuƛ} and \textit{ʔunw̓iiƛ} behave as verbs with a sentential complement, \textit{ʔuusaaḥi} typically requires a participant complement. The ungrammatical example (\ref{ex:whydidntsleep2}) can be made grammatical by switching out \textit{ʔuunuuƛ} for \textit{ʔuusaaḥi} (\ref{ex:whydidntsleep3}).

\ex \label{ex:whydidntsleep3}
\begingl
\glpreamble wikitaḥ ƛuł weʔič ʔuusaaḥi nay̓aqakʔisʔi. //
\gla wik=(m)it=(m)aˑḥ ƛuł weʔič ʔuusaaḥi nay̓aqak=ʔis=ʔiˑ //
\glb \textsc{neg}=\textsc{pst}=\textsc{real.1sg} good sleep because.of baby=\textsc{dim}=\textsc{art} //
\glft `I didn't sleep well because of the baby.' (\textbf{B}, Bob Mundy) //
\endgl
\xe

\begin{comment}
\ex \label{ex:uusahi1}
\begingl
\glpreamble ʔuusaaḥimta nay̓aqakʔi. wikitaḥ ƛuł weʔič. //
\gla ʔuusaaḥi=imt=(m)aˑ nay̓aqak=ʔiˑ. wik=(m)it=(m)aˑḥ ƛuł weʔič //
\glb because.of=\textsc{pst}=\textsc{real.3} baby=\textsc{art} \textsc{neg}=\textsc{pst}=\textsc{real.1sg} good sleep //
\glft `It was because of the baby; I didn't sleep well.' (\textbf{B}, Bob Mundy) //
\endgl
\xe

\ex~ \label{ex:uusahi2}
\begingl
\glpreamble *ʔuusaaḥimta ʕiḥak nay̓aqakʔi. wikitaḥ ƛuł weʔič. //
\gla ʔuusaaḥi=imt=(m)aˑ ʕiḥak nay̓aqak=ʔiˑ. wik=(m)it=(m)aˑḥ ƛuł weʔič //
\glb because.of=\textsc{pst}=\textsc{real.3} cry.\textsc{dr} baby \textsc{neg}=\textsc{pst}=\textsc{real.1sg} good sleep //
\glft Intended: `It was because of the crying baby; I didn't sleep well.' (\textbf{B}, Bob Mundy) //
\endgl
\xe
\end{comment}

The participant cause must occur immediately following \textit{ʔuusaaḥi}, as shown in (\ref{ex:wonbecausemedicine2}, \ref{ex:*wonbecausemedicine2}).

\ex \label{ex:wonbecausemedicine2}
\begingl
\glpreamble ʔuusaaḥi ʕuʔi hitaʔap. //
\gla ʔuusaaḥi ʕuʔi hitaʔap //
\glb because.of medicine win //
\glft `They won because of the medicine.' (\textbf{C}, \textit{tupaat} Julia Lucas) //
\endgl
\xe

\ex~ \label{ex:*wonbecausemedicine2}
\begingl
\glpreamble *ʔuusaaḥi hitaʔap ʕuʔi. //
\gla ʔuusaaḥi hitaʔap ʕuʔi //
\glb because.of win medicine //
\glft Intended: `They won because of the medicine.' (\textbf{C}, \textit{tupaat} Julia Lucas) //
\endgl
\xe

\textit{ʔuusaaḥi} may take a sentential causal complement only if the cause is preceded by the complementizer (\ref{ex:uusahi5}--\ref{ex:uusahi7}).

\ex \label{ex:uusahi5}
\begingl
\glpreamble ʔuusaaḥi hitaʔap ʔin ʕuy̓inak. //
\gla ʔuusaaḥi hitaʔap ʔin ʕuy̓i-naˑk //
\glb because.of win \textsc{comp} medicine-have  //
\glft `They won because they had medicine.' (\textbf{C}, \textit{tupaat} Julia Lucas) //
\endgl
\xe

\ex~ \label{ex:uusahi6}
\begingl
\glpreamble ʔuusaaḥis wik ƛuł waʔič ʔin waawaałyuqʷa ʕiniiƛ. //
\gla ʔuusaaḥi=s wik ƛuł waʔič ʔin wałyuq-LR2L.a ʕiniiƛ //
\glb because.of=\textsc{strg.1sg} \textsc{neg} good sleep \textsc{comp} bark-\textsc{rp} dog  //
\glft `I didn't sleep well because the dog was barking.' (\textbf{C}, \textit{tupaat} Julia Lucas) //
\endgl
\xe

\ex~ \label{ex:uusahi7}
\begingl
\glpreamble ʔuusaaḥimta ʔuusuqta ʔanis t̓iʕaaʔatimt. //
\gla ʔuusaaḥi=imt=maˑ ʔuusuqta ʔani=s t̓iʕaaʔatu=imt //
\glb because.of=\textsc{pst}=\textsc{real.3} be.hurt.\textsc{cv} \textsc{comp}=\textsc{1sg} fall.down=\textsc{pst}  //
\glft `I got hurt because I fell down.' (\textbf{B}, Bob Mundy) //
\endgl
\xe

\begin{comment}
[[uusahi plus linker ]]
\textit{ʔuusaaḥi} may only be able to take the linker when it is non-initial. Both consultants with whom I attempted to add a linker to an ʔuusaaḥi-initial sentence were uncertain if it was okay or not but felt it was weird (\ref{ex:uusahi7}, \ref{ex:uusahi8}).

\ex \label{ex:uusahi7}
\begingl
\glpreamble ?? ʔuusaaḥiqḥita nay̓aqakʔi wikitaḥ ƛuł weʔič. //
\gla ʔuusaaḥi-(q)ḥ=(m)it=(m)aˑ nay̓aqak=ʔiˑ wik=(m)it=(m)aˑḥ ƛuł weʔič //
\glb because.of-\textsc{link}=\textsc{pst}=\textsc{real.3} baby=\textsc{art} \textsc{neg}=\textsc{pst}=\textsc{real.1sg} good sleep //
\glft Intended: `I didn't sleep well because of the baby.' (\textbf{B}, Bob Mundy) //
\endgl
\xe

*? ʔuusaaḥiqḥʔiš ʔuusaqta wik̓aałukʷint

ʔuusaqtumtʔiš ʔuusaaḥiqḥ wik̓aałukʷint
\end{comment}

\textit{ʔuusaaḥi} is also able to take the linker, although like the use of the complementizer, this changes the syntactic category of its complement from a noun or participant to a clause.\footnote{This is an extremely rare case of linker attachment causing a syntactic shift when it attaches to a word.}

\ex \label{ex:uusahiqh}
\begingl
\glpreamble ʔuusuqtumtʔiš ʔuusaaḥiqḥ wik̓aałukʷint. //
\gla ʔuusuqta=umt=ʔiˑš ʔuusaaḥi-(q)ḥ wik-!aałuk=int //
\glb hurt=\textsc{pst}=\textsc{strg.3} because-\textsc{link} \textsc{neg}-look.after=\textsc{pst}  //
\glft `He got hurt because he wasn't paying attention.' (\textit{N}, Fidelia Haiyupis) //
\endgl
\xe

%Like \textit{ʔuunuuƛ}/\textit{ʔunw̓iiƛ}, \textit{ʔuusaaḥi} behaves in many ways like other verbs. It has two complements, one of which must be a noun phrase protasis (unlike \textit{ʔuunuuƛ}/\textit{ʔunw̓iiƛ}, which must have clausal protases). Like \textit{ʔuunuuƛ}/\textit{ʔunw̓iiƛ}, \textit{ʔuusaaḥi} shares its subject with its apodosis complement. It may be open to linker attachment, but this is unclear. The word does not occur in the Sapir-Thomas texts \citep{sapir1939, sapir1955}, so appeals to published historical Nuuchahnulth cannot resolve the matter. If \textit{ʔuusaaḥi} cannot accept the linker, it is one of very few verbs (if any) with this property, and is perhaps in the midst of a change in progress, from verb-like to preposition or conjunction-like.

%\textit{ʔuusaaḥi} occurs in the first two publications of the Sapir-Thomas texts \citep{sapir1939, sapir1955} 37 times. In only two of these is the cause a sentential complement.

Like \textit{ʔuunuuƛ}/\textit{ʔunw̓iiƛ}, \textit{ʔuusaaḥi} is a verb taking a single argument, a cause. This is associated with the result through either a serial verb construction or with a linker. Unlike \textit{ʔuunuuƛ}/\textit{ʔunw̓iiƛ}, \textit{ʔuusaaḥi} takes a nominal causal complement, although this can be changed to a sentential complement with either the introduction of the complementizer or by attaching the linker to \textit{ʔuusaaḥi}. This latter phenomenon, where a word's syntactic behavior changes on account of the linker, is unique in the language so far as I know, and I have not modeled it in my implemented analysis.

\subsection{\textit{ʔuyi}} \label{ch:link:uyi}

Of the possibly-verbal, possibly-adpositional words in Nuuchahnulth, \textit{ʔuyi} is one of the most ambiguous cases (Adam Werle, \textit{p.c.}). The meaning of \textit{ʔuyi} is `at (a time)' and it typically cooccurs with another predicative word in a sentence. In this case, the clausal enclitics scope over both predicates (\ref{ex:uyin}--\ref{ex:uyidrop}). The temporal complement of \textit{ʔuyi} can be a nominal either occurring after (\ref{ex:uyin}) or before \textit{ʔuyi} (\ref{ex:uyiobj}), it can be expressed in a clause with the possible mood (\ref{ex:uyipssb}) or the definite mood (\ref{ex:uyidef}), or it can be dropped from the clause entirely (\ref{ex:uyidrop}).

\ex \label{ex:uyin}
\begingl
\glpreamble ʔuyaw̓it̓siis saantii ʔucičƛ ciquuwłi. //
\gla ʔuya-w̓it̓s=(y)iis saantii ʔu-ci-čiƛ ciquwił=ʔiˑ //
\glb at.a.time-going.to=\textsc{weak.1sg} Sunday \textsc{x}-go.to-\textsc{mo} church=\textsc{art} //
\glft `I'm going to church on Sunday.' (\textbf{Q}, Sophie Billy) //
\endgl
\xe

\ex~ \label{ex:uyiobj}
\begingl
\glpreamble waałakin yuułuʔiłʔatḥ kuʔał ʔuyi. //
\gla wałaak-LS=(m)in yuułuʔiłʔatḥ kuʔał ʔuyi //
\glb go.to.\textsc{mo}-\textsc{gr}=\textsc{real.1pl} Ucluelet morning at.a.time //
\glft `We're going to Ucluelet in the morning.' (\textbf{B}, Bob Mundy) //
\endgl
\xe

\ex~ \label{ex:uyipssb}
\begingl
\glpreamble ʔuyimaḥʔaała n̓an̓aan̓ič kuʔiičiʔeƛquu. //
\gla ʔuyi=maˑḥ=ʔaała n̓an̓aan̓ič kuʔał-iˑčiƛ=!aƛ=quu //
\glb at.a.time=\textsc{real.1sg}=\textsc{habit} read morning-\textsc{in}=\textsc{now}=\textsc{pssb.3} //
\glft `I read in the mornings.' (\textbf{B}, Bob Mundy) //
\endgl
\xe

\ex~ \label{ex:uyidef}
\begingl
\glpreamble ʔuyimtaḥ ʕimtnaakšiƛ čakupšiʔeƛqas. //
\gla ʔuyi=imt=(m)aˑḥ ʕimt-naˑk-šiƛ čakup-šiƛ=!aƛ=qaˑs //
\glb at.a.time=\textsc{pst}=\textsc{real.1sg} name-have-\textsc{mo} man-\textsc{mo}=\textsc{now}=\textsc{defn.1sg} //
\glft `I was a full man when I got my name.' (\textbf{B}, Bob Mundy) //
\endgl
\xe

\ex~ \label{ex:uyidrop}
\begingl
\glpreamble ʔuyiʔum kitḥšiƛ siičił. //
\gla ʔuyi=!um kitḥ-šiƛ si-L.(č)ił //
\glb at.a.time=\textsc{cmfu.2sg} ring-\textsc{mo} \textsc{1sg}-do.to //
\glft `Call me then.' (\textbf{C}, \textit{tupaat} Julia Lucas) //
\endgl
\xe

\textit{ʔuyi} has a tendency to double in fluent speech: as the first predicate of a two-predicate utterance, then later following its object (\ref{ex:uyidouble1}, \ref{ex:uyidouble2}). The doubling is always grammatically optional, so that (\ref{ex:uyidouble2}) is grammatical without the doubling, as in (\ref{ex:uyidouble3}).

\ex \label{ex:uyidouble1}
\begingl
\glpreamble ʔuyimtinʔaała wałaak May ʔuyiʔeƛ. //
\gla ʔuyi=imt=(m)in=ʔaała wałaak May ʔuyi=!aƛ //
\glb at.a.time=\textsc{pst}=\textsc{real.1pl}=\textsc{habit} go.to.\textsc{mo} May at.a.time=\textsc{now} //
\glft `We would go (there) in May.' (\textbf{B}, Bob Mundy) //
\endgl
\xe

\ex~ \label{ex:uyidouble2}
\begingl
\glpreamble ʔuyisʔaał yaacuk kuʔał ʔuyi. //
\gla ʔuyi=s=ʔaał yaacuk kuʔał ʔuyi //
\glb at.a.time=\textsc{strg.1sg}=\textsc{habit} walk.\textsc{dr} morning at.a.time //
\glft `I walk in the morning.' (\textbf{C}, \textit{tupaat} Julia Lucas) //
\endgl
\xe

\ex~ \label{ex:uyidouble3}
\begingl
\glpreamble ʔuyisʔaał yaacuk kuʔał. //
\gla ʔuyi=s=ʔaał yaacuk kuʔał //
\glb at.a.time=\textsc{strg.1sg}=\textsc{habit} walk.\textsc{dr} morning //
\glft `I walk in the morning.' (\textbf{C}, \textit{tupaat} Julia Lucas) //
\endgl
\xe

Except for the strange case of doubling, the features of \textit{ʔuyi} so far are in line with other verbs. The clitic-sharing across predicates and the structure of (\ref{ex:uyidouble3}) in particular is identical to other serial verb constructions (\S\ref{ch:sv:data}).

However, a significant point of differentiation from typical verbs is that \textit{ʔuyi} does not accept the linker (\ref{ex:uyiqh1}, \ref{ex:uyiqh2}). When I presented (\ref{ex:uyiqh1}), Marjorie Touchie immediately corrected me and said that the way to say this would be with \textit{ʔuyi ʔam̓ii}. There are also no instances of linked \textit{ʔuyiqḥ} in \citet{sapir1939, sapir1955}.

\ex \label{ex:uyiqh1}
\begingl
\glpreamble *ʔuyiqḥʔaƛaḥ ʔam̓ii mamuuk hił makuwił. //
\gla ʔuyi-(q)ḥ=!aƛ=(m)aˑḥ ʔam̓ii mamuuk hił makuwił //
\glb at.a.time-\textsc{link}=\textsc{now}=\textsc{real.1sg} one.day.away work at.a.location store //
\glft Intended: `I will go to work at the store tomorrow.' (\textbf{B}, Bob Mundy \& Marjorie Touchie) //
\endgl
\xe

\ex~ \label{ex:uyiqh2}
\begingl
\glpreamble *ʔuyiqḥʔaƛs ʔam̓ii mamuuk hił makuwił. //
\gla ʔuyi-(q)ḥ=!aƛ=s ʔam̓ii mamuuk hił makuwił //
\glb at.a.time-\textsc{link}=\textsc{now}=\textsc{strg.1sg} one.day.away work at.a.location store //
\glft Intended: `I will go to work at the store tomorrow.' (\textbf{C}, \textit{tupaat} Julia Lucas) //
\endgl
\xe

\textit{ʔuyi} then behaves much like a verb in an SVC, but with two exceptions: (i) Unlike verbs (and all predicative words), it cannot accept the predicate linker; (ii) It can optionally double in constructions where it is separated from its direct object. I believe the reasons for this are that \textit{ʔuyi} is an historic verb that has undergone a grammatical category shift that has caused it to lose its status as a predicate.

The shape of \textit{ʔuyi} looks like the empty root \textit{ʔu-} combined with some verbal suffix. Although there is no contemporary productive suffix \textit{-yi} meaning `at a time,' \citet[p.~320]{sapir1939} list \textit{-(y)iya} with the definition `at $\ldots$ time, in $\ldots$ weather.' There are some instances of \textit{-(y)iya} in the Sapir-Thomas texts, such as \textit{ʔaḥʔaayiya} `at that time' \citep[p.~16]{sapir1939}, \textit{c̓awaayiya} `one time/one day' \citep[p.~19]{sapir1939}, and \textit{ʔuyiya} `at the time$\ldots$' \citep[p.~112]{sapir1939}. %Nuuchahnulth has another similar suffix \textit{-L.yuuya} `$\ldots$ of the time,' not listed in \citet{sapir1939}. It forms words like \textit{ʔuušyuuya} `sometimes,' \textit{hiišyuuya} `all the time.' These suffixes look like they may be related, and whether \textit{ʔuyi} came from \textit{ʔu-} + \textit{-(y)iˑ}, \textit{ʔu-} + \textit{-(y)iya}, or something else, 
It seems likely to me that \textit{ʔuyi} historically derives from this suffix verb \textit{-(y)iya}.

\textit{ʔuyi} retains many verbal properties, entering into SVCs as though it were an adposition-like verb (\S\ref{ch:sv:data:type3}), including being split from its direct object by the intervening VP. It is set apart from other verbs in having lost its status as a predicate, as seen from its inability to accept the predicate linker. This is, I believe, a case of grammaticalization in progress, with \textit{ʔuyi} moving from a verb to an adposition.

This process can also be seen in the word's ``doubling" in the right contexts. This could be analyzed as a simple repetition with argument-dropping, but that doesn't explain why this structure only occurs with this one word. The better explanation is that \textit{ʔuyi} is a word in-between two syntactic categories. As a verb, it can enter into serial verb constructions. But as an adposition, it can take a nominal complement. In ``doubling" contexts, \textit{ʔuyi} occurs twice: once entering into an SVC as a verb, and then again taking a complement as an adposition. In these contexts, the verbal \textit{ʔuyi} takes as its complement an adpositional phrase headed by adpositional \textit{ʔuyi}. Under this analysis, this doubling phenomenon only appears because this word is in a transitory step of a grammaticalization process, which ends with \textit{ʔuyi} becoming an adposition and losing its remaining verbal properties.

\subsection{Adposition-like words} \label{ch:link:adpositive}

In her dissertation, \citet{woo2007b} examines the syntax of what she terms ``prepositional predicates" and ultimately agrees with previous researchers that these words are verbs. The words she considers are: (1) \textit{ʔuuḥw̓ał} `using', (2) \textit{ʔuuʔink} `using', (3) \textit{ʔucḥin} benefactive, (4) \textit{ʔuuʔatup} benefactive/recipient, (5) \textit{ʔuukčamałčiqḥ} `do together with someone', (6) \textit{ʔukʷink} `go with', (7) \textit{ʔuukʷił} `do to', (8) \textit{ʔuḥtaa} `do to', and (9) \textit{ʔuḥ} subject marker.

\citeauthor{woo2007b} separates these words into two categories. The first six of these prepositional predicates introduce an extra argument into the clause, and using the Minimalist Framework, \citeauthor{woo2007b} categorizes them as full verbs (V) which, when working in concert with a main verb, coordinate at the level of \textit{v}P. By my definition, this would be a serial verb construction (\S\ref{ch:sv:def}). This full verb analysis is supported in part by the fact that the first set of words can occur as the sole predicate of a sentence.

However, the  last three words (\textit{ʔuukʷił}, \textit{ʔuḥtaa}, and \textit{ʔuḥ}) optionally mark arguments already inherent in the main verb. They require a main predicate to form a grammatical sentence (or may only be used alone in special circumstances, like question-answering). These \citeauthor{woo2007b} categorizes as flavors of little-\textit{v}.

Although I approach my analysis from within a different framework, I agree with \citeauthor{woo2007b}'s broad categorization. I checked speaker's intuitions about attaching the linker \textit{-(q)ḥ} to these adposition-like words and the judgments I received support \citeauthor{woo2007b}'s bifurcation into two categories, the first of which is verbal. Structurally, verbs should be able to coordinate either covertly through a serial verb construction or overtly with a linker morpheme. If a word is a member of a grammatical category (like an adposition or Minimalism's little-\textit{v}), it is non-predicative, and therefore in my analysis the predicate linker should not be able to attach.

Not all speakers recognize or use all of the adposition-like words \citeauthor{woo2007b} lists, so I was not able to test all of these words with all speakers. There is also a morphophonological problem testing \textit{ʔuḥ} (which would be *?\textit{ʔuḥḥ} with the linker). However, I have collected data on these from her list: (1) \textit{ʔuuḥw̓ał}, (3) \textit{ʔucḥin}, (4) \textit{ʔuuʔatup} and \textit{ʔuupaał(ḥ)} (not in \citeauthor{woo2007b}'s list), (6) \textit{ʔukʷink} with, (7) \textit{ʔuukʷił}, and (8) \textit{ʔuḥtaa}. In short, the words \citeauthor{woo2007b} calls verbs mostly accept the linker, while her ``little-\textit{v}" words do not.

\subsubsection{\textit{ʔuuḥw̓ał}} \label{ch:link:uuhwal} The adposition-like verb \textit{ʔuuḥw̓ał} `using' can accept the linker in a sentence without any change of meaning.

\ex \label{ex:uuhwal}
\begingl
\glpreamble wikcuk̓ʷapʔic ƛiisƛiisa ʔuuḥw̓ał ƛiisc̓uuy̓ak. //
\gla wikcuk=!ap=ʔic ƛis-LR2L.a ʔuuḥw̓ał ƛiisc̓uuy̓ak //
\glb easy=\textsc{caus}=\textsc{strg.2sg} write-\textsc{rp} using computer //
\glft `It's easy for you to write using a computer.' (\textbf{T}, Fidelia Haiyupis) //
\endgl
\xe

\ex~ \label{ex:uuhwalh}
\begingl
\glpreamble wikcuk̓ʷapʔic ƛiisƛiisa ʔuuḥw̓ałḥ ƛiisc̓uuy̓ak. //
\gla wikcuk=!ap=ʔic ƛis-LR2L.a ʔuuḥw̓ał-(q)ḥ ƛiisc̓uuy̓ak //
\glb easy=\textsc{caus}=\textsc{strg.2sg} write-\textsc{rp} using-\textsc{link} computer //
\glft `It's easy for you to write using a computer.' (\textbf{T}, Fidelia Haiyupis) //
\endgl
\xe

\subsubsection{\textit{ʔucḥin}} \label{ch:link:uuchin} The adposition-like verb \textit{ʔucḥin} `for, on the behalf of' can also freely accept the linker. %, although my consultant was less sure about it. She said that I could ``get away with" (\ref{ex:uuchinqh}) but thought it was unnecessary.

\ex \label{ex:uuchin}
\begingl
\glpreamble ʔucḥins mamuuk ʔuušḥy̓umsukqs. //
\gla ʔucḥin=s mamuuk ʔuuš-(q)ḥy̓uˑ-mis=uk=qs //
\glb \textsc{benef}=\textsc{strg.1sg} work some-related.or.friend-\textsc{nmlz}=\textsc{poss}=\textsc{defn.1sg} //
\glft `I'm working for my friend.' (\textbf{T}, Fidelia Haiyupis) //
\endgl
\xe

\ex~ \label{ex:uuchinqh}
\begingl
\glpreamble ʔucḥinqḥʔaƛs mamuuk ʔuušḥy̓umsukqs. //
\gla ʔucḥin-(q)ḥ=!aƛ=s mamuuk ʔuuš-(q)ḥy̓uˑ-mis=uk=qs //
\glb \textsc{benef}-\textsc{link}=\textsc{now}=\textsc{strg.1sg} work some-related.or.friend-\textsc{nmlz}=\textsc{poss}=\textsc{defn.1sg} //
\glft `I'm working for my friend.' (\textbf{T}, Fidelia Haiyupis) //
\endgl
\xe

\subsubsection{\textit{ʔuuʔatup}} \label{ch:link:uatup} There is speaker disagreement on whether the adpositive verb \textit{ʔuuʔatup} `on the behalf of, for the benefit of' freely accepts the linker. My consultant \textit{tupaat} Julia Lucas, a Central speaker, accepted it (\ref{ex:uatup}, \ref{ex:uatuph}) but my Barkley Sound consultants Bob Mundy and Marjorie Touchie did not (\ref{ex:uatup2}, \ref{ex:uatuph2}). This may be another case of a change in progress, where for my Barkley consultants, \textit{ʔuuʔatup} is in the midst of a grammaticalization process and becoming an adposition.

\ex \label{ex:uatup}
\begingl
\glpreamble ʔak̓ułis suw̓a ḥiy̓aḥi č̓apac ʔuuʔatup ḥaakʷaaƛukʔitk. //
\gla ʔak̓ułi=s suw̓a ḥiy̓aḥi č̓apac ʔuuʔatup ḥaakʷaaƛ=uk=ʔitk. //
\glb loan=\textsc{strg.1sg} \textsc{2sg} \textsc{d1} canoe \textsc{benef} young.woman=\textsc{poss}=\textsc{defn.2sg} //
\glft `I'm loaning you that canoe for your daughter.' (\textbf{C}, \textit{tupaat} Julia Lucas) //
\endgl
\xe

\ex~ \label{ex:uatuph}
\begingl
\glpreamble ʔak̓ułis suw̓a ḥiy̓aḥi č̓apac ʔuuʔatupḥ ḥaakʷaaƛukʔitk. //
\gla ʔak̓ułi=s suw̓a ḥiy̓aḥi č̓apac ʔuuʔatup-(q)ḥ ḥaakʷaaƛ=uk=ʔitk. //
\glb loan=\textsc{strg.1sg} \textsc{2sg} \textsc{d1} canoe \textsc{benef}-\textsc{link} young.woman=\textsc{poss}=\textsc{defn.2sg} //
\glft `I'm loaning you that canoe for your daughter.' (\textbf{C}, \textit{tupaat} Julia Lucas) //
\endgl
\xe

\ex~ \label{ex:uatup2}
\begingl
\glpreamble huyaałaḥ ʔuuʔatup t̓aatn̓eʔis. //
\gla huyaał=(m)aˑḥ ʔuuʔatup t̓aatn̓a=ʔis. //
\glb dance.\textsc{dr}=\textsc{real.1sg} \textsc{benef} child.\textsc{pl}=\textsc{dim} //
\glft `I dance for the children.' (\textbf{B}, Bob Mundy, Marjorie Touchie) //
\endgl
\xe

\ex~ \label{ex:uatuph2}
\begingl
\glpreamble *huyaałaḥ ʔuuʔatupḥ t̓aatn̓eʔis. //
\gla huyaał=(m)aˑḥ ʔuuʔatup-(q)ḥ t̓aatn̓a=ʔis //
\glb dance.\textsc{dr}=\textsc{real.1sg} \textsc{benef}-\textsc{link} child.\textsc{pl}=\textsc{dim} //
\glft Intended: `I am dancing for the children.' (\textbf{B}, Bob Mundy, Marjorie Touchie) //
\endgl
\xe

\subsubsection{\textit{ʔuupaał(ḥ)}} Though this word does not appear in \citet{woo2007b}, it is another adposition-like verb that appears to have the same meaning as \textit{ʔukʷink} `with'. Only Julia Lucas recognized \textit{ʔuupaał} as an independent word which could optionally take the linker \textit{-(q)ḥ} (\ref{ex:uupaal1}, \ref{ex:uupaalqh1}). For my other consultants who knew the word (Bob Mundy, Marjorie Touchie, and Fidelia Haiyupis), they only recognized \textit{ʔuupaałḥ} and not \textit{ʔuupaał}. They could articulate this straightforwardly (i.e., ``\textit{ʔuupaał} is not a word") but also rejected \textit{ʔuupaał} in examples (\ref{ex:uupaalh2}--\ref{ex:*uupaal3}).

\textit{ʔuupaał} (but not \textit{ʔuupaałḥ}) occurs in \citet{sapir1939, sapir1955}, so my interpretation of this is that for some speakers, \textit{ʔuupaał} has relexicalized to include what was formerly a separate linker morpheme. That is, a relexicalization process occurred that looks like:

\textit{ʔuupaał} > \textit{ʔuupaał} + \textit{-(q)ḥ} > \textit{ʔuupaałḥ}.

\ex \label{ex:uupaal1}
\begingl
\glpreamble ciiqmałapiw̓it̓asniš ʔuupaał y̓ukʷiiqsakqs. //
\gla ciq-mał-L.api-w̓it̓as=niˑš ʔuupaał y̓ukʷiiqsu=ʔak=qs. //
\glb speak-move.\textsc{dr}-above-going.to=\textsc{strg.1pl} with younger.sibling=\textsc{poss}=\textsc{defn.1sg} //
\glft `I am going to speak along with my younger sister.' (\textbf{C}, \textit{tupaat} Julia Lucas) //
\endgl
\xe

\ex~ \label{ex:uupaalqh1}
\begingl
\glpreamble ciiqmałapiw̓it̓asniš ʔuupaałqḥ y̓ukʷiiqsakqs. //
\gla ciq-mał-L.api-w̓it̓as=niˑš ʔuupaał-(q)ḥ y̓ukʷiiqsu=ʔak=qs. //
\glb speak-move.\textsc{dr}-above-going.to=\textsc{strg.1pl} with-\textsc{link} younger.sibling=\textsc{poss}=\textsc{defn.1sg} //
\glft `I am going to speak along with my younger sister.' (\textbf{C}, \textit{tupaat} Julia Lucas) //
\endgl
\xe

\ex~ \label{ex:uupaalh2}
\begingl
\glpreamble ʔuupaałḥitweʔin t̓an̓eʔisukqas łuučm̓uupukʔi pisat̓asw̓it̓as. //
\gla ʔuupaałḥ=(m)it=weˑʔin t̓an̓a=ʔis=uk=qaˑs łuučm̓uup=uk=ʔiˑ pisat-!as-w̓it̓as //
\glb with=\textsc{pst}=\textsc{hrsy.3} child=\textsc{dim}=\textsc{poss}=\textsc{defn.1sg} sister=\textsc{poss}=\textsc{art} play-outside.\textsc{dr}-going.to //
\glft `My child went with his sister to go play.' (\textbf{B}, Bob Mundy) //
\endgl
\xe

\ex~ \label{ex:*uupaal2}
\begingl
\glpreamble *ʔuupaałitweʔin t̓an̓eʔisukqas łuučm̓uupukʔi pisat̓asw̓it̓as. //
\gla ʔuupaał=(m)it=weˑʔin t̓an̓a=ʔis=uk=qaˑs łuučm̓uup=uk=ʔiˑ pisat-!as-w̓it̓as //
\glb with=\textsc{pst}=\textsc{hrsy.3} child=\textsc{dim}=\textsc{poss}=\textsc{defn.1sg} sister=\textsc{poss}=\textsc{art} play-outside.\textsc{dr}-going.to //
\glft Intended: `My child went with his sister to go play.' (\textbf{B}, Bob Mundy) //
\endgl
\xe

\ex~ \label{ex:uupaalh3}
\begingl
\glpreamble ʔuupaałḥintʔiš mamuuk ƛiisƛiisaʔapt̓. //
\gla ʔuupaałḥ=int=ʔiˑš mamuuk ƛiisƛiisaʔapt̓ //
\glb with=\textsc{pst}=\textsc{strg.3} work.\textsc{dr} Adam //
\glft `I worked with Adam.' (\textbf{T}, Fidelia Haiyupis) //
\endgl
\xe

\ex~ \label{ex:*uupaal3}
\begingl
\glpreamble *ʔuupaałintʔiš mamuuk ƛiisƛiisaʔapt̓. //
\gla ʔuupaał=int=ʔiˑš mamuuk ƛiisƛiisaʔapt̓ //
\glb with=\textsc{pst}=\textsc{strg.3} work.\textsc{dr} Adam //
\glft Intended: `I worked with Adam.' (\textbf{T}, Fidelia Haiyupis) //
\endgl
\xe

\subsubsection{\textit{ʔukʷink}} \label{ch:link:ukwink} \textit{ʔukʷink} `with' freely accepts the linker without a change in meaning (\ref{ex:ukwink1}--\ref{ex:ukwinkh2}).

\ex \label{ex:ukwink1}
\begingl
\glpreamble ʔucačiƛw̓it̓asaḥ nučiiʔi ʔukʷink ʔeʔiič̓imʔakqas. //
\gla ʔu-ca-čiƛ-w̓it̓as=(m)aˑḥ nučii=ʔiˑ ʔukʷink ʔeʔiič̓im=ʔak=qaˑs //
\glb \textsc{x}-go-\textsc{mo}-going.to=\textsc{real.1sg} mountain=\textsc{art} with parent.\textsc{pl}=\textsc{poss}=\textsc{defn.1sg} //
\glft `I'm going to the mountains with my parents.' (\textbf{B}, Bob Mundy, Marjorie Touchie) //
\endgl
\xe

\ex~ \label{ex:ukwinkh1}
\begingl
\glpreamble ʔucačiƛw̓it̓asaḥ nučiiʔi ʔukʷinkḥ ʔeʔiič̓imʔakqas. //
\gla ʔu-ca-čiƛ-w̓it̓as=(m)aˑḥ nučii=ʔiˑ ʔukʷink-(q)ḥ ʔeʔiič̓im=ʔak=qaˑs //
\glb \textsc{x}-go-\textsc{mo}-going.to=\textsc{real.1sg} mountain=\textsc{art} with-\textsc{link} parent.\textsc{pl}=\textsc{poss}=\textsc{defn.1sg} //
\glft `I'm going to the mountains with my parents.' (\textbf{B}, Bob Mundy, Marjorie Touchie) //
\endgl
\xe

\ex~ \label{ex:ukwink2}
\begingl
\glpreamble ʔukʷinkints ƛiisƛiisaʔapt̓ ʔucačiƛ yuułuʔiłʔatḥ. //
\gla ʔukʷink=int=s ƛiisƛiisaʔapt̓ ʔu-ca-čiƛ yuułuʔiłʔatḥ //
\glb with=\textsc{pst}=\textsc{strg.1} Adam \textsc{x}-go-\textsc{mo} Ucluelet //
\glft `I went with Adam to Ucluelet.' (\textbf{T}, Fidelia Haiyupis) //
\endgl
\xe

\ex \label{ex:ukwinkh2}
\begingl
\glpreamble ʔukʷinkḥints ƛiisƛiisaʔapt̓ ʔucačiƛ yuułuʔiłʔatḥ. //
\gla ʔukʷink-(q)ḥ=int=s ƛiisƛiisaʔapt̓ ʔu-ca-čiƛ yuułuʔiłʔatḥ //
\glb with-\textsc{link}=\textsc{pst}=\textsc{strg.1} Adam \textsc{x}-go-\textsc{mo} Ucluelet //
\glft `I went with Adam to Ucluelet.' (\textbf{T}, Fidelia Haiyupis) //
\endgl
\xe

\subsubsection{\textit{ʔuukʷił}} \label{ch:link:uukwil} Unlike the fully predicative verbs above, \textit{ʔuukʷił} `do to' does not accept the linker.

\ex \label{ex:tugofwar1}
\begingl
\glpreamble hałiiłintʔiš ʔiiḥatisʔatḥ ʔuukʷił c̓išaaʔatḥ čiicst̓ałw̓it̓as. //
\gla hałiił=int=ʔiˑš ʔiiḥatisʔatḥ ʔu-L.(č)ił c̓išaaʔatḥ čiicst̓ał-w̓it̓as //
\glb ask=\textsc{pst}=\textsc{strg.3} Ehattisaht \textsc{do.to} Tseshaht do.tug.of.war-going.to //
\glft `The Ehattesahts invited the Tseshahts to play tug of war.' (\textbf{T}, Fidelia Haiyupis) //
\endgl
\xe

\ex~ \label{ex:tugofwar2}
\begingl
\glpreamble *hałiiłintʔiš ʔiiḥatisʔatḥ ʔuukʷiłḥ c̓išaaʔatḥ čiicst̓ałw̓it̓as. //
\gla hałiił=int=ʔiˑš ʔiiḥatisʔatḥ ʔu-L.(č)ił-(q)ḥ c̓išaaʔatḥ čiicst̓ał-w̓it̓as //
\glb ask=\textsc{pst}=\textsc{strg.3} Ehattisaht \textsc{do.to}-\textsc{link} Tseshaht do.tug.of.war-going.to //
\glft Intended: `The Ehattesahts invited the Tseshahts to play tug of war.' (\textbf{T}, Fidelia Haiyupis) //
\endgl
\xe

\paragraph{\textit{ʔuḥtaa}} \label{ch:link:uhta} Like the more common object marker \textit{ʔuukʷił}, the marker \textit{ʔuḥtaa} does not appear to accept the linker. \textit{ʔuḥtaa} is an archaic word in modern Nuuchahnulth. My only consultant who recognized it was Julia Lucas, when listening to a recording of her older sister who used the word fluently in natural speech. She recognized the word without the linker, but rejected rephrases with the linker attached.

\begin{comment}
\vspace{5pt}

\noindent Context for (\ref{ex:uhta}, \ref{ex:uhtaqh}), discussing family relations.

\ex \label{ex:uhta}
\begingl
\glpreamble ʔuḥtaa Jane ʔuʔukʷił Alexandra y̓ukʷiiqsu. //
\gla ʔuḥtaa Jane ʔuʔukʷił Alexandra y̓ukʷiiqsu //
\glb only.\textsc{do.to} Jane call Alexandra younger.sibling //
\glft `Only Jane can call Alexandra younger.' (\textbf{C}, \textit{tupaat} Julia Lucas) //
\endgl
\xe

\ex~ \label{ex:uhtaqh}
\begingl
\glpreamble *ʔuḥtaaqḥ Jane ʔuʔukʷił Alexandra y̓ukʷiiqsu. //
\gla ʔuḥtaa-(q)ḥ Jane ʔuʔukʷił Alexandra y̓ukʷiiqsu //
\glb only.\textsc{do.to}-\textsc{link} Jane call Alexandra younger.sibling //
\glft Intended: `Only Jane can call Alexandra younger.' (\textbf{C}, \textit{tupaat} Julia Lucas) //
\endgl
\xe
\end{comment}

\subsection{Summary of the linker and class-ambiguous words}

Data about the attachment of the predicate linker can help shed light on the syntactic category of words whose categoricity is unclear. \textit{ʔuunuuƛ}, \textit{ʔunw̓iiƛ}, and \textit{ʔuusaaḥi} `because' all behave like verbs, and the free attachment of the predicate linker helps determine their argument structure. \textit{ʔuyi} does not accept the linker, and is in the process of transitioning to an adposition. The adposition-like words that can accept the linker seem to be clearly verbal. However, the argument-marking words \textit{ʔuukʷił} and \textit{ʔuḥtaa} do not accept the linker, as is expected if they belong to a non-predicative and functional category, whether they are called little-\textit{v} within Minimalism \citep{woo2007b}, or perhaps adpositions in other frameworks.

\section{HPSG analysis and implementation} \label{ch:link:analysis}

In this section I will go over my implementation for the linker morpheme (\S\ref{ch:link:analysis:linker}), the becausatives (\S\ref{ch:link:analysis:because}), and \textit{ʔuyi} (\S\ref{ch:link:analysis:uyi}). The implementation of adposition-like verbs has already been given (\S\ref{ch:sv:analysis:htype}), where they are treated as verbs. I have in this section given corroborating evidence of their inherently verbal qualities. The special adpositions/little-\textit{v} elements \textit{ʔuukʷił} and \textit{ʔuḥtaa} are not addressed here.

\subsection{The predicate linker lexeme(s)} \label{ch:link:analysis:linker}

The predicate linker attaches either directly to the predicate it is coordinating, or to a preceding modifier of that predicate (\S\ref{ch:link:attach}). Its syntactic position is in this way very much in line with that of second-position suffixes (\S\ref{ch:clause:2pv}), and so I will use the same analysis I developed for auxiliary predicate second-position suffixes like \textit{-w̓it̓as} and \textit{-maḥsa} in \S\ref{ch:clause:analysis:auxpred}. That is, I reuse the preparatory lexical rules for auxiliary predicate suffixes, \textit{pred-incorporation-lex-rule} (\ref{pred-incorporation-lex-rule}) and \textit{adv-incorporation-lex-rule} (\ref{adv-incorporation-pred-lex-rule}), the output of which can then be the daughter of the rule which adds the predicate linker. The reason for the two-step process is to accommodate attachment to different parts of speech without having multiple lexical rules for adding the suffix itself. My modeling of the linker morpheme thus treats it morphologically as an auxiliary predicate suffix, although it will have some different semantic and syntactic properties.

There are two main ways that the predicate linker differs from other suffixes in this morphological category. The first is that it adds more syntactic information: for example, it adds a complement (the coordinated predicate\footnote{The normal means of coordination is through syntactic features \textsc{lcoord} and \textsc{rcoord} (see \S\ref{ch:sv:analysis:coord}). In (\ref{2p-linker-verb-lex-rule-super}), coordination is taking place through the \textsc{comps} list instead, which allows the predicate linker rules to inherit from lexical supertypes for second-position suffixes that do not involve coordination.}), and requires matching subjects between the two coordinated predicates. The second way it differs is that there are two positions the linker can be in: on the initial coordinand, with the second-position enclitics, or on the later coordinand without the enclitics. Notionally, the predicate linker is simply an ``and" coordination and falls on one of the coordinands. However, I am going to need two versions of the suffix, one for attaching to the first coordinand and one for attaching to the second. In the case where the linker falls on the first coordinand along with the second-position enclitics, it needs to select for a later complement (the other coordinated predicate) that still has a non-cancelled subject on its valence list and is nonfinite. In the case where the linker falls on the second coordinand and without the second-position enclitics, it needs to select for an earlier complement (the other predicate) that is governed by a second-position enclitic complex and is finite.

I create a predicate linker supertype {\textit{2p-linker-verb-lex-rule-super}} for both of these subtypes (linker-first and linker-last). This supertype is part of a type hierarchy that includes suffix-verb types such as {\textit{2p-suffix-pred-verb-lex-rule}} (\ref{2p-suffix-pred-verb-lex-rule}). The type definition in (\ref{2p-linker-verb-lex-rule-super}) shows its inherited properties.

This rule states that the linker, when added through incorporation, adds the relation \textsc{and} which coordinates two events (an \textsc{l-index} and an \textsc{r-index}). All these events share the same mood and tense, semantic properties which are constrained through the second-position enclitics. The first event (the \textsc{l-index}) is identified with the daughter, which is the element that the linker has been added to, and the \textsc{r-index} is identified with the index of a new item on the \textsc{comps} list, that is, the to-be-linked element. This new element is placed first on the \textsc{comps} list, before the complements of the original verb.\footnote{This is a constraint imposed by the limitations of the DELPH-IN formalism. It is possible that the daughter is intransitive, in which case the daughter's complements list is null. I cannot append a contentful list to the end of a null list, so I must first add the element that is known to be non-null (the other coordinand) followed by the possibly-null element. This does not cause any problems with constructing a string, since the second complement can be realized first through a {\textit{head-comp-2}} rule.} Finally, the subjects of both coordinands are identified with each other by identifying their \textsc{xarg}s with each other.

\begin{singlespacing}
\ex \label{2p-linker-verb-lex-rule-super}
\adjustbox{max width=\textwidth - 0.2in}{
\begin{avm}
\[\asort{2p-linker-verb-lex-rule-super}
 synsem.local & \[ cat & \[ val & \[ subj & \< \avmbox{S} \> \\
                      comps & \< \[ local & \[ cat.val.comps & \q< \q> \\
                                               cont.hook & \[ index & \avmbox{2} \\
                                                   xarg & \avmbox{3} \] \] \] \> $\oplus$ \avmbox{C} \] \] \\
                   cont.hook & \[ index & \avmbox{1} \\
                                  xarg & \avmbox{3} \] \] \\
 c-cont & \< \[ pred & \textsc{and} \\
                c-arg & \avmbox{0} \textit{e}\[ mood & \avmbox{4} \\
                                                  tense & \avmbox{5} \] \\
                l-index & \avmbox{1} \textit{e}\[ mood & \avmbox{4} \\
                                                  tense & \avmbox{5} \] \\
                r-index & \avmbox{2} \textit{e}\[ mood & \avmbox{4} \\
                                                  tense & \avmbox{5} \] \\ \] \> \\
 daughter & \[ synsem.local & \[ cat & \[ head.prd & + \\
                                       val & \[ subj & \< \avmbox{S} \> \\
                                              comps & \avmbox{C} \\ \] \] \\
                                 cont.hook & \[ index & \avmbox{1} \\
                                                xarg & \avmbox{3} \] \] \] \]
\end{avm}
}
\xe
\end{singlespacing}

 The two rules for attaching the linker to the first or the second coordinand are given in (\ref{2p-linker-first-lex-rule}, \ref{2p-linker-last-lex-rule}) below and inherit from (\ref{2p-linker-verb-lex-rule-super}) above.

\begin{singlespacing}
\ex \label{2p-linker-first-lex-rule}
\adjustbox{max width=\textwidth - 0.2in}{
\begin{avm}
\[\asort{2p-linker-first-lex-rule}
 synsem.local.cat & \[ head & \[ aux & -- \\
                                 form & nonfinite \] \\
                       val.comps & \< \[ local.cat & \[ val.subj & cons \\
                                                    posthead & + \] \], $\ldots$ \> \]
 \]
\end{avm}
}
\xe

\ex~ \label{2p-linker-last-lex-rule}
\adjustbox{max width=\textwidth - 0.2in}{
\begin{avm}
\[\asort{2p-linker-last-lex-rule}
 synsem.local.cat & \[ head & \[ aux & + \\
                                 form & finite \] \\
                       val.comps & \< \[ local.cat & \[ head & \[ aux & + \\
                              form & finite \] \\
%                          val.subj & \< \[ opt & + \] \> \\
                          posthead & -- \] \], $\ldots$ \> \]
 \]
\end{avm}
}
\xe
\end{singlespacing}

When the linker attaches to the first coordinand (\ref{2p-linker-first-lex-rule}), the complement needs to still be looking for its subject, be nonfinite, and occur after the linker. The mother node is nonfinite and not an auxiliary so that it can be the complement of the second-position enclitic complex (\S\ref{ch:clause:analysis:2p}).

When the linker attaches last (\ref{2p-linker-last-lex-rule}), it is looking for a complement with the second-position enclitics attached (a finite auxiliary) and which will be to the left of the linker. The linker will be the head in this construction, so I constrain it to be a finite auxiliary in order to head the sentence. (\ref{ex:someonefindtree}) below is a tree structure for the sentence in (\ref{ex:someonefind}) and shows the linker attaching to the first coordinand, and (\ref{ex:speakoutsidefidelia2tree}) is a tree for (\ref{ex:speakoutsidefidelia2}), showing the linker attaching to the second coordinand.

\begin{singlespacing}
\ex \label{ex:someonefindtree}
\adjustbox{max width=\textwidth - 0.2in}{
\begin{forest}
[PredP \\ \begin{avm}
         \[ \asort{head-comp}
 	        subj & \avmbox{1} \\
 	        comps & \q< \q> \\
 	        index & \avmbox{A} \]
         \end{avm}
 [PredP \\ \begin{avm}
         \[ \asort{comp-head}
 	        head.prd & + \\
 	        subj & \avmbox{1} \\
 	        comps & \< \avmbox{4} \[ head & \[ aux & + \\
 	                                form & nonfinite \\
 	                                prd & + \] \\
 	                      subj & \< \avmbox{1} \> \\
 	                      comps & \q< \q> \\
 	                      index & \avmbox{R} \] \> \\
 	       index & \avmbox{A} \]
         \end{avm}
   [Pred \\ \begin{avm}
 \avmbox{2} \[ \asort{2p-linker-first-lex-rule}
 	        head.prd & + \\
 	        subj & \avmbox{1} \\
 	        comps & \avmbox{3} \< \[ head & \[ aux & + \\
 	                                form & nonfinite \\
 	                                prd & + \] \\
 	                      subj & \avmbox{1} \\
 	                      comps & \q< \q> \\
 	                      index & \avmbox{R} \] \> \\
 	        index & \avmbox{A} \\
 	        rel & {\textsc{some}(\textit{e}\avmbox{L}, \textit{x}\avmbox{1}), \textsc{and}(\textit{e}\avmbox{A}, \avmbox{L}, \avmbox{R})} \]
         \end{avm}
     [ʔuušqḥ \\ some.\textsc{link} ]
   ]
   [Aux \\ \begin{avm}
 	               \[ \asort{2p-mood-lex}
 	                  head.prd & + \\
 	                  subj & \avmbox{1} 3rd \\
 	                  comps & \< \avmbox{2} \[ head.prd & + \\
 	                                           subj & \avmbox{1} \\
 	                                           comps & \avmbox{3} \] \>\ $\oplus$ \avmbox{3} \]
                   \end{avm}
     [{=!aƛ} \\ \textsc{now} ]
   ]
 ]
 [VP \\ \begin{avm}
 \avmbox{4} \[ \asort{head-opt-comp}
 	        head & \[ aux & + \\
 	                  form & nonfinite \\
 	                  prd & + \] \\
 	        subj & \avmbox{1} \\
 	        comps & \q<  \q> \\
 	        index & \avmbox{R} \]
         \end{avm}
   [VP \\ \begin{avm}
 \avmbox{2} \[ \asort{neutral-non-head-verb-lex}
 	        head & \[ aux & + \\
 	                  form & nonfinite \\
 	                  prd & + \] \\
 	        subj & \avmbox{1} \\
 	        comps & \< \[index & \avmbox{5} \] \> \\
 	        index & \avmbox{R} \\
 	        rel & {\textsc{find}(\textit{e}\avmbox{R}, \textit{x}\avmbox{1}, \textit{x}\avmbox{5})} \]
         \end{avm}
     [ʔuuwaƛ \\ find ]
   ]
 ]
]
\end{forest}}
\xe

\ex~ \label{ex:speakoutsidefidelia2tree}
\adjustbox{max width=\textwidth - 0.2in}{
\begin{forest}
[VP \\ \begin{avm}
         \[ \asort{comp-head}
 	        subj & \< \avmbox{1} \> \\
 	        comps & \q< \q> \\
 	        index & \avmbox{A} \]
         \end{avm}
    [VP \\ \begin{avm}
\avmbox{4} \[ \asort{comp-head}
 	        head & \avmbox{5} \\
 	        subj & \< \avmbox{1} \[opt & + \] \> \\
 	        comps & \q< \q> \\
 	        index & \avmbox{R} \\
 	        xarg & \avmbox{1} \]
         \end{avm}
      [V \\ \begin{avm}
 \avmbox{2} \[ \asort{verb}
 	        head.prd & + \\
 	        subj & \< \avmbox{1} \> \\
 	        comps & \avmbox{3} \q< \q> \\
 	        index & \avmbox{R} \\
 	        xarg & \avmbox{1}
 	        rel & {\textsc{speak}(\textit{e}\avmbox{R}, \textit{x}\avmbox{1})} \]
         \end{avm}
        [ ciiqciiqa \\ speak.\textsc{rp} ]
      ]
      [Aux \\ \begin{avm}
 	               \[ \asort{2p-mood-lex}
 	                  head & \avmbox{5}\[ aux & + \\
 	                                      form & finite \\
 	                                      prd & + \] \\
 	                  subj & \< \avmbox{1} 1pl \> \\
 	                  comps & \< \avmbox{2} \[ head.prd & + \\
 	                                           subj & \< \avmbox{1} \> \\
 	                                           comps & \avmbox{3} \] \>\ $\oplus$ \avmbox{3} \]
                   \end{avm}
        [ {=mitniš} \\ \textsc{strg.1pl.pst} ]
      ]
  ]
  [V \\ \begin{avm}
         \[ \asort{2p-linker-last-lex-rule}
 	        head & \[ aux & + \\
 	                  form & finite \] \\
 	        subj & \< \avmbox{1} \> \\
 	        comps & \< \avmbox{4} \[ head & \[ aux & + \\
 	                                form & finite \\
 	                                prd & + \] \\
 	                      subj & \< \[opt & + \] \> \\
 	                      comps & \q< \q> \\
 	                      index & \avmbox{R} \\
 	                      xarg & \avmbox{1} \] \> \\
 	        index & \avmbox{A} \\
 	        rel & {\textsc{outside}(\textit{e}\avmbox{L}, \textit{x}\avmbox{1}), \textsc{and}(\textit{e}\avmbox{A}, \avmbox{L}, \avmbox{R})} \]
         \end{avm}
    [ ƛ̓aaʔaasḥ \\ outside.\textsc{link} ]
  ]
]
\end{forest}}
\xe
\end{singlespacing}

\subsection{The becausative lexemes} \label{ch:link:analysis:because}

The because words, as defined in \S\ref{ch:link:because}, can be treated simply as adposition-like verbs. \textit{ʔuusaaḥi} is a plain transitive verb that takes a nominal complement,\footnote{The analysis presented here does not account for the syntactic changes \textit{ʔuusaaḥi} undergoes when the linker attaches, or when its complement is introduced by a complementizer (\S\ref{ch:link:because}).} and \textit{ʔuunuuƛ} is a verb that takes a clausal complement that is not a main clause. These verbs can then take a linker like any other verb, or enter into adposition-like SVCs. Type definitions for these verbs are given in (\ref{uusahi-lexeme}, \ref{uunuutl-lexeme}). 

\begin{singlespacing}
\ex \label{uusahi-lexeme}
\adjustbox{max width=\textwidth - 0.2in}{
\begin{avm}
\[\asort{ʔuusaaḥi}
phon & ``ʔuusaaḥi" \\
synsem.local & \[ cat & \[ head & \[\asort{verb} htype & adpositive \] \\
               val & \[ subj & \< \[ $\ldots$index & \avmbox{1} \] \> \\
                        comps & \< \[ local & \[ cat & \[ head & noun \\
                                           val & \[ spr & \q< \q> \\
                                                    comps & \q< \q> \] \\
                                           posthead & + \] \\
                                           $\ldots$index & \avmbox{2} \] \] \> \] \] \\
           cont & \[ hook.index & \avmbox{0} \\
                     rels & \< \[ pred & \textsc{cause} \\
                             arg0 & \avmbox{0} \textit{e} \\
                             arg1 & \avmbox{1} \textit{x} \\
                             arg2 & \avmbox{2} \textit{x} \] \> \] \]
\]
\end{avm}
}
\xe

\vspace{-10pt}

\ex~ \label{uunuutl-lexeme}
\adjustbox{max width=\textwidth - 0.2in}{
\begin{avm}
\[\asort{ʔuunuuƛ}
phon & ``ʔuunuuƛ" \\
synsem.local & \[ cat & \[ head & \[\asort{verb} htype & adpositive \] \\
               val & \[ subj & \< \[ $\ldots$index & \avmbox{1} \] \> \\
                        comps & \< \[ local & \[ cat & \[ head & \[\asort{verb}
                                                            aux & + \\
                                                            form & finite \] \\
                                           val & \[ subj & \q< \q> \\
                                                    comps & \q< \q> \] \\
                                           posthead & + \\
                                           mc & -- \] \\
                                     $\ldots$index & \avmbox{2} \] \] \> \] \] \\
           cont & \[ hook.index & \avmbox{0} \\
                     rels & \< \[ pred & \textsc{cause} \\
                             arg0 & \avmbox{0} \textit{e} \\
                             arg1 & \avmbox{1} \textit{x} \\
                             arg2 & \avmbox{2} \textit{e} \] \> \] \]
\]
\end{avm}
}
\xe
\end{singlespacing}

\subsection{The \textit{ʔuyi} lexeme(s)} \label{ch:link:analysis:uyi}

\vspace{-5pt}

I have claimed that \textit{ʔuyi} is a word in transition: It is an historic verb that has lost some of its predicativeness and is undergoing a change to an adposition. It is possible for the word appear twice, once in an SVC-like construction, and then again following its object.

The way my grammar has been constructed, if I simply constrain \textit{ʔuyi} to be [\textsc{prd} --], the second-position enclitics will not be able to attach to it, since they select for a [\textsc{prd} +] complement (\S\ref{ch:clause:analysis:2p}). And yet I still want second-position enclitics to attach, but I want to prevent linker attachment. So my analysis requires more moving parts than just a [\textsc{prd} --] constraint.

To model \textit{ʔuyi} I constrain it not just to be [\textsc{prd} --] but also [\textsc{head} \textit{adposition}]. This lexical entry will not yet be able to form a clause (it cannot be the complement of a second-position enclitic), but it will be able to go through lexical incorporation rules, which will allow second-position suffixes like \textit{-w̓it̓as} `going to' (\S\ref{ch:clause:analysis:2pv}) to apply. While it can undergo auxiliary predicate suffixation, the [\textsc{prd} --] constraint will block the application of the linker rule (\ref{2p-linker-verb-lex-rule-super}). The lexical type for \textit{ʔuyi} is (\ref{uyi-lexeme}) below.

\vspace{-25pt}

\begin{singlespacing}
\ex \label{uyi-lexeme}
\adjustbox{max width=\textwidth - 0.2in}{
\begin{avm}
\[\asort{ʔuyi}
phon & ``ʔuyi" \\
synsem.local & \[ cat & \[ head & \[\asort{adposition}
                              htype & adpositive \\
                              prd & -- \] \\
               val & \[ subj & \< \[ $\ldots$index & \avmbox{1} \] \> \\
                        comps & \< \[ local.cat.head & noun \\
                                      opt & -- \\
                                      $\ldots$index & \avmbox{2} \] \> \] \] \\
           cont & \[ hook.index & \avmbox{0} \\
                     rels & \< \[ pred & \textsc{at-a-time} \\
                             arg0 & \avmbox{0} \textit{e} \\
                             arg1 & \avmbox{1} \textit{x} \\
                             arg2 & \avmbox{2} \textit{x}\] \> \] \]
\]
\end{avm}
}
\xe
\end{singlespacing}

This lexical specification assumes a nominal complement. However, \textit{ʔuyi} can take a clause in the possible or definite mood. I know of no way to underspecify between a noun (with a semantic individual) and a clause (with a semantic event), so I have to define a separate version of \textit{ʔuyi} that takes this kind of complement. I give this version of the lexeme in (\ref{uyi-lexeme-clausal-comp}), only giving the parts that differ from (\ref{uyi-lexeme}).

\begin{singlespacing}
\ex \label{uyi-lexeme-clausal-comp}
\adjustbox{max width=\textwidth - 0.2in}{
\begin{avm}
\[\asort{ʔuyi-clausal-comp}
phon & ``ʔuyi" \\
synsem.local & \[ cat.val.comps & \< \[ local.cat & \[ head.prd & + \\
                       cont.hook.index & \avmbox{2} \] \\
                                        opt & -- \] \> \\
           cont & \[ hook.index & \avmbox{0} \\
                     rels & \< \[ pred & \textsc{at-a-time} \\
                             arg0 & \avmbox{0} \textit{e}\\
                             arg1 & \avmbox{1} \textit{x}\\
                             arg2 & \avmbox{2} \textbf{\textit{e}}\[mood & possible-or-definite \] \] \> \] \]
\]
\end{avm}
}
\xe
\end{singlespacing}

While (\ref{uyi-lexeme}) and (\ref{uyi-lexeme-clausal-comp}) can go through the unary lexical incorporation rules that add second-position suffixes (see \S\ref{ch:clause:analysis:2pv}), as [\textsc{prd} --] lexical items they cannot yet be the complement of a second-position enclitic. To address this, I create a lexical rule \textit{adp-to-pred-verb-lex-rule} (\ref{adp-to-pred-verb-lex-rule}) that turns a [\textsc{prd} --] adposition or adpositive verb\footnote{Once a lexical entry for \textit{ʔuyi} goes through the incorporation rules to acquire a second-position suffix, it will become a [\textsc{prd} --] verb, but stay [\textsc{htype} adpositive].} into a [\textsc{prd} +] verb. Once this rule applies, the word will be blocked from acquiring further second-position suffixes because it will no longer inherit from any morphological type that can be a daughter to these rules ({\textit{predicate-lex}}, {\textit{common-noun-lex}}, and so on). But it will be able to be the complement of the auxiliary second-position enclitics, since it is now [\textsc{prd} +].

\begin{singlespacing}
\ex \label{adp-to-pred-verb-lex-rule}
\adjustbox{max width=\textwidth - 0.2in}{
\begin{avm}
\[\asort{adp-to-pred-verb-lex-rule}
synsem & \[ local.cat.head & \[\asort{verb}
                    prd & + \\
                    htype & adpositive \] \] \\
daughter & \[ synsem.local.cat.head & \[\asort{verb-or-adposition} 
                                   prd & -- \\
                                   htype & adpositive \] \\
            inflected & infl-satisfied \]
\]
\end{avm}
}
\xe
\end{singlespacing}

These rules are sufficient to model cases where \textit{ʔuyi} does not double, that is, appear twice with a single complement as in (\ref{ex:uyidouble1}, \ref{ex:uyidouble2}). To accommodate doubling I will need a second form of \textit{ʔuyi}.

I define another version of the word (\ref{uyi-lexeme-2}) which will be a verb that does not supply any semantic content and selects specifically for the adposition version of \textit{ʔuyi}. This lexical entry is also defined as [\textsc{prd} --] to prevent the linker from attaching, and must go through the {\textit{adp-to-pred-verb-lex-rule}} (\ref{adp-to-pred-verb-lex-rule}) in order to be used in the syntax. This version of \textit{ʔuyi} can be involved in an SVC, however it must have a complement headed by the adposition form of itself. It will only appear in sentences with a double \textit{ʔuyi}, and only as the initial instance of the word.

\begin{singlespacing}
\ex \label{uyi-lexeme-2}
\adjustbox{max width=\textwidth - 0.2in}{
\begin{avm}
\[\asort{ʔuyi-2}
phon & ``ʔuyi" \\
synsem.local & \[ cat & \[ head & \[\asort{verb}
                              aux & -- \\
                              htype & adpositive \\
                              prd & -- \] \\
               val & \[ subj & \< \[ $\ldots$index & \avmbox{2} \] \> \\
                        comps & \< \[ local.cat.head & \[\asort{adposition}
                                            htype & adpositive \\
                                            prd & -- \] \\
                                      opt & -- \\
                                      $\ldots$index & \avmbox{1} \\
                                      $\ldots$xarg & \avmbox{2} \] \> \] \] \\
           cont & \[ hook.index & \avmbox{1} \\
                     rels & \q< \q> \] \]
\]
\end{avm}
}
\xe 
\end{singlespacing}

Example trees of \textit{ʔuyi} without doubling (\ref{ex:uyidouble3tree}) and with doubling (\ref{ex:uyidouble2tree}) are given below (sentences taken from (\ref{ex:uyidouble3}) and (\ref{ex:uyidouble2}) respectively).

\begin{singlespacing}
\ex \label{ex:uyidouble3tree}
\adjustbox{max width=\textwidth - 0.2in}{
\begin{forest}
[VP \\ \begin{avm}	
         \[\asort{head-comp}
           subj & \avmbox{1} \\
           comps & \q< \q> \]
           \end{avm}
  [VP \\ \begin{avm}	
         \[\asort{svc3-top-coord}
           subj & \avmbox{1} \\
           comps & \< \avmbox{3} \> \\
 	       lcoord & \avmbox{L} \\
 	       rcoord & \avmbox{R} \]
           \end{avm}
    [VP \\ \begin{avm}
 \avmbox{L} \[ \asort{comp-head}
 	          subj & \avmbox{1} \\
 	          comps & \< \avmbox{3} \> \]
            \end{avm}
      [V \\ \begin{avm}
 \avmbox{v1}  \[ \asort{adp-to-pred-verb-lex-rule}
 	            head & \[\asort{verb} prd & + \] \\
 	            subj & \avmbox{1} \\
 	            comps & \avmbox{2} \< \avmbox{3} \> \]
             \end{avm}
        [Adp \\ \begin{avm}
 	               \[ \asort{adposition}
 	                  head.prd & -- \\
 	                  subj & \avmbox{1} \\
 	                  comps & \< \avmbox{3} \> \\
 	                  rel & {\textsc{at}(\textit{e}, \avmbox{1}\textit{x}, \avmbox{3}\textit{x})} \]
                   \end{avm}
          [ ʔuyi \\ at.a.time ]
        ]
      ]
      [Aux \\ \begin{avm}
 	               \[ \asort{2p-mood-lex}
 	                  head.prd & + \\
 	                  subj & \avmbox{1} 1sg \\
 	                  comps & \< \avmbox{v1} \[ subj & \avmbox{1} \\
 	                                           comps & \avmbox{2} \] \>\ $\oplus$ \avmbox{2} \]
                   \end{avm}
        [ {=sʔaał} \\ \textsc{strg.1sg.habit} ]
      ]
    ]
    [VP \\ \begin{avm}
 	   \avmbox{R} \[ \asort{svc3-bottom-coord}
 	        subj & \avmbox{1} \\
 	        coord-rel.pred & \textsc{meanwhile} \\
 	        nonconj-dtr & \avmbox{v2}
 	      \]
          \end{avm}
      [V \\ \begin{avm}
 \avmbox{v2} \[ \asort{verb}
 	            subj & \avmbox{1} \\
 	            comps & \q< \q> \\
 	            rel & {\textsc{walk}(\textit{e}, \avmbox{1}\textit{x})} \]
             \end{avm}
        [ yaacuk \\ walk.\textsc{dr} ]
      ]
    ]
  ]
  [NP \\ \begin{avm}
 \avmbox{3} \[ \asort{bare-np-phrase}
 	        rel & {\textsc{morning}(\textit{e}, \avmbox{3}\textit{x})} \]
             \end{avm}
    [ kuʔał \\ morning ]
  ]
]
\end{forest}}
\xe
\end{singlespacing}

\begin{minipage}{\textwidth}
\ex~ \label{ex:uyidouble2tree}
\xe
\vspace{-30pt}
\begin{singlespacing}
\adjustbox{max width=\textheight-0.7in,angle=90}{
\begin{forest}
[VP \\ \begin{avm}	
         \[\asort{head-comp}
           subj & \avmbox{1} \\
           comps & \q< \q> \]
           \end{avm}
  [VP \\ \begin{avm}	
         \[\asort{svc3-top-coord}
           subj & \avmbox{1} \\
           comps & \< \avmbox{3} \> \\
 	       lcoord & \avmbox{L} \\
 	       rcoord & \avmbox{R} \]
           \end{avm}
    [VP \\ \begin{avm}
 \avmbox{L} \[ \asort{comp-head}
 	          subj & \avmbox{1} \\
 	          comps & \< \avmbox{3} \> \]
            \end{avm}
      [V \\ \begin{avm}
 \avmbox{v1}  \[ \asort{adp-to-pred-verb-lex-rule}
 	            prd & + \\
 	            subj & \avmbox{1} \\
 	            comps & \avmbox{2} \< \avmbox{3} \> \]
             \end{avm}
        [V \\ \begin{avm}
 	               \[ \asort{verb}
 	                  prd & -- \\
 	                  subj & \avmbox{1} \\
 	                  comps & \< \avmbox{3} \[\asort{adposition} \] \> \]
                   \end{avm}
          [ ʔuyi \\ at.a.time ]
        ]
      ]
      [Aux \\ \begin{avm}
 	               \[ \asort{2p-mood-lex}
 	                  prd & + \\
 	                  subj & \avmbox{1} 1sg \\
 	                  comps & \< \avmbox{v1} \[ subj & \avmbox{1} \\
 	                                           comps & \avmbox{2} \] \> $\oplus$\avmbox{2} \]
                   \end{avm}
        [ {=sʔaał} \\ {\textsc{strg.1sg.habit}} ]
      ]
    ]
    [VP \\ \begin{avm}
 	   \avmbox{R} \[ \asort{svc3-bottom-coord}
 	        subj & \avmbox{1} \\
% 	        coord-rel.pred & \textsc{meanwhile} \\
 	        nonconj-dtr & \avmbox{v2}
 	      \]
          \end{avm}
      [V \\ \begin{avm}
 \avmbox{v2} \[ \asort{verb}
 	            subj & \avmbox{1} \\
 	            comps & \q< \q> \\
 	            rel & {\textsc{walk}(\textit{e}, \avmbox{1}\textit{x})} \]
             \end{avm}
        [ yaacuk \\ walk.\textsc{dr} ]
      ]
    ]
  ]
  [AdpP \\ \begin{avm}
 \avmbox{3} \[ \asort{comp-head}
 	         prd & -- \\
 	         subj & \avmbox{1} \\
 	         comps & \q< \q> \]
          \end{avm}
    [NP \\ \begin{avm}
   \avmbox{4} \[ \asort{bare-np-phrase}
 	        rel & {\textsc{morning}(\textit{e}, \avmbox{4}\textit{x})} \]
             \end{avm}
      [ kuʔał \\ morning ]
    ]
    [Adp \\ \begin{avm}
 	        \[ \asort{adposition}
 	           prd & -- \\
 	           subj & \avmbox{1} \\
 	           comps & \< \avmbox{4} \> \\
 	           rel & {\textsc{at}(\textit{e}, \avmbox{1}\textit{x}, \avmbox{4}\textit{x})} \]
            \end{avm}
      [ ʔuyi \\ at.a.time ]
    ]
  ]
]
\end{forest}}
\end{singlespacing}
\end{minipage}

\subsection{Summary} \label{ch:link:analysis:summary}

My analysis of the linker (\S\ref{ch:link:analysis:linker}) requires two versions of the morpheme: One for the case where the linker appears on the initial coordinand and accepts the second-position enclitics, and one for the case where the linker appears on the second coordinand. In both cases, the predicate with the linker attached is the head of the clause, which is necessary so that coordinands are associated with the \textsc{and} relation the linker introduces. When the linker appears on the first coordinand, it selects for a complement that does not have the second-position enclitics and is lacking a subject. When the linker appears on the second coordinand, it selects for a complement that has already picked up the clausal enclitics (is headed by a finite auxiliary). Both versions of the linker are applied as the last step of a lexical incorporation process as first defined in \S\ref{ch:clause:analysis:2p}. This incorporation process allows the linker to attach either to a predicate or to a preceding modifying adverb.

The because words (\S\ref{ch:link:analysis:because}) do not require any special rules. I simply constrain them to be complement-taking verbs that are [\textsc{htype} adpositive], which ensures that they enter into the appropriate serial verb construction.

Finally, \textit{ʔuyi} is defined in three flavors (\S\ref{ch:link:analysis:uyi}): two that are non-predicative ([\textsc{prd} --]) adpositions, one selecting for a nominal complement and another for clausal complements, and a third that is a non-predicative verb that selects for a phrase headed by one of the adpositive \textit{ʔuyi}s. The [\textsc{prd} --] definition shared by all versions allows them to pass through the incorporation rules that attach auxiliary predicate suffixes, but blocks linker attachment. In order to be used as a predicate in the syntax (for example, as the target of second-position enclitics), \textit{ʔuyi} must go through a rule that converts it into a verbal predicate, the application of which means it is no longer able to be the target of incorporation. The adpositional versions of \textit{ʔuyi} introduce the relation \textsc{at-a-time} that provides the semantics of the word. The verbal version of \textit{ʔuyi} introduces no semantics, and only occurs in clauses where there is doubling.		