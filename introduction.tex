\acknowledgments{
   {\narrower\noindent
    Though one name appears on the author page, dissertations are not purely solitary exercises. It is impossible for this list to be exhaustive, and it only represents the people whose assistance I was most aware of when this document was written.
    
    I would first like to thank the Nuuchahnulth elders and native speakers who have shared with me their time and language, often sitting through long sessions where I asked questions about bizarre sentences, strange scenarios, odd ways of framing things, or asked (for the second or third time) what some sentence on a recording meant. Without their patience and willingness to work with me, none of this would be possible. I particularly want to thank: Julia Lucas (Nuuchahnulth name \textit{tupaat}), the late Simon Lucas (\textit{yuułnaak}), Bob Mundy (\textit{ʔamaawatuʔa}), Marjorie Touchie (\textit{ʔaʔasmacy̓ak}), Fidelia Haiyupis (\textit{čiiʔiłumqa}), and Sophie Billy (\textit{m̓aam̓aqiinšał}). There were other elders and speakers who I worked with on an ad-hoc or unofficial basis, or who I worked with but whose data I have not used on request. Although their names aren't acknowledged here out of respect for their privacy, they also helped me understand their language and helped sharpen my analyses and I am grateful to them as well.
    
    I would also like to thank my advisor Emily Bender, who helped me develop technical analyses, guided the set of questions for me to ask, and supported me in the long process of conducting research and writing a dissertation.
    
    I want to thank Adam Werle, who first put me in contact with Nuuchahnulth speakers and helped me navigate the community. He has included me in research programs and introduced me to other researchers, and my understanding of the language and research direction have been dramatically improved through informal discussions about linguistic phenomena. Adam organizes the Somass Valley Language Circle, which I have participated in and benefited from. He has also assisted on multiple occasions with data collection.
    
    \newpage
    
    The Department of Linguistics at the University of Washington and its faculty have also been supportive in this work. I am in particular grateful to Sharon Hargus for her feedback both on my dissertation as a document and in guiding my research direction and helping to scope my work.
    
    I am grateful to Henry Kammler, who has shared with me some of his work with Ucluelet elders and assisted me with some transcription work. I am also thankful to Matthew Davidson, who provided me with his digitized copy of the first two volumes of the Sapir-Thomas Nootka Texts \citep{sapir1939, sapir1955}, which enabled me to quickly look for example constructions in those volumes to compare against modern Nuuchahnulth.
    
    I would also like to thank those who helped fund this work. When learning Nuuchahnulth, I was supported for the summer of 2016 by a Foreign Language and Area Studies (FLAS) Fellowship from the University of Washington. For my initial fieldwork, I was funded through a Field Research Support Award. Some of the sessions I spent with Fidelia Haiyupis and Sophie Billy were funded by the Ehattesaht-Chinehkint Tribe, who in return received my notes and recordings. The final portion of my fieldwork was funded through a 2018-2019 Jacobs Research Funds group grant, "Wakashan Documentation", with Henry Kammler and Adam Werle. Finally, I received the Excellence in Linguistics Research Graduate Fellowship from the University of Washington Department of Linguistics, which enabled me to focus entirely on this dissertation for a full quarter. I am thankful to everyone involved in providing financial support for my research.
    
    Finally, I want to thank those who are also learning the Nuuchahnulth language along with me. First, I want to thank Amie DeJong, a fellow PhD student at the University of Washington with whom I carpooled many times up to Vancouver Island, and who began studying Nuuchahnulth before I did. I also want to thank community members learning their language, many of whom I have studied with, who have helped me, and who have invited me to learn along with them from fluent elders. This includes \textit{t̓aqumsʔaqƛ} Josh Shaw, \textit{n̓aasʔałuk} John Rampanen, \textit{čuucqa} Layla Rorick, \textit{taaʔisumqa} Dawn Foxcroft, \textit{yaacuʔisʔaqs} Linsey Haggard, \textit{ḥakaƛ} Chrissie John, Moira Barney, \textit{ƛ̓iiḥƛ̓iiḥaʔaqsa} Verena Cootes, \textit{čiiʔiłimq} Gerri Thomas, and \textit{ḥiikuusinapšiił} Victoria Wells. 
   \par}
}

\chapter{Introduction} \label{ch:introduction}


\section{The research question} \label{ch:intro:researchquestion}

In this dissertation, I approach the following research question: What are the morphosyntactic properties of Nuuchahnulth clauses that contain more than one predicate? To answer this, I focus on clauses without coordinating word forms with meanings like `and' or `or.' I examine two such constructions: serial verb constructions (\cref{ch:sv}) and a morpheme called the predicate linker (\cref{ch:link}).

Both of these strategies are used to coordinate multiple elements. I give a typical serial verb construction below in (\ref{ex:svcintro}), where the verbs \textit{nuutkšiƛ} `go around (in a circle)' and \textit{kamatqšiƛ} `run' are juxtaposed, and a typical linker construction in (\ref{ex:linkintro}), where the linker \textit{-(q)ḥ} coordinates the verbs \textit{t̓iqʷaas} `sit' and \textit{n̓ačaał} `read.'

\ex \label{ex:svcintro}
\begingl
\glpreamble hišuk̓aqƛsuu nuutkšiƛ kamatqšiƛ. //
\gla hišuk=!aqƛ=suu nuutkšiƛ kamatq-šiƛ //
\glb all=\textsc{fut}=\textsc{neut.2pl} go.around.\textsc{mo} run-\textsc{mo} //
\glft `All of you will run around (the circle).' (\textbf{C}, \textit{tupaat} Julia Lucas) //
\endgl
\xe

\ex~ \label{ex:linkintro}
\begingl
\glpreamble t̓iqʷaasḥʔiš n̓ačaał łucsacʔi. //
\gla t̓iqʷ-aˑs-(q)ḥ=ʔiˑš n̓ačaał łucsac=ʔiˑ //
\glb sit-on.horizontal.surface-\textsc{link}=\textsc{strg.3} read girl=\textsc{art} //
\glft `The girl is sitting and reading.' (\textbf{T}, Fidelia Haiyupis) //
\endgl
\xe

These forms of covert (in the case of a serial verb construction) or non-canonical (in the case of the linker) coordination are sometimes taken for granted in other syntactic literature on the language, and have not yet been a focus of study on their own. Most recent researchers sketch out the basics of the Nuuchahnulth clause structure, as I will in \cref{ch:clause}, but do not include coordination strategies in these sketches.\footnote{See Chapter 3 in \citet{wojdak2005}, Chapter 2 in \citet{woo2007b}, and Chapter 2 in \citet{waldie2012}.} There are however other researchers who have addressed one or both of these coordination strategies: \citet{rose1981} provides an abbreviated discussion on both strategies, \citet{jacobsen1993} mentions both in his overview of the Nuuchahnulth clause, \citet{nakayama2001} has a discussion on serialization, and \citet{davidson2002} also discusses serialization, although by a different name.\footnote{\citeauthor{davidson2002} objects to the term ``serialization" for this part of the grammar, but he is describing the same phenomenon as the other researchers.} The work I describe here goes beyond what is present in these previous accounts.

With respect to serialization, \citet[p.~153--154]{rose1981} discusses ``noncued coordinate clauses". However, in her brief account she mixes examples of two clauses with examples of only one clause, and does not give the syntactic properties of this construction. %I take considerable care to delineate the clause boundaries in my review of serialization.
\citet[p.~248--252]{jacobsen1993} addresses what he calls ``verb serialization" within his larger discussion on the Nuuchahnulth clause. He defines this as multiple verbs that share second-position subject information.\footnote{This is functionally the same definition used by \citeauthor{nakayama2001} and \citeauthor{davidson2002}, and the one I will use in \S\ref{ch:sv:def}.} However, his analysis is limited due to its brevity (three and a half pages), and there are some cases where I think his categorization is incorrect: he analyzes some adverbs as verbs (\textit{ʔiiḥ} in its adverbial use `very',\footnote{\textit{ʔiiḥ} can also be used as an adjective, where it means `big.'} \textit{ʔaanaqḥ} `really', \textit{qii} `for a long time') and he categorizes as serialization some verbs that are plausibly subject-raising (the negator \textit{wik}, \textit{ḥasiik} `finish'). \citet[p.~102--109]{nakayama2001} gives the longest and most complete treatment of the phenomenon, which he calls ``serialization." He approaches this task from a primarily semantic and functional perspective, whereas my approach prioritizes the syntactic properties of serialization strategies and gives a formal account of their properties. Nonetheless, there is a surprising degree of overlap in our serialization categories.\footnote{See \S\ref{ch:sv:data} and especially footnote 3 of \cref{ch:sv}.} Finally, \citet[p.~149--152]{davidson2002} discusses the same phenomenon, which he calls ``bare absolute constructions" and analyzes as a type of adverbial clause. I will argue that these constructions are a form of coordination, and detail the differences between syntactic categories within this group. The work I present in \cref{ch:sv} represents a much more detailed accounting of the syntactic properties of serial verb constructions in Nuuchahnulth than has been given in the literature so far.

The linker morpheme has received considerably less attention than serialization in the literature. \citet[p.~241]{sapir1939} list it as an incremental suffix with the simple gloss `meanwhile' and the description ``sometimes used as a formative suffix". \citet[p.~151]{rose1981} gives one example of this morpheme in her discussion on coordination, which she also glosses as `meanwhile'. \citet[p.~245]{jacobsen1993} briefly mentions the linker, and defines it as `while'. As I will discuss in \cref{ch:link}, the syntactic properties of the linker are more complex than the English gloss `while' or `meanwhile' would indicate, and clauses containing a linker morpheme share some syntactic properties with serial verb constructions. To my knowledge, these three mentions of the linker are the only published original linguistic research on the morpheme. The linker is extremely common in fluent Nuuchahnulth. It occurs in syntactic examples in other work, although it is rarely examined as having its own syntactic properties. For instance, \citet{nakayama2001} gives an example of serialization where one verb has the linker morpheme attached.\footnote{Ex (257) on p.102} An opposition between bare serialization and linker constructions has been noted by Werle, who is actively working with speakers of the language (p.c.). On the analysis I present here, the addition of a linker morpheme creates an entirely different syntactic structure from serialization. My description of this morpheme presented in this dissertation is the first extended look at its morphosyntactic properties and the most complete in the literature to date.

My documentation and analysis of these two multi-predicate constructions fills a gap in the existing literature. It illuminates properties of the Nuuchahnulth clause, particularly what elements within the language can be coordinated and how. My analysis of the linker morpheme also sheds light on syntactic categories within the language and their properties (\S\ref{ch:link:application}). With my research question defined and situated within existing Nuuchahnulth linguistic work, I will now give some background situating the language itself.

\section{Background on Nuuchahnulth} \label{ch:intro:ncn}

Nuuchahnulth (ISO 639-3 nuk, formerly called Nootka\footnote{The name Nootka was applied to Nootka Sound and the people living there by Captain James Cook, likely upon being told as he approached in his ship to `circle round' (\textit{nuutkaa}) to the other side of the harbor \citep[p.~396]{campbell1997}. The word Nuuchahnulth (\textit{nuučaan̓uł}) is a more recent endonym created by members of the Nuuchahnulth tribes and means `along the mountains.'}) is a Wakashan language of the South Wakashan branch. It is spoken along the west coast of Vancouver Island, from the Kyuquot-Checleseht nation in the north to the Huu-ay-aht nation in the south. The language is a dialect continuum with a significant range of lexical and, to a lesser extent, grammatical variation among dialects. My Ucluelet consultants (in the south) expressed some difficulty with Kyuquot-Checleseht speech, and vice versa my Kyuquot-Checleseht consultant found southern speech strange and difficult at times. Despite this, there is a large amount of overlap between the dialects and speakers of one dialect can accommodate themselves to the speech of other dialects.

I follow Werle's dialect definitions \citep{werle2013dialects, werle2015b}, a schema which breaks the language into four dialect regions: Kyuquot-Checleseht (abbreviated \textbf{Q}), Northern (\textbf{T}), Central (\textbf{C}), and Barkley Sound (\textbf{B}). I will mark all data in this dissertation with the speaker, their name, and which dialect region they are from.

As a result of economic changes and more importantly the policy of residential schools in Canada, all native speakers of Nuuchahnulth are older, with the youngest speaker I know of in her later sixties. Most of the speakers I worked with were put in residential schools in their childhood, although two speakers I worked with were not.

Linguistic work on Nuuchahnulth began with \citet{sapir1911}. Along with the collaboration in particular of Tseshaht speaker and linguist Alex Thomas, Sapir collected language texts beginning in 1910 to around 1923. The first publication of this data was \citet{sapir1924}, followed much later, with the help of editor Morris Swadesh, by the ``Nootka Texts" volume \citep{sapir1939} and later still by the follow-up volume ``Native Accounts of Nootka Ethnography" \citep{sapir1955}. A planned third and final volume was intended to be entitled ``Nootka Legends and Stories", however Sapir passed on after the first volume and Swadesh passed on after the second. Their remaining work was published posthumously in several volumes beginning in 2000 \citep{whalingindians2000, whalingindians2004, wolfritual2007, whalingindians2009}. This delayed and multi-part ``third volume" was published after the exonym \textit{Nootka} had been replaced by the endonym \textit{Nuuchahnulth}, which caused changes to the titles of these publications. However, these volumes were also published with subtitles indicating they were some numbered part of the ``Sapir-Thomas Nootka Texts." The name ``Nootka Texts" has thus become ambiguous with respect to the first volume of this work or the work as a whole. In an attempt to differentiate, I will refer to the entire collection as either the ``Sapir-Thomas Nootka Texts" or the ``Sapir-Thomas texts", and reserve plain ``Nootka Texts" for the first published volume \citep{sapir1939}. It should be kept in mind that the Nuuchahnulth published in these volumes is approximately 100 years older than the Nuuchahnulth of my consultants, and represents the language at an earlier stage of its development. In addition, some of the speakers represented in these texts were monolingual in Nuuchahnulth, while all of my consultants were bilingual, and most of them spend most of their time using English.

Linguistic research on Nuuchahnulth has continued since the Sapir-Thomas Nootka Texts, although no collected publication of transcribed fluent Nuuchahnulth has yet matched their work in size. Most linguistic publications following the Sapir-Thomas texts have been in the form of grammars and theoretical (as opposed to textual) work. I reviewed much of it in \S\ref{ch:intro:researchquestion}, and will frequently reference both the theoretical and textual literature on Nuuchahnulth throughout this dissertation.

\section{Structure of the dissertation} \label{ch:intro:structure}

I will begin by describing my methodology (\cref{ch:methodology}), which will cover both my fieldwork and my grammar implementation work. I will then describe the basics of the language's clause structure (\cref{ch:clause}), and then the two types of multi-predicate constructions I examined: serial verbs (\cref{ch:sv}) and the predicate linker (\cref{ch:link}). Each of these three chapters is divided into two components (possibly comprising multiple sections): the first focusing on data and theory-neutral descriptions of the phenomena, and the second describing how I have modeled and implemented this within a computationally readable grammar in the Head-driven Phrase Structure Grammar (HPSG, \citealt{pollardsag1994}) formalism. The serial verb chapter also includes a discussion on phenomena included and excluded by the definition, and the linker chapter includes a section applying facts about the linker to words whose syntactic category or argument structure is unclear. For readers most interested in the descriptive facts and not the details of my HPSG analysis, they can focus on the descriptive sections, which will be sufficient for this purpose. For readers who are interested in syntactic modeling, the implementation sections should be accessible to both linguists working within HPSG as well as syntacticians working in other frameworks, although the details of my analyses will be framework-specific. I then give an overview of the results of my implemented grammar on targeted test suites (\cref{ch:results}), and finally conclude with a discussion of my findings about Nuuchahnulth, the HPSG analyses I have developed, and directions this for future investigation opened up by this work (\cref{ch:conclusion}).

There are two appendices included. The first defines the orthographic conventions and pronunciation of orthographic symbols (\cref{appendix:orthography}). The second describes the glossing conventions I have adopted, perhaps most significantly among them those conventions used to describe some of Nuuchahnulth's vowel-length-altering morphological rules, the meaning of special symbols like !\ and °, as well as a list of grams (\cref{appendix:glossing}). If some grams or notational conventions are confusing to the reader, they can consult this appendix.

Although the two multi-predicate constructions (serial verbs and the predicate linker) that I analyze share some similar properties, the chapters are largely separate from each other. It is possible for the interested reader to just focus on one chapter or the other, although I will be assuming basic facts about the Nuuchahnulth clause structure, outlined in \cref{ch:clause}.