Serial 

Previous Definitions of Serial Verbs

Background: The phenomena under consideration,

- Subj TransV IntransV Obj
- Subj V V
- V-qḥ V

Thesis: The definition of ``serial verbs" is too imprecise for my purposes. It appears that multiple different phenomena are subsumed under the name ``serial verb", and the tests are too imprecise for my purposes.

\cite{aikhenvalddixon2006} give a survey of serial verb phenomena across the world's languages, and some criteria to separate serial verbs from other linguistic phenomena. Their definition depends on the (prototypical) serial verb construction possessing six features:

\begin{enumerate}
\item a single predicate
\item syntactic monoclausality
\item prosodic monoclausality
\item shared tense, aspect, mood, and polarity
\item a single event
\item argument sharing
\end{enumerate}

Several of these features are not easily distinguishable or testable: in particular, the claims to monoclausality and serial verbs sharing a single predicate and event.


> Aikhenvald \& Dixon

> Miriam Butts

Phenomena in Nuuchahnulth

Methods (corpus, elicitation)

Basic structure of the Nuuchahnulth clause