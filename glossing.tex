\chapter{Segmentation and Glossing Conventions} \label{appendix:glossing}

Many of the segmentation and glossing conventions I use are non-standard and adapted particularly to the challenges of Nuuchahnulth. I will first address some of the special notations in the segmentation line, and then give the grams I use. In this section, I have attempted to give all non-Leipzig standard \citep{leipzig} glosses in the dissertation. These glossing conventions are a version of those presented in \citet{inmanwerle2016a}, and most grams are also developed in \citet{werle2016}. This section is only meant to give enough of a background to make the interlinear glossed text (IGT) in this dissertation intelligible and interpretable to linguists.

\section{Segmentation symbols}

There are four parts of Nuuchahnulth phonology that require special symbols in the morpheme segmentation: two types of consonant mutations, variable-length vowels, and segments that only appear after a vowel or nasal.

Consonant mutation is triggered by certain affixes, following patterns called ``hardening" and ``softening" \citep{werle2010}. A ``hardening" suffix causes the preceding segment to become glottalized, resulting in an ejective in the case of stops and affricates, and otherwise inserts a glottal stop. The hardening pattern for fricatives differs between hardening suffixes and hardening clitics. Suffix hardening typically converts the fricative into a glottalized glide, whereas clitic hardening inserts a glottal stop. There is a special morphophonemic rule that the \textit{ƛ} of the momentaneous aspect (\S\ref{ch:clause:aspect}) under hardening always becomes \textit{ʔ} instead of \textit{ƛ̓}.

A ``softening" suffix causes the preceding segment to weaken, which converts fricatives to glides and otherwise inserts a glottal stop. Nuuchahnulth only has suffix (and not clitic) softening.

Following \citet{werle2010}, I use !\ to represent hardening, and ° to represent softening, across both clitics and suffixes. The !\ notation was first introduced by \citet{boas1900}, and ° by \citet{jacobsen1973}. Like \citeauthor{werle2010}, I abandon Sapir's use of\ \ ' and ` for mutations, and use the same symbols for both suffix and clitic mutations. Examples of all three types of suffix and clitic hardening and softening are shown in (\ref{ex:stopoutside}--\ref{ex:gatherindoors}).

\ex \label{ex:stopoutside}
\begingl
\glpreamble wiinapasʔap̓i. //
\gla wiinapi-!as=!ap=!iˑ //
\glb stop-outside.\textsc{dr}=\textsc{caus}=\textsc{cmmd.2sg} //
\glft `Stop (the car or driver of the car).' (\textbf{C}, \textit{tupaat} Julia Lucas) //
\endgl
\xe

\ex~ \label{ex:lineupoutside}
\begingl
\glpreamble c̓iy̓iiƛ //
\gla c̓is-!iˑƛ //
\glb line-outside.\textsc{mo} //
\glft `line up outside' (\textbf{T}, Fidelia Haiyupis) //
\endgl
\xe

\ex~ \label{ex:gatherindoors}
\begingl
\glpreamble hišumyiłʔaqƛniš hawiiʔaƛqu ʔapw̓in n̓aas. //
\gla hišumł-°ił=ʔaqƛ=niˑš hawiiƛ=!aƛ=quˑ ʔapw̓in n̓aas //
\glb gather.together-indoors.\textsc{dr}=\textsc{fut}=\textsc{strg.1pl} finish.\textsc{mo}=\textsc{now}=\textsc{pssb.3} half day //
\glft `Let's get together at midday.' (\textbf{T}, Fidelia Haiyupis) //
\endgl
\xe

Nuuchahnulth also has vowels that may be long or short depending on where they fall in the word. These vowels are long in the first two syllables of a word, and short in the third syllable or later. It is hard to say what their ``underlying length" property may be: they are a third length category. Following the established system in Wakashan studies, I represent these syllables in the morpheme line as the vowel followed by a ˑ.\footnote{This innovation is thanks to \citet{rose1981}, who amended \citet{sapir1939}'s more cumbersome \u{$\cdot$}} Both long and short realizations of variable-length vowels are shown for the ending \textit{=maˑ} in (\ref{ex:tworoads}, \ref{ex:newoneisnice}).

\ex \label{ex:tworoads}
\begingl
\glpreamble ʔaƛiičiʔaƛma t̓ašii. //
\gla ʔaƛa-iˑčiƛ=!aƛ=maˑ t̓ašii //
\glb two-\textsc{in}=\textsc{now}=\textsc{real.3} road //
\glft `There are two roads (now).' (\textbf{B}, Bob Mundy) //
\endgl
\xe

\ex~ \label{ex:newoneisnice}
\begingl
\glpreamble ƛułmaa c̓ušukʔi. //
\gla ƛuł=maˑ c̓ušuk=ʔiˑ //
\glb good=\textsc{real.3} new=\textsc{art} //
\glft `The new one is nice.' (\textbf{B}, Bob Mundy) //
\endgl
\xe

Many affixes in Nuuchahnulth have a leading consonant that regularly disappears under certain phonological conditions, typically when preceded by a non-nasal consonant. Again, following the established literature in South Wakashan and first introduced by Sapir, I write these disappearing consonants in parentheses. Both realizations for the suffix \textit{-L.(č)ił} are shown in (\ref{ex:whatwatching}, \ref{ex:givemedicine}).

\ex \label{ex:whatwatching}
\begingl
\glpreamble ʔaaqičiłk n̓aacsa. //
\gla ʔaqi-L.(č)ił=k n̓aacsa //
\glb what-\textsc{do.to}=\textsc{ques.2sg} see.\textsc{cv} //
\glft `What are you watching?' (\textbf{C}, \textit{tupaat} Julia Lucas) //
\endgl
\xe

\ex~ \label{ex:givemedicine}
\begingl
\glpreamble ʕuʔikʷiƛs suutił. //
\gla ʕuʔikʷiƛ\footnotemark=s sut-L.(č)ił //
\glb give.medicine.\textsc{mo}=\textsc{strg.1sg} \textsc{2sg}-do.to //
\glft `I'm giving you medicine.' (\textbf{C}, \textit{tupaat} Julia Lucas) //
\endgl
\xe

\footnotetext{The momentaneous ending is typically \textit{-kʷiƛ} after \textit{u}, \textit{-čiƛ} after other vowels and nasals, and \textit{-šiƛ} after other consonants. This is a rare instance of \textit{-kʷiƛ} occurring after something other than a \textit{u}, and might be an indication that there was a \textit{u} here in an earlier stage of the language.}

Some of these disappearing consonants change based on their environment. A \textit{č} regularly becomes a \textit{k} after a \textit{u}. This correspondence is not shown. Which consonants of an affix are disappearing also changes from dialect to dialect. I have attempted to segment disappearing consonants as appropriate for each dialect.\footnote{Notably, my consultant Julia Lucas fairly consistently pronounces the /q/ in the linker suffix \textit{-(q)ḥ}. I still transcribe the suffix in the segmentation line with the parentheses, as she sometimes fails to produce the /q/ when attached to quantifiers.}

\section{Template notation}

Nuuchahnulth has a set of vowel length and reduplication templates, typically triggered by a suffix containing segmental phonology. These templates specify reduplication and vowel length of up to the first two syllables of the word. I gloss these templates with the symbols \textit{L}, \textit{S}, \textit{R}, and \textit{R2}, attached to the suffix which triggers the template. %There are two exceptions to the general rule that templatic morphology is associated with segmental phonology. The first is the graduative aspect, which I gloss as though it were an aspect suffix consisting only of the template. The other is certain kinds of plurals, which may consist of only reduplication or lengthening. I gloss these as prefixes consisting of only the templatic information.

In my notation, \textit{L} and \textit{S} indicate \textit{L}ong and \textit{S}hort vowels, and are ordered with respect to their occurrence: LS for a long first vowel and a short second vowel, SS for two short vowels, and so on. \textit{R} indicates an onset-nucleus reduplication pattern, and \textit{R2} a pattern that is onset-nucleus for polysyllabic roots, and full reduplication for monosyllabic roots. R2 is a pattern that only occurs with the iterative and repetitive aspects, and a limited number of plurals. In this notation, vowel length is always specified prior to reduplication: LR means a long reduplicant (followed by a vowel whose length is unaltered), and RL means a reduplicant followed by a lengthened base. If multiple templates apply, the vowel length specifications of the final morpheme win out, and reduplication remains.

There are two cases where a template is not attached to segmental morphology, yet I always gloss the template as though it were a prefix or suffix. The first is certain templatic patterns for plurals, such as R2 or LR, which I segment as if they were prefixes. Non-segmental morphology doesn't naturally have a place, so it is a little arbitrary. However there is one reason to prefer an analysis of plural templates as prefixal: Some of these plural templates include an infix <t> which always inserts after the first vowel (so a left-peripheral infix, closer to prefix than suffix). The one other case is the graduative aspect, which is simply realized as the template LS. I segment this as a suffix occurring after other aspect forms. The graduative only occurs following other aspect marking, and after the graduative applies, other aspect suffixes may follow. Segmenting this template as though it were a suffix preserves the morphological ordering of the aspect system.

\cref{table:typicaltemplates} gives a list of most types of templates found in the language, including an example of the two patterns for the R2 template.

\begin{table}[ht]
\centering
\caption{List of lexical suffix templates}
\label{table:typicaltemplates}
\begin{tabular}{lll}
template                  & gloss                                                                                                        & surface form                                                                                      \\ \hline
\multicolumn{1}{|l|}{L}   & \multicolumn{1}{l|}{\begin{tabular}[c]{@{}l@{}}ču-L.ʔatu\\ dive-sink.into.water\end{tabular}}               & \multicolumn{1}{l|}{\begin{tabular}[c]{@{}l@{}}čuuʔatu\\ dive down into water\end{tabular}}      \\ \hline
\multicolumn{1}{|l|}{LS}  & \multicolumn{1}{l|}{\begin{tabular}[c]{@{}l@{}}hašił-LS.sa\\ have.news-\textsc{aug1}\end{tabular}}                   & \multicolumn{1}{l|}{\begin{tabular}[c]{@{}l@{}}haašiłsa\\ interesting\end{tabular}}              \\ \hline
\multicolumn{1}{|l|}{SS}  & \multicolumn{1}{l|}{\begin{tabular}[c]{@{}l@{}}ʔaya-iˑčiƛ-SS.(q)aq\\ many-\textsc{in}-\textsc{aug2}\end{tabular}}           & \multicolumn{1}{l|}{\begin{tabular}[c]{@{}l@{}}ʔayičiƛaq\\ became very many\end{tabular}}        \\ \hline
\multicolumn{1}{|l|}{R}   & \multicolumn{1}{l|}{\begin{tabular}[c]{@{}l@{}}t̓uc-R.!iiḥ\\ sea.urchin-go.after.food\end{tabular}}         & \multicolumn{1}{l|}{\begin{tabular}[c]{@{}l@{}}t̓ut̓uc̓iiḥ\\ getting sea urchins\end{tabular}}   \\ \hline
\multicolumn{1}{|l|}{LR}  & \multicolumn{1}{l|}{\begin{tabular}[c]{@{}l@{}}kuḥʷ-LR.inqił\\ hole-at.ribs\end{tabular}}                   & \multicolumn{1}{l|}{\begin{tabular}[c]{@{}l@{}}kuukuḥinqił\\ hole at the ribs\end{tabular}}      \\ \hline
\multicolumn{1}{|l|}{LRS} & \multicolumn{1}{l|}{\begin{tabular}[c]{@{}l@{}}qʷi-LRS.ity̓ak\\ what-fear\end{tabular}}                          & \multicolumn{1}{l|}{\begin{tabular}[c]{@{}l@{}}qʷiiqʷity̓ak\\ whatever one fears\end{tabular}}    \\ \hline
\multicolumn{1}{|l|}{RL} & \multicolumn{1}{l|}{\begin{tabular}[c]{@{}l@{}}ʔu-RL.čiy̓ał\\ \textsc{x}-pursue\end{tabular}}                   & \multicolumn{1}{l|}{\begin{tabular}[c]{@{}l@{}}ʔuʔuukʷiy̓ał\\ pursue it\end{tabular}}       \\ \hline
\multicolumn{1}{|l|}{R2}   & \multicolumn{1}{l|}{\begin{tabular}[c]{@{}l@{}}R2-nuuk\\ \textsc{pl}-song\end{tabular}}               & \multicolumn{1}{l|}{\begin{tabular}[c]{@{}l@{}}nuuknuuk\\ songs\end{tabular}}      \\ \hline
\multicolumn{1}{|l|}{LR2L}   & \multicolumn{1}{l|}{\begin{tabular}[c]{@{}l@{}}t̓apat-LR2L.a\\ think-\textsc{rp}\end{tabular}}               & \multicolumn{1}{l|}{\begin{tabular}[c]{@{}l@{}}t̓aat̓aapata\\ consider\end{tabular}}      \\ \hline
\multicolumn{1}{|l|}{LR2L}   & \multicolumn{1}{l|}{\begin{tabular}[c]{@{}l@{}}huuł-LR2L.a\\ dance-\textsc{rp}\end{tabular}}               & \multicolumn{1}{l|}{\begin{tabular}[c]{@{}l@{}}huułhuuła\\ dance\end{tabular}}      \\ \hline
\end{tabular}
\end{table}

\newpage

\section{Grams} \label{appendix:grams}

\subsection{Aspect}

In my glosses, I use the older and more traditional categorization of aspect, although there is a reanalysis of the system that I accept (\S\ref{ch:clause:aspect}). The table below is adapted from a system I helped Adam Werle devise. I use his grams for the conservative names of the aspects. There is a straightforward collapse from the conservative aspect system to the (hypothesized) revised aspect system, which I include in the table. By using the most conservative glossing I avoid losing information. Although not properly aspect, I include in the table the resultative morpheme, which, when used, occurs in lieu of aspect morphemes.

\begin{table}[ht]
\centering
\caption{Aspects and resultative}
\label{table:aspects}
\begin{tabular}{llll}
revised analysis & conservative analysis & gram                    & forms                                                         \\ \hline
\multicolumn{1}{|l|}{\multirow{2}{*}{perfective}}   & \multicolumn{1}{l|}{momentaneous} & \multicolumn{1}{l|}{\textsc{mo}} & \multicolumn{1}{l|}{-čiƛ, -šiƛ, -kʷiƛ, -uƛ}                 \\ \cline{2-4} 
\multicolumn{1}{|l|}{}                            & \multicolumn{1}{l|}{inceptive}    & \multicolumn{1}{l|}{\textsc{in}} & \multicolumn{1}{l|}{-°ačiƛ, -iˑčiƛ}                         \\ \hline
\multicolumn{1}{|l|}{durative}                    & \multicolumn{1}{l|}{durative}     & \multicolumn{1}{l|}{\textsc{dr}} & \multicolumn{1}{l|}{-(ʔ)ak, -(ʔ)uk, -L.ḥiˑ}  \\ \hline
\multicolumn{1}{|l|}{continuative}                & \multicolumn{1}{l|}{continuative} & \multicolumn{1}{l|}{\textsc{cv}} & \multicolumn{1}{l|}{-(y)aˑ}                   \\ \hline
\multicolumn{1}{|l|}{graduative}                     & \multicolumn{1}{l|}{graduative}   & \multicolumn{1}{l|}{\textsc{gr}} & \multicolumn{1}{l|}{-LS}                        \\ \hline
\multicolumn{1}{|l|}{repetitive}                  & \multicolumn{1}{l|}{repetitive}   & \multicolumn{1}{l|}{\textsc{rp}} & \multicolumn{1}{l|}{-LR2L.a}                   \\ \hline
\multicolumn{1}{|l|}{iterative}                  & \multicolumn{1}{l|}{iterative}    & \multicolumn{1}{l|}{\textsc{it}} & \multicolumn{1}{l|}{-R2.č, -R2.š} \\ \hline \hline
\multicolumn{1}{|l|}{resultative}                 & \multicolumn{1}{l|}{resultative}  & \multicolumn{1}{l|}{\textsc{rs}} & \multicolumn{1}{l|}{-yuˑ, -čuˑ}              \\ \hline
\end{tabular}
\end{table}

As discussed in \S\ref{ch:clause:aspect}, these aspects can be divided into perfective and imperfective categories. Verbs ending in momentaneous or inceptive aspect are perfective, while the rest are imperfective. The durative and continuative aspect are weakly differentiated, and plausibly there is a supertype, continuous, that subsumes both. In Werle's notation, this is \textsc{ct}. To avoid confusion between ``continuous" and ``continuative," I have assigned every morpheme either continuative or durative aspect, and avoided the underspecified continuous.


\subsection{Mood} \label{sec:grams:mood}

The category traditionally called ``Mood" in Nuuchahnulth is not the same as ``mood" as usually used by linguists, which stands in opposition to aspect and tense. Instead, Nuuchahnulth ``mood" is a morphological category that fuses mood and evidential information with subject person and number, as well as containing other propositional information such as interrogative and imperative marking \citep{jacobsen1986}. The fullest accounting of the semantics and syntax of these particles is given in \citet[Chapter 4]{waldie2012}, and this section is merely a small sketch. These moods can be split into matrix clause moods, dependent clause moods, and commands. Commands are special matrix clause moods that contain object agreement, while all other moods only contain subject agreement. \cref{table:moods} gives a list of the moods, their abbreviations, and their third person forms. I use a mix of the ``practical names" and ``technical names" given in \citet{werle2015} for the mood complex, selecting the name that most closely correlates with the gram. For commands, I list second person singular forms without an object, or a third person object, as third person agreement is null.

The meanings of these moods are mostly but not entirely consistent across dialects. The strong mood and real mood have the same meaning: a strong claim to reality, with the real mood used in the Barkley Sound dialect and the strong mood used in the Central and Northern dialects. The strong mood is in free variation in the Kyuquot-Checleseht dialect with the weak mood, which has come to be used as a matrix clause mood.

Typically a clause can have only one mood ending, although there are some exceptions: the possible mood in the third person can be followed by the hearsay, yielding a matrix mood meaning something like `what is typically done, so I hear', and the hearsay mood can be followed by the dubitative.

\newpage

%\begin{table}[ht]
%\centering
%\caption{Moods}
\begin{table}[ht]
\centering
\caption{Mood enclitics}
\label{table:moods}
\begin{tabular}{lll}
name                                       & gram                         & third person                           \\ \hline
\multicolumn{3}{|c|}{Matrix Moods}       \\ \hline
\multicolumn{1}{|l|}{real}               & \multicolumn{1}{l|}{\textsc{real}} & \multicolumn{1}{l|}{\textit{=maˑ}}             \\ \hline
\multicolumn{1}{|l|}{strong}                 & \multicolumn{1}{l|}{\textsc{strg}} & \multicolumn{1}{l|}{\textit{=ʔiˑš}}            \\ \hline
\multicolumn{1}{|l|}{neutral}              & \multicolumn{1}{l|}{\textsc{neut}} & \multicolumn{1}{l|}{=$\emptyset$\footnotemark}       \\ \hline
\multicolumn{1}{|l|}{question}             & \multicolumn{1}{l|}{\textsc{ques}} & \multicolumn{1}{l|}{\textit{=ḥaˑ}}             \\ \hline
\multicolumn{1}{|l|}{hearsay}              & \multicolumn{1}{l|}{\textsc{hrsy}} & \multicolumn{1}{l|}{\textit{=weˑʔin}, \textit{=waˑʔiš}} \\ \hline
\multicolumn{1}{|l|}{inferential}        & \multicolumn{1}{l|}{\textsc{infr}} & \multicolumn{1}{l|}{\textit{=č̓aˑʕaš}}        \\ \hline
\multicolumn{1}{|l|}{dubitative}           & \multicolumn{1}{l|}{\textsc{dubt}} & \multicolumn{1}{l|}{\textit{=qaˑč̓a}}          \\ \hline
\multicolumn{3}{|c|}{Dependent Moods} \\ \hline
\multicolumn{1}{|l|}{weak}                 & \multicolumn{1}{l|}{\textsc{weak}} & \multicolumn{1}{l|}{\textit{=(y)ii}}            \\ \hline
\multicolumn{1}{|l|}{definite}             & \multicolumn{1}{l|}{\textsc{defn}} & \multicolumn{1}{l|}{\textit{=ʔiˑtq}}            \\ \hline
\multicolumn{1}{|l|}{possible}             & \multicolumn{1}{l|}{\textsc{pssb}} & \multicolumn{1}{l|}{\textit{=quu}}              \\ \hline
\multicolumn{1}{|l|}{dubitative formative} & \multicolumn{1}{l|}{\textsc{unk1}} & \multicolumn{1}{l|}{\textit{=(w)uus}}           \\ \hline
\multicolumn{1}{|l|}{dubitative relative}  & \multicolumn{1}{l|}{\textsc{unk2}} & \multicolumn{1}{l|}{\textit{=(w)uusi}}          \\ \hline
\multicolumn{1}{|l|}{embedded}             & \multicolumn{1}{l|}{\textsc{embd}} & \multicolumn{1}{l|}{\textit{=qaˑ}}              \\ \hline
\multicolumn{1}{|l|}{purposive}            & \multicolumn{1}{l|}{\textsc{purp}} & \multicolumn{1}{l|}{\textit{=!eeʔita}, \textit{=!aaḥi}}  \\ \hline
\multicolumn{1}{|l|}{article\footnotemark}              & \multicolumn{1}{l|}{\textsc{artl}} & \multicolumn{1}{l|}{\textit{=ʔiˑ}}              \\ \hline
\multicolumn{1}{|l|}{hearsay article}      & \multicolumn{1}{l|}{\textsc{arth}} & \multicolumn{1}{l|}{\textit{=č̓a}}             \\ \hline
\multicolumn{3}{|c|}{Command Moods} \\ \hline
\multicolumn{1}{|l|}{command}              & \multicolumn{1}{l|}{\textsc{cmmd}} & \multicolumn{1}{l|}{\textit{=!iˑ}}              \\ \hline
\multicolumn{1}{|l|}{`go' command}         & \multicolumn{1}{l|}{\textsc{cmgo}} & \multicolumn{1}{l|}{\textit{=čiˑ}}             \\ \hline
\multicolumn{1}{|l|}{`come' command}       & \multicolumn{1}{l|}{\textsc{cmcm}} & \multicolumn{1}{l|}{\textit{=!iˑk}}             \\ \hline
\multicolumn{1}{|l|}{future command}       & \multicolumn{1}{l|}{\textsc{cmfu}} & \multicolumn{1}{l|}{\textit{=!im}}              \\ \hline
\end{tabular}
\end{table}

\addtocounter{footnote}{-1}
\footnotetext{Although the third person neutral is null-marked, the first and second person neutral mood forms are non-null. In the IGT, I do not actually gloss third-person neutral with a $\emptyset$, out of an aversion to inserting unpronounced items into an analysis, and due to the fact that my implemented grammar does not make use of null-marked elements in the gloss line.}
\addtocounter{footnote}{1}
\footnotetext{The article in Nuuchahnulth is also part of the mood complex, occupying the same morphological position and complementary with the other moods. More on this can be found in \citet{inman2018}.} 

\subsection{Other clausal morphemes} \label{sec:otherclause}

Other clausal morphemes that occur in the second-position enclitic complex (\S\ref{ch:clause:cliticnormal}) include tense (and some related notions) and valence-changing morphemes, given in \cref{table:tense}.

\begin{table}[ht]
\centering
\caption{Tense, valence-changing, and other clausal morphemes}
\label{table:tense}
\begin{tabular}{lll}
meaning                        & gram               & morph                        \\ \hline
\multicolumn{1}{|l|}{now}      & \multicolumn{1}{l|}{\textsc{now}}   & \multicolumn{1}{l|}{\textit{=!aƛ}}     \\ \hline
\multicolumn{1}{|l|}{future}   & \multicolumn{1}{l|}{\textsc{fut}}   & \multicolumn{1}{l|}{\textit{=ʔaaqƛ}, \textit{=!aaqƛ}}   \\ \hline
\multicolumn{1}{|l|}{past}     & \multicolumn{1}{l|}{\textsc{pst}}   & \multicolumn{1}{l|}{\textit{=mit}}     \\ \hline
\multicolumn{1}{|l|}{habitual} & \multicolumn{1}{l|}{\textsc{hab}} & \multicolumn{1}{l|}{\textit{=ʔaała}}    \\ \hline
\multicolumn{1}{|l|}{plural\footnotemark} & \multicolumn{1}{l|}{\textsc{pl}} & \multicolumn{1}{l|}{\textit{=ʔał}}    \\ \hline
\multicolumn{1}{|l|}{causative}  & \multicolumn{1}{l|}{\textsc{caus}} & \multicolumn{1}{l|}{\textit{=!ap}}      \\ \hline
\multicolumn{1}{|l|}{passive\footnotemark}    & \multicolumn{1}{l|}{\textsc{pass}} & \multicolumn{1}{l|}{\textit{=!at}}      \\ \hline
\multicolumn{1}{|l|}{possessive} & \multicolumn{1}{l|}{\textsc{poss}} & \multicolumn{1}{l|}{\textit{=ʔak}, \textit{=uk}} \\ \hline
\end{tabular}
\end{table}

\addtocounter{footnote}{-1}
\footnotetext{This plural is separate from the plural that occurs as part of the mood portmanteaus, and may refer to the plurality of the subject or object of the verb. It is the only way to express the plurality of a dropped third person subject.}
\addtocounter{footnote}{1}
\footnotetext{The passive morpheme is also used for inalienable possession and generic statements. I do not gloss it differently according to its use.}

The ``now" morpheme (\textsc{now}) should not be understood as a simple present, as it is often used in conjunction with the past and future tense, and can occur in a sentence that takes place at any time. It indicates that an event is occurring next in a sequence, and that the current clause is the next in some progression.

\subsection{Other predicative morphemes}

There are other elements that modify predicates in some way: the linker, and the root-maker or stem formative. The linker is described in detail in \cref{ch:link}. The stem formative \textit{-q} (\textsc{stem}) is used to create a bound root from a free word so that certain affixes can attach. Examples are \textit{saantiquwił} `church', from the word \textit{saantii} `Sunday' + -\textit{uwił} `indoor room'. There is also \textit{ḥimwic̓aqy̓ak} `myth', from \textit{ḥimwic̓a} `myth telling' + -\textit{y̓ak} `instrument, device for'. Although it is restricted to verbs (unlike the linker and the stem formative), I include reciprocal \textit{-(c)st̓ał} in this list.

\begin{table}[ht]
\centering
\caption{Predicate-bound morphemes}
\label{table:predicate}
\begin{tabular}{lll}
meaning                         & gram              & morph                       \\ \hline
\multicolumn{1}{|l|}{linker}    & \multicolumn{1}{l|}{\textsc{link}} & \multicolumn{1}{l|}{\textit{-(q)ḥ}} \\ \hline
\multicolumn{1}{|l|}{root-maker}    & \multicolumn{1}{l|}{\textsc{stem}} & \multicolumn{1}{l|}{\textit{-q}} \\ \hline
\multicolumn{1}{|l|}{reflexive} & \multicolumn{1}{l|}{\textsc{recp}} & \multicolumn{1}{l|}{\textit{-(c)st̓ał}} \\ \hline
\end{tabular}
\end{table}

\subsection{Augmentative and diminutive}

Nuuchahnulth has two augmentatives and at least two diminutives. The \textit{-SS.(q)aq} augmentative straightforwardly means `big' while the \textit{-LS.sa} augmentative has a broader augmentative meaning, including `real', `true', and `very'. I give the more common \textit{-sa} the \textsc{aug1} label. The diminutives have no appreciable difference in meaning, so I gloss both as \textsc{dim}.

\begin{table}[ht]
\centering
\caption{Augmentative and diminutive}
\label{table:augdim}
\begin{tabular}{lll}
meaning                            & gram               & morph                                         \\ \hline
\multicolumn{1}{|l|}{augmentative (``real")} & \multicolumn{1}{l|}{\textsc{aug1}}   & \multicolumn{1}{l|}{\textit{-LS.sa}}                   \\ \hline
\multicolumn{1}{|l|}{augmentative (``big")} & \multicolumn{1}{l|}{\textsc{aug2}}   & \multicolumn{1}{l|}{\textit{-SS.(q)aq}}                   \\ \hline
\multicolumn{1}{|l|}{diminutive}   & \multicolumn{1}{l|}{\textsc{dim}} & \multicolumn{1}{l|}{\textit{$\langle$čk$\rangle$}, \textit{-ʔis}} \\ \hline
\end{tabular}
\end{table}

\subsection{Semantically empty roots} \label{sec:empty}

Many suffixes in Nuuchahnulth contain complex semantic content, and often attach to semantically light or meaningless roots. Two semantically contentless roots are used in such cases: \textit{hita-}/\textit{hina-} and \textit{ʔu-}.

The root \textit{ʔu-} is used in place of an object for transitive suffixes to attach to. Many transitive verbs in Nuuchahnulth are suffixes that may attach to the first syntactic word of their direct object (\S\ref{ch:clause:2pv:mainpredicate}). In lieu of attaching to their object, these suffixes may attach to semantically empty \textit{ʔu-} instead. After attaching to \textit{ʔu-}, the direct object of the verb may be expressed as a separate word, or dropped altogether. Because of its nature as a ``placeholder" for a syntactic object, I use \textsc{x} as the gloss for this morpheme.

\ex \label{havename1}
\begingl
\glpreamble \textit{ʕumtnaak}//
\gla ʕumt-naˑk //
\glb name-have //
\glft `having a name' //
\endgl
\xe

\ex~ \label{havename2}
\begingl
\glpreamble \textit{ʔunaak ʕumt-ii}//
\gla ʔu-naˑk ʕumt-iˑ //
\glb \textsc{x}-have name-\textsc{nmlz} //
\glft `having a name' //
\endgl
\xe

The roots \textit{hita-}/\textit{hina-}\footnote{There appears to be no way to predict whether \textit{hita-} or \textit{hina-} is used for a particular word, although there is a clear phonological resemblance.} are more unpredictable in their distribution. They tend to be a place of attachment for location suffixes (\S\ref{ch:clause:2pv:loc}), although occasionally other suffixes can attach to them.\footnote{An example is \textit{hiniic} `carry', \textit{hina-iic}.} To distinguish these grams from \textit{ʔu-} \textsc{x}, I gloss this root as \textsc{empty}, as can be seen in (\ref{ex:entervessel}, \ref{ex:outofcanoe}).

\ex \label{ex:entervessel}
\begingl
\glpreamble \textit{hitaqsiƛ}//
\gla hita-qsiƛ //
\glb \textsc{empty}-in.a.vessel.\textsc{mo} //
\glft `enter into a vessel' //
\endgl
\xe

\ex~ \label{ex:outofcanoe}
\begingl
\glpreamble \textit{hinułta}//
\gla hina-ułta //
\glb \textsc{empty}-out.of.canoe //
\glft `get out of the canoe' //
\endgl
\xe

\begin{table}[ht]
\centering
\caption{Semantically empty roots}
\label{table:augdim}
\begin{tabular}{lll}
meaning                            & gram               & morph                                         \\ \hline
\multicolumn{1}{|l|}{---} & \multicolumn{1}{l|}{\textsc{empty}}   & \multicolumn{1}{l|}{\textit{hita}, \textit{hina}}                   \\ \hline
\multicolumn{1}{|l|}{---} & \multicolumn{1}{l|}{\textsc{x}}   & \multicolumn{1}{l|}{\textit{ʔu}}                   \\ \hline
\end{tabular}
\end{table}

\subsection{Deictics}

Nuuchahnulth dialects each have a set of demonstrative deictics. In the Central, Northern, and Kyuquot-Checleseht dialects there are six: four locative deictics and two non-locative deictics. The Barkley dialect only has one non-locative deictic: the topical deictic, and so has five altogether. For the locative deictics I use a numbering scheme 1-4, with 1 being the closest and 4 the furthest away. For the shared topical deictic I use \textsc{dtop}, and \textsc{ddyn} for the topical `this.' This distinction among deictics originates from \cite{werle2015}. I use the Central deictics to demonstrate the glossing schema below.

\begin{table}[ht]
\centering
\caption{Deictics, Central dialect}
\label{table:deictics}
\begin{tabular}{lll}
meaning                            & gram               & morph                                         \\ \hline
\multicolumn{1}{|l|}{this} & \multicolumn{1}{l|}{\textsc{d1}}   & \multicolumn{1}{l|}{\textit{ʔaḥkuu}}                   \\ \hline
\multicolumn{1}{|l|}{that by you} & \multicolumn{1}{l|}{\textsc{d2}}   & \multicolumn{1}{l|}{\textit{ʔaḥn̓ii}}                   \\ \hline
\multicolumn{1}{|l|}{that} & \multicolumn{1}{l|}{\textsc{d3}}   & \multicolumn{1}{l|}{\textit{ḥaay̓aḥi}}                   \\ \hline
\multicolumn{1}{|l|}{that (far)} & \multicolumn{1}{l|}{\textsc{d4}}   & \multicolumn{1}{l|}{\textit{ḥuuy̓aḥi}}                   \\ \hline
\multicolumn{1}{|l|}{this (dynamic)} & \multicolumn{1}{l|}{\textsc{ddyn}}   & \multicolumn{1}{l|}{\textit{ḥiy̓aḥi}}                   \\ \hline
\multicolumn{1}{|l|}{that (topical)} & \multicolumn{1}{l|}{\textsc{dtop}}   & \multicolumn{1}{l|}{\textit{ʔaḥʔaa}}                   \\ \hline
\end{tabular}
\end{table}