\chapter{Methodology} \label{ch:methodology}

This work has proceeded along two tracks. The first has been gathering primary data through field work as well as using published corpora in the language to uncover grammatical facts. The second is the implementation of the analysis of these grammatical facts through a computational syntactic framework. I will address my methods for each part of this separately.

\section{Gathering data in Nuuchahnulth} \label{ch:method:ncn}

Before I began my project on serial verbs and the linker, I first learned enough Nuuchahnulth to become at least conversant in the language. I did this by reading the published literature (especially \citealt{sapir1939}), attending language learning classes in Port Alberni (many of them with my colleague, Amie DeJong), and direct study with Adam Werle, some of which was funded through summer Foreign Language Acquisition Scholarships (FLAS). The language lessons I participated in were taught by Adam Werle and often included elders and native speakers who would assist, correct, and aid in teaching. It was through this venue that I first met fluent Nuuchahnulth elders.

In the summer of 2016, Adam and I traveled to Hot Springs Cove and collected texts from some Hesquiaht elders. On request, that data is not presented in this dissertation, but some of that work has informed my analysis, which I have confirmed with other speakers.

\section{Data sources} \label{ch:method:sources}

I began learning and working with Nuuchahnulth at the start of 2015. Before I collected my own data, I looked at data from a variety of sources to generate appropriate questions. My sources were previous syntactic work on the language, especially \cite{jacobsen1993}, \cite{nakayama2001}, \cite{wojdak2003}, \cite{waldie2004}, and \cite{woo2007b}. I also relied on corpora published by linguists, especially the Nootka Texts \citep{sapir1924, sapir1939, sapir1955, whalingindians2000, whalingindians2004, whalingindians2009}. Matthew Davidson has digitized two of these volumes \citep{sapir1939,sapir1955} and has provided me access to it. Without this work, searching through the texts for grammatical constructions would have been much harder.

In addition to these resources, I looked at community-produced texts such as ``Son of Thunderbird" and texts I received from linguists Adam Werle and Henry Kammler. The largest of these was an in-progress Bible translation Adam Werle and Sophie Billy were working on and several recordings Henry Kammler made with the late Barbara Touchie. I looked through these sources for examples of the phenomena I was looking for, annotated and cataloged them, and used some of these examples as prompts for speakers.

\section{Elicitation methods} \label{ch:method:elicitation}

I spent January, February, and part of March of 2018 in Port Alberni working with native speakers and gathering data specifically for this dissertation. In that period of time I worked with Julia Lucas (Nuuchahnulth name \textit{tupaat}, Hesquiaht by marriage but speaks Ahousaht, which is central dialect), Bob Mundy (Uclueleht tribe, Barkley Sound dialect), Marjorie Touchie (Uclueleht tribe, Barkley Sound dialect), Fidelia Haiyupis (Ehattesaht tribe, northern dialect), and Sophie Billy (Checkleseht tribe, Kyuquot-Checkleseht dialect). I also present data I gathered earlier from Simon Lucas (Nuuchahnulth name \textit{yuułnaak}, Heshquiaht tribe, northern dialect), the late husband of Julia Lucas. I later spent much of March and April of 2019 in Uclueleht, working with the same speakers, and some of this work was funded by and the First Nations Education Foundation, who received copies of transcriptions and translations.

I have made an effort to make my work, especially my recordings and transcriptions, available to the communities I have worked with. Some of my work with Fidelia Haiyupis and Sophie Billy was funded by the Ehattesaht tribe, which has received copies of my notes and recordings. The Uclueleht Nation has received the notes and recordings I made with Bob Mundy and Marjorie Touchie. I have also made recordings and transcriptions available online to language learners. Some of this information is restricted to people who have the right password to access the folder. I take precautions not to collect data that is sensitive to audience restrictions, and so for most of these materials, password-restricted access is not done out of a concern with rights management, but with the fact that many of these materials are works in progress and I do not want possibly-inaccurate transcriptions to be disseminated widely among people who are lower-level language learners.

When working with speakers, I tended to work two to four hours at a time and tried to structure sections in three parts: grammatical questions and elicitations, vocabulary questions and clarification questions on existing texts, and text elicitation. The purpose of this was to avoid wearing speakers out with too many grammatical questions in a row, and to collect other important data. While there has been good primary linguistic documentation in Nuuchahnulth, particularly in \cite{sapir1939} and \cite{rose1981}, there are many differences across the language's wide spread of dialects that remain undocumented and unknown. Although one of my speakers did not like giving lengthy texts, I was able to collect connected, fluent texts from all other speakers, which is a lasting artifact and can be used to answer questions beyond the scope of my dissertation. In total, I have about seven hours of recordings from my field work, about three hours of which are texts that are continuous or mostly continuous Nuuchahnulth.


\section{Methods of Elicitation}

I used eight methods of elicitation, the aim of which was to obtain the most natural Nuuchahnulth examples or grammatical judgments relevant to the phenomenon under investigation. Some methods worked better than others, although none of them worked all of the time. Anecdotally, I found that staying in Nuuchahnulth for longer periods of time helped more than anything else, although this was quite difficult. Only one of my consultants was literate in the language, and while she would correct my pronunciation sometimes by writing out a word, she preferred to work in an oral environment and have me read my notes back to her. These elicitation sessions then occurred in either a completely or nearly-completely oral context. All Nuuchahnulth speakers I worked with were bilingual in English.

\subsection{Describing Images}

The aim of this methodology is to avoid the metalanguage (English) through the visual medium. The speaker is presented with a series of images and asked to describe what is going on using only Nuuchahnulth. One set I used was a series of photos I took of dogs at a reserve. The dogs are standing at a pier. They begin barking at the water. A boat approaches the pier. The dogs go up to meet the man in the boat, who pets them. The purpose of this was to elicit a few serial verb constructions, the equivalent of ``The dogs are at the wharf" (locations are verbs in Nuuchahnulth), and ``The man pilots the boat to the dock" (which would require two verbs). In addition to photo series, I also used hand-drawn pictures on index cards, and existing picture-story books.

I found this method occasionally fruitful but limiting. Sometimes (especially with my hand-drawn cards), speakers would spend a lot of time questioning what the picture was meant to represent. Even with photos, they wanted to know what to focus on: Who is the man in the photo, and who is he related to? While broad grammatical structures could be gathered this way, other methods were more fruitful for eliciting targeted phenomena.

\subsection{Answering Questions}

Another way of getting natural speech is by asking questions to elicit the phenomenon. In this method, I would tell a short story and ask a question about what happened. I hoped to elicit a response that used the grammatical phenomenon I was investigating.

For instance, one of my setups was the following (spoken in Nuuchahnulth): ``I saw two creatures come out of the forest. One was a dog, one was a wolf. The dog approached me. The other went back into the forest. He ran. It was the dog that approached me. What did the wolf do?"

The expected answer is ``The wolf ran into the forest," which requires coordinating the two verbs `run' and `into the forest.' I had very low success rates with this kind of elicitation and quickly abandoned it. Speakers would select the most semantically salient verb, in this case `into the forest,' and drop the other verb in the construction. For instance, one response I got to this prompt was (\ref{ex:wolfintoforest}).

\ex \label{ex:wolfintoforest}
\begingl
\glpreamble hitaaqƛ̓iʔaƛ qʷayac̓iik. //
\gla hitaaqƛ̓iƛ=!aƛ qʷayac̓iik //
\glb in.forest.\textsc{mo}=\textsc{now} wolf //
\glft `The wolf went into the forest.' (\textbf{N}, Fidelia Haiyupis) //
\endgl
\xe

I had similar issues with other question-answering. Speakers preferred to answer as succinctly as possible, which was not useful for the phenomena I was investigating. There may be a more fruitful way of using this kind of elicitation method, but I was unable to find it.

\subsection{Recording Texts}

My fieldwork also involved recording fluent texts from Nuuchahnulth speakers. This work is a valuable endeavor in itself, but it also allows speakers to give examples of these phenomena in a fluent context. Both linker and serial verb constructions occur naturally in running texts, and in the relevant chapters I will give counts of grammatical phenomena in texts I collected as well as some historical texts.

\subsection{Rephrasing Stories}

The typical person is interested in language as a means of communication and not a set of abstract grammatical rules. Rephrasing traditional stories or short narratives is one way of trying to get natural versions of grammatical phenomena, especially if the original requires them. I tried three forms of retelling: (1) asking a speaker to summarize in a few sentences a text I had previously gotten from them (in Nuuchahnulth); (2) asking a speaker to summarize my own story (in Nuuchahnulth and English); (3) asking a speaker to retell a traditional story they know.

I did not have good results with (3), but I did better with (1) and (2). Not every consultant I worked with had the patience to resummarize their own text, but those that did could be persuaded to give a few-sentence quick summary. For retelling my own stories, I told stories both in Nuuchahnulth and English. With Nuuchahnulth stories, speakers quickly got frustrated (``But you already said it"), so I got furthest by giving succinct stories in English and asked for a retelling in Nuuchahnulth. For example, ``I like to walk in the forest in the mornings. There are lots of blue jays in the forest. They must like me, because they follow me around the forest." The first sentence has the opportunity for three verbal expressions in a sentence: location, action, and time. The final sentence also has the possibility for a serial verb construction: a location and an action.

\subsection{Forced Choice}

Forced choice gives the speaker a few examples to choose from when trying to select the best way to describe something. This strategy was a mixed bag. Very frequently speakers would reject both, or offer a third way of saying it and refuse to choose between those presented. On the other hand, giving speakers a limited set of choices sometimes led them to describe what made one or all sentences bad, for example, ``It could mean something else." Speakers would more frequently volunteer reasons why a sentence is bad if there were other options present, or if all options were bad.

One case where I used this was a situation where I am being witnessed doing something by another party. I've left home to go after whales, and my wife sees me hunting while she is standing on the shore.

\ex \label{ex:goingafterwhale1}
\begingl
\glpreamble ? n̓aacsaʔataḥ łuucmaakqas ʔuʔuʔiiḥ ʔiiḥtuup. //
\gla n̓aacsa=!at=(m)aˑḥ łuucma=ʔak=qaˑs ʔu-R.ʔiiḥ ʔiiḥtuup //
\glb see.\textsc{dr}=\textsc{pass}=\textsc{real.1sg} woman=\textsc{poss}=\textsc{defn.1sg} \textsc{x}-go.after whale //
\glft ? `My wife saw me going after a whale.' //
\endgl
\xe

\ex~ \label{ex:goingafterwhale2}
\begingl
\glpreamble ? n̓aacsaʔataḥ łuucmaakqas ʔuʔuʔiiḥʔat ʔiiḥtuup. //
\gla n̓aacsa=!at=(m)aˑḥ łuucma=ʔak=qaˑs ʔu-R.ʔiiḥ=!at ʔiiḥtuup //
\glb see.\textsc{dr}=\textsc{pass}=\textsc{real.1sg} woman=\textsc{poss}=\textsc{defn.1sg} \textsc{x}-go.after=\textsc{pass} whale //
\glft ? `My wife saw me going after a whale.' //
\endgl
\xe


\begin{comment}
\ex \label{becauseofbaby1}
\begingl
\glpreamble ? ʔuusaaḥimta nay̓aqakʔi wikitaḥ ƛuł weʔič. //
\gla ʔuusaaḥi=(m)it=maˑ nay̓aqak=ʔiˑ wik=(m)it=(m)aˑḥ ƛuł weʔič //
\glb because.of=\textsc{pst}=\textsc{real.3} baby=\textsc{art} \textsc{neg}=\textsc{pst}=\textsc{real.1sg} good sleep //
\glft ? `I didn't sleep well because of the baby.' //
\endgl
\xe

\ex~ \label{becauseofbaby2}
\begingl
\glpreamble ? ʔuusaaḥiqḥita nay̓aqakʔi wikitaḥ ƛuł weʔič. //
\gla ʔuusaaḥi-(q)ḥ=(m)it=maˑ nay̓aqak=ʔiˑ wik=(m)it=(m)aˑḥ ƛuł weʔič //
\glb because.of-\textsc{link}=\textsc{pst}=\textsc{real.3} baby=\textsc{art} \textsc{neg}=\textsc{pst}=\textsc{real.1sg} good sleep //
\glft ? `I didn't sleep well because of the baby.' //
\endgl
\xe
\end{comment}

In this case, my consultant Bob Mundy strongly rejected (\ref{ex:goingafterwhale2}). This gave me evidence about the limits of ``clitic spreading" (see discussion at the end of \S\ref{ch:clause:cliticnormal}). Enclitics that sometimes copy across an utterance (like passive \textit{=!at} occasionally does) cannot spread into an embedded clause.

I also attempted to put together a list of standard sentences testing for ordering preferences. I presented the below to Checkleseht speaker Sophie Billy.

\ex \label{ex:workatmaatmaas1}
\begingl
\glpreamble mamuukw̓it̓sin hił maatmaas. //
\gla mamuuk-w̓it̓s=(y)in hił maatmaas //
\glb work-going.to=\textsc{weak.1pl} be.at house.\textsc{pl} //
\glft ? `We will work at Mahtmahs.' //
\endgl
\xe

\ex~ \label{ex:workatmaatmaas2}
\begingl
\glpreamble hiłw̓it̓sin maatmaas mamuuk. //
\gla hił-w̓it̓s=(y)in maatmaas mamuuk //
\glb be.at-going.to=\textsc{weak.1pl} work house.\textsc{pl} //
\glft ? `We will work at Mahtmahs.' //
\endgl
\xe

Sophie Billy preferred (\ref{ex:workatmaatmaas2}). I believe both utterances are grammatical, but there is an overall preference to express locations first. I will go into more depth about this preference in \S\ref{ch:sv}.

\subsection{Translation}

I also used translation from English, which I consider a less preferable form of elicitation due to the possibility that the speaker will adopt English-like syntactic structures instead of Nuuchahnulth-like structures. However, some speakers were most comfortable with this kind of elicitation task, and it is easier to do. With one speaker, we worked slowly over a couple of sessions through an abridged translation of The Little Prince.

There were other, shorter versions of this kind of elicitation. For instance, ``We are going to go camping. I want the children to help their mother. I want them to pack. I want them to carry the luggage. What should I tell them?" The purpose of this was to get a command form, which is always marked with overt second position inflection, with a serialized verb construction where the verbs must necessarily share the command mood. The construction would minimally have two sequential verbs and perhaps the benefactive verb to express ``for your mother."

\subsection{Grammatical judgments}

It is a cultural Nuuchahnulth value not to overtly correct people, and especially not to do so in public. While this is perhaps a good cultural practice for creating a healthy community, it is bad for linguists trying to learn what is and is not grammatical in a language.

Straight grammatical judgments---is this utterance a part of the language or not---are necessary in linguistic descriptions. These are also necessarily linguist-constructed, so I would put together a sentence, and ask about it. In my first few attempts, speakers would typically respond with ``I suppose you could say it that way," or ``I understand you." Even asking ``Would you say that?" or ``Am I saying it correctly?" speakers were typically hesitant to offer a correction unless the sentence was completely unintelligible.

The way I attempted to get around this was by asking speakers if what I said sounded like their dialect, like something they might say, and if they could repeat it. If a speaker consistently would rephrase the utterance when repeating it, I took it to mean that my version was likely ungrammatical. If I gave an example sentence out of the blue, I tried to provide context. I often have the best success with getting clear judgments by rephrasing speaker utterances when we were working on something else. I would add, remove, or move an element, and sometimes change the setup. Speakers were much more willing to give a firm yes or no in this context.

\subsection{Constructing a sentence}

There were many instances where I would ask speakers, ``Can you think of a case where you would use this word?" I constructed this method on the fly, as speakers would reject examples I thought were grammatical, or I could not come up with a context that would elicit the construction I was looking for. In most of these cases, I was trying to get an example of a word with a linker morpheme attached (\S\ref{ch:link}).

%My methods of elicitation were: linguist-constructed sentences with preceding context verified or corrected by my consultant, summarization of short stories, finishing an incomplete narrative, requests to rephrase previous speaker utterances, direct translation from English, and description of picture stories. Elicitation was done in Nuuchahnulth when possible, but was often done in English as well. Many of these elicitation methods focused on single actors performing multiple salient actions at once. An example is: ``A wolf and a dog came from the forest. One approached me and one ran back. The dog approached me. What did the wolf do?" The expression `into the forest' in Nuuchahnulth is the verb \textit{hitaaqƛ̓aʔiƛ} and the word for `run' is \textit{kamitquk}. The purpose was to get both in a serial verb construction together. Another example is, ``We are going to go camping. I want the children to help their mother. I want them to pack. I want them to carry the luggage. What should I tell them?" The purpose of this was to get a command form, which is always marked with second position inflection, with a serialized verb construction where the verbs must necessarily share the command mood. The construction would minimally have two (at least pragmatically temporally) sequential verbs and perhaps the benefactive verb to express ``for your mother." Linker constructions were much more difficult to elicit in a roundabout manner like this, and so I often ended up asking directly whether I could attach it to a word, and attempted rephrases of speaker sentences with an added linker morpheme. Sometimes these took the form of ``Is there a way to say this with [word with a linker attached]?"

%I also attempted to elicit stories that would have a greater likelihood of illustrating serial verb or linker phenomenon, which meant settings of simultaneous action. This was not as successful as I had hoped, and I found instead that the best way to gather example constructions was through elicitation methods and gathering as much fluent text as possible. 

\section{Data Collation} \label{ch:method:collation}

I collated the examples of the grammatical phenomena I was interested in. These came from a set of stories I had previously interlinearized, from a subset of Nootka Texts stories I was familiar with, from my elicitation sessions with consultants, and from my transcriptions of elicited texts. I entered these examples into a spreadsheet that was tagged with the phenomenon that the example illustrated, and used this to help me find patterns in the grammatical data. To port this data to a test suite that the implemented grammar can run on, I simply ran a script that would generate a format readable by the implemented grammar.

\section{Implementation through the DELPH-IN framework} \label{ch:method:delphin}

My grammatical analysis has been through the DELPH-IN\footnote{\url{http://www.delph-in.net}} framework, which is a computationally-implemented formalism of Head-driven Phrase Structure Grammar (HPSG, \citealt{pollardsag1994}) using Minimal Recursion Semantics (MRS, \citealt{copestake2005}). My implementation is built on a base that uses the Grammar Matrix \citep{bender2002, flickinger2003, benderetal2010}.

My first step in the grammar development was to answer a questionnaire on the Grammar Matrix webpage, which generates a baseline grammar in the form of text files in the type description language (TDL). TDL is a series of declarative statements that describe grammatical rules, and the Grammar Matrix is a database of common grammatical rules across the world's languages. For instance, below I replicate a part of the TDL that describes the basic form of a head-complement rule.

\begin{verbatim}
basic-head-comp-phrase := head-valence-phrase & head-compositional &
              binary-headed-phrase &
  [ SYNSEM phr-synsem-min &
           [ LOCAL.CAT [ VAL [ SUBJ #subj,
                               SPEC #spec,
                               SPR #spr ],
                         POSTHEAD #ph,
                         HC-LIGHT #light ],
             LIGHT #light ],
    HEAD-DTR.SYNSEM [ LOCAL.CAT [ VAL [ SUBJ #subj,
                                        SPEC #spec,
                                        SPR #spr ],
                                  HC-LIGHT #light,
                                  POSTHEAD #ph ]],
    NON-HEAD-DTR.SYNSEM canonical-synsem ].
\end{verbatim}

This rule first states that head-complement rules inherit all the constraints of \textit{head-valence-phrase}, \textit{head-compositional}, and \textit{binary-headed-phrase}. I will gloss over what is present in these rules. Then this rule adds to the constraints of the rules it inherits from, stating that, minus the \textsc{comps} list (where complements are stored), the mother node inherits the valence and \textsc{cat}(egory) values of its head-daughter. The non-head-daughter is specified only to be some kind of syntactic-semantic item. A further rule, \textit{basic-head-1st-comp-phrase}, inherits from \textit{basic-head-comp-phrase} and specifies what happens to the head-daughter's complements.

\begin{verbatim}
basic-head-1st-comp-phrase := basic-head-comp-phrase &
  [ SYNSEM.LOCAL.CAT.VAL.COMPS #comps,
    HEAD-DTR.SYNSEM.LOCAL.CAT.VAL.COMPS < #synsem . #comps >,
    NON-HEAD-DTR.SYNSEM #synsem ].
\end{verbatim}

This code states that the non-head-daughter is identified with whatever the first thing is on the head-daughter's complements list, and the mother node's complements list is reduced by one. In the case where the head-daughter only has a complements list with one item on it, the value \texttt{\#comps} above will be a null element, and the mother node will have an empty comps list. This means that the parent node is no longer looking for any complements. This process is called cancellation, and is how HPSG keeps track of the saturation of a verb's argument structure.

All of the above rule specifications are from the Grammar Matrix, and part of the provided analyses when the system generates an output grammar based on a user's answer to questions. So far, \textit{\justify basic-head-1st-comp-phrase} says nothing about whether the head or non-head appears first. In my generated grammar, I have a \textit{head-comp-phrase} that inherits from both the \textit{\justify basic-head-1st-comp-phrase} above, as well as the \textit{head-initial} constraint, which simply says that the head is the leftmost element in the structure. Together with a few other constraints, this defines the basic head-complement rule in my Nuuchahnulth grammar.

Once this output from the Grammar Matrix was generated, I could then develop my own, more complex syntactic analyses. For instance, I added the below rule which allows for a dropped subject:

\begin{verbatim}
nuk-head-opt-subj-phrase := decl-head-opt-subj-phrase &
[ HEAD-DTR.SYNSEM [ LOCAL.CAT [ HEAD [ FORM finite,
                                      AUX + ],
                                VAL.SUBJ < synsem-min > ],
                    NON-LOCAL.SLASH <! !> ] ].
\end{verbatim}

\noindent This rule inherits from the \textit{decl-head-opt-subj-phrase} rule in the Grammar Matrix. It further specifies that the node having its subject dropped (the head daughter) needs to be finite, an auxiliary (which in my grammar means headed by a second position enclitic, \S\ref{ch:clause:cliticnormal}), and that it has no gapped elements (an empty \textsc{slash} list). This rule definition is not generated by the Grammar Matrix.

I have limited the scope of my work in two major ways. Firstly, I am not modeling the morphophonology. There are two reasons for this. Morphophonology is theoretically separate from morphosyntax, and as a result of that assumption the DELPH-IN toolsets are focused on the morphosyntax. Because this is a project modeling multi-predicate constructions, the morphophonology is also not the most relevant component of the grammar. What this means is that a sentence like \textit{ʔuumac̓uk̓ʷaƛaḥ quʔušin} `I am talking about Raven' is represented in my grammar in its already-segmented form, ``ʔu-L.mac̓uk =!aƛ=(m)aˑḥ quʔušin."

Secondly, I am not separating dialect features into different grammatical models. My data comes from many different dialects of Nuuchahnulth, which each have different morphemes and slightly different grammatical rules. In my grammar's lexicon, I have simply entered all dialect variations. This means that on generation, the grammar is happy to mismatch morphology from different dialects, which is an overgeneration. A larger project would catalog this information by dialect in a larger metagrammar which could then produce separate grammars targeting each dialect. While worthwhile, this project was set aside so I could focus on the multi-predicate constructions.

%In an ideal situation, I would have a grammar for each dialect, generated from a metagrammar that holds information about which grammatical rules and lexical entries belong to each dialect. However, due to time and scope constraints of the project, I simply entered different lexical entries for different dialects into the same grammar. This means that my grammar will happily mismatch morphology from different dialects, which is an incorrect overgeneration. Also due to scope constraints, my grammar also does not include a morphophonological component. This means that the second line of the IGT is what this grammar works with. That is, a sentence like \textit{ʔuumac̓uk̓ʷaƛaḥ quʔušin} `I am going to talk about Raven' is represented in the grammar as ``ʔu-L.mac̓uk =!aƛ=(m)aˑḥ quʔušin." The reasons for this are the theoretical separation of morphophonology from morphosyntax and the focus of the DELPH-IN machinery, which is squarely on morphosyntax and compositional semantics. To generate the surface string from the representation above would require the coding of a morphophonological analyzer (ideally a finite state transducer) which can go from the surface string to the segmented line and vice versa.

Development was done against multiple test suites of example sentences, which included both grammatical and ungrammatical examples. The three test suites I used are: (1) a basic test suite for basic grammatical sentences; (2) a test suite of serial verbs; (3) a test suite of linker constructions. For the first test suite, basic components of the grammar, I used simple example sentences from stories, sessions with consultants, and sentences whose grammaticality or ungrammaticality I was very confident about. In the end, a lot of this test suite was sentences that I created. For the second and third test suites, the phenomena under investigation in this document, I used only grammatical examples from my elicitation and corpora work, and ungrammatical examples from my elicitation sessions. These came from my collated data (\S\ref{ch:method:collation}).  All test suites were loaded into a \texttt{[incr tsdb()]} database \citep{oepen2001}. This test suite of sentences could be run against each version of the implemented grammar and checked for changes to the parse coverage. Beyond parsing/not-parsing, each example sentence was tested for semantic faithfulness. Semantic validation had to be done manually, but regression tests allowed for parsing results to be compared with previous iterations of the grammar rather than independently reverified every time the grammar changed.

I have focused on the parsing component of the grammar. Future work will involve focusing on generation, for which the grammatical tool sets I have used are descriptively adequate. The challenges here involve restricting dialect variation, as mentioned above, as well as restricting certain second position elements which may recurse (an issue explored in more depth in \citealt{bender2010reweaving}). These issues do not affect the descriptive validity of the analyses presented here.

%Although the DELPH-IN tool set is descriptive for the purposes of both parsing and generating sentences, I focused only on parsing. My grammar as it stands has some issues with generation. These are caused by insufficient semantic constraints introduced by some morphemes (which can thus be hypothesized endlessly by the generator), and particular difficulties introduced by inflectional second position elements, explored in depth in \cite{bender2010reweaving}. Fixing these issues in generation will require further development of the grammar.

The result of the implemented grammar is a series of files that detail the grammatical rules, the lexicon, and rules for generation. The format for most of these files is TDL, which is a series of grammatical descriptions which are equivalent to HPSG attribute-value matrices. The regression tests in \texttt{[incr tsdb()]} \citep{oepen2001} are also outputted to readable databases which show the resulting coverage of the grammar run over test cases. All of these materials are available at \url{https://bitbucket.org/davinman/nuuchahnulth-grammar/}.

