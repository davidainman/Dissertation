\chapter{Conclusion} \label{ch:conclusion}

In this work, I have collected data on and analyzed multi-predicate constructions in Nuuchahnulth, and provided descriptions of the syntactic properties of clauses under these forms of coordination. I have also given a formal analysis of these phenomena within the HPSG formalism. Though there have been a flurry of syntactic studies on Nuuchahnulth in the past 20 years \citep{nakayama2001, davidson2002, waldie2004, wojdak2005, woo2007b, waldie2012}, none of them have gone into this level of detail on the properties of these particular types of constructions.

\section{Summary of findings}

The two multi-predicate constructions I have described are serial verb constructions and predicate linker constructions. Serial verb constructions come historically in four types, which split on semantic interpretation (required simultaneity versus possible sequentiality) and verb type (locations, adposition-like verbs, and others). The modern system appears to be losing a requirement for one type of serialization to have verbs match in perfectivity, in which case the number of constructions is simplifying from four down to three.

The predicate linker coordinates any two elements that are predicates. This provides significant supporting evidence that Nuuchahnulth has a broad category of syntactic predicates which encompasses verbs, adjectives, and common nouns. All of these lexical categories introduce events in the syntax and have subjects.

Both strategies share certain properties. With the exception of 	sequential/separable serial verb constructions, both serial verb and linker constructions permit complement separation, that is, a structure like \textit{Verb1 Verb2 Verb1\_Object}. Both constructions show the causative and passive scoping narrowly over each coordinated predicate, positioned within what I've termed the \textit{maximal predicate phrase}---the largest constituent consisting of a predicate, its complements, and modifiers. Maximal predicate phrases are the units coordinated in linker constructions and SVCs (in which case they must be verbal predicates), and exclude subject, tense, and mood information, which appear at the syntactic domain of the clause and scope over any coordinated elements.

This work directly poses other questions about Nuuchahnulth grammar. In particular, if the causative and passive morphemes scope narrowly over the maximal predicate phrase, what about the other elements of the ``second-position enclitic complex"? The variable domain properties of these enclitics has been noted before (most directly in \citealt[p.~106--109,253--255]{davidson2002}), but I believe this work is the first to describe the precise syntactic domain of some of them. What are the domain properties the possessive, which is also valence-increasing, and the ``now" morpheme \textit{=!aƛ}, which frequently ``copies" or ``spreads" across a clause? Perhaps these morphemes also have a syntactic domain smaller than the full clause, a feature which can be investigated by looking at multi-predicate clauses. %Since I've shown the linker to be yet another in a class of second-position suffixes, are there other apparently simple suffixes in the language that also show a pattern of moving to a preceding modifying adverb? Perfective morphology may, in fact, behave similarly.

%The findings here also offer a place to compare with other languages. A key component of these coordination strategies is complement-stranding. Is this a feature common to serial verbs and other coordination strategies in related languages, or around the world? Similarly, are there reasons in other languages to understand adjectives and nouns, along with verbs, as part of a predicative class, complete as semantic events with syntactic subjects? Do other languages share coordination strategies like the linker that can coordinate across syntactic categories? This research will contribute to answering questions like these.

For typologists, this dissertation provides a descriptive resource for coordination structures in Nuuchahnulth and can be used for comparative work on serialization and coordination. I am myself interested in whether certain properties common to these coordination strategies---especially verb-object interruption by an intervening coordinand, and coordinating across different (predicative) syntactic categories---are more widespread within the Pacific Northwest and across the world.

\section{Summary of HPSG analysis}

Although I have presented the HPSG analysis and the more theory-neutral facts separately, this work was actually heavily intertwined. Without the work of implementing a grammar, there are several descriptive facts in this dissertation that I would not have noticed. This is how I became convinced of the basic eventiveness of common nouns, adjectives, and verbs (\S\ref{ch:clause:predp}) and how I noticed that suffix verbs have different semantic relationships with their stem depending on its lexical category (\S\ref{ch:clause:2pv}). It is also how I first noticed that even when collocated with the rest of the second-position enclitics, the causative and passive morpheme still scope narrowly over a single predicate (\S\ref{ch:sv:valence}, \S\ref{ch:link:second}), and it drove the shape of my analysis for the grammaticalization process I claim generates doubled \textit{ʔuyi} (\S\ref{ch:link:uyi}). Despite the organization of this dissertation, my description and grammar implementation were interdependent processes that informed one another.

The analyses I have presented generalize over related syntactic phenomena by creating abstract supertypes that cover all common traits. I have developed several components that may be of broader interest to those working within the formalism. The first is my analysis of second-position suffixes as a type of lexical incorporation which proceeds in two steps, the first of which prepares a word for suffixing and the second of which adds the syntactic and semantic properties of the suffix. This process requires the creative use of some valence lists and semantic pointers, and may be useful for other phenomena that resemble category-flexible lexical incorporation.%, or could even shed light on a shortcoming in the morphological paradigm of this particular implementation of the theory. %I have also developed a method for representing a null morpheme that is later overwritten by a phonologically contentful morpheme.

My analyses also depend on defining new head properties or using existing ones in new ways, such as \textsc{prd} for predicative types, and \textsc{htype} for keeping track of verbal types. These categories function in and of themselves for the analyses presented, but further evidence of their utility within Nuuchahnulth would support their utility as conceptual categories within the language (especially \textsc{htype}).

My analysis of serial verbs as a kind of coordination opens up questions about modeling serialization more generally. Are serial verbs modelable as coordination in other languages as well, or is there some property of serial verbs in Nuuchahnulth that make them more amenable to this analysis? I made some modifications to the coordination structures present in \citet{drellishakbender2005} to accommodate the properties of Nuuchahnulth serialization: in specific, the capacity for the first verb to be separated from its complement by the intervening verb phrase. It may be useful for other researchers to add this to their typological accounts of coordination.

\section{Contributions and future directions}

The main contributions coming out of this work for Nuuchahnulth and South Wakashan studies are: a fuller account of the syntax of serial verb constructions (\cref{ch:sv}); an account of the syntax and morphological properties of the linker morpheme (\cref{ch:link}); strong evidence for a targetable phrase below the level of the clause, which I call the maximal predicate phrase and which is the element involved in these coordination strategies (\S\ref{ch:sv:valence}, \S\ref{ch:link:second}); and new morphological tests for determining syntactic category using the linker (\S\ref{ch:link:application}). I have also added to the body of evidence showing that common nouns, adjectives, and verbs all introduce semantic events (\S\ref{ch:clause:predp}) and are (in my terminology) predicative. This is an analysis strongly supported by the behavior of the linker morpheme, which freely coordinates elements from these categories. My account of the lexical properties of suffix verb attachment (\S\ref{ch:clause:2pv}) and linker attachment (\S\ref{ch:link:attach}, \S\ref{ch:link:application}) should enable further investigation into the subtle distinctions of lexical categories within South Wakashan.

My implemented grammar contains several novel analyses, sometimes by using features in new ways and sometimes by developing new strategies, and I have documented those analyses both here in prose and in my implemented grammar which is publicly available.\footnote{The full grammar can be downloaded by visiting \url{http://bitbucket.org/davinman/nuuchahnulth-grammar/}} By approaching multi-predicate constructions in Nuuchahnulth from within the particular syntactic framework of head-driven phrase structure grammar (HPSG) and using a computer-readable implementation in the DELPH-IN architecture, I have made contributions both to the understanding of Nuuchahnulth grammatical structures and to particular methods for linguistic modeling.

%These contributions should inform investigation into further clausal phenomena in Nuuchahnulth, such as the variable locations and domain properties of the enclitics and their spreading or copying across the clause (\S\ref{ch:clause:cliticnormal}, \citealt[p.~253--255]{davidson2002}). 

%For those interested in HPSG analyses of syntactic phenomena, the analyses I have presented here represent some novel uses of elements of the framework to capture linguistic facts. %Most significantly in this domain are my use of the \textsc{head} feature \textsc{prd} to track the introduction of semantic events (\S\ref{ch:clause:analysis:predpart} and my two-step process for adding suffix verbs, which is a strategy for modeling lexical incorporation for suffixes that can affix to a wide variety of parts of speech (\S\ref{ch:clause:analysis:2pv}).
%Since my analysis of both multi-predicate constructions depends on coordination (\S\ref{ch:sv:analysis}, \S\ref{ch:link:analysis}), it is a natural question whether this analysis is extensible to other types of serial verb constructions in other languages, and if so what elements of the analysis are necessary for describing Nuuchahnulth specifically and which generalize.
