
%\setcounter{page}{-1}
\abstract{%
This dissertation documents and models two types of multi-predicate constructions in Nuuchahnulth: serial verb constructions, and a construction involving a suffix called the predicate linker.
I define a serial verb construction (SVC) as any clause with two verbs present and no overt coordinating element. I document the circumstances under which this occurs, give its grammatical constraints, and classify SVCs in Nuuchahnulth into four syntactically distinct categories.
I also examine the linker suffix and provide a grammatical description for it. Unlike SVCs, the linker coordinates two elements which serve as predicates in the syntax, a category which includes more than just verbs. I use the properties of the linker and SVCs to shed light on words that are category-ambiguous.
Finally, this is all implemented inside of a DELPH-IN style HPSG computational grammar. The analyses are then tested against a set of speaker-vetted sentences illustrating the phenomena.
}
